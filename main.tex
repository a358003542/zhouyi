% !Mode:: "TeX:UTF-8"

\documentclass[12pt,oneside]{book}

\newlength{\textpt}
\setlength{\textpt}{11pt}
    
\newcommand{\flypage}[1]{\begin{titlepage}\begin{center}\vspace*{\stretch{1}}#1\vspace*{\stretch{1}}\end{center}\end{titlepage}}
    
%========基本必备的宏包========%
\RequirePackage{calc,float,moresize}
%\RequirePackage[onehalfspacing]{setspace}
\linespread{1.5}
%1.3 onehalfspacing
%试卷或需要文字紧凑的
%1.6 doublespacing

%===========加入目录 某章或某节=====%
\makeatletter

\newcommand{\addchtoc}[1]{
        \cleardoublepage
        \phantomsection
        \addcontentsline{toc}{chapter}{#1}}

\newcommand{\addsectoc}[1]{
        \phantomsection
        \addcontentsline{toc}{section}{#1}}

%===========全文基本格式==========%
\setlength{\parskip}{1.6ex plus 0.2ex minus 0.2ex}   %段落間距
\setlength{\parindent}{\textpt * \real{2}}

%=========页面设置=========%
\RequirePackage[a4paper, %a4paper size 297:210 mm
  bindingoffset=10mm,%裝訂線
  top=35mm,  %上邊距 包括頁眉
  bottom=30mm,%下邊距 包括頁腳
  inner=10mm,  %左邊距or inner
  outer=10mm,  %右邊距or  outer
  headheight=10mm,%頁眉
  headsep=15mm,%
  footskip=15mm,%
  marginparsep=10pt, %旁註與正文間距
  marginparwidth=6em,includemp=true% 旁註寬度計入width%旁註寬度
  ]{geometry}

%color
\RequirePackage[table,svgnames]{xcolor}

%================字體================%
%设置数学字体
\RequirePackage{amssymb,amsmath}
\RequirePackage{stmaryrd}
\everymath{\displaystyle}

\RequirePackage{fontspec}
%設置英文字體
\setmainfont[Mapping=tex-text]{DejaVu Serif}
\setsansfont[Mapping=tex-text]{DejaVu Sans}
\setmonofont[Mapping=tex-text]{DejaVu Sans Mono}


%中文環境
\RequirePackage[]{xeCJK}
\xeCJKsetup{PunctStyle=plain}
\xeCJKDeclareSubCJKBlock{LIUSHISIGUA}{ "4DC0 -> "4DFF}
\setCJKmainfont[FallBack=DejaVu Serif, ItalicFont=TW-Kai,LIUSHISIGUA=DejaVu Sans]{Source Han Serif CN}
\setCJKsansfont[FallBack=DejaVu Sans]{Source Han Sans CN}
\setCJKmonofont[FallBack=DejaVu Sans Mono]{TW-Kai}


%%===============中文化=========%
\renewcommand\contentsname{目~录}
\renewcommand\listfigurename{插图目录}
\renewcommand\listtablename{表格目录}
\renewcommand\bibname{参~考~文~献}
\renewcommand\indexname{索~引}
\renewcommand\figurename{图}
\renewcommand\tablename{表}
\renewcommand\partname{部分}
\renewcommand\appendixname{附录}
\renewcommand{\today}{\number\year{}年\number\month{}月\number\day{}日}


%=======页眉页脚格式=========%
\RequirePackage{fancyhdr}   %頁眉頁腳
\RequirePackage{zhnumber}  %计数器中文化
\pagestyle{fancy}
\renewcommand{\sectionmark}[1]
{\markright{第\zhnumber{\arabic{section}}节~~#1}{}}

\fancypagestyle{plain}{%
    \fancyhf{}
    \renewcommand{\headrulewidth}{0pt}
    \renewcommand{\footrulewidth}{0pt}
    \fancyhf[HR]{\ttfamily \footnotesize \rightmark }
    \fancyhf[FR]{\thepage}}
\pagestyle{plain}


%=========章節標題設計=========%
\RequirePackage{titlesec}
%修改part
\titleformat{\part}{\huge\sffamily}{}{0em}{}
%修改chapter
\titleformat{\chapter}{\LARGE\sffamily}{}{0em}{}
%修改section
\titleformat{\section}{\Large\sffamily}{}{0em}{}
%修改subsection
\titleformat{\subsection}{\large\sffamily}{}{0em}{}
%修改subsubsection
\titleformat{\subsubsection}{\normalsize\sffamily}{}{0em}{}


%================目录===============%
%toc label to contents space   dynamic adjust
\RequirePackage{tocloft}%
\renewcommand{\numberline}[1]{%
  \@cftbsnum #1\@cftasnum~\@cftasnumb%
}

%==============超鏈接===============%
\RequirePackage[colorlinks=true,linkcolor=blue,citecolor=blue]{hyperref} %設置書簽和目錄鏈接等
\newcommand{\hlabel}[1]{\phantomsection \label{#1}}%某一小段的引用


%=================文字強調=========%
\RequirePackage{ulem} %下劃線,加點
\normalem%normal em , not instead of the uline

%modified udot command from the dotuline
\def\udot{\bgroup
  \UL@setULdepth
  \markoverwith{\begingroup
     \advance\ULdepth0.1ex
     \lower\ULdepth\hbox{\kern.25em . \kern.045em}%
     \endgroup}%
  \ULon}

\let\oldemph\emph % Save emph in oldemph
\renewcommand{\emph}[1]{\textcolor{red}{#1}}  

%==================插入圖片=======%
\RequirePackage{wrapfig}
\RequirePackage{graphicx}
\graphicspath{{figures/}}
%change the caption style a little like 1-1
\renewcommand{\thefigure}{\arabic{chapter}-\arabic{figure}}


%==============插入表格========%
\RequirePackage{booktabs}
\renewcommand{\thetable}{\arabic{chapter}-\arabic{table}}
\RequirePackage{caption}
%\renewcommand{\arraystretch}{1.3}
%如果用setspace宏包而不是linespread调整行间距,那么才需要额外的表格行距拉大。

%插入代码
\RequirePackage{fancyvrb} 
\fvset{frame=lines,tabsize=4 ,baselinestretch=1.8, fontsize=\footnotesize}


%==========其他宏包===========%
\RequirePackage{tikz} 
\usetikzlibrary{calc}

%========脚注=========%
\newcommand*\circled[1]{%
  \tikz[baseline=(char.base)]\node[shape=circle,draw,inner sep=0.4pt,minimum size=4pt] (char) {#1};}
\newcommand*\circledarabic[1]{\circled{\arabic{#1}}}

\RequirePackage{perpage} %the perpage package
\MakePerPage{footnote} %the perpage package command

\renewcommand*{\thefootnote}{\protect\circledarabic{footnote}}


\renewcommand\@makefntext[1]
{\vspace{5pt}
\noindent
\makebox[20pt][c]{\@makefnmark}
\fontsize{10pt}{12pt}\selectfont #1}

\setlength{\skip\footins}{20pt plus 10pt}
%main body 与脚注之间的距离


%framed环境
\RequirePackage{framed}

\newenvironment{shici}{
\begin{verse}
\centering\large\hspace{12pt}}
{\end{verse}}

\RequirePackage{indentfirst} 

\makeatother



\title{周易}
\author{万泽}
\hypersetup{
  pdfkeywords={},
  pdfsubject={制作者邮箱:a358003542@outlook.com},
  pdfcreator={万泽}}
  
\newcommand{\bookcover}[1]{\tikz[remember picture,overlay]{\node[inner sep=0] at (current page.center)
{\includegraphics[width=\paperwidth,height=\paperheight]{#1}}}} 
 

  
\begin{document}
\frontmatter 

\thispagestyle{empty}

\bookcover{book_cover.png}

\cleardoublepage

\flypage{感谢上天}


\addchtoc{引言}
\chapter*{引言}
天人感应是周易预测的基石,只有良好的天人感应,上天以卦数示人,人才可能继续做出正确的预测。没有天人感应学说,后面的一切都是站不住脚的。

我们学到的各个周易预测学派都是前人内心修身,天人感应,外在体察的经验总结。各个学派彼此是不相干的,甚至是彼此跨越几百年完全不搭架的,将这些不同的学派和不同领域的应用企图统一起来的想法是极具野心的,学者应该慎重。

我发觉试图将五行和八卦混合起来的做法可能是错误的,五行学说更适合的是在中医领域,而观星领域发展起来的紫微斗数,或者和建筑相关的奇门之术。它们在具体领域对于从卦象到更具体的物象都有各种各样的约定,所有的这些约定都是源于前人先贤们内在修身,天人感应,体察外在万物,然后解读卦象的结果。所有这些学派的发展都是遵循如下过程:\textbf{内在修身静心,诚心发问,起卦得卦象或者观察万物得卦象,用心体察解读卦象}。

这个过程是最核心本质的部分,而至于具体得出来的结论和前人总结的各个约定反倒只是一些术了。这些术你甚至可以看作某种统计规律,但离周易的本质已经很远了。

所以就以最简单的周易六十四卦预测来说,个人的修身静心和用心体察解读卦象,这个过程是缺一不可的。没有这个过程,而是试图从周易预测的各个术中寻找科学或者某种规律的东西,那就实在是南辕北辙了。

中国人的这种直觉思维模式是很有别于西方的那种逻辑思维模式的,目前来看西方的逻辑思维对于目前我们的科学贡献更大,但是也许那也只是对于宇宙存在的某一方面而言确实西方的逻辑体系说准了宇宙存在的某些东西,能够更好地对外部存在进行建模和判断。但谁也无法保证中国人的这种整体的直觉思维模式不会在下个阶段,对于宇宙存在做出更好的解释。



\section{阴阳学说}
\begin{quote}
有太极,是生两仪, 两仪生四象, 四象生八卦,八卦定吉凶,吉凶生大业。
\end{quote}


阴阳学说是如此的浅显,以至于我觉得无法再补充说些什么了,就好像就是那样,它就在哪里,存在于我们的基因中,存在于我们的记忆中,觉得不需要再多说点什么了。但实际上阴阳学说是很深的哲理,西方哲学家们从莱布尼兹到黑格尔到马克思,慢慢把辩证法发展起来付出了极大的努力,而这个努力的基石,莱布尼兹自己也承认过,是受到了中国阴阳学说的启发。

任何事物内部都存在着矛盾对立面,这个矛盾对立面就是阴和阳。然后阴阳衍生出四象。四象很是巧妙地把事物内部矛盾演化的过程进一步细化,于是可比春夏秋冬,可比市场经济繁荣危机的GDP波动图等。实际上任何事物发展演化都可以进一步细化分析,从而得到四象。

\section{八卦和物象}
\begin{quote}
乾(qián)、坤(kūn)、坎(kǎn)、离(lí)、兑(duì)、艮(gèn)、巽(xùn)、震(zhèn)
\end{quote}


假如我们问上天的问题是“是或者不是”这样简单的问题,那么是就是阳,而不是就是阴,一个爻的卦象就够用了。但某些问题必然要求具有更大的信息量,于是人们提出八卦,并将八卦和这个世界不同的万事万物的物象对应起来。读者可以随便搜一搜就可以搜到八卦和物象的映射关系,实际上我们如果问上天的问题最后答案是指明某个物的话,那么分析卦象里面的八卦对应的物象,就能够大致猜测出来上天启示你的那个物品是什么。

我对八卦和这些物象的映射关系持保留意见,可做参考吧。类似的还有后天八卦映射方位的理论,可以在你想特别预测某个方位的时候权做参考吧。





\section{如何摇卦}
蓍草占卦就不说了,要铜钱六十四卦方法因为以前有兴趣学了一下,下面说明一下。

我看到有人批判说电脑计算随机数摇卦是不准的,一定要用铜钱。我觉得说这话的人一定要用蓍草占卜,否则对不起他的纯真坚持的心。重要的是天人感应,是铜钱也好,电脑生成的随机数也好,都不过是过程是手段是术罢了,只要保证这个过程不是掩耳盗铃,就是可行的。

具体是一个爻要抛三次硬币或者生成三个0或1的随机数:

\begin{itemize}
\item 如果加和为3,也就是都是阳,则为老阳爻,本卦得阳爻,变卦得阴爻
\item 如果加和为2,也就是两阳一阴,则为少阳爻,本卦得阳爻,变卦得阳爻
\item 如果加和为1,也就是一阳两阴,则为少阴爻,本卦得阴爻,变卦得阴爻
\item 如果加和为0,也就是都是阴,则为老阴爻,本卦为阴爻,变卦为阳爻
\end{itemize}


如此重复得到周易六十四卦预测的本卦和变卦。

\section{易者谨记}
除了自身的修身静心和用心体察解读卦象之外,无疑不占也是一大准则。若是你的内心的疑问困惑并不是一个真正的问题,你的内心是知道答案的,那么强行去占卦只是试图遮蔽自己的内心,占卦也将是没有意义的;因为天人感应指的是上天和你的内心发生感应。

天道无亲,常与善人,常罚恶人。

周易预测可知天命,天道有常然人命无常,人子善恶一念皆可改命。此易者不可不牢记在心!

王阳明:“人但得好善如好好色,恶恶如恶恶臭,便是圣人。”

善善恶恶,还有更多的道理吗,我看是没有了啊。

\section{六十四卦解卦入门}
在得到本卦和变卦之后,我们就可以查阅周易一书里面的内容了。从我个人的经验来看,变爻的一般为一个到两个,这些情况基本上是没有争议的:

\begin{itemize}
\item 无变爻 以本卦卦辞断之
\item 一个变爻 以该爻的爻辞断之
\item 两个变爻 以两个爻的爻辞共同断之,这里朱熹说 \verb+以上者断之+ ,我觉得说得不够确切,因为周易的各个爻辞明显可以看出是在讲述一个事件的发展经过的过程,所以更确切来说是这两个爻是你预测的该事件在那两个发展阶段存在变数的情况,\verb+以上为主+ 含义是以事情最终那个变化情况为主,但我觉得最好说成是由这两个爻的爻辞共同断之。
\item 更多的变爻的时候我会觉得这次摇卦不是很准了,那个时候我还没有接触朱熹的变爻断法,我觉得可以作为参考\footnote{但也只是参考,除开预测者的非常规预测需求,以一般的事件预测来说,我觉得还是要以本卦为主,只是事情会在很短时间内发生很多变化,所以该卦的时效性可能会很短} 。

具体朱熹关于多个变爻的断法如下所示:
\begin{quotation}
三个变爻以本卦与变卦卦辞断;本卦为贞(体),变卦为悔(用)

四个变爻以变卦之两不变爻爻辞断,但以下者为主

五个变爻以变卦之不变爻爻辞断

六个变爻以变卦之卦辞断,乾坤两卦则以「用」辞断
\end{quotation}

\end{itemize}



\section{关于本书基本术语}
经是周易最开始的内容,

具体解卦最先的是卦象,根据你提出的问题分析卦象。卦象也就是那两个八卦叠起来的象。相传卦辞是文王写的,爻辞是周公写的。然后下面还有彖(tuàn)辞、象辞是孔子和他的门人写的。


\addchtoc{目录}
\setcounter{tocdepth}{2}    
\tableofcontents



\mainmatter
\part{周易}

\chapter{乾卦}
\section{原文}
乾  {\Large ䷀}


\subsection{经}
乾:元亨,利贞。

初九:潜龙,勿用。

九二:见龙在田,利见大人。

九三:君子终日乾乾,夕惕若厉,无咎。

九四:或跃在渊,无咎。

九五:飞龙在天,利见大人。

上九:亢龙有悔。

用九:见群龙无首,吉。


\subsection{彖}
大哉乾元,万物资始,乃统天。云行雨施,品物\footnote{参看汉典“品物”词条,品物即万物。}流形。大明终始,六位时成,时乘六龙以御天。乾道变化,各正性命,保合大和,乃利贞。首出庶物,万国咸宁。

\subsection{象}
天行健,君子以自强不息。

初九:潜龙勿用,阳在下也。

九二:见龙在田,德施普也。

九三:终日乾乾,反复道也。

九四:或跃在渊,进无咎也。

九五:飞龙在天,大人造也。

上九:亢龙有悔,盈不可久也。

用九:用九,天德不可为首也。


\section{初讲}
卦辞说:乾卦,大亨通,利于贞正之道。

彖说:伟大啊,乾元,万物因此开始,于是统领天下。云气流行,雨水布施,万物运动变化而各具成形。从上到下对应万物的结束到开始,其都是阳爻,六爻应时而形成,时乘六龙来驾御天道。天道的变化,万物各个正定其本性和运命,保全太和之气,是故“利贞”。始出万物,万国皆得安宁。

象辞说:天行健,君子以自强不息。

初九:潜龙勿用,阳在下也。此时你需要隐忍,还没有大展才干的时候。

九二:龙出现在了田野,适合去见到大人。龙在田不得位也。此利见大人是因为龙不得位,和九五的利见大人是有区别的,

九三:君子终日自强不息,到了晚上也不放松警惕如入危险之境,无咎。

九四:或腾跃而起,或退居深渊,无咎。

九五:飞龙在天,利见大人。九五的利见大人是因为你的飞龙在天是大人成就的【大人造也】。

上九:亢龙有悔。亢龙有悔,盈不可久也。

用九:用九是乾卦全部是阳爻,也即是乾卦本卦。一群龙却没有首领,吉祥。为什么吉祥呢?天生万物不居功不为首,是故吉祥。


\chapter{坤卦}
坤 {\Large ䷁}

\section{原文}
\subsection{经}
坤:元亨,利牝马之贞。君子有攸往,先迷后得主,利西南得朋,东北丧朋。安贞吉。

初六:履霜,坚冰至。

六二:直方大,不习无不利。

六三:含章可贞。或从王事,无成有终。

六四:括囊;无咎,无誉。

六五:黄裳,元吉。

上六:龙战于野,其血玄黄。

用六:利永贞。

\subsection{彖}
至哉坤元,万物资生,乃顺承天。坤厚载物,德合无疆。含弘光大,品物咸亨。牝马地类,行地无疆,柔顺利贞。君子攸行,先迷失道,后顺得常。西南得朋,乃与类行;东北丧朋,乃终有庆。安贞之吉,应地无疆。

\subsection{象}
地势坤,君子以厚德载物。

初六:履霜坚冰,阴始凝也。驯致其道,至坚冰也。

六二:六二之动,直以方也。不习无不利,地道光也。

六三:含章可贞;以时发也。或从王事,知光大也。

六四:括囊无咎,慎不害也。

六五:黄裳元吉,文在中也。

上六:龙战于野,其道穷也。

用六:用六永贞,以大终也。

\section{初讲}
卦辞说:坤卦,大亨通,利牝马之贞。君子有所前往,先迷而后得主。西南得朋有利,东北丧朋。安贞吉\footnote{安贞不可浅解为安于现状,牝马之贞指贞正之道偏坤德,安贞指贞正之道偏安守。}。

彖说:伟大啊,坤元。万物因你而生,顺承天道。坤用厚德载养万物,德性与天相合而无边无疆。包容博厚而广大,万物皆得亨通。牝马属于地上的生物,奔行于地而没有疆界,牝马性柔顺而利于贞正之道。君子有所前往,先迷失其道,后顺从于常道\footnote{即天道}。西南得到朋友,乃与之同行;东北失去朋友,最终有喜庆之事。安贞何以吉祥,应和地道的无边无疆。

象辞说:坤象征大地,君子应该效仿大地,胸怀宽广,包容万物。


初六:脚踩于霜而知气候变冷,坚冰将至。脚踩于霜,说明阴气开始凝结,照这个情况发展下去,必然迎来坚冰。本爻在提醒占者要见微知著,防微杜渐。

六二:直方大,此大地的形状。六二的行动如同大地的形状一样正直方正,即使不用修习也是无不利的。这是因为他将地道发扬光大了。

六三:胸怀美德和才华可以守正道,或许能够跟从君王做事,没有功名成就事情却有好的结果。

六四:将口袋扎紧,无咎亦无誉。此爻有守口如瓶之象,如此谨慎才得无害。

六五:黄色的下衣,吉祥。为何吉祥?温文之德在心中。

上六:与龙相战于旷野,龙流下了玄黄色的血。阴气极盛,地道也发展到头了。此爻提醒占者阴气过头复转为阳的道理。

用六:用六是坤卦全部是阴爻,也即是坤卦本卦。利于永远坚守正道,用六的永远坚守贞正之道,可用来得到大的善终。





\chapter{水雷屯(zhūn)卦}
屯 {\Large ䷂}

\section{原文}
\subsection{经}
屯:元亨,利贞,勿用有攸往,利建侯。

初九:磐桓;利居贞,利建侯。

六二:屯如邅(zhān)如,乘马班如。匪寇婚媾,女子贞不字,十年乃字。

六三:既鹿无虞,惟入于林中,君子几,不如舍,往吝。

六四:乘马班如,求婚媾,往吉,无不利。

九五:屯其膏,小贞吉,大贞凶。

上六:乘马班如,泣血涟如。

\subsection{彖}
屯,刚柔始交而难生,动乎险中,大亨贞。雷雨之动满盈,天造草昧,宜建侯而不宁。

\subsection{象}
云,雷,屯;君子以经纶。

初九:虽磐桓,志行正也。以贵下贱,大得民也。

六二:六二之难,乘刚也。十年乃字,反常也。

六三:既鹿无虞\footnote{虞:虞人,掌管山泽之官。},以从禽也。君子舍之,往吝穷也。

六四:求而往,明也。

九五:屯其膏,施未光也。

上六:泣血涟如,何可长也。

\section{初讲}
卦辞说:屯卦,大亨通,利于正道。不要有所前往,适宜建国立候。【屯卦还是很吉祥的,也利于君子去守正道,只是不要轻举妄动,适合做一些大事情的筹备工作。】

彖说:屯卦,阳刚之气和阴柔之气刚开始交汇,困难也随之而生,危险也在其中涌动,但终是大亨通,利贞的。雷雨之动满布大地,而上天的造化开物还处于蒙昧状态,适宜于建国立候,但不安宁。

象辞说:屯卦上面是云下面是雷,君子观此象而经纶天下。

【屯卦指事情才刚开始,寓意天地也是云雷互动而困难重重,此时不要轻举妄动,但也不要倦怠安宁,而应该效法此时云雷互动中上天经纬造化万物之象,立大志,定大计,定好大的方向和规划。经天纬地,如此安能不是大亨通和利贞的呢。】

初九:徘徊不前,利于安居的贞德,利于规划建侯般的伟业。虽然徘徊不前,但是志向的操行都是纯正的。珍贵基层的平民百姓,会大大得到民心。

六二:困难啊难以前行啊,骑着马徘徊不前,不是匪寇是来商量嫁娶事宜的。女子坚守贞道不嫁,十年后才出嫁。六二之难,柔乘刚也。十年后才出嫁,回归正常罢了。

六三:追逐野鹿却没有山林之官作向导,只是跟着进入山林之中。君子明察之,不如舍弃吧,前往会有悔恨。即鹿无虞,只是跟着野兽跑。君子请舍弃吧,前往会有悔恨并陷入穷困。

六四:骑着马徘徊不前,去求婚,前往吉祥,无不利。为求婚而前往,是明智的选择。

九五:自我囤积财富,小贞吉,大贞凶。为什么会大贞凶呢?并没有将布施之德发扬光大。

上六:骑着马徘徊不前,悲泣不已,流泪如流血一样,涟涟不断。泣血涟如,这种状况怎么能长久呢。



\chapter{山水蒙卦}
蒙 {\Large ䷃}

\section{原文}
\subsection{经}
蒙:亨。 匪我求童蒙\footnote{童蒙:蒙昧的孩童。},童蒙求我。初筮告,再三渎,渎则不告。利贞。

初六:发蒙,利用刑人,用说\footnote{说,脱也。}桎梏,以往吝\footnote{参看重庆文理学院学报2006年1月出版的第五卷第1期田小中,“以往吝”衍文呢考,其认为这个以字只是一个衍文即多余的字,也就是这里应该解读为往吝。个人认为有一定的可信度。}。

九二:包蒙吉;纳妇吉;子克家。

六三:勿用娶女;见金夫,不有躬,无攸利。

六四:困蒙,吝。

六五:童蒙,吉。

上九:击蒙;不利为寇,利御寇。

\subsection{彖}
蒙,山下有险,险而止,蒙。蒙亨,以亨行时中也。匪我求童蒙,童蒙求我,志应也。初筮告,以刚中也。再三渎,渎则不告,渎蒙也。蒙以养正,圣功也。


\subsection{象}
山下出泉,蒙;君子以果行育德。

初六:利用刑人,以正法也。

九二:子克家,刚柔接也。

六三:勿用娶女,行不顺也。

六四:困蒙之吝,独远实也。

六五:童蒙之吉,顺以巽也。

上九:利用御寇,上下顺也。

\section{初讲}
卦辞说:蒙,亨通。不是我有求于童蒙,而是童蒙有求于我。【上天教育我们也好比教育童蒙一样】初次占筮就告诉,再三反复占筮就是亵渎了,亵渎则上天不会予以告知了。利贞。

彖说:蒙卦,上为艮为山为止,下为坎为水为险,故曰山下有险,险而止,这就是蒙卦了。蒙卦是亨通的,是因为他以亨道行动,得“时中”也。匪我求童蒙,童蒙求我,是因为童蒙有志于此。初筮告,是因为以刚正居于其中。再三渎,渎则不告,这个亵渎正是蒙昧的表现。将蒙昧培养入正道,这是圣人的功绩啊。

象辞说:山下冒出泉水,这就是蒙卦的卦象了。君子见此而知要以果敢的行动来培育良好的品德。

初六:启发蒙昧,利于用刑罚来约束人,让其将来免脱桎梏之苦。还在发蒙阶段贸然前往会有悔恨。利用刑人,是用来确立正确的法度。

九二:能够包容蒙昧,吉祥。娶媳妇吉祥,子女们能够持家了。子女们能够持家了,是因为刚爻与柔爻相接的缘故。

六三:不要娶这样的女人啊,她见到有钱的男人就委身于他,没什么好处。勿用娶女,是因为六三以柔乘九二之刚,没有顺从的美德\footnote{参考了\href{http://www.guoxuez.com/64gua/menggua/23431.html}{这个网页} 。}。

六四:困在蒙昧之中,悔恨啊。六四被困在蒙昧之中,是因为它这个柔爻独自远离于充实的阳爻。

六五:蒙昧的孩童,吉祥。童蒙之吉,是因为他能够对上面的阳爻也就是老师采用谦逊的态度来顺从他。

上九:上九击打蒙昧。不要把蒙昧当作匪寇一般击打,最好如艮山一般被动防御。利用御寇,是因为上下一心相互顺从的缘故。




\chapter{水天需卦}
需 {\Large ䷄}

\section{原文}

\subsection{经}
需:有孚,光亨,贞吉。利涉大川。

初九:需于郊。利用恒,无咎。

九二:需于沙。小有言,终吉。

九三:需于泥,致寇至。

六四:需于血,出自穴。

九五:需于酒食,贞吉。

上六:入于穴,有不速之客三人来,敬之终吉。

\subsection{彖}
需,须\footnote{此处须作等待之意。}也;险在前也。刚健而不陷,其义不困穷矣。需有孚,光亨,贞吉。位乎天位,以正中也。利涉大川,往有功也。

\subsection{象}
云上于天,需;君子以饮食宴乐。

初九:需于郊,不犯难行也。利用恒,无咎;未失常也。

九二:需于沙,衍\footnote{衍,宽衍,指内心宽宏大量。}在中也。虽小有言,以吉终也。

九三:需于泥,灾在外也。自我致寇,敬慎不败也。

六四:需于血,顺以听也。

九五:酒食贞吉,以中正也。

上六:不速之客来,敬之终吉。虽不当位,未大失也。

\section{初讲}
卦辞说:有诚信,光明亨通,贞吉。利涉大川。

彖说:需卦,等待的意思。需卦上为坎为险,下为乾为刚健,故危险在前方,但刚健而不陷,因天道正义遇险能通,而不会困穷。需有孚,光亨,贞吉。是因为九五爻位居天子位,居正而得中道。利涉大川是因为前往会有功绩。

象辞说:需卦上为坎为水,下为乾为天,水在天上称之为云,故曰需卦是云在天上之象。君子观此卦象,借饮食宴乐来积蓄精力,等待时机。


初九:等待于郊外,利于持之以恒,无咎。等待于郊外,没有犯难而行。利于恒无咎,未失天地常理也。

九二:等待于沙滩,小人有所言,终吉。等待于沙滩,内心宽宏大量,虽然小人有所言,但最终是吉祥的。

九三:等待于泥沼中,导致贼寇到来。等待于泥沼中,灾祸就在外面,是自我招来的贼寇,敬告人们要谨慎小心才不会陷入失败。

六四:等待于血泊\footnote{阴阳会战之地乃见血}中,刚刚从洞穴\footnote{坎为险为阴暗险陷之所。}中出来。此时六四身处杀伤之地,要柔顺听从九五的劝告方能脱险。

九五:等待于酒食中,贞吉。酒食贞吉是因为九五爻居中而得正的缘故。

上六:进入洞穴之中,有不速之客三人来,敬之,终吉。不速之客来,敬之终吉,三个不速之客,指内卦的三阳爻,象曰不当位,指上六入于穴,未大失也,仍是终吉之意。



\chapter{天水讼卦}
讼 {\Large ䷅}

\section{原文}
\subsection{经}
讼:有孚,窒惕\footnote{参见汉典窒惕词条,为汉语词汇,意为恐惧。},中吉。终凶。利见大人,不利涉大川。

初六:不永所事,小有言,终吉。

九二:不克讼,归而逋\footnote{bū,逃亡。},其邑人三百户,无眚\footnote{灾祸}。

六三:食旧德,贞厉\footnote{厉,危也。},终吉,或从王事,无成。

九四:不克讼,复即命,渝\footnote{yú,改变,比如忠贞不渝。},安贞吉。

九五:讼元吉。

上九:或锡\footnote{xī,锡通赐}之鞶带\footnote{pán dài,古代官员的服饰},终朝三褫\footnote{chǐ,褫夺}之。

\subsection{彖}
讼,上刚下险,险而健,讼。讼有孚,窒惕中吉,刚来而得中也。终凶;讼不可成也。 利见大人;尚中正也。不利涉大川;入于渊也。

\subsection{象}
天与水违行,讼;君子以作事谋始。

初六:不永所事,讼不可长也。虽有小言,其辩明也。

九二:不克讼,归逋窜也。自下讼上,患至掇也。

六三:食旧德,从上吉也。

九四:复即命,渝,安贞;不失也。

九五:讼元吉,以中正也。

上九:以讼受服,亦不足敬也。


\section{初讲}
卦辞说:有诚信,内心恐惧,中吉,终凶。利见大人,不利涉大川。

彖说:讼卦上面为乾卦为刚,下面为坎卦为险。虽遇险但有刚健,这就是讼卦了。讼,有孚,窒惕,中吉,这是因为上面刚健的三爻来到坎险之地而居九二之中的缘故。终凶,争讼之事没有成功。利见大人,这是因为崇尚九五的中正之德。不利涉大川,这是因为入于坎险的深渊之地。

象辞说:坎卦上为天下为水,天从东向西转动\footnote{此处大概取太阳东升西落之意},水从西向东流,天水相背而行,这就是讼卦的卦象了。君子观此卦象而知道作事须谋始。

初六:不纠缠于争讼之事\footnote{此讼事很小,故曰事而不言讼。},虽然小人有所言论,终吉。不永所事,此争讼之事并不长久。虽有小小的责难之言,但通过辩解就能将是非曲直说明白了。

九二:不能胜讼,回来后又马上逃亡出去了。他所在村邑下有三百户人家,并没有受到牵连的灾祸。不克讼,是故回来后又马上逃窜中。以九二之下和上面有权有势的人争讼,灾祸是自己找的啊。

六三:享用旧有的德业,贞厉,守正道是危险的,但最终会获得吉祥。可能会随君王作事,但并没有功名成就。食旧德,六三柔顺顺从上面的刚健三爻,还是能收获吉祥的。

九四:不能胜讼,继而复归到原来的命运状态,改变后,固守安贞而得吉祥。“复即命,渝,安贞”,并没有什么损失。

九五:争讼之事结果大吉祥。讼元吉,是因为九五的中正之德。

上九:可能会受到君王赐赏官服大带,一天之内却多次被褫夺。因为争讼而受到赏赐,这没什么值得尊敬的。



\chapter{地水师卦}
师 {\Large ䷆}

\section{原文}
\subsection{经}
师:贞,大人\footnote{有文作丈人,高岛断易认为此处应该按照子夏传所说的是大人吉,个人认为是正确的。假设是丈人则周易全文只有此处出现丈人一词,此其一也;再就是从师卦的含义来看,能够领众以做行军之事的,也必是一个大人了。}吉,无咎。

初六:师出以律,否臧凶。

九二:在师中,吉,无咎,王三锡命。

六三:师或舆尸\footnote{以车运尸},凶。

六四:师左次,无咎。

六五:田有禽,利执言,无咎。长子帅师,弟子舆尸,贞凶。

上六:大君有命,开国承家,小人勿用。

\subsection{彖}
师,众也,贞,正也,能以众正,可以王矣。刚中而应,行险而顺,以此毒\footnote{通“督”}天下,而民从之,吉又何咎矣。

\subsection{象}
地中有水,师;君子以容民畜众。

初六:师出以律,失律凶也。

九二:在师中吉,承天宠也。王三锡命,怀万邦也。

六三:师或舆尸,大无功也。

六四:左次无咎,未失常也。

六五:长子帅师,以中行也。弟子舆尸,使不当也。

上六:大君有命,以正功也。小人勿用,必乱邦也。

\section{初讲}
卦辞说:师,贞,大人吉,无咎。

彖说:师,众也,贞,正也。师卦下为坎为水为众,上为坤为地为众,故曰师为众;贞,守正道也。能够使众人皆守正道,可以王天下矣。九二刚爻居中而应六五,行于坎险之地而得坤顺。以此治理天下,而民众从之,吉又何咎矣。

象辞说:师卦上为坤为地,下为坎为水,地中有水,这就是师卦的卦象了。君子观此卦象而知容民畜众\footnote{畜养民众}。

初六:出师征战必须要有严明的纪律,否则就会蕴藏凶险。师出以律,失律凶也。

九二:九二在军中任统帅,吉,无咎,君王多次嘉奖。在师中吉,是因为承受六五君王的宠幸。王三锡命,是因为君王如坤胸怀万邦。

六三:出兵可能会有舆尸之败,凶险。师或舆尸,大败是也。

六四:率军撤退,无咎。左次无咎,不违背用兵有进有退的常理是也。

六五:田野里有野禽,利于发表言论,无咎。九二长子带兵出征,六三弟子以车载尸,贞凶。长子帅师,是因为九二以中正而行之,六三弟子舆尸,是六三这个弟子任用不当所致啊。

上六:六五君王有赏命于上六,开国承家\footnote{孔颖达疏:“若其功大,使之开国为诸侯;若其功小,使之承家为卿大夫。”},小人勿用。大君有命,以正功也。小人勿用,必乱邦也。



\chapter{水地比卦}
比 {\Large ䷇}

\section{原文}
\subsection{经}
比:吉。原筮,元\footnote{参见汉典元字解,元有始大的意思,此处应作始解。}永贞,无咎。不宁方来,后夫凶。

初六:有孚比之,无咎。有孚盈缶\footnote{fǒu,古代一种盛酒的瓦器},终来有它吉。

六二:比之自内,贞吉。

六三:比之匪人。

六四:外比之,贞吉。

九五:显比,王用三驱,失前禽。 邑人不诫,吉。

上六:比之无首,凶。

\subsection{彖}
比,吉也,比,辅也,下顺从也。原筮,元永贞,无咎,以刚中也。不宁方来,上下应也。后夫凶,其道穷也。

\subsection{象}
地上有水,比;先王以建万国,亲诸侯。

初六:比之初六,有它吉也。

六二:比之自内,不自失也。

六三:比之匪人\footnote{个人认为王弼的注解是正确的,四自外比,二为五贞,近不相得,远则无应,所与比者,皆非已亲,故曰“比之匪人”。},不亦伤乎!

六四:外比於贤,以从上也。

九五:显比之吉,位正中也。舍逆取顺,失前禽也。邑人不诫,上使中也。

上六:比之无首,无所终也。

\section{初讲}
卦辞说:比卦,吉祥。原来就占卜过了的,从一开始就永守正道,无咎。因上为坎卦为险,故曰不安宁的事将要来到,后到的人有凶险啊。

彖说:比卦是吉祥的,比卦有辅佐的意思。下卦为坤为顺故曰下顺从也。原筮,元永贞,无咎,这是因为九五以刚健而居中正之位,不宁方来,因为上为坎为险所以不安宁的事将要到来,继而上下五阴爻和九五相应,后到的人有凶险啊,这是指上六爻亲比辅佐之道走到头了,竟然居于九五之上,以阴乘阳,傲慢而后到,自然凶险可知矣。

象辞说:比卦下为坤为地,上为坎为水,故曰地上有水,这就是比卦的卦象了。先王以封建万国来亲近诸侯。

初六:初六有诚信来亲比九五,无咎。其诚信就好比美酒从容器中满盈而出,终于将来有其他的吉祥来到。比之初六,有它吉也。

六二:比之自内,贞吉。六二爻乃内卦之主,故曰亲比来自内,比之自内,这是因为六二爻没有自我迷失正道啊。

六三:比之匪人,不亦伤乎。九五与六二内比,与六四外比,孤独六三却不是九五亲近的人,不是很令人悲伤吗。

六四:外比之,贞吉。下卦为内,上卦为外,六四爻向外亲比九五,贞吉。外比于九五贤君,来顺从尊上。

九五:九五彰显比卦之道。君王用三驱之礼狩猎,失去了前面的禽兽,封地上的人也不警戒,吉祥。显比之吉,是因为九五爻位居中正的缘故。不逆天而行而顺其自然,所以失掉前面的禽兽了。邑人不诫,这是因为九五君上以中正治国的缘故。

上六:上六爻比之无首,凶。比之无首,不得善终。



\chapter{风天小畜(xù)卦}
小畜\footnote{畜,止也,止则聚矣——程颐} {\Large ䷈}

\section{原文}

\subsection{经}
小畜:亨。密云不雨,自\footnote{从或由}我西郊。

初九:复自道,何其咎,吉。

九二:牵\footnote{牵连的意思,参考了\href{http://baike.yidao5.com/jingzhuan/xiaoxugua/6428.shtml}{这个网页}。}复,吉。

九三:舆说\footnote{通脱}辐\footnote{轮中直木},夫妻反目。

六四:有孚,血\footnote{血者,恤也。可解释为忧愁远去,但血在周易中是有特殊含义的,此处即指九三和六四阴阳相斗。}去惕出,无咎。

九五:有孚挛\footnote{挛(luán),係(xì 通系,名为系物的绳子)也。此处指如同绳子系起来一样紧密相连的样子}如,富以其邻。

上九:既雨既处\footnote{既雨,和也。既处,止也。阴之畜阳,不和则不能止。既和而止,畜之道成矣。——程颐。},尚德载\footnote{载者,积也。},妇贞厉。月几望,君子征凶

\subsection{彖}
小畜;柔得位,而上下应之,曰小畜。健而巽,刚中而志行,乃亨。密云不雨,尚\footnote{通上}往也。自我西郊,施未行也。

\subsection{象}
风行天上,小畜;君子以懿\footnote{懿为美。君子观小畜之象,以懿美其文德,——程颐。}文德。

初九:复自道,其义吉也。

九二:牵复在中,亦不自失也。

九三:夫妻反目,不能正室也。

六四:有孚惕出,上合志也。

九五:有孚挛如,不独富也。

上九:既雨既处,德积载也。君子征凶,有所疑也。


\section{初讲}
卦辞说:小畜,亨通。浓云密布不下雨,从我西郊而来。

彖说:小畜,六四柔爻得位\footnote{柔爻居阴位},而上下诸阳爻与之相应,这就是小畜卦了。内卦为乾为刚健,外卦为巽为逊顺。九二九五阳刚之爻居中位而志在必行,是故亨通。密云不雨,此云向上行的缘故。自我西郊,雨施还未实行。

象辞说:小畜,上为巽为风,下为乾为天,风在天上行,这就是小畜的卦象了。君子观此卦象来修美自己的文章才艺之德。

初九:初九阳爻得位本欲进,但与六四相应而止,返回自己原先的道路,那里来的过错呢,吉祥。

九二:九二爻与初九爻\footnote{王弼解为和九五爻相牵连,个人赞同高岛断易的观点,是和初九爻相牵连。}相牵连而返回原处,得内卦之中位,[和初九爻一样]也不自失其道,吉祥。

九三:车脱其幅,夫妻反目。这是因为九三阳爻被六四阴爻凌驾而止,丈夫不能端正家室的缘故。

六四:有诚信,[血斗的]忧愁远去,从惧怕中出来,无咎。这是因为九五君上和六四爻心志相合的缘故。

九五:九五爻[和六四爻]彼此有诚信,紧密相连的样子。九五爻让它的邻居[六四爻]更富有,这是因为九五爻不独富的缘故。

上九:雨已下也已经停了,这是小畜之德积载发展的自然结果。妇贞厉,月几望,君子征凶。君子征凶,这是有所疑虑的缘故。


\chapter{天泽履(lǚ)卦}
履 {\Large ䷉}

\section{原文}

\subsection{经}
履:履虎尾,不咥\footnote{dié,咬}人,亨。

初九:素履往,无咎。

九二:履道坦坦\footnote{道之平也},幽人贞吉。

六三:眇\footnote{眇,一目小也——《说文》}能视,跛能履,履虎尾,咥人,凶。武人为于大君。

九四:履虎尾,愬愬\footnote{shuò,恐惧的样子},终吉。

九五:夬\footnote{guài,决也}履,贞厉。

上九:视履考祥,其旋\footnote{旋,回首,三省吾身之谓也。——《执象易注》}元吉。

\subsection{彖}
履,柔履刚也。说而应乎乾,是以履虎尾,不咥人,亨。刚中正,履帝位而不疚,光明也。

\subsection{象}
上天下泽,履;君子以辩上下,定民志。

初九:素履之往,独行愿也。

九二:幽人贞吉,中不自乱也。

六三:眇能视;不足以有明也。跛能履;不足以与行也。咥人之凶;位不当也。武人为于大君;志刚也。

九四:愬愬终吉,志行也。

九五:夬履贞厉,位正当也。

上九:元吉在上,大有庆也。

\section{初讲}
卦辞说:履卦,走路踩到老虎尾巴,老虎却不咬人,亨通。

彖说:履卦,下为兑为柔弱,上为乾为刚健,故曰以柔履刚。下卦为兑为悦,上卦为乾为天,心悦诚服地应对上天\footnote{王弼在此处作六三柔爻讲,不是很认同,因为从周易全文行文风格思路来看这里多以上下卦来分析的。},所以就算踩到老虎尾巴,老虎也不咬人。亨通。九五刚正而居中,登上帝位而不愧疚,这是因为内心光明的缘故。

象辞说:履卦上为乾为天,下为兑为泽,这就是履卦的卦象了。君子观此卦象来分辨上下之别,安定人民之志\footnote{夫上下之分明,然后民志有定,民志定,然后可以言治。——程颐。}。

初九:穿着朴素的鞋\footnote{此处素履直义是朴素的鞋,但也有衍生之义即素行。}前往,无咎。素履之往,独行己之愿是也。

九二:行进之路坦坦,幽居之人贞吉。幽人贞吉,这是九二爻居中位而不自乱其操行的缘故也。

六三:眼睛有病眇之勉强能看,腿脚有病跛之勉强能走路。踩到老虎尾巴,老虎咬人,凶。好武之人要做大君。眇能视;不足以有明也。跛能履;不足以与行也。咥人之凶;位不当也\footnote{此处指六三爻以阴爻居阳位,不得位。}。武人\footnote{此武人色厉内荏,内为阴爻,外为爻变的阳。}为于大君;志刚也\footnote{六三爻为履卦唯一的阴爻,本不得位,再爻变为阳爻,意欲刚强统领其他诸阳爻,这本是九五大君该做的。}。

九四:踩到老虎尾巴,很是恐惧的样子,最终会吉祥。愬愬终吉,这是九四爻志向得到实行的缘故。

九五:九五爻刚决而行,贞厉。夬履贞厉,这是因为九五爻得位而且尊位,如果刚决而行,则会失之专断,这是很危险的,故曰贞厉。

上九:审视履行的道路,考察祸福吉凶之预兆,这样的反身自省,大吉祥。上九的元吉,是因为履道大成,必有庆也。



\chapter{地天泰卦}
泰 {\Large ䷊}

\section{原文}

\subsection{经}
泰\footnote{泰者,通也。——《序卦》}:小往大来,吉,亨。

初九:拔茅茹\footnote{茅之为物,拔其根而相牵引者也。“茹”,相牵引之貌也。——王弼},以其汇\footnote{汇者,类也。},征吉。

九二:包荒,用冯河\footnote{不敢暴虎,不敢冯河——《诗·小雅》,暴虎冯河是一个成语,空手打虎,徒步渡河,喻冒险蛮干。},不遐遗。朋亡\footnote{无偏无私,不朋党。},得尚\footnote{尚,配也}于中行。

九三:无平不陂\footnote{bēi,斜坡},无往不复,艰贞无咎。勿恤其孚,于食有福。

六四:翩翩不富,以其邻,不戒以孚\footnote{六四处泰之过中,以阴在上,志在下复,上二阴亦志在趋下。翩翩,疾飞之貌,四翩翩就下,与其邻同也。邻,其类也,谓五与上。夫人富而其类从者,为利也,不富而从者,其志同也。三阴皆在下之物,居上乃失其实,其志皆欲下行,故不富而相从,不待戒告而诚意相合也。——程颐}。

六五:帝乙归妹,以祉\footnote{zhǐ,福也。——《说文》}元吉。

上六:城复\footnote{通覆}于隍\footnote{隍者,城壕也,无水为隍,有水为池——高岛断易},勿用师。自邑告命,贞吝。

\subsection{彖}
泰,小往大来,吉,亨。则是天地交而万物通也;上下交而其志同也。内阳而外阴,内健而外顺,内君子而外小人,君子道长,小人道消也。

\subsection{象}
天地交,泰,后\footnote{后,继君体也——《说文》}以财\footnote{通裁}成天地之道,辅相天地之宜\footnote{财成,谓体天地交泰之道,而财制成其施为之方也。辅相天地之宜,天地通泰则万物茂遂。人君体之而为法制。使民用天时,因地利,辅助化育之功,成其丰美之利也。——程颐},以左右\footnote{治理也}民。

初九:拔茅征吉,志在外也。

九二:包荒,得尚于中行,以光大也。

九三:无往不复,天地际也。

六四:翩翩不富,皆失实也。不戒以孚,中心愿也。

六五:以祉元吉,中以行愿也。

上六:城复于隍,其命乱也。

\section{初讲}
卦辞说:泰卦,小往大来\footnote{阳爻代表阳气为大,阴爻代表阴气为小,阴气都离去到了外卦,阳气都来到了内卦,故曰小往大来。},吉祥,亨通。

彖说:泰卦,小往大来,吉祥,亨通。于是天地阴阳之气交感而万物通泰。君民上下结交然后心志相同。内卦为乾为阳,外卦为坤为阴,内刚健而外柔顺,内卦为君子而外卦为小人,君子之道盛长,小人之道消亡是也\footnote{是故吉祥而亨通。}。

象辞说:天地阴阳之气交感,这就是泰卦的卦象了。君王当效仿此天地通泰之道来成立体制,继而辅之以天地化育之功宜,以此来治理万民。

初九:就好像拔茅草一样根连根,初九爻这个贤士还连着他的同类九二九三贤士,征吉\footnote{泰卦征吉,否卦贞吉,征吉是主动出击行动吉祥,而贞吉是固守现状吉祥。}。拔茅征吉,是因为初九爻其志向是在外发展。

九二:包容八荒,以徒步渡河之勇,来无所遐弃偏远之才。无偏无私,来配合九二爻的中正而行之义。包荒,得尚于中行,来光明正大泰道【九二主泰】。

九三:没有那个平路走着走着不遇到斜坡的,没有谁一直前往前往而不返回的,艰贞无咎。不要担忧九三爻的诚信,他在饮食上是有福的。无往不复,是因为九三爻处天地交接之际,当明白往复屈伸之理。

六四:六四爻翩翩向下而飞,有六五爻和上六爻这两个阴爻与之为邻,他们不相戒备而诚信相待。翩翩不富是因为他们都是阴爻皆失实也,不戒以孚,是因为他们皆中心愿意。

六五:帝乙嫁出自己的妹妹【给九二】,因此得到福祉,大吉。以祉元吉,是因为六五爻居中而能行其愿是也。

上六:城墙倾覆于城壕中,不要用兵,上六爻自己在城邑中宣布君命,贞吝。城复于隍,泰道将灭,上下不交,其命将乱也。


\chapter{天地否(pǐ)卦}
否 {\Large ䷋}

\section{原文}

\subsection{经}
否\footnote{否,隔也——《广雅》闭塞不通之意。}:否之匪人\footnote{天地不交则不生万物,是无人道,故曰匪人,谓非人道也。——程颐},不利君子贞,大往小来。

初六:拔茅茹,以其汇,贞吉亨。

六二:包承。小人吉,大人否亨。

六三:包羞。

九四:有命无咎,畴\footnote{畴,通俦,同类。——执象易注}离祉\footnote{离,依附。祉,福祉。}。

九五:休否,大人吉。其亡其亡,系于苞桑。

上九:倾否,先否后喜。

\subsection{彖}
否之匪人,不利君子贞。大往小来,则是天地不交,而万物不通也;上下不交,而天下无邦也。内阴而外阳,内柔而外刚,内小人而外君子。小人道长,君子道消也。

\subsection{象}
天地不交,否;君子以俭德辟难,不可荣以禄。

初六:拔茅贞吉,志在君也。

六二:大人否亨,不乱群也。

六三:包羞,位不当也。

九四:有命无咎,志行也。

九五:大人之吉,位正当也。

上九:否终则倾,何可长也。


\section{初讲}
卦辞说:否卦非人道,不利君子贞。大往小来\footnote{可参考泰卦小往大来的解释。阳气都离去到了外卦,阴气都来到了内卦。}。

彖说:否卦非人道,不利君子贞。大往小来,于是天地阴阳之气不相交感而万物闭塞不通。君民上下不结交而天下无邦。内卦为坤为阴,外卦为乾为阳,内柔顺而外刚健,内卦为小人而外卦为君子,小人之道盛长,君子之道消亡是也。

象辞说:天地阴阳之气不相交感,这就是否卦的卦象了。君子当以节俭为美德,避免祸难,不可荣居厚禄之位。

初六:就好像拔茅草一样根连根,初六爻这个小人还连着他的同类六二和六三,初六爻固守现状则吉祥亨通。拔茅贞吉,是因为初六爻此时还是有忠君爱国之心的。

六二:六二小人包容初六顺承六三,小人吉祥,大人固守否道,则亨通。大人否亨,是因为大人没去扰乱内卦三爻的小人之群。

六三:六三小人包容六二初六的可羞之事,是因为六三爻自身居位不当的缘故。

九四:九四受九五[休否]之命,无咎,其和九五上九同类相互依附,共享福祉。有命无咎,是因为九四爻其志向得到实行的缘故\footnote{乾行刚健}。

九五:九五爻休止否运,大人吉祥。否运将要亡了,否运将要亡了,却又好像系于苞桑之上,否运根深蒂固,怎么也亡不了。大人之吉,是因为九五爻得位而居中的缘故。

上九:否运倾覆,先否后喜。否终则倾,怎么可能长久不变呢。


\chapter{天火同人卦}
同人 {\Large ䷌}

\section{原文}

\subsection{经}
同人\footnote{即同仁,志同道合之人。}于野,亨。利涉大川,利君子贞。

初九:同人于门,无咎。

六二:同人于宗,吝。

九三:伏戎于莽,升其高陵,三岁不兴。

九四:乘其墉,弗克攻,吉。

九五:同人,先号咷\footnote{同“号啕大哭”的啕}而后笑。大师克相遇。

上九:同人于郊,无悔。

\subsection{彖}
同人,柔得位得中,而应乎乾,曰同人。同人曰:“同人于野,亨。利涉大川,”乾行也。文明以健,中正而应,君子正也。唯君子为能通天下之志。

\subsection{象}
天与火,同人;君子以类族辨物。

初九:出门同人,又谁咎也。

六二:同人于宗,吝道也。

九三:伏戎于莽,敌刚也。三岁不兴,安\footnote{安,辞也——王弼,语气辞。}行也。

九四:乘其墉,义弗克也,其吉,则困而反则也\footnote{以不克困苦而反归其法则,故得吉也。——孔颖达}。

九五:同人之先,以中直也。大师相遇,言相克也。

上九:同人于郊,志未得也。

\section{初讲}
卦辞说:在旷野与人亲同,亨通。利涉大川,利君子贞。

彖说:同人卦,六二柔爻得位又得中,而与外卦乾卦之主九五爻相应,称之为同人。同人卦说:“同人于野,亨。利涉大川,”。【为何呢】,因为乾卦刚健而善行。【为何利君子贞呢】因为内卦离卦文明而外卦乾卦刚健,二五爻皆中正\footnote{居中得位}而相应,君子之正道如是也。唯君子能通达天下人的心志。

象辞说:上为乾卦为天,下为离卦为火,上天下火,这就是同人的卦象了。君子观此卦象而知道族以类似之,物以辩别之的道理。【亦即对人求同存异,对物辨别分明;类族辨物,君子做人格物的道理就在其中了。】

初九:刚出门即与人亲同,无咎。出门同人,谁会去罪责他呢。【反观六二的过于吝啬只同人于宗族之中,谁又能指责他呢。此爻有近人友爱之象。】

六二:只与九五本宗族人亲同,吝。同人于宗,这是吝啬可鄙之道。

九三:埋伏兵戎于草莽之中,等上高陵,三年不敢兴兵打仗。伏戎于莽,[九三爻与九五爻敌对而欲强行亲同六二]是因为九五爻很是刚健。三岁不兴,这怎能行得通啊。

九四:九四爻登上九五爻的城墙欲攻打之,最终没有进攻,吉祥。乘其墉,按照道义是不能攻打的。其吉祥,是因为九四爻陷入困苦后而能反归其正当法则。

九五:九五亲同六二[六二,同人之主],【因有九三九四阻拦】,先号啕大哭而后欢笑。大军克敌制胜才得相遇啊。同人之先,以中直也。大师相遇,言相克也。九五同人之前所以号啕大哭者,是因为九五爻中正而直率。大师相遇,是说九五爻终能克敌制胜而和六二爻相遇。

上九:在郊野与人亲同,无悔\footnote{郊者,外之极也。处“同人”之时,最在于外,不获同志,而远于内争,故虽无悔吝,亦未得其志。——王弼}。同人于郊,志未得也\footnote{此处高岛断易结合卦辞对比郊野和旷野和通天下人之志来做解释,个人觉得解之过大。}。


\chapter{火天大有卦}
大有 {\Large ䷍}

\section{原文}

\subsection{经}
大有:元亨。

初九:无交害,匪咎,艰则无咎。

九二:大车以载,有攸往,无咎。

九三:公用亨\footnote{亨,享也。参考高岛断易。}于天子,小人弗克\footnote{克,能也。参考执象易注。}。

九四:匪其彭\footnote{“彭”者,盛多貌。参考高岛断易。},无咎。

六五:厥\footnote{厥,其也。参考执象易注。}孚交如,威如;吉。

上九:自天祐之,吉无不利。

\subsection{彖}
大有,柔得尊位,大中而上下应之,曰大有。其德刚健而文明,应乎天而时行,是以元亨。

\subsection{象}
火在天上,大有;君子以遏恶扬善,顺天休\footnote{休,美也。君子观大有之象,以遏绝众恶,扬明善类,以奉顺天休美之命。——程颐。}命。

初九:大有初九,无交害也。

九二:大车以载,积中不败也。

九三:公用亨于天子,小人害也。

九四:匪其彭,无咎;明辨晢\footnote{皙,明智也。参考执象易注。}也。

六五:厥孚交如,信以发志也。威如之吉,易而无备也。

上九:大有上吉,自天祐也。


\section{初讲}
卦辞说:大有,大亨通。

彖说:大有,六五阴柔爻得尊位,居中为大而上下诸阳爻与之相应,故曰大有【五刚之大,皆为尊位所有\footnote{参考高岛断易}。】。内卦乾卦刚健,外卦离卦文明,故曰其德性刚健而文明;德应于天道,行合乎时宜,时行也本就是顺应天道,六五人君有此德性,是以元亨。

象辞说:上为离为火,下为乾为天,火在天上,这就是大有卦的卦象了。君子观此卦象以遏恶扬善来顺天美命。

初九:大有初九,并未涉害。若要没有灾难,则艰难自守可无咎矣。大有初九,并未涉害是也。

九二:九二爻为六五所用,大车以载,其才足以重任。有所前往,无咎。总的情况是任重不危。大有九二所积用大车装载于其中,所积于中才不会败坏是也,故而无咎。

九三:九三王公爻享用六五天子爻的宴请款待,小人则不能这样。小人得到重用将会成为祸害。【此在告诫九三爻所居位置重要,要做正人君子,不要做小人。】

九四:不是九四爻的盛大多有,无咎。九四爻虽处盛地然非己之盛,只要明智地明辨这一点则无咎矣。

六五:六五爻诚信交往的样子,威严的样子,吉祥啊。厥孚交如,信以发志。【上能诚信待下而下有协助之志\footnote{参见高岛断易}】威严的样子是吉祥是六五若无威严则会为人所易慢,无戒备之心\footnote{参考《程传》}。

上九:上九爻居大有之世之极,自有天佑,吉无不利。



\chapter{地山谦卦}
谦 {\Large ䷎}

\section{原文}

\subsection{经}
谦:亨,君子有终。

初六:谦谦君子,用涉大川,吉。

六二:鸣\footnote{鸣者,声名闻之谓也。——王弼}谦,贞吉。

九三:劳谦,君子有终,吉。

六四:无不利,撝\footnote{huī,挥也。}谦。

六五:不富,以其邻,利用侵伐,无不利。

上六:鸣谦,利用行师,征邑国。

\subsection{彖}
谦,亨,天道下济\footnote{利泽下施,长养万物。——汉典}而光明,地道卑而上行。天道亏盈而益谦,地道变\footnote{倾变}盈而流\footnote{流聚}谦,鬼神害盈而福谦,人道恶盈而好谦。谦尊而光,卑而不可逾,君子之终也。

\subsection{象}
地中有山,谦;君子以裒\footnote{(póu)减少。“多者用谦以为裒,少者用谦以为益,随物而与,施不失平也。”——王弼}多益寡,称物平施。

初六:谦谦君子,卑以自牧\footnote{牧,养也。——王弼}也。

六二:鸣谦贞吉,中心得也。

九三:劳谦君子,万民服也。

六四:无不利,撝谦;不违则也。

六五:利用侵伐,征不服也。

上六:鸣谦,志未得也。可用行师,征邑国也。

\section{初讲}
卦辞说:谦,亨通,君子有善终\footnote{礼记曰:“君子曰终,小人曰死”,又有“老曰终,少曰死”。大体终字在古代除了结局终了之意思外还有褒义的意思,所以此处君子有终,大体就是君子有善终之意。}。

彖说:谦,亨通,天之道下施而万物光明,地之道卑下而上行。天道亏损盈者而增益谦者,地道倾变盈者而流聚谦者,鬼神祸害盈者而福佑谦者,人道厌恶盈者而喜好谦者。【天地鬼神人,有益谦,有流谦,有福谦,有好谦。简言之,天地鬼神人皆对谦者善。】故而谦虚能够让尊者更光荣,让卑者不可逾越,这就是卦辞所言君子有善终的原因。

象辞说:谦卦上为坤为地,下为艮为山,故曰地中有山,如山一般厚重而静静地待在大地之下,谦虚啊。君子观谦卦的卦象而知道减有余而补不足,称量财物平均施舍于人的道理。

初六:初六这个谦虚而又谦虚的君子,用此谦道来跋涉大川,吉祥。谦谦君子,初六这个谦谦君子于卑下之处自养谦德。

六二:六二谦德声名在外,贞吉。鸣谦贞吉,是因为六二爻[居中而得位],内心有谦德自然所得。

九三:九三爻[谦卦之主]劳苦功高而又谦虚,君子有善终,吉祥。劳谦君子,大家都会心悦诚服于他。

六四:无不利,挥扬谦德。六四爻挥扬自己的谦德吧,无不利,只要不违背法则。

六五:六五爻并不富足【阴爻】,因居于尊位而用以其邻【六四爻和上六爻】,利于用来侵伐,无不利。利用侵伐,是因为征伐的是骄逆不服之徒。

上六:上六爻谦德名声在外,利于用来兴兵打仗,征伐邑国。上六爻虽然鸣谦,但因为居谦卦之极,反而内心谦德之志并未得遂\footnote{不同于六二的中心得也。}。可用行师,征邑国\footnote{邑国,诸侯的封地,征邑国,自治也。}也。【当时之时可以兴兵打仗,来治理好自己的治下封地。】


\chapter{雷地豫卦}
豫 {\Large ䷏}

\section{原文}

\subsection{经}
豫\footnote{豫,乐也。——尔雅;豫,喜豫说乐之貌也。——周易郑注。}:利建侯行师。

初六:鸣豫,凶。

六二:介\footnote{介,独立操守,确乎其不可拔之谓,孟子所谓“不以三公易其介”者是也。——执象易注。}于石,不终日,贞吉。

六三:盱\footnote{(xū),张目上望之状。}豫,悔。迟有悔。

九四:由豫,大有得。勿疑。朋盍\footnote{盍,合也——王弼}簪\footnote{(zān)簪,疾也。——王弼。}。

六五:贞疾,恒不死。

上六:冥\footnote{冥,昏也——执象易注。}豫,成\footnote{成,终也——执象易注。}有渝\footnote{渝,变也——执象易注。},无咎。

\subsection{彖}
豫,刚应而志行,顺以动,豫。豫顺以动,故天地如\footnote{如,从也。}之,而况建侯行师乎?天地以顺动,故日月不过\footnote{过,失度。},而四时不忒\footnote{忒,差错。};圣人以顺动,则刑罚清而民服。豫之时义大矣哉!

\subsection{象}
雷出地奋\footnote{震动。奋,动也——广雅。},豫。先王以作乐崇德,殷\footnote{殷,盛也——执象易注。}荐\footnote{进献}之上帝,以配\footnote{配享}祖考\footnote{祖先}。

初六:初六鸣豫,志穷凶也。

六二:不终日,贞吉;以中正也。

六三:盱豫有悔,位不当也。

九四:由豫,大有得;志大行也。

六五:六五贞疾\footnote{贞疾,常病,痼疾。——高岛断易},乘刚也。恒不死,中未亡也。

上六:冥豫在上,何可长也。

\section{初讲}
卦辞说:豫卦,利建候行师。

彖说:豫卦,九四刚爻为众阴爻响应而其志得行,豫卦下为坤为顺,上为震为动,[众阴爻]顺而[九四爻]动之,这就是豫卦了。悦豫顺从地行动,所以连天地都会如此,则何况建候行师[有所不顺]乎?天地以之顺动之道,所以日月运行不失其度,而四时更替没有差错;圣人以之顺动之道,则刑罚清而万民服。豫卦运行之时和含义很是重大啊。

象辞说:豫卦上为震卦为雷,下为坤卦为地,雷在地上震动,这就是豫卦的卦象了。先王通过制作音乐来崇尚道德,以盛大的仪式进献给上帝,以及配享祖先。

初六:初六爻安享悦豫之初,与九四相应而自鸣得意,凶险。初六鸣豫,其没什么志向必遭凶险。

六二:独立操守其介如石,见几而作不俟终日\footnote{言君子见事之几微,则须动作而应之,不得待终其日,言赴机之速也。——孔颖达},贞吉。不终日,贞吉;是因为六二爻居中而得位,有中正之德。

六三:六三爻张目仰视九四,想要豫悦九四,[但九四爻见其不得位而鄙弃之,是以]有悔,迟疑不决则必生悔恨。盱豫有悔,是因为六三爻以柔居阳位,位不得当的缘故。

九四:由于九四爻而来的豫悦\footnote{豫之所以为豫者,由九四也,为动之主,动而众阴悦顺,为豫之义。——程颐。},大有得益。不要疑虑,朋友会很快合聚。由豫,大有得;是因为九四爻其志得以大行的缘故。

六五:六五爻患有痼疾,久病而不死。六五贞疾,是因为六五爻以柔乘[九四]刚的缘故。恒不死,是因为六五爻居中而处尊位,未可得亡\footnote{四以刚动为豫之主,专权执制,非已所乘,故不敢与四争权,而又居中处尊,未可得亡,是以必常至于“贞疾,恒不死”而已。——王弼}。

上六:上六爻昏聩豫乐,终有变,无咎。冥豫在上,何可长也\footnote{豫极则变,人终也须变而应对之。}。



\chapter{泽雷随卦}
随 {\Large ䷐}

\section{原文}

\subsection{经}
随:元亨,利贞,无咎。

初九:官有渝,贞吉。出门交有功。

六二:系\footnote{系,牵挂。刚爻能自立曰随,柔爻不足自立曰系——高岛断易。}小子,失丈夫。

六三:系丈夫,失小子。随有求得,利居贞。

九四:随有获,贞凶。有孚在道,以明,何咎。

九五:孚于嘉,吉。

上六:拘系\footnote{拘禁,管束。——汉典。}之,乃从维\footnote{绳两系称维——虞翻。}之,王\footnote{周文王}用亨\footnote{通享,享祭。}于西山\footnote{岐山}。

\subsection{彖}
随,刚来而下柔,动而说,随。大亨贞,无咎,而天下随时,随时之义大矣哉!

\subsection{象}
泽中有雷,随;君子以向\footnote{向,接近。}晦\footnote{晦,夜晚。}入宴息\footnote{宴息,休息。}。

初九:官有渝,从正吉也。出门交有功,不失也。

六二:系小子,弗兼与\footnote{交往——汉典。}也。

六三:系丈夫,志舍下也。

九四:随有获,其义凶也。有孚在道,明功也。

九五:孚于嘉吉,位正中也。

上六:拘系之,上穷也。


\section{初讲}
卦辞说:随卦,大亨通,利于正道,无咎。

彖说:随卦,初九刚爻来到内卦,来到柔爻之下\footnote{此刚来而下柔众说纷纭,刚爻指初九这是确定的,柔有说是上六,但从爻辞来看上六并不是随卦之主,故我舍弃了这一说法。}。下为震为动,上为兑为悦,行动并可喜悦,这就是随卦了。大亨通,[利于]正道,无咎,而天下万物皆随时而动,随时而动的含义很重大啊!

象辞说:随卦上为兑为泽,下为震为雷,泽中有雷,这就是随卦的卦象了。君子应当到了快晚上的时候就入室休息\footnote{君子观象以随时而动。随时之宜,万事皆然,取其最明且近者言之。君子以向晦入宴息:君子昼则自强不息,及向昏晦,则入居于内,宴息以安其身,起居随时,适其宜也。——程颐。}。

初九:初九爻主官之事有所变动,贞吉,出门与人交往有功。官有渝,其所从得正则吉祥。出门交有功,因为其所随不失正也。

六二:六二爻随系于初九这个小子就会失去九五这个大丈夫。系小子,是不能同时两个一起交往的。

六三:六三爻随系于九四这个大丈夫,就会失去初九这个小子。随有所求则有所得,利居贞\footnote{虽然,固不可非理枉道以随于上,苟取爱说以遂所求。如此,乃小人邪谄趋利之为也,故云利居贞。自处于正,则所谓有求而必得者,乃正事君子之随也。}。系丈夫,六三爻其志舍下初九爻而不从也。

九四:九四爻被人随从,有所收获,贞凶。有诚信于正道,以明其功,何咎\footnote{既能著信在于正道,是明立其功,故无咎也。——孔颖达。}。随有获,其义凶也\footnote{九四以阳刚之才,处臣位之极,若于随有获,则虽正亦凶。有或,谓得天下之心隨于己。为臣之道,当使恩威一出于上,众心皆隨于君。若人心从己,危疑之道也,故凶。——程颐。}。有孚在道,以明其功。[则无咎矣。]

九五:九五有诚信随于嘉善,吉祥。孚于嘉吉,是因为九五爻位得正而居中\footnote{位正者德必正,时中者道必中。——执象易注。}。

上六:对上六拘系之,又从而维系之\footnote{拘系之,谓隨之极,如拘持縻系之。乃从维之,又从而维系之也,谓隨之固結如此。——程颐。},[但上六还是会如同周文王一样],用享祭祀于岐山之上\footnote{昔者太王用此道,亨王业于西山。太王避狄之难,去豳来岐,豳人老稚扶携以隨之如归市,——程颐。}。拘系之,是因为上六随道穷极将变。


\chapter{山风蛊(gǔ)卦}
蛊 {\Large ䷑}

\section{原文}

\subsection{经}
蛊:元亨,利涉大川。先甲三日,后甲三日。

初六:干父之蛊,有子,考无咎,厉终吉。

九二:干母之蛊,不可贞。

九三:干父之蛊,小有悔,无大咎。

六四:裕父之蛊,往见吝。

六五:干父之蛊,用誉。

上九:不事王侯,高尚其事。

\subsection{彖}
蛊,刚上而柔下,巽而止,蛊。蛊,元亨,而天下治也。利涉大川,往有事也。先甲三日,后甲三日,终则有始,天行也。

\subsection{象}
山下有风,蛊;君子以振民育德。

初六:干父之蛊,意承考也。

九二:干母之蛊,得中道也。

九三:干父之蛊,终无咎也。

六四:裕父之蛊,往未得也。

六五:干父之蛊;承以德也。

上九:不事王侯,志可则也。

\section{讲解}
卦辞说:

彖说:

象辞说:

\chapter{地泽临卦}
临 {\Large ䷒}

\section{原文}

\subsection{经}
临:元,亨,利,贞。 至于八月有凶。

初九:咸临,贞吉。

九二:咸临,吉无不利。

六三:甘临,无攸利。 既忧之,无咎。

六四:至临,无咎。

六五:知临,大君之宜,吉。

上六:敦临,吉无咎。

\subsection{彖}
临,刚浸而长。 说而顺,刚中而应,大亨以正,天之道也。 至于八月有凶,消不久也。

\subsection{象}
泽上有地,临; 君子以教思无穷,容保民无疆。

初九:咸临贞吉,志行正也。

九二:咸临,吉无不利;未顺命也。

六三:甘临,位不当也。既忧之,咎不长也。

六四:至临无咎,位当也。

六五:大君之宜,行中之谓也。

上六:敦临之吉,志在内也。

\section{讲解}
卦辞说:

彖说:

象辞说:


\chapter{风地观卦}
观 {\Large ䷓}

\section{原文}

\subsection{经}
观:盥而不荐,有孚顒若。

初六:童观,小人无咎,君子吝。

六二:窥观,利女贞。

六三:观我生,进退。

六四:观国之光,利用宾于王。

九五:观我生,君子无咎。

上九:观其生,君子无咎。

\subsection{彖}
大观在上,顺而巽,中正以观天下。观,盥而不荐,有孚顒若,下观而化也。观天之神道,而四时不忒,圣人以神道设教,而天下服矣。

\subsection{象}
风行地上,观;先王以省方,观民设教。

初六:初六童观,小人道也。

六二:窥观女贞,亦可丑也。

六三:观我生,进退;未失道也。

六四:观国之光,尚宾也。

九五:观我生,观民也。

上九:观其生,志未平也。

\section{讲解}
卦辞说:祭祀前洗手却没有献上祭品,有诚心,温和肃敬的样子。【心诚最重要】

彖说:最上爻刚而大,观之在上位,坤顺而巽入,五爻中正以观天下。观,祭祀前洗手却没有献上祭品,有诚心,温和肃敬的样子,下面的人观之而受教化。观天之神道,四时运行没有差错,圣人以神道设教,天下自然顺服。

象辞说:风吹在地上,这就是观卦的卦象;先王省察四方,观察民情并设立教化。

六三:观察自己一生的所作所为,来决定接下来的进退。未失道也。本爻更多是强调自我观察,自己要有自己的判断和主见。

\chapter{火雷噬嗑(shìhé)卦}
噬嗑 {\Large ䷔}

\section{原文}

\subsection{经}
噬嗑:亨。利用狱。

初九:屦校灭趾,无咎。

六二:噬肤灭鼻,无咎。

六三:噬腊肉,遇毒;小吝,无咎。

九四:噬干胏,得金矢,利艰贞,吉。

六五:噬干肉,得黄金,贞厉,无咎。

上九:何校灭耳,凶。

\subsection{彖}
颐中有物,曰噬嗑,噬嗑而亨。刚柔分,动而明,雷电合而章。柔得中而上行,虽不当位,利用狱也。

\subsection{象}
雷电噬嗑;先王以明罚敕法。

初九:屦校灭趾,不行也。

六二:噬肤灭鼻,乘刚也。

六三:遇毒,位不当也。

九四:利艰贞吉,未光也。

六五:贞厉无咎,得当也。

上九:何校灭耳,聪不明也。

\section{讲解}
卦辞说:

彖说:

象辞说:

\chapter{山火贲(bì)卦}
贲 {\Large ䷕}

\section{原文}
\subsection{经}
贲:亨。 小利有所往。

初九:贲其趾,舍车而徒。

六二:贲其须。

九三:贲如濡如,永贞吉。

六四:贲如皤如,白马翰如,匪寇婚媾。

六五:贲于丘园,束帛戋戋,吝,终吉。

上九:白贲,无咎。

\subsection{彖}
贲,亨;柔来而文刚,故亨。分刚上而文柔,故小利有攸往。刚柔交错,天文也;文明以止,人文也。观乎天文,以察时变;观乎人文,以化成天下。
\subsection{象}
山下有火,贲;君子以明庶政,无敢折狱。

初九:舍车而徒,义弗乘也。

六二:贲其须,与上兴也。

九三:永贞之吉,终莫之陵也。

六四:六四,当位疑也。 匪寇婚媾,终无尤也。

六五:六五之吉,有喜也。

上九:白贲无咎,上得志也。

\section{讲解}
卦辞说:

彖说:

象辞说:


\chapter{山地剥卦}
剥 {\Large ䷖}

\section{原文}
\subsection{经}
剥:不利有攸往。

初六:剥床以足,蔑贞凶。

六二:剥床以辨,蔑贞凶。

六三:剥之,无咎。

六四:剥床以肤,凶。

六五:贯鱼,以宫人宠,无不利。

上九:硕果不食,君子得舆,小人剥庐。

\subsection{彖}
剥,剥也,柔变刚也。不利有攸往,小人长也。 顺而止之,观象也。君子尚消息盈虚,天行也。
\subsection{象}
山附地上,剥;上以厚下,安宅。

初六:剥床以足,以灭下也。

六二:剥床以辨,未有与也。

六三:剥之无咎,失上下也。

六四:剥床以肤,切近灾也。

六五:以宫人宠,终无尤也。

上九:君子得舆,民所载也。 小人剥庐,终不可用也。

\section{讲解}
卦辞说:

彖说:

象辞说:

\chapter{地雷复卦}
复 {\Large ䷗}

\section{原文}
\subsection{经}
复:亨。出入无疾,朋来无咎。反复其道,七日来复,利有攸往。

初九:不远复,无祗悔,元吉。

六二:休复,吉。

六三:频复,厉无咎。

六四:中行独复。

六五:敦复,无悔。

上六:迷复,凶,有灾眚。用行师,终有大败,以其国君,凶;至于十年,不克征。

\subsection{彖}
复亨;刚反,动而以顺行,是以出入无疾,朋来无咎。反复其道,七日来复,天行也。利有攸往,刚长也。 复其见天地之心乎?

\subsection{象}
雷在地中,复;先王以至日闭关,商旅不行,后不省方。

初九:不远之复,以修身也。

六二:休复之吉,以下仁也。

六三:频复之厉,义无咎也。

六四:中行独复,以从道也。

六五:敦复无悔,中以自考也。

上六:迷复之凶,反君道也。

\section{讲解}
卦辞说:

彖说:

象辞说:

\chapter{天雷无妄卦}
无妄 {\Large ䷘}

\section{原文}
\subsection{经}
无妄:元,亨,利,贞。 其匪正有眚,不利有攸往。

初九:无妄,往吉。

六二:不耕获,不菑畲,则利有攸往。

六三:无妄之灾,或系之牛,行人之得,邑人之灾。

九四:可贞,无咎。

九五:无妄之疾,勿药有喜。

上九:无妄行,有眚,无攸利。

\subsection{彖}
无妄,刚自外来,而为主於内。动而健,刚中而应,大亨以正,天之命也。其匪正有眚,不利有攸往。无妄之往,何之矣?天命不祐,行矣哉?
\subsection{象}
天下雷行,物与无妄;先王以茂对时,育万物。

初九:无妄之往,得志也。

六二:不耕获,未富也。

六三:行人得牛,邑人灾也。

九四:可贞无咎,固有之也。

九五:无妄之药,不可试也。

上九:无妄之行,穷之灾也。

\section{讲解}
卦辞说:

彖说:

象辞说:


\chapter{山天大畜(xù)卦}
大畜\footnote{小畜是小有畜止小有畜积,而大畜因上为艮卦为止,畜止已经明确,故大畜多言畜养之意,取其重者也。} {\Large ䷙}

\section{原文}

\subsection{经}
大畜:利贞,不家食吉,利涉大川。

初九:有厉利已\footnote{(yǐ),已,停止,艮也。——执象易注}。

九二:舆说輹\footnote{(fù),古代在车轴下面束缚车轴的东西——汉典;輹,车轴缚也。——说文。}。

九三:良马逐,利艰贞。日闲\footnote{闲,通娴,有词闲习,熟悉演练练习之意。}舆卫,利有攸往。

六四:童牛之牿\footnote{(gù),绑在牛角上使其不能抵入的横木——汉典},元吉。

六五:豶豕\footnote{豶(fén)豕,阉割过的公猪}之牙,吉。

上九:何\footnote{(hè)古义可通荷,比如何天之宠,何天之休,但这里王弼和程颐皆认为这是一个语气辞。}天之衢,亨。

\subsection{彖}
大畜,刚健,笃实,辉光,日新其德。刚上而尚贤\footnote{谓上九也。处上而大通,刚来而不距,尚贤之谓也。——王弼},能止健,大正也。不家食吉\footnote{有大畜之实,以之养贤,令贤者不家食,乃吉也。——王弼},养贤也。利涉大川,应乎天也\footnote{大正应天,不忧险难,故利涉大川——王弼}。

\subsection{象}
天在山中,大畜;君子以多识前言往行,以畜其德。

初九:有厉利已,不犯灾也。

九二:舆说輹,中无尤也。

九三:利有攸往,上合志也。

六四:六四元吉,有喜也。

六五:六五之吉,有庆也。

上九:何天之衢,道大行也。

\section{初讲}
卦辞说:大畜,利于正道,不食于家【贤士食于外而得尊养之。】吉祥,利涉大川。

彖说:乾体刚健,艮山笃实,其内有才若光辉,而能日日更新己德。上九刚爻居上而尊尚贤士,上卦艮卦能止乾健,大正道是也。不家食吉,尊养贤士是也。利涉大川,顺应天道是也。

象辞说:下卦为乾卦为天,上卦为艮卦为山,天在山中,这就是大畜卦的卦象了。君子应当多识前言往行,以畜其德。

初九:有危险,利于停止。有厉利已,不去触犯灾难是也【初九为六四所止】。

九二:车子的輹脱落。舆说輹,九二居中不行也并没有什么好怨恨的。【九二为六五所止】。

九三:良马驰逐,利艰贞。日常训练车马防卫,利于有所前往。利有攸往,是因为九三爻与上九爻心志相合【上九畜极则通故利于有所前往】。

六四:六四爻畜刚健之初爻者初九爻若童牛之牯,大吉祥。六四元吉,有喜也\footnote{故畜止于微小之前,则大善而吉。不劳而无伤,故可喜也。——程颐}。

六五:六五爻畜刚健之中爻者九二爻若豶豕之牙,吉祥。六五之吉,有庆也\footnote{豕,刚躁之物,而牙为猛利,若强制其牙,则用力劳而不能止其躁猛,虽絷之维之,不能使之变也。若豶去其势,则牙虽存而刚躁自止,其用如此,所以吉也。……若知其本,制之有道,则不劳无伤而俗革,天下之福庆也。——程颐}。

上九:若天路一般四通八达,亨通。何天之衢,是因为畜极则通,道路大通行是也。

\chapter{山雷颐(yí)卦}
颐 {\Large ䷚}

\section{原文}

\subsection{经}
颐:贞吉。 观颐,自求口实。

初九:舍尔灵龟,观我朵颐,凶。

六二:颠颐,拂经,于丘颐,征凶。

六三:拂颐,贞凶,十年勿用,无攸利。

六四:颠颐吉,虎视眈眈,其欲逐逐,无咎。

六五:拂经,居贞吉,不可涉大川。

上九:由颐,厉吉,利涉大川。

\subsection{彖}
颐贞吉,养正则吉也。观颐,观其所养也;自求口实,观其自养也。天地养万物,圣人养贤,以及万民;颐之时大矣哉!
\subsection{象}
山下有雷,颐;君子以慎言语,节饮食。

初九:观我朵颐,亦不足贵也。

六二:六二征凶,行失类也。

六三:十年勿用,道大悖也。

六四:颠颐之吉,上施光也。

六五:居贞之吉,顺以从上也。

上九:由颐厉吉,大有庆也。

\section{讲解}
卦辞说:

彖说:

象辞说:


\chapter{泽风大过卦}
大过 {\Large ䷛}

\section{原文}

\subsection{经}
大过:栋桡\footnote{栋桡,意思是屋梁脆弱曲折——汉典。桡指木头弯曲。},利有攸往,亨。

初六:藉\footnote{藉(jiè)藉,祭藉也。——说文。即祭祀时用来垫祭品的衬垫。}用白茅\footnote{白茅,取其洁也——高岛断易},无咎。

九二:枯杨生稊\footnote{tí,杨柳新长出的嫩芽。},老夫得其女妻,无不利。

九三:栋桡,凶。

九四:栋隆,吉,有它吝。

九五:枯杨生华,老妇得士夫\footnote{年轻男子},无咎无誉。

上六:过涉灭顶,凶,无咎。

\subsection{彖}
大过,大者过也。栋桡,本末弱也。刚过而中,巽而说行,利有攸往,乃亨。大过之时大矣哉!

\subsection{象}
泽灭木,大过;君子以独立不惧,遁世无闷。

初六:藉用白茅,柔在下也。

九二:老夫女妻,过以相与\footnote{相处,相交往。他近日和衔玉的那位令郎相与甚厚。——红楼梦。}也。

九三:栋桡之凶,不可以有辅也。

九四:栋隆之吉,不桡乎下也。

九五:枯杨生华,何可久也。老妇士夫,亦可丑也。

上六:过涉之凶,不可咎也。

\section{初讲}
卦辞说:房屋栋梁弯曲,利于有所前往,亨通。

彖说:大过,大阳过于阴也。栋桡,本末两阴爻过于柔弱也。四刚爻虽过而居中\footnote{这里我同意高岛断易的观点,这里应该是从整个卦象来谈论的。},下卦为巽上卦为兑为悦,巽顺和悦地前行,故能利于有所前往,乃亨通。大过卦运作之时很重大啊\footnote{是君子有为之时也——王弼;如立非常之大事,兴百事之大功,成绝俗之大德,皆大过之事也。——程颐}!

象辞说:大过卦上为兑为泽,下为巽为木,泽水淹没树木\footnote{泽水本当润养树木,乃至灭没树木,则过甚矣——程颐},这就是大过的卦象了。君子观此卦象,应当做到独立无惧,遁世无闷。

初六:祭藉用的是白茅草,无咎。藉用白茅,之所以无咎是因为初六柔爻居于下卦巽卦之下,卑以处下,过于敬慎。

九二:枯萎的杨柳树又长出新的嫩芽,九二爻这个老男人得到了一个少女\footnote{九二爻为下卦之主,九二阳爻于大过中阳过则为老男人,下卦巽卦为处女。},无不利。老夫女妻,他们相互交往还是过分了些\footnote{谓九二初六阴阳相与之和过于常也。——程颐}。

九三:房屋栋梁弯曲,凶险。栋桡之凶,是因为九三爻这个位置[好比房屋栋梁的中间部位]是不可以有辅助的\footnote{栋当室之中,不可加助,是不可以有辅也。——程颐}。

九四:房屋栋梁隆起,吉祥,不过可惜有其他憾事\footnote{大家皆言此处的它吝是因为九四爻应于初六爻所致。}。栋隆之吉,是因为房屋栋梁没有向下弯曲\footnote{栋隆,九四爻能自胜其任也。}。

九五:枯萎的杨树长了花,上六这个老妇人得到九五这个年轻男子\footnote{这里的年轻是指九五相较上六很年轻}为丈夫,无咎也无誉。枯杨生华,怎么能够长久呢。老妇士夫,亦可丑也\footnote{老妇而得士夫,岂能成生育之功,为可丑矣。——程颐}。

上六:徒步过河被水淹没了头顶,凶险,无咎\footnote{处太过之极,过之甚也。涉难过甚,故至于灭顶凶。——王弼}。上六过涉之凶,不应该责备它的\footnote{虽凶无咎,不害义也。——王弼;乃自为之,不可以有咎也。——程颐}。


\chapter{坎卦}
坎 {\Large ䷜}

\section{原文}

\subsection{经}
坎:习坎,有孚,维心亨,行有尚。

初六:习坎,入于坎窞,凶。

九二:坎有险,求小得。

六三:来之坎坎,险且枕,入于坎窞,勿用。

六四:樽酒簋贰,用缶,纳约自牖,终无咎。

九五:坎不盈,祗既平,无咎。

上六:系用徽纆,置于丛棘,三岁不得,凶。

\subsection{彖}
习坎,重险也。水流而不盈,行险而不失其信。维心亨,乃以刚中也。行有尚,往有功也。天险不可升也,地险山川丘陵也,王公设险以守其国,险之时用大矣哉!

\subsection{象}
水洊至,习坎;君子以常德行,习教事。

初六:习坎入坎,失道凶也。

九二:求小得,未出中也。

六三:来之坎坎,终无功也。

六四:樽酒簋贰,刚柔际也。

九五:坎不盈,中未大也。

上六:上六失道,凶三岁也。

\section{讲解}
卦辞说:

彖说:

象辞说:

\chapter{离卦}
离 {\Large ䷝}

\section{原文}


\subsection{经}
离:利贞,亨。 畜牝牛,吉。

初九:履错然,敬之无咎。

六二:黄离,元吉。

九三:日昃之离,不鼓缶而歌,则大耋之嗟,凶。

九四:突如其来如,焚如,死如,弃如。

六五:出涕沱若,戚嗟若,吉。

上九:王用出征,有嘉折首,获匪其丑,无咎。

\subsection{彖}
离,丽也;日月丽乎天,百谷草木丽乎土,重明以丽乎正,乃化成天下。柔丽乎中正,故亨;是以畜牝牛吉也。

\subsection{象}
明两作离,大人以继明照于四方。

初九:履错之敬,以辟咎也。

六二:黄离元吉,得中道也。

九三:日昃之离,何可久也。

九四:突如其来如,无所容也。

六五:六五之吉,离王公也。

上九:王用出征,以正邦也。

\section{讲解}
卦辞说:

彖说:

象辞说:

\chapter{泽山咸卦}
咸 {\Large ䷞}

\section{原文}

\subsection{经}
咸:亨,利贞,取女吉。

初六:咸其拇。

六二:咸其腓,凶,居吉。

九三:咸其股,执其随,往吝。

九四:贞吉悔亡,憧憧往来,朋从尔思。

九五:咸其脢,无悔。

上六:咸其辅,颊,舌。

\subsection{彖}
咸,感也。柔上而刚下,二气感应以相与,止而说,男下女,是以亨利贞,取女吉也。天地感而万物化生,圣人感人心而天下和平;观其所感,而天地万物之情可见矣!

\subsection{象}
山上有泽,咸;君子以虚受人。

初六:咸其拇,志在外也。

六二:虽凶,居吉,顺不害也。

九三:咸其股,亦不处也。 志在随人,所执下也。

九四:咸其股,亦不处也。 志在随人,所执下也。

九五:咸其脢,志末也。

上六:咸其辅,颊,舌,滕口说也。

\section{讲解}
卦辞说:

彖说:

象辞说:

\chapter{雷风恒卦}
恒 {\Large ䷟}

\section{原文}

\subsection{经}
恒:亨,无咎,利贞,利有攸往。

初六:浚恒,贞凶,无攸利。

九二:悔亡。

九三:不恒其德,或承之羞,贞吝。

九四:田无禽。

六五:恒其德,贞,妇人吉,夫子凶。

上六:振恒,凶。

\subsection{彖}
恒,久也。刚上而柔下,雷风相与,巽而动,刚柔皆应,恒。恒亨无咎,利贞;久於其道也,天地之道,恒久而不已也。利有攸往,终则有始也。日月得天,而能久照,四时变化,而能久成,圣人久於其道,而天下化成;观其所恒,而天地万物之情可见矣!

\subsection{象}
雷风,恒;君子以立不易方。

初六:浚恒之凶,始求深也。

九二:九二悔亡,能久中也。

九三:不恒其德,无所容也。

九四:久非其位,安得禽也。

六五:妇人贞吉,从一而终也。夫子制义,从妇凶也。

上六:振恒在上,大无功也。

\section{讲解}
卦辞说:

彖说:

象辞说:

\chapter{天山遁(dùn)卦}
遁 {\Large ䷠}

\section{原文}

\subsection{经}
遁:亨,小利贞。

初六:遁尾,厉,勿用有攸往。

六二:执之用黄牛之革,莫之胜说。

九三:系遁,有疾厉,畜臣妾吉。

九四:好遁君子吉,小人否。

九五:嘉遁,贞吉。

上九:肥遁,无不利。

\subsection{彖}
遁亨,遁而亨也。刚当位而应,与时行也。小利贞,浸而长也。遁之时义大矣哉!

\subsection{象}
天下有山,遁;君子以远小人,不恶而严。

初六:遁尾之厉,不往何灾也。

六二:执用黄牛,固志也。

九三:系遁之厉,有疾惫也。畜臣妾吉,不可大事也。

九四:君子好遁,小人否也。

九五:嘉遁贞吉,以正志也。

上九:肥遁,无不利;无所疑也。

\section{讲解}
卦辞说:

彖说:

象辞说:


\chapter{雷天大壮卦}
大壮 {\Large ䷡}

\section{原文}

\subsection{经}
大壮:利贞。

初九:壮于趾,征凶,有孚。

九二:贞吉。

九三:小人用壮,君子用罔,贞厉。 羝羊触藩,羸其角。

九四:贞吉悔亡,藩决不羸,壮于大舆之輹。

六五:丧羊于易,无悔。

上六:羝羊触藩,不能退,不能遂,无攸利,艰则吉。

\subsection{彖}
大壮,大者壮也。刚以动,故壮。大壮利贞;大者正也。正大而天地之情可见矣!

\subsection{象}
雷在天上,大壮;君子以非礼弗履。

初九:壮于趾,其孚穷也。

九二:九二贞吉,以中也。

九三:小人用壮,君子罔也

九四:藩决不羸,尚往也。

六五:丧羊于易,位不当也。

上六:不能退,不能遂,不祥也。艰则吉,咎不长也。

\section{讲解}
卦辞说:

彖说:

象辞说:

\chapter{火地晋卦}
晋 {\Large ䷢}

\section{原文}

\subsection{经}
晋:康侯用锡马蕃庶,昼日三接。

初六:晋如,摧如,贞吉。 罔孚,裕无咎。

六二:晋如,愁如,贞吉。 受兹介福,于其王母。

六三:众允,悔亡。

九四:晋如鼫鼠,贞厉。

六五:悔亡,失得勿恤,往吉无不利。

上九:晋其角,维用伐邑,厉吉无咎,贞吝。

\subsection{彖}
晋,进也。明出地上,顺而丽乎大明,柔进而上行。是以康侯用锡马蕃庶,昼日三接也。

\subsection{象}
明出地上,晋;君子以自昭明德。

初六:晋如,摧如;独行正也。裕无咎;未受命也。

六二:受之介福,以中正也。

六三:众允之,志上行也。

九四:鼫鼠贞厉,位不当也。

六五:失得勿恤,往有庆也。

上九:维用伐邑,道未光也。

\section{讲解}
卦辞说:

彖说:

象辞说:

\chapter{地火明夷(yí)卦}
明夷 {\Large ䷣}

\section{原文}

\subsection{经}
明夷:利艰贞。

初九:明夷于飞,垂其翼。 君子于行,三日不食, 有攸往,主人有言。

六二:明夷,夷于左股,用拯马壮,吉。

九三:明夷于南狩,得其大首,不可疾贞。

六四:入于左腹,获明夷之心,于出门庭。

六五:箕子之明夷,利贞。

上六:不明晦,初登于天,后入于地。

\subsection{彖}
明入地中,明夷。内文明而外柔顺,以蒙大难,文王以之。利艰贞,晦其明也,内难而能正其志,箕子以之。

\subsection{象}
明入地中,明夷;君子以莅众,用晦而明。

初九:君子于行,义不食也。

六二:六二之吉,顺以则也。

九三:南狩之志,乃大得也。

六四:入于左腹,获心意也。

六五:箕子之贞,明不可息也。

上六:初登于天,照四国也。 后入于地,失则也。

\section{讲解}
卦辞说:

彖说:

象辞说:

\chapter{风火家人卦}
家人 {\Large ䷤}

\section{原文}

\subsection{经}
家人:利女贞。

初九:闲有家,悔亡。

六二:无攸遂,在中馈,贞吉。

九三:家人嗃嗃,悔厉吉;妇子嘻嘻,终吝。

六四:富家,大吉。

九五:王假有家,勿恤吉。

上九:有孚威如,终吉。

\subsection{彖}
家人,女正位乎内,男正位乎外,男女正,天地之大义也。家人有严君焉,父母之谓也。父父,子子,兄兄,弟弟,夫夫,妇妇,而家道正;正家而天下定矣。

\subsection{象}
风自火出,家人;君子以言有物,而行有恒。

初九:闲有家,志未变也。

六二:六二之吉,顺以巽也。

九三:家人嗃嗃,未失也;妇子嘻嘻,失家节也。

六四:富家大吉,顺在位也。

九五:王假有家,交相爱也。

上九:威如之吉,反身之谓也。

\section{讲解}
卦辞说:

彖说:

象辞说:

\chapter{火泽睽(kuí)卦}
睽 {\Large ䷥}

\section{原文}

\subsection{经}
睽:小事吉。

初九:悔亡,丧马勿逐,自复;见恶人,无咎。

九二:遇主于巷,无咎。

六三:见舆曳,其牛掣,其人天且劓\footnote{天髠首也,劓截鼻也。三从正应而四隔止之,三虽阴柔处刚而志行,故力进以犯之,是以伤也。——程颐},无初有终。

九四:睽孤,遇元夫\footnote{指初九},交孚,厉无咎。

六五:悔亡,厥\footnote{其}宗噬肤,往何咎。

上九:睽孤,见豕负涂,载鬼一车, 先张之弧,后说\footnote{通脱}之弧,匪寇婚媾,往遇雨则吉。

\subsection{彖}
睽,火动而上,泽动而下;二女同居,其志不同行;说\footnote{通悦}而丽\footnote{依附}乎明,柔进而上行,得中而应乎刚;是以小事吉。天地睽而其事同也;男女睽而其志通也;万物睽而其事类\footnote{指事物的类似性}也;睽之时用\footnote{王弼注:非用之常,用有时也。}大矣哉!

\subsection{象}
上火下泽,睽;君子以同而异。

初九:见恶人,以辟\footnote{通避}咎也。

九二:遇主于巷,未失道也。

六三:见舆曳,位不当也。无初有终,遇刚也。

九四:交孚无咎,志行也。

六五:厥宗噬肤,往有庆也。

上九:遇雨之吉,群疑亡也。

\section{初讲}
卦辞说:睽卦,小事吉祥。

彖说:睽卦上为离为火,火动向上,下为兑为泽,泽动向下;又离为中女,兑为少女,是以二女同居,志向不同,很难一起行动;内心愉悦而依附于光明,六五爻柔顺而向上进,得中位而于九二相应,【有此三德\footnote{王弼,周易注}】,是以小事吉。天地分离但它们养育万物之事是相同的,男女分离但它们交感之志是相通的,万物分离但它们之间还是很类似的。睽卦有时候用处很大啊!

象辞说:睽卦的卦象是上面是火下面是泽,君子观此卦象而知道求同存异的道理。


初九:悔恨消失,丢失的马不要去追寻,它自己会回来;见到交恶之人,无咎。见恶人,不避咎故而无咎。

九二:九二在小巷遇到六五,无咎。遇主于巷,未失道也。

六三:上互卦为坎为舆,车子有两爻向上而一爻向下,下互卦为离为牛,牛两爻向下一爻向上,故有车子拖曳受困,牛拉车却反而掣阻车子。六三其人受到了剃光头和割鼻子的刑罚。起初不好但最终有好的结局的。见舆曳,是因为六三爻位置不好。无初有终,因为六三与上九相应和故终有好结局。

九四:九四这个睽爻很孤独,但和初九这个阳爻是相应的,遇之,以诚信结交之,虽然很危险但不会有灾祸的。交孚无咎,因为九四爻其志得行的缘故。

六五:悔恨消失,九二厥宗设宴吃肉,六五爻前往会之,何咎?厥宗噬肤,六五爻前往会有喜庆之事。

上九:上九这个睽爻很孤独,看见猪背上涂满污泥,又看见车上满载着打扮得像鬼一样的人,先张弓欲射,后又放下弓箭;不是强盗来了,是迎亲的队伍,前往遇到下雨天会吉祥。遇雨之吉,是因为之前的满腹疑虑都烟消云散了。


\chapter{水山蹇(jiǎn)卦}
蹇 {\Large ䷦}

\section{原文}

\subsection{经}
蹇:利西南,不利东北;利见大人,贞吉。

初六:往蹇,来誉。

六二:王臣蹇蹇,匪躬之故。

九三:往蹇来反。

六四:往蹇来连。

九五:大蹇朋来。

上六:往蹇来硕,吉;利见大人。

\subsection{彖}
蹇,难也,险在前也。 见险而能止,知矣哉!蹇利西南, 往得中也;不利东北,其道穷也。 利见大人,往有功也。 当位贞吉,以正邦也。 蹇之时用大矣哉!

\subsection{象}
山上有水,蹇;君子以反身修德。

初六:往蹇来誉,宜待也。

六二:王臣蹇蹇,终无尤也。

九三:往蹇来反,内喜之也。

六四:往蹇来连,当位实也。

九五:大蹇朋来,以中节也。

上六:往蹇来硕,志在内也。利见大人,以从贵也。

\section{讲解}
卦辞说:

彖说:

象辞说:

\chapter{雷水解(xiè)卦}
解\footnote{查阅之后确认解卦读音为xiè,因为《易·杂卦传》里有:谢,缓也。即谢,通松懈的懈。} {\Large ䷧}

\section{原文}

\subsection{经}
解:利西南,无所往,其来复吉。有攸往,夙\footnote{夙,早也。}吉。

初六:无咎。

九二:田\footnote{田,猎也。}获三狐,得黄矢,贞吉。

六三:负且乘,致寇至,贞吝。

九四:解而拇\footnote{震为足也,此处拇指大脚趾。},朋至斯孚。

六五:君子维\footnote{维犹系也——闻一多。}有解,吉;有孚于小人。

上六:公用射隼\footnote{sǔn,一种凶猛的鸟。},于高墉\footnote{yōng,城墙。}之上,获之,无不利。

\subsection{彖}
解,险以动,动而免乎险,解。解利西南,往得众也,其来复吉,乃得中也。有攸往夙吉,往有功也。天地解,而雷雨作,雷雨作,而百果草木皆甲坼\footnote{chè,甲坼,指草木发芽时种子外皮裂开。},解之时大矣哉!

\subsection{象}
雷雨作,解;君子以赦过宥罪。

初六:刚柔之际,义无咎也。

九二:九二贞吉,得中道也。

六三:负且乘,亦可丑也,自我致戎,又谁咎也。

九四:解而拇,未当位也。

六五:君子有解,小人退也。

上六:公用射隼,以解悖也。

\section{初讲}
卦辞说:解卦,利西南,无所往,返回原来的地方就吉祥【蹇解而难已平,无难则无所往,休养生息之\footnote{参见高岛断易}】。有所往早早行动吉祥【蹇解而难犹在,有难则早早前往之\footnote{参见高岛断易}】。

彖说:解卦,下为坎为险,上为震为动。是故遇到危险而行动,行动而免于危险,这就是解卦了。解卦利西南,往得众也,其来复吉,乃得中也。【坎险已解,九二爻仍返回原处,故曰其来复,爻变为坤为西南,故曰利西南,坤为众也,故曰往得众。九二爻安居于内卦之中,故曰乃得中道\footnote{参考了\href{https://www.eee-learning.com/book/neweee40}{这个网页}}。】有所往早早行动吉祥,前往可获成功。天地阴阳之气缓和而雷雨大作,雷雨大作然后百果草木种子皆发芽,疏解之时也是很伟大的啊。

象辞说:解卦上为震为雷,下为坎为雨,上雷下雨,雷雨大作,这就是解卦的卦象了。君子观此卦象而知道天地疏解的道理,故而赦免那些有过错的人,宽恕那些有罪的人。


初六:初六柔爻与九四刚爻相交,理应无咎。

九二:九二爻猎获了很多狐狸\footnote{狐者,隐伏之物也,隐喻隐藏的某些奸诈小人或者麻烦。},得到了六五君上的黄色箭矢的赏赐,贞吉。九二贞吉,是因为它守中正之道的缘故。

六三:六三爻作小人之事却乘君子之器\footnote{子曰:作易者,其知盜乎!易曰:「負且乘,致寇至。」負也者,小人之事也。乘也者,君子之器也。},招来贼寇的到来,贞吝。负且乘,亦可丑也,自我致戎,又谁咎也。

九四:九四爻疏解自己的大脚趾,因为它位置不当的缘故,被六三爻小人弄得疲惫不堪,六五朋友来了才是可以相信的。

六五:六五君子虽被约束但有解,吉祥,有诚信于小人\footnote{居尊履中而应乎刚,可以有解而获吉矣。以君子之道解难释险,小人虽间,犹知服之而无怨矣。故曰“有孚于小人”也。——王弼}。君子有解,小人自退。

上六:上六王公爻射隼于高高的城墙之上,获之,无不利。公用射隼,以解悖也。上六疏解之极也,尤有悖逆者,非射杀之不能解也。


\chapter{山泽损卦}
损 {\Large ䷨}
\section{原文}

\subsection{经}
损:有孚,元吉,无咎,可贞,利有攸往。曷之用?二簋可用享。

初九:已事遄往,无咎,酌损之。

九二:利贞,征凶,弗损益之。

六三:三人行,则损一人;一人行,则得其友。

六四:损其疾,使遄有喜,无咎。

六五:或益之,十朋之龟弗克违,元吉。

上九:弗损益之,无咎,贞吉,利有攸往,得臣无家。

\subsection{彖}
损,损下益上,其道上行。损而有孚,元吉,无咎,可贞,利有攸往。 曷之用? 二簋可用享;二簋应有时。损刚益柔有时,损益盈虚,与时偕行。

\subsection{象}
山下有泽,损;君子以惩忿窒欲。

初九:已事遄往,尚合志也。

九二:九二利贞,中以为志也。

六三:一人行,三则疑也。

六四:损其疾,亦可喜也。

六五:六五元吉,自上佑也。

上九:弗损益之,大得志也。

\section{讲解}
卦辞说:

彖说:

象辞说:

\chapter{风雷益卦}
益 {\Large ䷩}
\section{原文}

\subsection{经}
益:利有攸往,利涉大川。

初九:利用为大作,元吉,无咎。

六二:或益之,十朋之龟弗克违,永贞吉。 王用享于帝,吉。

六三:益之用凶事,无咎。 有孚中行,告公用圭。

六四:中行,告公从。 利用为依迁国。

九五:有孚惠心,勿问元吉。 有孚惠我德。

上九:莫益之,或击之,立心勿恒,凶。

\subsection{彖}
益,损上益下,民说无疆,自上下下,其道大光。利有攸往,中正有庆。利涉大川,木道乃行。 益动而巽,日进无疆。天施地生,其益无方。 凡益之道,与时偕行。

\subsection{象}
风雷,益;君子以见善则迁,有过则改。

初九:元吉无咎,下不厚事也。

六二:或益之,自外来也。

六三:益用凶事,固有之也。

六四:告公从,以益志也。

九五:有孚惠心,勿问之矣。惠我德,大得志也。

上九:莫益之,偏辞也。或击之,自外来也。

\section{讲解}
卦辞说:

彖说:

象辞说:

\chapter{泽天夬(guài)卦}
夬 {\Large ䷪}
\section{原文}

\subsection{经}
夬:扬于王庭,孚号,有厉,告自邑,不利即戎,利有攸往。

初九:壮于前趾,往不胜为咎。

九二:惕号,莫夜有戎,勿恤。

九三:壮于頄,有凶。君子夬夬,独行遇雨,若濡有愠,无咎。

九四:臀无肤,其行次且。 牵羊悔亡,闻言不信。

九五:苋陆夬夬,中行无咎。

上六:无号,终有凶。

\subsection{彖}
夬,决也,刚决柔也。健而说,决而和,扬于王庭,柔乘五刚也。孚号有厉,其危乃光也。 告自邑,不利即戎,所尚乃穷也。利有攸往,刚长乃终也。

\subsection{象}
泽上于天,夬;君子以施禄及下,居德则忌。

初九:不胜而往,咎也。

九二:有戎勿恤,得中道也。

九三:君子夬夬,终无咎也。

九四:其行次且,位不当也。闻言不信,聪不明也。

九五:中行无咎,中未光也。

上六:无号之凶,终不可长也。

\section{讲解}
卦辞说:

彖说:

象辞说:

\chapter{天风姤(gòu)卦}
姤 {\Large ䷫}

\section{原文}

\subsection{经}
姤:女壮,勿用取女。

初六:系于金柅,贞吉,有攸往,见凶,羸豕孚蹢躅。

九二:包有鱼,无咎,不利宾。

九三:臀无肤,其行次且,厉,无大咎。

九四:包无鱼,起凶。

九五:以杞包瓜,含章,有陨自天。

上九:姤其角,吝,无咎。


\subsection{彖}
姤,遇也,柔遇刚也。勿用取女,不可与长也。天地相遇,品物咸章也。刚遇中正,天下大行也。姤之时义大矣哉!

\subsection{象}
天下有风,姤;后以施命诰四方。

初六:系于金柅,柔道牵也。

九二:包有鱼,义不及宾也。

九三:其行次且,行未牵也。

九四:无鱼之凶,远民也。

九五:九五含章,中正也。有陨自天,志不舍命也。

上九:姤其角,上穷吝也。


\section{讲解}
卦辞说:

彖说:

象辞说:


\chapter{泽地萃(cuì)卦}
萃 {\Large ䷬}
\section{原文}

\subsection{经}
萃:亨。 王假有庙,利见大人,亨,利贞。 用大牲吉,利有攸往。

初六:有孚不终,乃乱乃萃,若号一握为笑,勿恤,往无咎。

六二:引吉,无咎,孚乃利用禴。

六三:萃如,嗟如,无攸利,往无咎,小吝。

九四:大吉,无咎。

九五:萃有位,无咎。 匪孚,元永贞,悔亡。

上六:赍咨涕洟,无咎。

\subsection{彖}
萃,聚也;顺以说,刚中而应,故聚也。王假有庙,致孝享也。利见大人亨,聚以正也。用大牲吉,利有攸往,顺天命也。观其所聚,而天地万物之情可见矣。

\subsection{象}
泽上於地,萃;君子以除戎器,戒不虞。

初六:乃乱乃萃,其志乱也。

六二:引吉无咎,中未变也。

六三:往无咎,上巽也。

九四:大吉无咎,位不当也。

九五:萃有位,志未光也。

上六:赍咨涕洟,未安上也。

\section{讲解}
卦辞说:

彖说:

象辞说:

\chapter{地风升卦}
升 {\Large ䷭}

\section{原文}

\subsection{经}
升:元亨,用见大人,勿恤,南征吉。

初六:允升,大吉。

九二:孚乃利用禴,无咎。

九三:升虚邑。

六四:王用亨于岐山,吉无咎。

六五:贞吉,升阶。

上六:冥升,利于不息之贞。


\subsection{彖}
柔以时升,巽而顺,刚中而应,是以大亨。用见大人,勿恤;有庆也。 南征吉,志行也。

\subsection{象}
地中生木,升;君子以顺德,积小以高大。

初六:允升大吉,上合志也。

九二:九二之孚,有喜也。

九三:升虚邑,无所疑也。

六四:王用亨于岐山,顺事也。

六五:贞吉升阶,大得志也。

上六:冥升在上,消不富也。

\section{讲解}
卦辞说:

彖说:

象辞说:

\chapter{泽水困卦}
困 {\Large ䷮}

\section{原文}

\subsection{经}
困:亨,贞,大人吉,无咎,有言不信。

初六:臀困于株木,入于幽谷,三岁不觌。

九二:困于酒食,朱绂方来,利用亨祀,征凶,无咎。

六三:困于石,据于蒺藜,入于其宫,不见其妻,凶。

九四:来徐徐,困于金车,吝,有终。

九五:劓刖,困于赤绂,乃徐有说,利用祭祀。

上六:困于葛藟,于臲卼,曰动悔。 有悔,征吉。

\subsection{彖}
困,刚掩也。险以说,困而不失其所,亨;其唯君子乎?贞大人吉,以刚中也。有言不信,尚口乃穷也。

\subsection{象}
泽无水,困;君子以致命遂志。

初六:入于幽谷,幽不明也。

九二:困于酒食,中有庆也。

六三:据于蒺藜,乘刚也。入于其宫,不见其妻,不祥也。

九四:来徐徐,志在下也。虽不当位,有与也。

九五:劓刖,志未得也。乃徐有说,以中直也。利用祭祀,受福也。

上六:困于葛藟,未当也。动悔,有悔吉,行也。

\section{讲解}
卦辞说:

彖说:

象辞说:

\chapter{水风井卦}
井 {\Large ䷯}

\section{原文}

\subsection{经}
井:改邑不改井,无丧无得,往来井井。汔至亦未繘\footnote{繘(jú):井上汲水的绳子,繘井即用绳汲取井水。}井,羸其瓶,凶。

初六:井泥不食,旧井无禽。

九二:井谷射鲋,瓮敝漏。

九三:井渫不食,为我心恻,可用汲,王明,并受其福。

六四:井甃,无咎。

九五:井冽,寒泉食。

上六:井收勿幕,有孚元吉。

\subsection{彖}
巽乎水而上水,井;井养而不穷也。改邑不改井,乃以刚中也。汔至亦未繘井,未有功也。羸其瓶,是以凶也。

\subsection{象}
木上有水,井;君子以劳民劝相。

初六:井泥不食,下也。旧井无禽,时舍也。

九二:井谷射鲋,无与也。

九三:井渫不食,行恻也。求王明,受福也。

六四:井甃无咎,修井也。

九五:寒泉之食,中正也。

上六:元吉在上,大成也。

\section{讲解}
卦辞说:村邑会改迁但井制不会改变,人们来来往往打取井水,井自身没有什么损失也没什么获得。用绳打取井水几乎快要到了,但还没有打到井水,此时绳子还缠绕着水瓶,凶险。

彖说:巽作用于水然后把水弄上来,这就是井了;井养而不穷也。改邑不改井,乃以刚中也。用绳打取井水几乎快要到了,但还没有打到井水,所以称未有功也。此时绳子还缠绕着水瓶,而水瓶有触碰井壁的风险,是以凶也。

象辞说:木上面有水,这就是井的卦象了。井是民众共同劳作出来的,有了井水民众共同享用而得互相帮助,故君子观井之成和井之用而效仿之,这就是劳民劝相的道理了。

国易堂说:占得此卦者,在事业上处于平稳状态,当下不宜贸然前进,也不必后退,而应以积极的态度努力进修,提高自己,充实个人实力,待机而起。要注意与人的合作,相互协助。

在求名方面,应特别注意向贤德的人求教,以便被发现而受到推荐。同时要学习水井的精神,真诚奉献,不断丰富自己的才能,这样一定会受到社会的重若想外出,则要提前做好准备,若没有十分的必要和充分的把握不可随意出行。

在婚恋方面,不必因着急而结婚,会有般配的伴侣出现。

九三:井已经清理干净但是人们却不来饮用,为此我心甚是伤悲。可以来汲取井水了,盼望君王贤明,并受其福啊。此爻有怀才不遇之象,高岛断易说再过二爻时间\footnote{所谓二爻时间是按照你起卦时心中所想的时间刻度为基准,比如占之年,则再过二爻即再过两年;占之月,则再过两月。}可有转机。


\chapter{泽火革卦}
革 {\Large ䷰}

\section{原文}

\subsection{经}
革:巳日乃孚,元亨利贞,悔亡。

初九:巩用黄牛之革。

六二:巳日乃革之,征吉,无咎。

九三:征凶,贞厉,革言三就,有孚。

九四:悔亡,有孚改命,吉。

九五:大人虎变,未占有孚。

上六:君子豹变,小人革面,征凶,居贞吉。

\subsection{彖}
革,水火相息,二女同居,其志不相得,曰革。巳日乃孚;革而信之。文明以说,大亨以正,革而当,其悔乃亡。天地革而四时成,汤武革命,顺乎天而应乎人,革之时大矣哉!

\subsection{象}
泽中有火,革;君子以治历明时。

初九:巩用黄牛,不可以有为也。

六二:巳日革之,行有嘉也。

九三:革言三就,又何之矣。

九四:改命之吉,信志也。

九五:大人虎变,其文炳也。

上六:君子豹变,其文蔚也。小人革面,顺以从君也。

\section{讲解}
卦辞说:

彖说:

象辞说:

\chapter{火风鼎卦}
鼎 {\large ䷱}
\section{原文}

\subsection{经}
鼎:元吉,亨。

初六:鼎颠趾,利出否,得妾以其子,无咎。

九二:鼎有实,我仇有疾,不我能即,吉。

九三:鼎耳革,其行塞,雉膏不食,方雨亏悔,终吉。

九四:鼎折足,覆公餗,其形渥,凶。

六五:鼎黄耳金铉,利贞。

上九:鼎玉铉,大吉,无不利。

\subsection{彖}
鼎,象也。以木巽火,亨饪也。圣人亨以享上帝,而大亨以养圣贤。巽而耳目聪明,柔进而上行,得中而应乎刚,是以元亨。

\subsection{象}
木上有火,鼎;君子以正位凝命。

初六:鼎颠趾,未悖也。利出否,以从贵也。

九二:鼎有实,慎所之也。我仇有疾,终无尤也。

九三:鼎耳革,失其义也。

九四:覆公餗,信如何也。

六五:鼎黄耳,中以为实也。

上九:玉铉在上,刚柔节也。

\section{讲解}
卦辞说:鼎卦,大吉祥,亨通。

彖说:鼎卦的形状就像一个鼎,因为巽为木,离为火,以木生火烧鼎,可以进行烹饪之事。圣人用鼎烹饪用来祭拜上天,而大量的烹饪食物是用来供养圣贤。巽为风无孔不入表明此人耳目聪明,风很柔软表明此人柔顺上进,只要行动适中,就会受到刚强者的响应,因此卦辞上说元亨。

象辞说:木上有火,便是鼎卦的卦象。君子从这个卦象得到启发,应当端正自己的位置,重视上天赋予的使命。

国易堂说:占得此卦者,已经具备开拓事业的各种条件。个人条件很好,聪明冷静,但要注意应以端正的态度去为人处世,严于律已,无轻举妄动和邪思,刚中自守。要注意培养和吸收人才,为己所用。要与有才德的人合作,这样更利成功。

在求名上,首先应严于律已,不陷入与他人的怨仇之中,柔而上行,循序渐进,具备随时应变和随势应变的能力。如果得到知人者的善用,更是前途广大。即使暂时不受重视,无出路也无防,最终可实现抱负。

在外出方面,无重大事情不宜外出,如果为了工作和事业而出差,则会很顺

在婚恋方面,应该说个人条件比较不错,但是不要自视甚高,在选择另一半时要切合自己的实际。对于已婚人士来说,要注意不要出现三角恋爱、第三者插足的情况。

在身体健康方面,因为鼎除能烹煮食物外,还像煮药的砂锅,所以自己或家人可能会有因病而吃汤药的。




\chapter{震卦}
震 {\large ䷲}
\section{原文}

\subsection{经}
震:亨。 震来虩虩,笑言哑哑。 震惊百里,不丧匕鬯。

初九:震来虩虩,后笑言哑哑,吉。

六二:震来厉,亿丧贝,跻于九陵,勿逐,七日得。

六三:震苏苏,震行无眚。

九四:震遂泥。

六五:震往来厉,亿无丧,有事。

上六:震索索,视矍矍,征凶。 震不于其躬,于其邻,无咎。 婚媾有言。

\subsection{彖}
震,亨。震来虩虩,恐致福也。笑言哑哑,后有则也。震惊百里,惊远而惧迩也。 出可以守宗庙社稷,以为祭主也。

\subsection{象}
洊雷,震;君子以恐惧修省。

初九:震来虩虩,恐致福也。笑言哑哑,后有则也。

六二:震来厉,乘刚也。

六三:震苏苏,位不当也。

九四:震遂泥,未光也。

六五:震往来厉,危行也。其事在中,大无丧也。

上六:震索索,未得中也。虽凶无咎,畏邻戒也。


\section{讲解}
卦辞说:

彖说:

象辞说:



\chapter{艮卦}
艮 {\Large ䷳}


\section{原文}

\subsection{经}
艮:艮其背,不获其身,行其庭,不见其人,无咎。

初六:艮其趾,无咎,利永贞。

六二:艮其腓,不拯其随,其心不快。

九三:艮其限,列其夤,厉薰心。

六四:艮其身,无咎。

六五:艮其辅,言有序,悔亡。

上九:敦艮,吉。


\subsection{彖}
艮,止也。时止则止,时行则行,动静不失其时,其道光明。艮其止,止其所也。上下敌应,不相与也。 是以不获其身,行其庭不见其人,无咎也。

\subsection{象}
兼山,艮;君子以思不出其位。

初六:艮其趾,未失正也。

六二:不拯其随,未退听也。

九三:艮其限,危薰心也。

六四:艮其身,止诸躬也。

六五:艮其辅,以中正也。

上九:敦艮之吉,以厚终也。

\section{讲解}
卦辞说:

彖说:

象辞说:

\chapter{风山渐卦}
渐 {\Large ䷴}


\section{原文}

\subsection{经}
渐:女归吉,利贞。

初六:鸿渐于干,小子厉,有言,无咎。

六二:鸿渐于磐,饮食衎衎,吉。

九三:鸿渐于陆,夫征不复,妇孕不育,凶;利御寇。

六四:鸿渐于木,或得其桷,无咎。

九五:鸿渐于陵,妇三岁不孕,终莫之胜,吉。

上九:鸿渐于陆,其羽可用为仪,吉。

\subsection{彖}
渐之进也,女归吉也。进得位,往有功也。进以正,可以正邦也。其位刚,得中也。止而巽,动不穷也。

\subsection{象}
山上有木,渐;君子以居贤德,善俗。

初六:小子之厉,义无咎也。

六二:饮食衎衎,不素饱也。

九三:夫征不复,离群丑也。 妇孕不育,失其道也。利用御寇,顺相保也。

六四:或得其桷,顺以巽也。

九五:终莫之胜,吉;得所愿也。

上九:其羽可用为仪,吉;不可乱也。


\section{讲解}
卦辞说:

彖说:

象辞说:

\chapter{雷泽归妹卦}
归妹 {\Large ䷵}


\section{原文}

\subsection{经}
归妹:征凶,无攸利。

初九:归妹以娣,跛能履,征吉。

九二:眇能视,利幽人之贞。

六三:归妹以须,反归以娣。

九四:归妹愆期,迟归有时。

六五:帝乙归妹,其君之袂,不如其娣之袂良,月几望,吉。

上六:女承筐无实,士刲羊无血,无攸利。

\subsection{彖}
归妹,天地之大义也。天地不交,而万物不兴,归妹人之终始也。说以动,所归妹也。征凶,位不当也。 无攸利,柔乘刚也。

\subsection{象}
泽上有雷,归妹;君子以永终知敝。

初九:归妹以娣,以恒也。跛能履吉,相承也。

九二:利幽人之贞,未变常也。

六三:归妹以须,未当也。

九四:愆期之志,有待而行也。

六五:帝乙归妹,不如其娣之袂良也。其位在中,以贵行也。

上六:上六无实,承虚筐也。

\section{讲解}
卦辞说:

彖说:

象辞说:

\chapter{雷火丰卦}
丰 {\Large ䷶}

\section{原文}

\subsection{经}
丰:亨,王假之,勿忧,宜日中。

初九:遇其配主,虽旬无咎,往有尚。

六二:丰其蔀,日中见斗,往得疑疾,有孚发若,吉。

九三:丰其沛,日中见沫,折其右肱,无咎。

九四:丰其蔀,日中见斗,遇其夷主,吉。

六五:来章,有庆誉,吉。

上六:丰其屋,蔀其家,窥其户,阒其无人,三岁不觌,凶。

\subsection{彖}
丰,大也。明以动,故丰。王假之,尚大也。勿忧宜日中,宜照天下也。日中则昃,月盈则食,天地盈虚,与时消息,而况人於人乎?况於鬼神乎?

\subsection{象}
雷电皆至,丰;君子以折狱致刑。

初九:虽旬无咎,过旬灾也。

六二:虽旬无咎,过旬灾也。

九三:丰其沛,不可大事也。折其右肱,终不可用也。

九四:丰其蔀,位不当也。日中见斗,幽不明也。遇其夷主,吉;行也。

六五:六五之吉,有庆也。

上六:丰其屋,天际翔也。窥其户,阒其无人,自藏也。

\section{讲解}
卦辞说:

彖说:

象辞说:

\chapter{火山旅卦}
旅 {\Large ䷷}


\section{原文}

\subsection{经}
旅:小亨,旅贞吉。

初六:旅琐琐,斯其所取灾。

六二:旅即次,怀其资,得童仆贞。

九三:旅焚其次,丧其童仆,贞厉。

九四:旅于处,得其资斧,我心不快。

六五:射雉一矢亡,终以誉命。

上九:鸟焚其巢,旅人先笑后号啕。丧牛于易,凶。

\subsection{彖}
旅,小亨,柔得中乎外,而顺乎刚,止而丽乎明,是以小亨,旅贞吉也。旅之时义大矣哉!

\subsection{象}
山上有火,旅;君子以明慎用刑,而不留狱。

初六:旅琐琐,志穷灾也。

六二:得童仆贞,终无尤也。

九三:旅焚其次,亦以伤矣。以旅与下,其义丧也。

九四:旅于处,未得位也。得其资斧,心未快也。

六五:终以誉命,上逮也。

上九:以旅在上,其义焚也。丧牛于易,终莫之闻也。

\section{讲解}
卦辞说:

彖说:

象辞说:

\chapter{巽(xùn)卦}
巽 {\Large ䷸}

\section{原文}

\subsection{经}
巽:小亨,利攸往,利见大人。

初六:进退,利武人之贞。

九二:巽在床下,用史巫纷若,吉无咎。

九三:频巽,吝。

六四:悔亡,田获三品。

九五:贞吉悔亡,无不利。 无初有终,先庚三日,后庚三日,吉。

上九:巽在床下,丧其资斧,贞凶。

\subsection{彖}
重巽以申命,刚巽乎中正而志行。柔皆顺乎刚,是以小亨,利有攸往,利见大人。

\subsection{象}
随风,巽;君子以申命行事。

初六:进退,志疑也。利武人之贞,志治也。

九二:纷若之吉,得中也。

九三:频巽之吝,志穷也。

六四:田获三品,有功也。

九五:九五之吉,位正中也。

上九:巽在床下,上穷也。丧其资斧,正乎凶也。

\section{讲解}
卦辞说:

彖说:

象辞说:


\chapter{兑卦}
兑 {\Large ䷹}

\section{原文}

\subsection{经}
兑:亨,利贞。

初九:和兑,吉。

九二:孚兑,吉,悔亡。

六三:来兑,凶。

九四:商兑,未宁,介疾有喜。

九五:孚于剥,有厉。

上六:引兑。

\subsection{彖}
兑,说也。刚中而柔外,说以利贞,是以顺乎天,而应乎人。说以先民,民忘其劳;说以犯难,民忘其死;说之大,民劝矣哉!

\subsection{象}
丽泽,兑;君子以朋友讲习。

初九:和兑之吉,行未疑也。

九二:孚兑之吉,信志也。

六三:来兑之凶,位不当也。

九四:来兑之凶,位不当也。

九五:孚于剥,位正当也。

上六:上六引兑,未光也。

\section{讲解}
卦辞说:

彖说:

象辞说:

\chapter{风水涣卦}
涣 {\Large ䷺}

\section{原文}

\subsection{经}
涣:亨。王假有庙,利涉大川,利贞。

初六:用拯马壮,吉。

九二:涣奔其机,悔亡。

六三:涣其躬,无悔。

六四:涣其群,元吉。 涣有丘,匪夷所思。

九五:涣汗其大号,涣王居,无咎。

上九:涣其血,去逖出,无咎。

\subsection{彖}
涣,亨。 刚来而不穷,柔得位乎外而上同。王假有庙,王乃在中也。利涉大川,乘木有功也。

\subsection{象}
风行水上,涣;先王以享于帝立庙。

初六:初六之吉,顺也。

九二:涣奔其机,得愿也。

六三:涣其躬,志在外也。

六四:涣其群,元吉;光大也。

九五:王居无咎,正位也。

上九:涣其血,远害也。

\section{讲解}
卦辞说:

彖说:

象辞说:

\chapter{水泽节卦}
节 {\Large ䷻}


\section{原文}

\subsection{经}
节:亨。苦节不可贞。

初九:不出户庭,无咎。

九二:不出门庭,凶。

六三:不节若,则嗟若,无咎。

六四:安节,亨。

九五:甘节,吉;往有尚。

上六:苦节,贞凶,悔亡。


\subsection{彖}
节,亨,刚柔分,而刚得中。苦节不可贞,其道穷也。说以行险,当位以节,中正以通。天地节而四时成,节以制度,不伤财,不害民。

\subsection{象}
泽上有水,节;君子以制数度,议德行。

初九:不出户庭,知通塞也。

九二:不出门庭,失时极也。

六三:不节之嗟,又谁咎也。

六四:安节之亨,承上道也。

九五:甘节之吉,居位中也。

上六:苦节贞凶,其道穷也。


\section{讲解}
卦辞说:

彖说:

象辞说:

\chapter{风泽中孚(fú)卦}
中孚 {\LARGE ䷼}


\section{原文}

\subsection{经}
中孚:豚鱼吉,利涉大川,利贞。

\subsection{彖}
中孚,柔在内而刚得中。说而巽,孚,乃化邦也。豚鱼吉,信及豚鱼也。利涉大川,乘木舟虚也。中孚以利贞,乃应乎天也。

\subsection{象}
泽上有风,中孚;君子以议狱缓死。

\chapter{雷山小过卦}
小过 {\LARGE ䷽}

\section{原文}
\subsection{经}
小过:亨,利贞,可小事,不可大事。飞鸟遗之音,不宜上,宜下,大吉。

初六:飞鸟以凶。

六二:过其祖,遇其妣;不及其君,遇其臣;无咎。

九三:弗过防之,从或戕之,凶。

九四:无咎,弗过遇之。 往厉必戒,勿用永贞。

六五:密云不雨,自我西郊,公弋取彼在穴。

上六:弗遇过之,飞鸟离之,凶,是谓灾眚。

\subsection{彖}
小过,小者过而亨也。过以利贞,与时行也。柔得中,是以小事吉也。刚失位而不中,是以不可大事也。有飞鸟之象焉,有飞鸟遗之音,不宜上宜下,大吉;上逆而下顺也。

\subsection{象}
山上有雷,小过;君子以行过乎恭,丧过乎哀,用过乎俭。

初六:飞鸟以凶,不可如何也。

六二:不及其君,臣不可过也。

九三:从或戕之,凶如何也。

九四:弗过遇之,位不当也。 往厉必戒,终不可长也。

六五:密云不雨,已上也。

上六:弗遇过之,已亢也。

\section{讲解}
卦辞说:

彖说:

象辞说:

序卦曰:“有其信者,必行之,故受之以小过。” 。有诚信的人必然会行动,因故而得小过卦。

象曰:“山上有雷,小过;”。这里指出小过的卦象就是山上有雷。山上有雷,雷会把山上的树木击坏,但没有雷怎有雨,没有雨山上的树木又怎得滋润和生长。

象曰:“君子以行过乎恭,丧过乎哀,用过乎俭。”。指的是所以君子参见小过的卦象之后应该行为上过于恭敬,丧事上过于悲哀,日用上过于节俭。此为小小过分之理也。

经谈“飞鸟”,大概因为小过卦从卦象看像一只展翅的飞鸟。

小过,亨通,利于坚守中正之道,可以做一些小事,不要去做大事。飞鸟飞过留下声音,不宜向上飞,宜于向下飞,如此则大为吉祥。因为上飞




\chapter{水火既济卦}
既济 {\Large ䷾}
\section{原文}

\subsection{经}
既济:亨,小利贞,初吉终乱。

初九:曳其轮,濡其尾,无咎。

六二:妇丧其髴,勿逐,七日得。

九三:高宗伐鬼方,三年克之,小人勿用。

六四:繻有衣袽,终日戒。

九五:东邻杀牛,不如西邻之禴祭,实受其福。

上六:濡其首,厉。

\subsection{彖}
既济,亨,小者亨也。利贞,刚柔正而位当也。初吉,柔得中也。终止则乱,其道穷也。

\subsection{象}
水在火上,既济;君子以思患而豫防之。

初九:曳其轮,义无咎也。

六二:七日得,以中道也。

九三:三年克之,惫也。

六四:终日戒,有所疑也。

九五:东邻杀牛,不如西邻之时也;实受其福,吉大来也。

上六:濡其首厉,何可久也。

\section{讲解}
卦辞说:

彖说:

象辞说:

\chapter{火水未济卦}
未济 {\Large ䷿}
\section{原文}

\subsection{经}
未济:亨,小狐汔济,濡(rú)其尾,无攸利。

初六:濡其尾,吝。

九二:曳其轮,贞吉。

六三:未济,征凶,利涉大川。

九四:贞吉,悔亡,震用伐鬼方,三年有赏于大国。

六五:贞吉,无悔,君子之光,有孚,吉。

上九:有孚于饮酒,无咎,濡其首,有孚失是。

\subsection{彖}
未济,亨;柔得中也。小狐汔济,未出中也。濡其尾,无攸利;不续终也。虽不当位,刚柔应也。

\subsection{象}
火在水上,未济;君子以慎辨物居方。

初六:濡其尾,亦不知极也。

九二:九二贞吉,中以行正也。

六三:未济征凶,位不当也。

九四:贞吉悔亡,志行也。

六五:君子之光,其晖吉也。

上九:饮酒濡首,亦不知节也。

\section{讲解}
卦辞说:亨通,小狐狸渡河快到了,尾巴濡湿了,没什么利益。

彖说:

象辞说:

九四:







\part{周易相关}
\chapter{基本术语}
\section{易经}
易经原有三,连山易,归藏易,周易,前两易已失传,

\section{爻的当位和不当位}
认为爻位从下往上数,奇数为阳,偶数为阴,于是奇数位为阳位,偶数位为阴位,若阳爻居阳位则为当位,若阴爻居阴位也为当位,反之为不当位。

\section{卦彖象}
严格意义上来说周易的作者不详,只能说是成书于周朝,最大的可能是周文王指定,其朝中文官收集史料编纂而成。其他彖、象、文言、系、序等是孔子所作的注解,当然可能会有其门人的部分贡献,但绝大部分应该都是孔子所作。

\section{变爻}
爻分为二,若为少阴少阳则该爻不动。若为老阴老阳则该爻为变爻,变动之爻。

\section{先天六十四卦顺序和后天六十四卦顺序}
周易一书的顺序或者说按照序卦传而来的顺序通常被人们称为后天六十四卦顺序,这个顺序更多的反应了作者认为事物发展的一种哲理性解释。

通常预测会按照先天六十四卦顺序来,先天六十四卦顺序就是在变爻到六爻的阶段,最终发展到不可逆转进而形成变卦。先天六十四卦更多的是揭示天理自然规律,而后天六十四严格意义上来说并没有顺序一说,只是方便大家阅读而解释出来的那个顺序。

\section{尊卑贵贱上下}
周易里面尊卑,贵贱,上下各自是分开的,虽然人们常谈尊贵,卑下,下贱,但正所谓上位也可能不尊,下位也有高贵之民,看看周易里面系辞部分说的很清楚:“天尊地卑,乾坤定矣。卑高以陈,贵贱位矣。”尊卑本只是无褒贬含义的高和低的意思,天高高在上,地低低在下,乾坤就这样确定下来了。然后注意下面的\emph{卑高}这个词的顺序 ,正是从人的角度去看,先看到地,再看到天,是言卑高。关于这块南怀瑾说的很好,人性就是如此,容易摸得着的东西就轻贱它,总是得不到的东西就觉得很珍贵。周易在这里说的很明白,本来天尊地卑,并没有贵贱概念,因为人性,所以出现了贵贱的概念了。

在周易里面上下有统治或管理上的上下位之分,再一次将上下和贵贱和尊卑混为一谈那是后来人的私心想法。周易里面谈上下更多是让人们注意到管理上的上下位之别,比如履卦的“君子辩上下”,其正对应孔子谈为政的核心原则就是:“君君,臣臣,父父,子子”。当然孔子的谈论更多的局限在古代社会的君臣父子这几大关系上做了比喻和延伸,就现代社会而言管理上的核心原则其实仍然是一致的,即上位要有上位的样子,下位要有下位的样子。各尽分工,各尽其职。

\section{大人}
周易谈论的大人肯定不是指官位上的大人,从周易常现的利见大人这几个字可以推断大人是指在本卦象上有能力帮助你的人。

\section{元亨利贞}
元更多的形容词含义,大的意思。比如“元亨”就是大亨通,“元吉”就是大吉祥。亨则是亨通的含义,并没有太大问题。利就是有利的意思。就是这个贞在解读上还是有点歧义的,比如有的地方是“小贞吉,大贞凶。”,有的地方是“利牝马之贞”。贞字在后的一般解读为贞正之道,这是没有问题的。然后我否定了认为贞就是占卜的意思这一说法,古人写文特别讲究言简意赅,本就在行占卜之事,若吉祥则最多用一个字,吉,而不会赘言贞吉。周易并没有一味要求人们去坚守贞正之道,在某些命运乖巧时运不济的时候,是不利于人们坚守贞正之道的,这是很实际的,同时在某些卦里面特别强调贞正之道的某些方面,比如“牝马之贞”或者“安贞”等。

那么贞字还有另外的含义吗,比如“小贞吉,大贞凶。”很多人都解释为小事吉祥,大事凶,又有人将贞字解释为占卜,我们现在假设小在古代还有小事的意思,那么他如果要表达小事吉祥,则只需要两个字小吉即可,而事实是小和大古今差异都不大,都是一个形容程度的词语。

要理解贞字最关键的是弄明白这个贞字原在人们中的头脑具象是何种情景,后面的名词动词形容词含义都是由这个头脑具象衍生出来的。我观察汉典上的字源字形,大体可以推测这更多的也是一个青铜器形象,比鼎小,考虑到真的形象也是类似的青铜器形象,而贞和蒸在读音上的接近,这些都不是偶然。贞字有些字形象会发现去掉了火,或者两个脚抬得很高,我可以推测这个青铜器后面渐渐成了一种祭祀相关用器,而且下面不再直接生火了。不管怎么说,我都不肯定贞字有占卜的含义在里面,唯一把贞字做占卜解释的那本书就是《周礼》,而这本书目前似乎人们已经断定它是成书于两汉之间,我们看到,在周易行文的时候,贞字基本上已经抽象为一种贞正的美德的含义了。所以我是赞同《周礼》这本书是伪托性质的,至少是成书在周易之后,然后作者因此误将贞字衍生出了占卜的意思。但我从贞字的形象是怎么看不到占卜一事的,至于说文解字将贞解作卜贝,又是希望将贞这个词的含义往占卜上靠,也可能是谬误的。

总的来说,我否认了贞字有占卜含义在里面。继而得出结论,贞作动词就是行贞正之道,作名词就是贞正之道的意思。我不管贞这个容器在很久以前具体代表是什么礼仪或者祭祀事宜,这个实在难以考究了。但至少在周易成书之时,该字已经完全抽象化了,而至于认为贞为占卜意思的说法都不过是后人杜撰假想出来的。

那么为什么有的时候是利贞,有的时候贞凶呢?文天祥从容就义是完全了贞正之道,贞正之道简言之是坚持了人的正气。天地人以人为重为尊,然非常态。在某些形式下,天地太和之气不调,人坚守自己的正气不一定是吉利的。贞固然是值得夸奖的人的美德,但有时不合于天道,此亦非善也。

\section{贞吉和贞凶和贞吝和贞厉}
贞吉和贞凶即吉祥和凶险的意思。而贞厉,厉通危也,故贞厉的意思是守正道会很危险。

贞吝这个吝不是吝啬的意思,吝通遴,难行也。即守正道会很难行。

\section{无咎}
无咎这个词在周易中出现了很多次,有的会将其解为灾祸的意思,这不是正确的。更正确的解释是过错,罪过的意思。而作动词则有怪罪,罪责的意思。

\section{利涉大川}
利涉大川,指当前形势利于克服艰难险阻。涉大川,指如同跋涉大山大川一般困难的事情;同时利涉大川似乎还有利于远行之意。


\section{先天八卦和后天八卦的区别}
先天八卦顺序并不是那么重要,先天八卦也没有方位一说。

后天八卦有顺序有方位,其可用于预测。


\section{皇极经世的时间刻度}
除去乾坤坎离四卦的先天六十四卦按照顺序每一卦管六运总共三百六十运。具体就是六十卦按照变爻从初爻变动到六爻分为六个阶段也就是这六个卦,这样三百六十运就分别对应了三百六十个卦,这个卦叫做值运之卦。一运360年。

在此值运之卦下,继续按照变爻从初爻变动到六爻分为六个阶段从而得到值世之卦,这个值世之卦管两世也就是六十年。

将这个值世之卦按照上面讨论的六十卦顺序从第一个值世之卦开始算起,一年一卦得值年之卦。



\part{附录}
\chapter{参考资料}
\begin{itemize}
\item \href{http://www.quanxue.cn/QT_XiaoYa/YiJingIndex.html}{劝学网小雅易经入门学习教程}
\item \href{http://www.guoyi360.com/zyqs/}{guoyi360国易堂网周易全解}
\item \href{http://www.xshiqi.com/category_zyzs/dgzs/gddy}{高岛断易}
\item \href{http://www.quanxue.cn/QT_MingXiang/ZhouYiZhuIndex.html}{王弼周易注}
\item \href{https://www.zdic.net/}{汉典}
\item \href{http://vsucai.cn/yizhuan/index.html}{国学经典网易传}
\item \href{http://www.xshiqi.com/category_zyzs/dgzs/zxyz/}{执象易注}
\item \href{https://www.eee-learning.com/article/897}{伊川易传}
\item \href{https://ctext.org/book-of-changes/zhs}{中国哲学书电子化计划周易}
\end{itemize}








% 编者:万泽
\end{document}


