% !Mode:: "TeX:UTF-8"

\documentclass[12pt,oneside]{book}

\newlength{\textpt}
\setlength{\textpt}{11pt}
    
\newcommand{\flypage}[1]{\begin{titlepage}\begin{center}\vspace*{\stretch{1}}#1\vspace*{\stretch{1}}\end{center}\end{titlepage}}
    
%========基本必备的宏包========%
\RequirePackage{calc,float,moresize}
%\RequirePackage[onehalfspacing]{setspace}
\linespread{1.5}
%1.3 onehalfspacing
%试卷或需要文字紧凑的
%1.6 doublespacing

%===========加入目录 某章或某节=====%
\makeatletter

\newcommand{\addchtoc}[1]{
        \cleardoublepage
        \phantomsection
        \addcontentsline{toc}{chapter}{#1}}

\newcommand{\addsectoc}[1]{
        \phantomsection
        \addcontentsline{toc}{section}{#1}}

%===========全文基本格式==========%
\setlength{\parskip}{1.6ex plus 0.2ex minus 0.2ex}   %段落間距
\setlength{\parindent}{\textpt * \real{2}}

%=========页面设置=========%
\RequirePackage[a4paper, %a4paper size 297:210 mm
  bindingoffset=10mm,%裝訂線
  top=35mm,  %上邊距 包括頁眉
  bottom=30mm,%下邊距 包括頁腳
  inner=10mm,  %左邊距or inner
  outer=10mm,  %右邊距or  outer
  headheight=10mm,%頁眉
  headsep=15mm,%
  footskip=15mm,%
  marginparsep=10pt, %旁註與正文間距
  marginparwidth=6em,includemp=true% 旁註寬度計入width%旁註寬度
  ]{geometry}

%color
\RequirePackage[table,svgnames]{xcolor}

%================字體================%
%设置数学字体
\RequirePackage{amssymb,amsmath}
\RequirePackage{stmaryrd}
\everymath{\displaystyle}

\RequirePackage{fontspec}
%設置英文字體
\setmainfont[Mapping=tex-text]{DejaVu Serif}
\setsansfont[Mapping=tex-text]{DejaVu Sans}
\setmonofont[Mapping=tex-text]{DejaVu Sans Mono}


%中文環境
\RequirePackage[]{xeCJK}
\xeCJKsetup{PunctStyle=plain}
\xeCJKDeclareSubCJKBlock{LIUSHISIGUA}{ "4DC0 -> "4DFF}
\setCJKmainfont[FallBack=DejaVu Serif, ItalicFont=方正楷体简体,LIUSHISIGUA=DejaVu Sans]{Source Han Serif CN}
\setCJKsansfont[FallBack=DejaVu Sans]{Source Han Sans CN}
\setCJKmonofont[FallBack=DejaVu Sans Mono]{方正楷体简体}


%%===============中文化=========%
\renewcommand\contentsname{目~录}
\renewcommand\listfigurename{插图目录}
\renewcommand\listtablename{表格目录}
\renewcommand\bibname{参~考~文~献}
\renewcommand\indexname{索~引}
\renewcommand\figurename{图}
\renewcommand\tablename{表}
\renewcommand\partname{部分}
\renewcommand\appendixname{附录}
\renewcommand{\today}{\number\year{}年\number\month{}月\number\day{}日}


%=======页眉页脚格式=========%
\RequirePackage{fancyhdr}   %頁眉頁腳
\RequirePackage{zhnumber}  %计数器中文化
\pagestyle{fancy}
\renewcommand{\sectionmark}[1]
{\markright{第\zhnumber{\arabic{section}}节~~#1}{}}

\fancypagestyle{plain}{%
    \fancyhf{}
    \renewcommand{\headrulewidth}{0pt}
    \renewcommand{\footrulewidth}{0pt}
    \fancyhf[HR]{\ttfamily \footnotesize \rightmark }
    \fancyhf[FR]{\thepage}}
\pagestyle{plain}


%=========章節標題設計=========%
\RequirePackage{titlesec}
%修改part
\titleformat{\part}{\huge\sffamily}{}{0em}{}
%修改chapter
\titleformat{\chapter}{\LARGE\sffamily}{}{0em}{}
%修改section
\titleformat{\section}{\Large\sffamily}{}{0em}{}
%修改subsection
\titleformat{\subsection}{\large\sffamily}{}{0em}{}
%修改subsubsection
\titleformat{\subsubsection}{\normalsize\sffamily}{}{0em}{}


%================目录===============%
%toc label to contents space   dynamic adjust
\RequirePackage{tocloft}%
\renewcommand{\numberline}[1]{%
  \@cftbsnum #1\@cftasnum~\@cftasnumb%
}

%==============超鏈接===============%
\RequirePackage[colorlinks=true,linkcolor=blue,citecolor=blue]{hyperref} %設置書簽和目錄鏈接等
\newcommand{\hlabel}[1]{\phantomsection \label{#1}}%某一小段的引用


%=================文字強調=========%
\RequirePackage{ulem} %下劃線,加點
\normalem%normal em , not instead of the uline

%modified udot command from the dotuline
\def\udot{\bgroup
  \UL@setULdepth
  \markoverwith{\begingroup
     \advance\ULdepth0.1ex
     \lower\ULdepth\hbox{\kern.25em . \kern.045em}%
     \endgroup}%
  \ULon}

\let\oldemph\emph % Save emph in oldemph
\renewcommand{\emph}[1]{\textcolor{red}{#1}}  

%==================插入圖片=======%
\RequirePackage{wrapfig}
\RequirePackage{graphicx}
\graphicspath{{figures/}}
%change the caption style a little like 1-1
\renewcommand{\thefigure}{\arabic{chapter}-\arabic{figure}}


%==============插入表格========%
\RequirePackage{booktabs}
\renewcommand{\thetable}{\arabic{chapter}-\arabic{table}}
\RequirePackage{caption}
%\renewcommand{\arraystretch}{1.3}
%如果用setspace宏包而不是linespread调整行间距,那么才需要额外的表格行距拉大。

%插入代码
\RequirePackage{fancyvrb} 
\fvset{frame=lines,tabsize=4 ,baselinestretch=1.8, fontsize=\footnotesize}


%==========其他宏包===========%
\RequirePackage{tikz} 
\usetikzlibrary{calc}

%========脚注=========%
\newcommand*\circled[1]{%
  \tikz[baseline=(char.base)]\node[shape=circle,draw,inner sep=0.4pt,minimum size=4pt] (char) {#1};}
\newcommand*\circledarabic[1]{\circled{\arabic{#1}}}

\RequirePackage{perpage} %the perpage package
\MakePerPage{footnote} %the perpage package command

\renewcommand*{\thefootnote}{\protect\circledarabic{footnote}}


\renewcommand\@makefntext[1]
{\vspace{5pt}
\noindent
\makebox[20pt][c]{\@makefnmark}
\fontsize{10pt}{12pt}\selectfont #1}

\setlength{\skip\footins}{20pt plus 10pt}
%main body 与脚注之间的距离


%framed环境
\RequirePackage{framed}

\newenvironment{shici}{
\begin{verse}
\centering\large\hspace{12pt}}
{\end{verse}}

\RequirePackage{indentfirst} 

\makeatother



\title{周易}
\author{万泽}
\hypersetup{
  pdfkeywords={},
  pdfsubject={制作者邮箱:a358003542@outlook.com},
  pdfcreator={万泽}}
  
\newcommand{\bookcover}[1]{\tikz[remember picture,overlay]{\node[inner sep=0] at (current page.center)
{\includegraphics[width=\paperwidth,height=\paperheight]{#1}}}} 
 

  
\begin{document}
\frontmatter 

\thispagestyle{empty}

\bookcover{book_cover.png}

\cleardoublepage

\flypage{感谢上天}


\addchtoc{引言}
\chapter*{引言}
天人感应是周易预测的基石,只有良好的天人感应,上天以卦数示人,人才可能继续做出正确的预测。没有天人感应学说,后面的一切都是站不住脚的。

我们学到的各个周易预测学派都是前人内心修身,天人感应,外在体察的经验总结。各个学派彼此是不相干的,甚至是彼此跨越几百年完全不搭架的,将这些不同的学派和不同领域的应用企图统一起来的想法是极具野心的,初学者应该慎重。

我发觉试图将五行和八卦混合起来的做法可能是错误的,五行学说更适合的是在中医领域,而观星领域发展起来的紫微斗数,或者和建筑相关的奇门之术。它们在具体领域对于从卦象到更具体的物象都有各种各样的约定,所有的这些约定都是源于前人先贤们内在修身,天人感应,体察外在万物,然后解读卦象的结果。所有这些学派的发展都是遵循如下过程:\textbf{内在修身静心,诚心发问,起卦得卦象或者观察万物得卦象,用心体察解读卦象}。

这个过程是最核心本质的部分,而至于具体得出来的结论和前人总结的各个约定反倒只是一些术了。这些术你甚至可以看到有点类似于科学的统计,但离周易的本质已经很远了。

所以就以最简单的周易六十四卦预测来说,个人的修身静心和用心体察解读卦象,这个过程是缺一不可的。没有这个过程,而是试图从周易预测的各个术中寻找科学或者某种规律的东西,那就实在是南辕北辙了。

中国人的这种直觉思维模式是很有别于西方的那种逻辑思维模式的,目前来看西方的逻辑思维对于目前我们的科学贡献更大,但是也许那也只是对于宇宙存在的某一方面而言确实西方的逻辑体系说准了宇宙存在的某些东西,能够更好地对外部存在进行建模和判断。但谁也无法保证中国人的这种整体的直觉思维模式不会在下个阶段,对于宇宙存在做出更好的解释。

\section{阴阳学说}
\begin{quote}
有太极,是生两仪, 两仪生四象, 四象生八卦,八卦定吉凶,吉凶生大业。
\end{quote}


阴阳学说是如此的浅显,以至于我觉得无法再补充说些什么了,就好像就是那样,它就在哪里,存在于我们的基因中,存在于我们的记忆中,觉得不需要再多说点什么了。但实际上阴阳学说是很深的哲理,西方哲学家们从莱布尼兹到黑格尔到马克思,慢慢把辩证法发展起来付出了极大的努力,而这个努力的基石,莱布尼兹自己也承认过,是受到了中国阴阳学说的启发。

任何事物内部都存在着矛盾对立面,这个矛盾对立面就是阴和阳。然后阴阳衍生出四象。四象很是巧妙地把事物内部矛盾演化的过程进一步细化,于是可比春夏秋冬,可比市场经济繁荣危机的GDP波动图等。实际上任何事物发展演化都可以进一步细化分析,从而得到四象。

\section{八卦和物象}
\begin{quote}
乾(qián)、坤(kūn)、坎(kǎn)、离(lí)、兑(duì)、艮(gèn)、巽(xùn)、震(zhèn)
\end{quote}


假如我们问上天的问题是“是或者不是”这样简单的问题,那么是就是阳,而不是就是阴,一个爻的卦象就够用了。但某些问题必然要求具有更大的信息量,于是人们提出八卦,并将八卦和这个世界不同的万事万物的物象对应起来。读者可以随便搜一搜就可以搜到八卦和物象的映射关系,实际上我们如果问上天的问题最后答案是指明某个物的话,那么分析卦象里面的八卦对应的物象,就能够大致猜测出来上天启示你的那个物品是什么。

我对八卦和这些物象的映射关系持保留意见,可做参考吧。类似的还有后天八卦映射方位的理论,可以在你想特别预测某个方位的时候权做参考吧。

\section{如何摇卦}
蓍草占卦就不说了,要铜钱六十四卦方法因为以前有兴趣学了一下,下面说明一下。

我看到有人批判说电脑计算随机数摇卦是不准的,一定要用铜钱。我觉得说这话的人一定要用蓍草占卜,否则对不起他的纯真坚持的心。重要的是天人感应,是铜钱也好,电脑生成的随机数也好,都不过是过程是手段是术罢了,只要保证这个过程不是掩耳盗铃,就是可行的。

具体是一个爻要抛三次硬币或者生成三个0或1的随机数:

\begin{itemize}
\item 如果加和为3,也就是都是阳,则为老阳爻,本卦得阳爻,变卦得阴爻
\item 如果加和为2,也就是两阳一阴,则为少阳爻,本卦得阳爻,变卦得阳爻
\item 如果加和为1,也就是一阳两阴,则为少阴爻,本卦得阴爻,变卦得阴爻
\item 如果加和为0,也就是都是阴,则为老阴爻,本卦为阴爻,变卦为阳爻
\end{itemize}


如此重复得到周易六十四卦预测的本卦和变卦。

\section{易者谨记}
天道无亲,常与善人,常罚恶人。

周易预测可知天命,天道有常然天命无常,人子善恶一念皆可改命。此易者不可不牢记在心!

王阳明:“人但得好善如好好色,恶恶如恶恶臭,便是圣人。”

善善恶恶,还有更多的道理吗,我看是没有了啊。

\section{六十四卦解卦入门}

在得到本卦和变卦之后,我们就可以查阅周易一书里面的内容了。从我个人的经验来看,变爻的一般为一个到两个,这些情况基本上是没有争议的:

\begin{itemize}
\item 无变爻 以本卦卦辞断之
\item 一个变爻 以该爻的爻辞断之
\item 两个变爻 以两个爻的爻辞共同断之,这里朱熹说 \verb+以上者断之+ ,我觉得说得不够确切,因为周易的各个爻辞明显可以看出是在讲述一个事件的发展经过的过程,所以更确切来说是这两个爻是你预测的该事件在那两个发展阶段存在变数的情况,\verb+以上为主+ 含义是以事情最终那个变化情况为主,但我觉得最好说成是由这两个爻的爻辞共同断之。
\item 更多的变爻的时候我会觉得这次摇卦不是很准了,那个时候我还没有接触朱熹的变爻断法,我觉得可以作为参考\footnote{但也只是参考,除开预测者的非常规预测需求,以一般的事件预测来说,我觉得还是要以本卦为主,只是事情会在很短时间内发生很多变化,所以该卦的时效性可能会很短} 。

具体朱熹关于多个变爻的断法如下所示:
\begin{quotation}
三个变爻以本卦与变卦卦辞断;本卦为贞(体),变卦为悔(用)

四个变爻以变卦之两不变爻爻辞断,但以下者为主

五个变爻以变卦之不变爻爻辞断

六个变爻以变卦之卦辞断,乾坤两卦则以「用」辞断
\end{quotation}

\end{itemize}

\section{关于本书基本术语}
经是周易最开始的内容,

具体解卦最先的是卦象,根据你提出的问题分析卦象。卦象也就是那两个八卦叠起来的象。相传卦辞是文王写的,爻辞是周公写的。然后下面还有彖(tuàn)辞、象辞是孔子和他的门人写的。


\addchtoc{目录}
\setcounter{tocdepth}{2}    
\tableofcontents



\mainmatter
\part{周易}

\chapter{乾卦}
\section{原文}
乾  {\Large ䷀}


\subsection{经}
乾:元亨,利贞。

初九:潜龙,勿用。

九二:见龙在田,利见大人。

九三:君子终日乾乾,夕惕若厉,无咎。

九四:或跃在渊,无咎。

九五:飞龙在天,利见大人。

上九:亢龙有悔。

用九:见群龙无首,吉。


\subsection{彖}
大哉乾元,万物资始,乃统天。云行雨施,品物\footnote{参看汉典“品物”词条,品物即万物。}流形。大明终始,六位时成,时乘六龙以御天。乾道变化,各正性命,保合大和,乃利贞。首出庶物,万国咸宁。

\subsection{象}
天行健,君子以自强不息。

初九:潜龙勿用,阳在下也。

九二:见龙在田,德施普也。

九三:终日乾乾,反复道也。

九四:或跃在渊,进无咎也。

九五:飞龙在天,大人造也。

上九:亢龙有悔,盈不可久也。

用九:用九,天德不可为首也。


\section{初讲}
卦辞说:乾卦,大亨通,利于坚守正道。

彖说:伟大啊,乾元,万物因此开始,于是统领天下。云气流行,雨水布施,万物运动变化而各具成形。从上到下对应万物的结束到开始,其都是阳爻,六爻应时而形成,时乘六龙来驾御天道。天道的变化,万物各个正定其本性和运命,保全太和之气,是故“利贞”。始出万物,万国皆得安宁。

象辞说:天行健,君子以自强不息。

初九:潜龙勿用,阳在下也。此时你需要隐忍,还没有大展才干的时候。

九二:龙出现在了田野,适合去见到大人。龙在田不得位也。此利见大人是因为龙不得位,和九五的利见大人是有区别的,

九三:君子终日自强不息,到了晚上也不放松警惕如入危险之境,无咎。

九四:或腾跃而起,或退居深渊,无咎。

九五:飞龙在天,利见大人。九五的利见大人是因为你的飞龙在天是大人成就的【大人造也】。

上九:亢龙有悔。亢龙有悔,盈不可久也。

用九:用九是乾卦全部是阳爻,也即是乾卦本卦。一群龙却没有首领,吉祥。为什么吉祥呢?天生万物不居功不为首,是故吉祥。


\section{高岛断易占}
\subsection{乾卦占}
【占】 得此卦者,要临事刚健,自强不息,犹天行也。又要包括“元亨利贞”之四德。乾有施德而不计利之意。

\begin{itemize}
\item 女子:筮得此卦,以阴居阳,有刚强过中之嫌。宜慎重也。
\item 天候:二三四五之中,变则必晴也。
\item 买卖:不利买而利卖也。
\item 祸福:谓积善余庆,积不善余殃,恐有不在当代而在后裔也。
\item 常人:有高其身而不知鄙事之虞。
\item 贤者:有知天命而独行是道,恐群阴潜伏,有群小构谗之惧。
\end{itemize}


\subsection{初九占}
【占】 问战征:乾为武人,有战征之象。初爻阳气始动于黄泉,犹是潜伏,故曰“潜龙”。在军事,为威令初发,大军未集,宜按兵以待也。吉。

○ 问营商:龙而潜,曰“勿用”,虽是一种好贸易,只可株守,未可骤动也。

○ 问功名:龙本飞腾发达之物,初爻曰潜,是未得风云之会也。故曰位在下也。

○ 问婚姻:《乾》初变《姤》,《姤》曰“女壮,勿用娶女,”是宜戒之。

○ 问家宅:按震为龙,震在东方,是宅之东,必有渊水,闭塞不济,宜修凿之。

○ 问六甲:生男。

\subsection{九二占}
【占】 问战征:龙本灵物,初爻曰“潜”,是谓伏兵;二爻曰“现”,则发现而出也。“在田”则必列阵于田野空旷之地。《传》曰“德施普也”,是必战胜而行赏也。

○ 问营商:爻曰“现龙在田”,知其货物大般是米麦丝棉之类。现者,谓物价发动开涨;“利见大人”者,谓当有官场出而购买也。

○ 问功名:谓伏处田间者,当乘时而进用也,且得贵人之助,故曰“利见大人”。

○ 问婚姻:二五相应,五居尊位,婿家必贵。曰“现龙”,必是新进少年也。大吉。

○ 问六甲:生男,且主贵。


\subsection{九三占}
【占】 问战征:危事也。爻曰“终曰乾乾,夕惕若”,是能临事而惧者也,故虽危无咎。

○ 问功名:九三处下卦之极,其位犹卑,功名未显也,故称君子;在忧危之地,故曰乾乾惕若,斯可免咎。

○ 问营商:居不中之位,履得刚之险,度其贸易必是危地,须日夜防备,可脱险而获利也。

○ 问家宅:观爻象,必须谨慎持身,勤俭保家,斯无灾害。

○ 问婚嫁:三以六为应,三位卑,六位尊,尊则不免亢而得悔,是不宜攀结高亲也。

○ 问六甲:生男。产时恐稍有危惧,恐终无咎。


\subsection{九四占}
【占】 问战征:观爻象,行军前进,必有渊水阻隔,宜设船筏;或临渊有敌军埋伏,宜预设备,乃得无咎。

○ 问营商:爻曰“或跃在渊”,若在贩运海货,恐罹波涛之险,或者物价一时腾涨。爻曰“无咎”,可保无害。

○ 问功名:有一举成名之象,大吉。

○ 问家宅:渊者水也,跃者飞升也,必家道有一时振兴之象。

○ 问六甲:生男。

\subsection{九五占}
【占】 问战征:九五尊位,必是天子亲征,王师伐罪,故曰“大人造也”。

○ 问营商:九五辰在申,上值毕,附星咸池。咸池者苍龙之舍,咸池亦名五车,主稻黍豆麦,度其贸易,定在五谷之属。曰“飞龙”者,知物价之飞生也;曰“利见大人”,知其贩运或出自政府之命也。

○ 问功名:有云宵直达之兆。

○ 问疾病:有上应天召之象,不吉。

○ 问六甲:生男,主贵。

\subsection{上九占}
【占】 问战征:上九居《乾》之极,阳极于上,故“亢”;亢则因胜而骄,是以“有悔”也。故《传》曰“盈不可久”,知不能持久也。

○ 问营商:“亢”者,太过也,凡卖买之道,不可过于求盈也,过盈则必有亏,故曰“不可久”也。

○ 问功名:上九之位已极,宜反而自退,否则必致满而遭损。

○ 问家宅:是必宅基太高,太高则危,亦可惧也。

○ 问疾病:是龙阳上升之症。《传》曰“盈不可久”,知命在旦夕间矣。可危。

○ 问婚嫁:不利。

○ 问六甲:生男,恐不育。


\section{国易堂占}
一般来说,得到乾卦的人事业发展都会比较顺利,可以说是如曰中天,甚至能够达到名利双收的效果,但是全盛之时经常也是衰败之始,所以要小心谨慎,以免乐极生悲。

初九爻变:平。事业可成。时机还没有成熟,正面临着无限可能,如果想在事业上有所成就,就必须从现在开始努力。

九二爻变:平。事业可成。机会已经来临,你已经筹划好了要扎扎实实地干一番大事业,这时还需要找一个可靠的合作伙伴,抓紧这个好机会吧!九二爻提示你,成功的希望非常大。

九三爻变:吉。事业易成。你的努力没有白费,你的事业已经有了起色。九四爻变:吉。事业易成。现在的工作进展已经非常顺利,你需要做的只是把目前的状态维持下去。

九五爻变:吉。事业易成。恭喜你了!经过努力,你的工作事业已经达到了一个巅峰的状态,可以说是名利双收!

上九爻变:平。事业可成。注意不要自视过高,惟我独尊,要有“既达目标,见好就收”的智慧,否则就会盛极转衰。


\chapter{坤卦}
坤 {\Large ䷁}

\section{原文}
\subsection{经}
坤:元亨,利牝马之贞。君子有攸往,先迷后得主,利西南得朋,东北丧朋。安贞吉。

初六:履霜,坚冰至。

六二:直方大,不习无不利。

六三:含章可贞。 或从王事,无成有终。

六四:括囊;无咎,无誉。

六五:黄裳,元吉。

上六:龙战于野,其血玄黄。

用六:利永贞。

\subsection{彖}
至哉坤元,万物资生,乃顺承天。坤厚载物,德合无疆。含弘光大,品物咸亨。牝马地类,行地无疆,柔顺利贞。君子攸行,先迷失道,后顺得常。西南得朋,乃与类行;东北丧朋,乃终有庆。安贞之吉,应地无疆。

\subsection{象}
地势坤,君子以厚德载物。

初六:履霜坚冰,阴始凝也。驯致其道,至坚冰也。

六二:六二之动,直以方也。不习无不利,地道光也。

六三:含章可贞;以时发也。或从王事,知光大也。

六四:括囊无咎,慎不害也。

六五:黄裳元吉,文在中也。

上六:龙战于野,其道穷也。

用六:用六永贞,以大终也。

\section{初讲}
卦辞说:坤卦,大亨通,利牝马之贞。君子有所前往,先迷而后得主。西南得朋有利,东北丧朋。安贞吉\footnote{安贞不可浅解为安于现状,牝马之贞指贞正之道偏坤德,安贞指贞正之道偏安守。}。

彖说:伟大啊,坤元。万物因你而生,顺承天道。坤用厚德载养万物,德性与天相合而无边无疆。包容博厚而广大,万物皆得亨通。牝马属于地上的生物,奔行于地而没有疆界,牝马性柔顺而利于贞正之道。君子有所前往,先迷失其道,后顺从于常道\footnote{即天道}。西南得到朋友,乃与之同行;东北失去朋友,最终有喜庆之事。安贞何以吉祥,应和地道的无边无疆。

象辞说:坤象征大地,君子应该效仿大地,胸怀宽广,包容万物。


初六:脚踩于霜而知气候变冷,坚冰将至。脚踩于霜,说明阴气开始凝结,照这个情况发展下去,必然迎来坚冰。本爻在提醒占者要见微知著,防微杜渐。

六二:直方大,此大地的形状。六二的行动如同大地的形状一样正直方正,即使不用修习也是无不利的。这是因为他将地道发扬光大了。

六三:胸怀美德和才华可以坚守正道,或许能够跟从君王做事,没有功名成就事情却有好的结果。

六四:将口袋扎紧,无咎亦无誉。此爻有守口如瓶之象,如此谨慎才得无害。

六五:黄色的下衣,吉祥。为何吉祥?温文之德在心中。

上六:与龙相战于旷野,龙流下了玄黄色的血。阴气极盛,地道也发展到头了。此爻提醒占者阴气过头复转为阳的道理。

用六:用六是坤卦全部是阴爻,也即是坤卦本卦。利于永远坚守正道,用六的永远坚守贞正之道,可用来得到大的善终。


\section{高岛断易占}
\subsection{坤卦占}
【占】 问战征:坤为地,为众,“势”者有力之称。在行军,既得其地,复得其势,又得其众,宜乎攻无不克矣。

○ 问功名:上者能法坤德之厚,积厚流光,自得声名显远。

○ 问营商:《坤》为富,为财,为积,为聚,皆营商吉兆也。曰“厚德载物”,德者得也,可必得满载而归也。

○ 问家宅:知此宅胜占地势,大吉。

○ 问婚嫁:《坤》顺也,柔顺而已,地道也,即妇道也。大吉。

○ 问六甲:生女。

\subsection{初六占}
【占】 问营商:初六阴气犹微,曰“履霜,坚冰至”,是由微而推至于盛也,犹商业由小至大,积渐而至于富。

○ 问功名:初爻是少年新进之时,由卑而尊,犹履霜以到坚冰,随时而来,未可躁进也。

○ 问战征:初爻阴之始,“履霜”之象,至上爻“龙战”,阴之极也,“坚冰”之象。曰“其血玄黄”,是两败也。所当先慎其始。

○ 问家宅:《坤》纯阴之卦,初爻阴气尚微,故曰“履霜”,“至坚冰”,则阴气盛矣。阴盛则衰,不吉之兆。

○ 问婚嫁:《坤》卦纯阴,曰霜,曰冰,皆阴象。纯阴无阳,不利。

○ 问六甲:生女。

○ 问疾病:恐是阴邪之症,初起可治矣,久则难医。

\subsection{六二占}
【占】 问营商:六二《坤》之本位,“直方”者地之性,“大”者地之用,知其营业必是地产,如谷米、木材、丝棉之类是也。“不习无不利”,习与袭通,谓不烦重筮而知其获利也。

○ 问功名:二爻居中得位,动而获利,言不待修营而功自成,其成名也必矣。

○ 问战征:战之一道,以得地势为要,动以其方,势大力强,可一战而定也。

○ 问家宅:六二中正,居宅得宜,故曰“地道光也”。

○ 问嫁娶:“直方大”,地道也,妻道通于地道,故婚娶亦利。

○ 问疾病:爻曰“直方大”,知其素体强壮,不药有喜[17]。

○ 问六甲:生女。


\subsection{六三占}
【占】 问战征:爻曰“含章可贞”,言平时含蓄才智,敛藏不露,一旦从事,自能制胜,即不成功,亦无大败。故曰“无成有终”。

○ 问营商:《坤》地也,百货皆生于地,商能蓄积百货,故曰“含章”。凡从事营商者,贸迁百货,以时发售,故曰“时发”。《坤》内卦至三而极,正盛满之地,故曰“光大”。是以一时虽或未成,知必有终也。吉。

○ 问功名:凡求名者,最宜待时,时未当发,“含章可贞”;时而当发,出从王事。知此道者,必能保功名以终也。吉。

○ 问疾病:玩“无成有终”句义,知不可药救矣。凶。

○ 问六甲:生女。


\subsection{六四占}
【占】 问营商:四巽爻,巽为商,为利,巽“近利市三倍”之谓也。兹爻曰“括囊”,是明亦以闭囊之象,知必昔日得利,财已入囊,不使复出也。故曰“括囊,无咎,无誉”。

○ 问战征:六四重阴,当闭塞之时,虽有智,囊其才,无所施其计谋也,是宜闭关不战,如囊之括其口也,斯无咎矣。

○ 问功名:四重卦,动当否位,《文言》曰“天地闭”,“括囊”者,闭口也。天地且闭,何有于功名?若妄意干进求名,适足致祸,有誉反有咎矣。宜慎。

○ 问家宅:六四以阴居阴,履非中位,是宅必在山谷幽僻之处,宜隐遁者居之。

○ 问六甲:生女,或得孪生二女。

\subsection{六五占}
【占】 问战征:《坤》臣道,五居尊位,为人臣之极贵者,如舜之摄位诛四凶,周之摄政诛二叔。爻曰“黄裳,元吉”,是以文德而发为武功者也,故《传曰》“文在中也”。

○ 问功名:六五辰在卯,得《震》气,《震》有功名奋兴之象。五又《离》爻,《离》为黄位,近午,上值七星,七星主衣裳文绣,故曰“黄裳”。《离》又为明,有文明发达之象,故曰“文在中也”。

○ 问营商:《坤》五变《比》,比吉也,辅也,商业必得比辅而成。《比》卦下《坤》上《坎》,《坤》为裳,故曰“黄裳”;《比》为美,故曰“文在中”,知其经商必是锦绣章服之品。曰“元吉”,必获利也。

○ 问疾病:《坤》为大腹,又黄为中色,裳下饰,可知其病在中下两焦。

○ 问六甲:生女。

\subsection{上六占}
【占】 问战征:象已明示是两败也。

○ 问功名:上处外卦之极,是穷老人闱,抑塞已久,一战复北,可哀也。

○ 问营商:上六《坤》卦之终,其道已穷,是资财既竭,血本又耗,商道穷矣。

○ 问疾病:必是阴亏之症,阴极抗阳,肝血暴动,命已穷矣。

○ 问六甲:阴尽变阳,可望男孩。


\section{国易堂占}
坤卦启示人们,在工作中不要太工于心计,首先要安于现状,甘心当配角。不强出头,才不致陷入痛苦。但这并不意味着什么都不做,此时正是增强自身实力,充实自己的好时机,待外在局势对自己有利时,就可一展身手了。

初六爻变:平。事业难成。像冬天的冰天雪地一样,你的处境不太好,事业发展不太顺利。有时候等待也是一个不错的选择,你需要耐心等待春天的到来。

六二爻变:吉。事业易成。事业发展比较顺利,但需要你有耐心、要能甘于寂寞。

六三爻变:平。事业难成。你的努力似乎没有什么成效,但这只是表面现象,事情的转机值得期待。

六四爻变:平。事业可成。没有收获,但也没有损失,在这种两难状态下,需要的仍然是等待,时间会把一切安排好。

六五爻变:吉。事业易成。盼望已久的机遇终于到来,这是坤卦中最吉利的一爻。

上六爻变:平。事业难成。两龙相战,意味着你的事业发展又遇到波折了,问题的根源,在于你过于急躁,不能忍受寂寞。


\chapter{水雷屯(zhūn)卦}
屯 {\Large ䷂}

\section{原文}
\subsection{经}
屯:元亨,利贞,勿用有攸往,利建侯。

初九:磐桓;利居贞,利建侯。

六二:屯如邅(zhān)如,乘马班如。匪寇婚媾,女子贞不字,十年乃字。

六三:既鹿无虞,惟入于林中,君子几,不如舍,往吝。

六四:乘马班如,求婚媾,往吉,无不利。

九五:屯其膏,小贞吉,大贞凶。

上六:乘马班如,泣血涟如。

\subsection{彖}
屯,刚柔始交而难生,动乎险中,大亨贞。雷雨之动满盈,天造草昧,宜建侯而不宁。

\subsection{象}
云,雷,屯;君子以经纶。

初九:虽磐桓,志行正也。以贵下贱,大得民也。

六二:六二之难,乘刚也。十年乃字,反常也。

六三:既鹿无虞\footnote{虞:虞人,掌管山泽之官。},以从禽也。君子舍之,往吝穷也。

六四:求而往,明也。

九五:屯其膏,施未光也。

上六:泣血涟如,何可长也。

\section{初讲}
卦辞说:屯卦,大亨通,利于坚守正道。不要有所前往,适宜建国立候。【屯卦还是很吉祥的,也利于君子去坚守正道,只是不要轻举妄动,适合做一些大事情的筹备工作。】

彖说:屯卦,阳刚之气和阴柔之气刚开始交汇,困难也随之而生,危险也在其中涌动,但终是大亨通,利贞的。雷雨之动满布大地,而上天的造化开物还处于蒙昧状态,适宜于建国立候,但不安宁。

象辞说:屯卦上面是云下面是雷,君子观此象而经纶天下。

【屯卦指事情才刚开始,寓意天地也是云雷互动而困难重重,此时不要轻举妄动,但也不要倦怠安宁,而应该效法此时云雷互动中上天经纬造化万物之象,立大志,定大计,定好大的方向和规划。经天纬地,如此安能不是大亨通和利贞的呢。】

初九:徘徊不前,利于安居的贞德,利于规划建侯般的伟业。虽然徘徊不前,但是志向的操行都是纯正的。珍贵基层的平民百姓,会大大得到民心。

六二:困难啊难以前行啊,骑着马徘徊不前,不是匪寇是来商量嫁娶事宜的。女子坚守贞道不嫁,十年后才出嫁。六二之难,柔乘刚也。十年后才出嫁,回归正常罢了。

六三:追逐野鹿却没有山林之官作向导,只是跟着进入山林之中。君子明察之,不如舍弃吧,前往会有悔恨。即鹿无虞,只是跟着野兽跑。君子请舍弃吧,前往会有悔恨并陷入穷困。

六四:骑着马徘徊不前,去求婚,前往吉祥,无不利。为求婚而前往,是明智的选择。

九五:自我囤积财富,小贞吉,大贞凶。为什么会大贞凶呢?并没有将布施之德发扬光大。

上六:骑着马徘徊不前,悲泣不已,流泪如流血一样,涟涟不断。泣血涟如,这种状况怎么能长久呢。


\section{高岛断易占}
\subsection{屯卦占}
【占】 问功名:内《震》外《坎》为《屯》,《震》为雷,《坎》为云,故曰“云雷”;《震》为出,《坎》为入,欲出而复入,故曰《屯》。又《震》为人,为上,《坎》为经,为法,故曰“君子以经纶”。是君子施经纶之才,而运当其屯也,宜待时而动。

○ 问战征:勤兵而守曰屯。“云雷”者,蓄其势也;“经纶”者,怀其才也。然当其屯,宜守不宜进。

○ 问营商:《彖》曰“刚柔始交而难生”,是必初次营商也。凡事始创者,多苦其难。经纶,治丝之事,知其业必在丝棉之类。

○ 问家宅:《震》东方,《坎》北方,《震》动也,《坎》陷也,恐是宅东北方有动作,宜经理修治之。

○ 问婚姻:雷阳气,云阴气,“刚柔始交而难生”,是初婚时,必不和洽,宜正人劝解之。

○ 问六甲:生男,恐始产不免有险难。

\subsection{初九占}
【占】 问战征:磐桓,不进之貌,曰“利居贞,利建侯”。尽尝屯难之时,内则居正以守,外则求贤以辅,斯民心归向,众志成城,而终无不利矣。

○ 问营商:初九爻,辰在子,北方,上值虚宿,曰元枵,枵之为言耗,虚亦耗意,不利行商。能以守贞任人,尚有利也。

○ 问功名:初爻是必初次求名也,“磐桓者”,是欲进不进也。要当志行正直,谦退自下,终有得也。

○ 问家宅:磐字从石,所谓安如磐石,知其宅基巩固也;曰“利居贞”,知其居之安;曰“利建侯”,知必是贵宅也。

○ 问婚嫁:曰“以贵下贱”,知为富贵下嫁之象,吉。

○ 问六甲:初爻生男。

\subsection{六二占}
【占】 问婚嫁:爻曰“匪寇婚媾”,是明言佳偶,非怨偶也。但曰“女子贞不字,十年乃字”,知于归尚有待也。

○ 问战征:六二以柔居柔,有濡滞之象,故曰“屯如”。《春秋传》:“有班马之声,齐《师》乃《遁》。”古者还师称班师,故曰“班如”,知行师未可遽进也,必养精蓄锐,十年乃可获胜。

○ 问营商:媾与购音同,义亦相通。以货物求购,有迟回不决之意,故曰“屯如邅如”。又曰“十年乃字”,十者据成数而言,货物未可久积,或者十日十月乎?

○ 问功名:士之求名,犹女子之求嫁也,曰“屯如”、“邅如”、“班如”,皆言一时未成也。“十年乃字”,此其时也。

○ 问六甲:生子。

\subsection{六三占}
【占】 问战征:爻曰“即鹿无虞,惟入于林中”,犹言行军而无向导,冒进险地也。当知几而退,否则必凶。

○ 问营商:玩爻辞,知其不谙商业,不熟地理,前往求货,不特无货,反有损失,舍而去之,尚无大害也。

○ 问婚嫁:是钻穴隙以求婚也,其道穷矣。

○ 问功名:梯荣乞宠,士道穷矣。

○ 问六甲:六三阴居阳位,生男。

\subsection{六四占}
【占】 问战征:“乘马班如”者,不明其进攻之路故也,明而前往,则所向无敌,故曰“往吉,无不利”。

○ 问功名:士者藏器待时,不宜躁进,迨于旌下逮,出而加民,“无不利”也。

○ 问婚嫁:《诗·关睢》云,“窈窕淑女,君子好逑”,逑,求也,必待君子来求,始为往嫁,故吉。

○ 问六甲:生女。

\subsection{九五占}
【占】 问功名:士之所赖以显扬者,全望上之施其恩膏也,若上“屯其膏”,而士复何望焉!

○ 问战征:上有厚赏,则下愿效死,若恩泽不下,势必离心离德,大事去矣。凶。

○ 问营商:膏者谓商业之资财也,“屯其膏”,谓蓄聚而不流通也,小买卖犹可固守,大经营未免困穷矣。凶。

○ 问疾病:膏者在人为脂血,屯而不通,是闭郁之症,初病治之尚易,久病危矣。

○ 问六甲:九五居尊,生男,且主贵。

\subsection{上六占}
【占】 问战征:上居屯之极,进退维谷,穷戚已甚,而至泣血,是军败国亡之日也。凶。

○ 问营商:“乘马班如”一句,上已三复言之,是商业之疑惑不决,已至再至三矣。极之泣血,知耗失已多,故曰“何可长也”。

○ 问功名:上居《坎》终,更无前进,得保其身幸矣。

○ 问疾病:知必是呕血之症。凶。

○ 问六甲:生女,又恐不能长大。


\section{国易堂占}
国易堂说:水雷屯卦是四大难卦之一,因此,得到屯卦就意味着事业发展很不顺利,困难重重。自身能力的不足加上过于急躁,使得局面很棘手。

初九爻变:平。事业可成。但此时不宜主动出击,只需按兵不动即可。

六二爻变:平。事业可成。“女子贞不字,十年乃字。”是说现在事业还处在困境中,需要耐心的等待,好运会来的。这个时间不一定是十年之后,要看形势的变化。

三爻变:凶。事业不成。现在的事业发展十分不顺利,如果硬要出手,反而会有损失,所以不如耐心等待,不要轻举妄动。

四爻变:吉。事业易成。现在是事业发展的最佳时机,所以积极行动吧。

九五爻变:平。事业可成。虽然不会有太大的麻烦,但是如果太心急了反而会失败,处之淡然比较好。也不要期望工作中会一直一帆风顺,那样的状态是不存在的。

上六爻变:凶。事业不成。十分不顺利,如果硬要出手,只会让自己陷于困境。


\chapter{山水蒙卦}
蒙 {\Large ䷃}

\section{原文}
\subsection{经}
蒙:亨。 匪我求童蒙\footnote{童蒙:蒙昧的孩童。},童蒙求我。初筮告,再三渎,渎则不告。利贞。

初六:发蒙,利用刑人,用说\footnote{说,脱也。}桎梏,以往吝\footnote{参看重庆文理学院学报2006年1月出版的第五卷第1期田小中,“以往吝”衍文呢考,其认为这个以字只是一个衍文即多余的字,也就是这里应该解读为往吝。个人认为有一定的可信度。}。

九二:包蒙吉;纳妇吉;子克家。

六三:勿用娶女;见金夫,不有躬,无攸利。

六四:困蒙,吝。

六五:童蒙,吉。

上九:击蒙;不利为寇,利御寇。

\subsection{彖}
蒙,山下有险,险而止,蒙。蒙亨,以亨行时中也。匪我求童蒙,童蒙求我,志应也。初筮告,以刚中也。再三渎,渎则不告,渎蒙也。蒙以养正,圣功也。


\subsection{象}
山下出泉,蒙;君子以果行育德。

初六:利用刑人,以正法也。

九二:子克家,刚柔接也。

六三:勿用娶女,行不顺也。

六四:困蒙之吝,独远实也。

六五:童蒙之吉,顺以巽也。

上九:利用御寇,上下顺也。

\section{初讲}
卦辞说:蒙,亨通。不是我有求于童蒙,而是童蒙有求于我。【上天教育我们也好比教育童蒙一样】初次占筮就告诉,再三反复占筮就是亵渎了,亵渎则上天不会予以告知了。利贞。

彖说:蒙卦,上为艮为山为止,下为坎为水为险,故曰山下有险,险而止,这就是蒙卦了。蒙卦是亨通的,是因为他以亨道行动,得“时中”也。匪我求童蒙,童蒙求我,是因为童蒙有志于此。初筮告,是因为以刚正居于其中。再三渎,渎则不告,这个亵渎正是蒙昧的表现。将蒙昧培养入正道,这是圣人的功绩啊。

象辞说:山下冒出泉水,这就是蒙卦的卦象了。君子见此而知要以果敢的行动来培育良好的品德。

初六:启发蒙昧,利于用刑罚来约束人,让其将来免脱桎梏之苦。还在发蒙阶段贸然前往会有悔恨。利用刑人,是用来确立正确的法度。

九二:能够包容蒙昧,吉祥。娶媳妇吉祥,子女们能够持家了。子女们能够持家了,是因为刚爻与柔爻相接的缘故。

六三:不要娶这样的女人啊,她见到有钱的男人就委身于他,没什么好处。勿用娶女,是因为六三以柔乘九二之刚,没有顺从的美德\footnote{参考了\href{http://www.guoxuez.com/64gua/menggua/23431.html}{这个网页} 。}。

六四:困在蒙昧之中,悔恨啊。六四被困在蒙昧之中,是因为它这个柔爻独自远离于充实的阳爻。

六五:蒙昧的孩童,吉祥。童蒙之吉,是因为他能够对上面的阳爻也就是老师采用谦逊的态度来顺从他。

上九:上九击打蒙昧。不要把蒙昧当作匪寇一般击打,最好如艮山一般被动防御。利用御寇,是因为上下一心相互顺从的缘故。


\section{高岛断易占}
\subsection{蒙卦占}
【占】 问战征:《象》曰“山下出泉”,是潜伏之水也,有伏兵之象。君子谓军中之将帅也。“果行育德”,果者果敢也,育者蓄养也,谓当蓄其锐势,而果决以进也。

○ 问营商:玩《象》辞,想是开凿矿山生意。当果决从事,吉。

○ 问功名:是士者素抱德行,伏处深山之象。曰“山下出泉”,终将出而用世也。

○ 问家宅:知是宅坐向坎艮。曰“山下”,必近山也;曰“出泉”,必有泉流出其下也。君子居之,其宅必吉。

○ 问婚嫁:坎辰在子,上值女,《圣冷符》曰,“须女者主嫁娶”。《艮》下《兑》上为《咸》,二气相感,故曰“取女吉”。“山下出泉,蒙”,是婚姻之始也。

○ 问疾病:艮止坎险,病势必热邪渐陷于内,待初爻发蒙,邪气外发,可保无虞。

○ 问六甲:生男。

\subsection{初六占}
【占】 问战征:爻曰“发蒙”,是为伐暴讨罪之师,如大禹之征有苗,格则罢师而还,故曰“以正法也”。

○ 问营商:初居内卦之始,是必初次谋办也。《坎》为难,爻曰“发蒙”,曰“用刑”,知营商必有阻碍,殆将兴讼,得直理直即止,若欲穷究,恐有害也,故曰“往吝”。

○ 问功名:欲往求荣,恐反受辱,宜自休止。

○ 问嫁娶:初居始位,爻曰“发蒙”,必在少年订婚。既多事变,罢婚可也。

○ 问六甲:初爻阴居阴位,生女,又恐生产有难。

\subsection{九二占}
【占】 问战征:二爻以阳居阴,爻曰“包蒙”,有包括群阴之象。《象》曰“刚柔接也”,刚柔者两军也,“接”,接战也,“克家”,犹言克敌也。占例妇为财,子为福,既克敌军,又纳其财,并受其福,大吉。

○ 问营商:二上以两阳包三阴,一阳在内,一阳在外,有包罗财物,出贩外地之象,故曰“包蒙吉”。“纳妇”者,是必旅居纳妇也,有妇复有子。‘克家”者,必其子能继父业也。

○ 问功名:想不在其身,而在其子也,故曰“子克家”。

○ 问家宅:曰“包蒙”,以《艮》包《坎》,是必山环水抱之地。曰“纳妇”,曰“克家”,是宅必有佳妇佳儿,克振家业,吉。

○ 问婚姻:玩爻辞,有二吉,明言有妇有子,吉莫大焉。

○ 问六甲:生男,主富贵。

\subsection{六三占}
【占】 问战征:二爻以阳居阴,爻曰“包蒙”,有包括群阴之象。《象》曰“刚柔接也”,刚柔者两军也,“接”,接战也,“克家”,犹言克敌也。占例妇为财,子为福,既克敌军,又纳其财,并受其福,大吉。

○ 问营商:二上以两阳包三阴,一阳在内,一阳在外,有包罗财物,出贩外地之象,故曰“包蒙吉”。“纳妇”者,是必旅居纳妇也,有妇复有子。‘克家”者,必其子能继父业也。

○ 问功名:想不在其身,而在其子也,故曰“子克家”。

○ 问家宅:曰“包蒙”,以《艮》包《坎》,是必山环水抱之地。曰“纳妇”,曰“克家”,是宅必有佳妇佳儿,克振家业,吉。

○ 问婚姻:玩爻辞,有二吉,明言有妇有子,吉莫大焉。

○ 问六甲:生男,主富贵。

\subsection{六四占}
【占】 问战征:行军宜深入显出,曰“困蒙”,是入阴险之地,而不能出也,故困。足以济困者,在初爻之阳,六四距初间隔二爻,阳为实,故“远实”。是知救兵在远,不能及也。凶。

○ 问营商:经商之道,宜亨不宜困,宜通不宜吝。“实”资本也,“远实”则伤其资矣。困蒙之吝,其道穷矣。

○ 问时运:“蒙”,暗昧也,“困”,厄穷也,蒙而困,其终困矣。

○ 问家宅:据爻辞观之,家业困苦,宅地亦幽僻,《象》曰“独远实也”,是必孤村而乏邻居也。

○ 问六甲:生女。是女必少兄弟,故曰独。

\subsection{六五占}
【占】 问战征:五互《坤》,辰在未,值井,弧矢九星在井东南,主伐叛。又东为子孙星,曰“童蒙”,是帅子弟以从军也,故吉。

○ 问营商:五为卦主,爻曰“童蒙”,是必店主尚在童年。五应二,正义云“委物以能”,谓委付事物于有能力之人,是委二也。盖五爻店主,自知年少,顺从二爻,以为经纪,故曰“童蒙吉”。

○ 问功名:年在“童蒙”,功未成,名未就,惟能顺听二爻师教,则成就未可量也,故曰“吉”。

○ 问婚姻:《蒙》上体艮,艮为少男,是以幼年定姻也,故曰“童蒙吉”。

○ 问六甲:生男。

\subsection{上九占}
【占】 问战征:上辰在戌,上值奎,奎主库兵,禁不违时,故曰“利御寇”。

○ 问营商:商业一道,全在利用,又贵顺取。逆取为寇,顺取则为御寇。“上下”者,卖买两家,卖买和洽,则上下顺矣。吉。

○ 问婚姻:“击蒙”,马郑作“系蒙”,恰合月下老人红丝系足之意。《屯》卦两言“匪寇婚媾”,是佳偶为婚,怨偶为仇之谓也。利用御寇,必为佳偶。妇道贵顺,《象》曰“上下顺也”,是必家室和平也。吉。

○ 问六甲:生男。此男童年,必宜严教。



\section{国易堂占}
处蒙卦之时,能力不足以应付危险,面对危险应当停下来静观其变,求教于有智慧、有经验的长者则是最好的策略。

初六爻变:平。事业可成。就像启蒙小孩子学习一样,事业发展一开始非常不顺利,但以后的发展还算不错。

九二爻变:吉。事业易成。爻辞已经明确提示了:“纳妇,吉。”就像小孩子学习取得成绩一样,时机到来时,你的事业发展也会比较顺利。

六三爻变:凶。事业不成。因为目前的形势不明朗,再加上自己的判断失误,造成了事业发展不顺利。

六四爻变:平。事业难成。不太顺利,因为形势依旧不明朗,此时不应贸然有所行动。

六五爻变:平。事业可成。像小孩子一样蒙昧不清,但这样自有好处,因为在工作上不会作出太大的变动,所以也是一种不错的状态。

上九爻变:凶。事业不成。很不顺利,因为你对自己工作的下一步还没有搞清楚,处在一种蒙昧状态。最好等想清楚了再做决断。


\chapter{水天需卦}
需 {\Large ䷄}

\section{原文}

\subsection{经}
需:有孚,光亨,贞吉。 利涉大川。

初九:需于郊。 利用恒,无咎。

九二:需于沙。 小有言,终吉。

九三:需于泥,致寇至。

六四:需于血,出自穴。

九五:需于酒食,贞吉。

上六:入于穴,有不速之客三人来,敬之终吉。

\subsection{彖}
需,须也;险在前也。刚健而不陷,其义不困穷矣。需有孚,光亨,贞吉。位乎天位,以正中也。 利涉大川,往有功也。

\subsection{象}
云上于天,需;君子以饮食宴乐。

初九:需于郊,不犯难行也。 利用恒,无咎;未失常也。

九二:需于沙,衍在中也。 虽小有言,以吉终也。

九三:需于泥,灾在外也。 自我致寇,敬慎不败也。

六四:需于血,顺以听也。

九五:酒食贞吉,以中正也。

上六:不速之客来,敬之终吉。 虽不当位,未大失也。

\section{讲解}
卦辞说:

彖说:

象辞说:

\chapter{天水讼卦}
讼 {\Large ䷅}

\section{原文}
\subsection{经}
讼:有孚,窒惕,中吉。终凶。利见大人,不利涉大川。

初六:不永所事,小有言,终吉。

九二:不克讼,归而逋其邑,人三百户,无眚。

六三:食旧德,贞厉,终吉,或从王事,无成。

九四:不克讼,复自命,渝,安贞,吉。

九五:讼元吉。

上九:或锡之鞶带,终朝三褫之。

\subsection{彖}
讼,上刚下险,险而健,讼。讼有孚,窒惕中吉,刚来而得中也。终凶;讼不可成也。 利见大人;尚中正也。不利涉大川;入于渊也。

\subsection{象}
天与水违行,讼;君子以作事谋始。

初六:不永所事,讼不可长也。虽有小言,其辩明也。

九二:不克讼,归而逋归逋窜也。自下讼上,患至掇也。

六三:食旧德,从上吉也。

九四:复即命渝,安贞;不失也。

九五:讼元吉,以中正也。

上九:以讼受服,亦不足敬也。


\section{讲解}
卦辞说:

彖说:

象辞说:

\chapter{地水师卦}
师 {\Large ䷆}

\section{原文}
\subsection{经}
师:贞,大人\footnote{有文作丈人,高岛断易认为此处应该按照子夏传所说的是大人吉,个人认为是正确的。假设是丈人则周易全文只有此处出现丈人一词,此其一也;再就是从师卦的含义来看,能够领众以做行军之事的,也必是一个大人了。} 吉,无咎。

初六:师出以律,否臧凶。

九二:在师中,吉无咎,王三锡命。

六三:师或舆尸,凶。

六四:师左次,无咎。

六五:田有禽,利执言,无咎。长子帅师,弟子舆尸,贞凶。

上六:大君有命,开国承家,小人勿用。

\subsection{彖}
师,众也,贞正也,能以众正,可以王矣。刚中而应,行险而顺,以此毒天下,而民从之,吉又何咎矣。

\subsection{象}
地中有水,师;君子以容民畜众。

初六:师出以律,失律凶也。

九二:在师中吉,承天宠也。王三锡命,怀万邦也。

六三:师或舆尸,大无功也。

六四:左次无咎,未失常也。

六五:长子帅师,以中行也。弟子舆尸,使不当也。

上六:大君有命,以正功也。小人勿用,必乱邦也。

\section{讲解}
卦辞说:

彖说:

象辞说:

\chapter{水地比卦}
比 {\Large ䷇}

\section{原文}
\subsection{经}
比:吉。原筮元永贞,无咎。不宁方来,后夫凶。

初六:有孚比之,无咎。有孚盈缶,终来有它,吉。

六二:比之自内,贞吉。

六三:比之匪人。

六四:外比之,贞吉。

九五:显比,王用三驱,失前禽。 邑人不诫,吉。

上六:比之无首,凶。

\subsection{彖}
比,吉也,比,辅也,下顺从也。原筮元永贞,无咎,以刚中也。不宁方来,上下应也。后夫凶,其道穷也。

\subsection{象}
地上有水,比;先王以建万国,亲诸侯。

初六:比之初六,有它吉也。

六二:比之自内,不自失也。

六三:比之匪人,不亦伤乎!

六四:外比於贤,以从上也。

九五:显比之吉,位正中也。舍逆取顺,失前禽也。 邑人不诫,上使中也。

上六:比之无首,无所终也。

\section{讲解}
卦辞说:

彖说:

象辞说:


\chapter{风天小畜(xù)卦}
小畜 {\Large ䷈}

\section{原文}

\subsection{经}
小畜:亨。 密云不雨,自我西郊。

初九:复自道,何其咎,吉。

九二:牵复,吉。

九三:舆说辐,夫妻反目。

六四:有孚,血去惕出,无咎。

九五:有孚挛如,富以其邻。

上九:既雨既处,尚德载,妇贞厉。 月几望,君子征凶

\subsection{彖}
小畜;柔得位,而上下应之,曰小畜。 健而巽,刚中而志行,乃亨。 密云不雨,尚往也。 自我西郊,施未行也。

\subsection{象}
风行天上,小畜;君子以懿文德。

初九:复自道,其义吉也。

九二:牵复在中,亦不自失也。

九三:夫妻反目,不能正室也。

六四:有孚惕出,上合志也。

九五:有孚挛如,不独富也。

上九:既雨既处,德积载也。 君子征凶,有所疑也。


\section{讲解}
卦辞说:

彖说:

象辞说:

\chapter{天泽履(lǚ)卦}
履 {\Large ䷉}

\section{原文}

\subsection{经}
履:履虎尾,不咥人,亨。

初九:素履往,无咎。

九二:履道坦坦,幽人贞吉。

六三:眇能视,跛能履,履虎尾,咥人,凶。武人为于大君。

九四:履虎尾,愬愬,终吉。

九五:夬履,贞厉。

上九:视履考祥,其旋元吉。

\subsection{彖}
履,柔履刚也。说而应乎乾,是以履虎尾,不咥人,亨。刚中正,履帝位而不疚,光明也。

\subsection{象}
上天下泽,履;君子以辩上下,定民志。

初九:素履之往,独行愿也。

九二:幽人贞吉,中不自乱也。

六三:眇能视;不足以有明也。跛能履;不足以与行也。咥人之凶;位不当也。武人为于大君;志刚也。

九四:愬愬终吉,志行也。

九五:夬履贞厉,位正当也。

上九:元吉在上,大有庆也。

\section{讲解}
卦辞说:履卦,走路踩到老虎尾巴了,老虎却不咬人,亨通。

彖说:履卦,以柔顺礼遇刚健。下卦为兑为悦,上卦为乾为天,心悦诚服地应对上天,所以就算踩到老虎尾巴,也不咬人。亨通。九五刚正而居中,登上帝位而不愧疚,因为内心正大光明。

象辞说:上天下泽,履卦;君子应当辨明上下之别,确定人民的志向。

国易堂说:占得此卦者,在事业上起初很不顺利,受到种种威胁,若能提高警惕,谨小慎微,脚踏实地,逐个地去克服困难,就能顺利地渡过危险。

在求名方面,持之以恒,孜孜以求,不为富贵所动,不为世俗干扰,虚心听取别人的建议,不逞强自负,就一定会有好的结果。

占得此卦者,可以外出,虽然可能遇到危险,但却是有惊无险。若是遇到了紧急情况,则可以缓行。

在婚恋方面,夫妻双方都能安贫乐道,则可和睦相处,若一方被财富利诱而变心,则会发生婚变。



\chapter{地天泰卦}
泰 {\Large ䷊}

\section{原文}

\subsection{经}
泰:小往大来,吉亨。

初九:拔茅茹,以其汇,征吉。

九二:包荒,用冯河,不遐遗,朋亡,得尚于中行。

九三:无平不陂,无往不复,艰贞无咎。 勿恤其孚,于食有福。

六四:翩翩不富,以其邻,不戒以孚。

六五:帝乙归妹,以祉元吉。

上六:城复于隍,勿用师。 自邑告命,贞吝。

\subsection{彖}
泰,小往大来,吉亨。则是天地交,而万物通也;上下交,而其志同也。内阳而外阴,内健而外顺,内君子而外小人,君子道长,小人道消也。

\subsection{象}
天地交,泰,后以财(=裁)成天地之道,辅相天地之宜,以左右民。

初九:拔茅征吉,志在外也。

九二:包荒,得尚于中行,以光大也。

九三:无往不复,天地际也。

六四:翩翩不富,皆失实也。不戒以孚,中心愿也。

六五:以祉元吉,中以行愿也。

上六:城复于隍,其命乱也。

\section{讲解}
卦辞说:

彖说:

象辞说:

\chapter{天地否(pǐ)卦}
否 {\Large ䷋}

\section{原文}

\subsection{经}
否:否之匪人,不利君子贞,大往小来。

初六:拔茅茹,以其汇,贞吉亨。

六二:包承。小人吉,大人否亨。

六三:包羞。

九四:有命无咎,畴离祉。

九五:休否,大人吉。其亡其亡,系于苞桑。

上九:倾否,先否后喜。

\subsection{彖}
否之匪人,不利君子贞。大往小来,则是天地不交,而万物不通也;上下不交,而天下无邦也。内阴而外阳,内柔而外刚,内小人而外君子。小人道长,君子道消也。

\subsection{象}
天地不交,否;君子以俭德辟难,不可荣以禄。

初六:拔茅贞吉,志在君也。

六二:大人否亨,不乱群也。

六三:包羞,位不当也。

九四:有命无咎,志行也。

九五:大人之吉,位正当也。

上九:否终则倾,何可长也。


\section{讲解}
卦辞说:

彖说:

象辞说:


\chapter{天火同人卦}
同人 {\Large ䷌}

\section{原文}

\subsection{经}
同人:同人于野,亨。利涉大川,利君子贞。

初九:同人于门,无咎。

六二:同人于宗,吝。

九三:伏戎于莽,升其高陵,三岁不兴。

九四:乘其墉,弗克攻,吉。

九五:同人,先号啕而后笑。大师克相遇。

上九:同人于郊,无悔。

\subsection{彖}
同人,柔得位得中,而应乎乾,曰同人。同人曰,同人于野,亨。利涉大川,乾行也。文明以健,中正而应,君子正也。唯君子为能通天下之志。

\subsection{象}
天与火,同人;君子以类族辨物。

初九:出门同人,又谁咎也。

六二:同人于宗,吝道也。

九三:伏戎于莽,敌刚也。三岁不兴,安行也。

九四:乘其墉,义弗克也,其吉,则困而反则也。

九五:同人之先,以中直也。大师相遇,言相克也。

上九:同人于郊,志未得也。

\section{讲解}
卦辞说:在野外遇到志同道合的人,亨通。利于克服艰难险阻,利于坚守君子的贞正之道。

彖说:同人卦,乾卦得六二,是谓柔顺六二得位又得中,而和乾卦相应。后面彖主要在解释何为君子贞,为何同人卦利君子贞。上天文明而又刚健,中正而相应,和谁相互回应,和君子的坚守正道相应啊。何为君子贞,唯君子为能通天下之志;既君子坚守贞正之道,得到上天的回应,又怎么不亨通呢,又怎么不有利呢。

象辞说:天火,就是同人的卦象了。君子观此卦象而知道族以类似之,物以辩别之的道理。亦即对人求同存异,对物辨别分明;类族辨物,君子做人格物的道理就在其中了。

初九:刚出门就遇到志同道合的人,谁又能指责他呢。反观六二的过于吝啬只同人于宗族之中,初九是没有这个问题的,只是刚出门恰好就遇到了同人,谁又能指责他呢。此爻有近人友爱之象。



\chapter{火天大有卦}
大有 {\Large ䷍}

\section{原文}

\subsection{经}
大有:元亨。

初九:无交害,匪咎,艰则无咎。

九二:大车以载,有攸往,无咎。

九三:公用亨于天子,小人弗克。

九四:匪其彭,无咎。

六五:厥孚交加,威如;吉。

上九:自天祐之,吉无不利。

\subsection{彖}
大有,柔得尊位,大中而上下应之,曰大有。其德刚健而文明,应乎天而时行,是以元亨。

\subsection{象}
火在天上,大有;君子以遏恶扬善,顺天休命。

初九:大有初九,无交害也。

九二:大车以载,积中不败也。

九三:公用亨于天子,小人害也。

九四:匪其彭,无咎;明辨晢也。

六五:厥孚交加,信以发志也。威如之吉,易而无备也。

上九:大有上吉,自天祐也。


\section{讲解}
卦辞说:

彖说:

象辞说:


\chapter{地山谦卦}
谦 {\Large ䷎}

\section{原文}

\subsection{经}
谦:亨,君子有终。

初六:谦谦君子,用涉大川,吉。

六二:鸣谦,贞吉。

九三:劳谦,君子有终,吉。

六四:无不利,撝谦。

六五:不富,以其邻,利用侵伐,无不利。

上六:鸣谦,利用行师,征邑国。

\subsection{彖}
谦,亨,天道下济而光明,地道卑而上行。天道亏盈而益谦,地道变盈而流谦,鬼神害盈而福谦,人道恶盈而好谦。谦尊而光,卑而不可逾,君子之终也。

\subsection{象}
地中有山,谦;君子以裒多益寡,称物平施。

初六:谦谦君子,卑以自牧也。

六二:鸣谦贞吉,中心得也。

九三:劳谦君子,万民服也。

六四:无不利,撝谦;不违则也。

六五:利用侵伐,征不服也。

上六:鸣谦,志未得也。 可用行师,征邑国也。

\section{讲解}
卦辞说:谦,亨通,君子有善终\footnote{礼记曰:“君子曰终,小人曰死”,又有“老曰终,少曰死”。大体终字在古代除了结局终了之意外还有褒义的意思,所以此处周易简要地说道君子有终,大体就是君子有善终之意。关于君子为何有善终,君子有何善终,下面彖继续说道。}。

彖说:谦,亨通,天道下济而光明,地道卑而上行。天道亏盈而益谦,地道变盈而流谦,鬼神害盈而福谦,人道恶盈而好谦。天地鬼神人,有益谦,有流谦,有福谦,有好谦。这就是君子为何有善终的答案,简言之,天地鬼神人皆对谦者善;继而谦虚能够让尊者更光荣,让卑者不可逾越,这就是君子的善终了。

象辞说:谦卦的卦象就是地中有山,如山一般厚重而静静地待在大地之下,谦虚啊\footnote{王弼的周易注说道:“多者用谦以为裒(póu),少者用谦以为益,随物而与,施不失平也。”,其对裒字的解释很是到位。}。所以君子观谦卦的卦象减有余而补不足,称量财物平均施舍于人。

初六:谦谦君子,用谦虚来跋涉大川,是吉祥的。象又继续解释道:谦谦君子,指的就是用谦卑来自我约束的君子。


\chapter{雷地豫卦}
豫 {\Large ䷏}

\section{原文}

\subsection{经}
豫:利建侯行师。

初六:鸣豫,凶。

六二:介于石,不终日,贞吉。

六三:盱豫,悔。迟有悔。

九四:由豫,大有得。勿疑。朋盍簪。

六五:贞疾,恒不死。

上六:冥豫成,有渝,无咎。

\subsection{彖}
豫,刚应而志行,顺以动,豫。豫,顺以动,故天地如之,而况建侯行师乎?天地以顺动,故日月不过,而四时不忒;圣人以顺动,则刑罚清而民服。 豫之时义大矣哉!

\subsection{象}
雷出地奋,豫。 先王以作乐崇德,殷荐之上帝,以配祖考。

初六:初六鸣豫,志穷凶也。

六二:不终日,贞吉;以中正也。

六三:盱豫有悔,位不当也。

九四:由豫,大有得;志大行也。

六五:六五贞疾,乘刚也。恒不死,中未亡也。

上六:冥豫在上,何可长也。

\section{讲解}
卦辞说:

彖说:

象辞说:


\chapter{泽雷随卦}
随 {\Large ䷐}

\section{原文}

\subsection{经}
随:元亨利贞,无咎。

初九:官有渝,贞吉。 出门交有功。

六二:系小子,失丈夫。

六三:系丈夫,失小子。 随有求得,利居贞。

九四:随有获,贞凶。有孚在道,以明,何咎。

九五:孚于嘉,吉。

上六:拘系之,乃从维之。 王用亨于西山。

\subsection{彖}
随,刚来而下柔,动而说,随。大亨贞,无咎,而天下随时,随时之义大矣哉!

\subsection{象}
泽中有雷,随;君子以向晦入宴息。

初九:官有渝,从正吉也。 出门交有功,不失也。

六二:系小子,弗兼与也。

六三:系丈夫,志舍下也。

九四:随有获,其义凶也。 有孚在道,明功也。

九五:孚于嘉,吉;位正中也。

上六:拘系之,上穷也。

\section{讲解}
卦辞说:

彖说:

象辞说:



\chapter{山风蛊(gǔ)卦}
蛊 {\Large ䷑}

\section{原文}

\subsection{经}
蛊:元亨,利涉大川。先甲三日,后甲三日。

初六:干父之蛊,有子,考无咎,厉终吉。

九二:干母之蛊,不可贞。

九三:干父之蛊,小有悔,无大咎。

六四:裕父之蛊,往见吝。

六五:干父之蛊,用誉。

上九:不事王侯,高尚其事。

\subsection{彖}
蛊,刚上而柔下,巽而止,蛊。蛊,元亨,而天下治也。利涉大川,往有事也。先甲三日,后甲三日,终则有始,天行也。

\subsection{象}
山下有风,蛊;君子以振民育德。

初六:干父之蛊,意承考也。

九二:干母之蛊,得中道也。

九三:干父之蛊,终无咎也。

六四:裕父之蛊,往未得也。

六五:干父之蛊;承以德也。

上九:不事王侯,志可则也。

\section{讲解}
卦辞说:

彖说:

象辞说:

\chapter{地泽临卦}
临 {\Large ䷒}

\section{原文}

\subsection{经}
临:元,亨,利,贞。 至于八月有凶。

初九:咸临,贞吉。

九二:咸临,吉无不利。

六三:甘临,无攸利。 既忧之,无咎。

六四:至临,无咎。

六五:知临,大君之宜,吉。

上六:敦临,吉无咎。

\subsection{彖}
临,刚浸而长。 说而顺,刚中而应,大亨以正,天之道也。 至于八月有凶,消不久也。

\subsection{象}
泽上有地,临; 君子以教思无穷,容保民无疆。

初九:咸临贞吉,志行正也。

九二:咸临,吉无不利;未顺命也。

六三:甘临,位不当也。既忧之,咎不长也。

六四:至临无咎,位当也。

六五:大君之宜,行中之谓也。

上六:敦临之吉,志在内也。

\section{讲解}
卦辞说:

彖说:

象辞说:


\chapter{风地观卦}
观 {\Large ䷓}

\section{原文}

\subsection{经}
观:盥而不荐,有孚顒若。

初六:童观,小人无咎,君子吝。

六二:窥观,利女贞。

六三:观我生,进退。

六四:观国之光,利用宾于王。

九五:观我生,君子无咎。

上九:观其生,君子无咎。

\subsection{彖}
大观在上,顺而巽,中正以观天下。观,盥而不荐,有孚顒若,下观而化也。观天之神道,而四时不忒,圣人以神道设教,而天下服矣。

\subsection{象}
风行地上,观;先王以省方,观民设教。

初六:初六童观,小人道也。

六二:窥观女贞,亦可丑也。

六三:观我生,进退;未失道也。

六四:观国之光,尚宾也。

九五:观我生,观民也。

上九:观其生,志未平也。

\section{讲解}
卦辞说:祭祀前洗手却没有献上祭品,有诚心,温和肃敬的样子。【心诚最重要】

彖说:最上爻刚而大,观之在上位,坤顺而巽入,五爻中正以观天下。观,祭祀前洗手却没有献上祭品,有诚心,温和肃敬的样子,下面的人观之而受教化。观天之神道,四时运行没有差错,圣人以神道设教,天下自然顺服。

象辞说:风吹在地上,这就是观卦的卦象;先王省察四方,观察民情并设立教化。

六三:观察自己一生的所作所为,来决定接下来的进退。未失道也。本爻更多是强调自我观察,自己要有自己的判断和主见。

\chapter{火雷噬嗑(shìhé)卦}
噬嗑 {\Large ䷔}

\section{原文}

\subsection{经}
噬嗑:亨。利用狱。

初九:屦校灭趾,无咎。

六二:噬肤灭鼻,无咎。

六三:噬腊肉,遇毒;小吝,无咎。

九四:噬干胏,得金矢,利艰贞,吉。

六五:噬干肉,得黄金,贞厉,无咎。

上九:何校灭耳,凶。

\subsection{彖}
颐中有物,曰噬嗑,噬嗑而亨。刚柔分,动而明,雷电合而章。柔得中而上行,虽不当位,利用狱也。

\subsection{象}
雷电噬嗑;先王以明罚敕法。

初九:屦校灭趾,不行也。

六二:噬肤灭鼻,乘刚也。

六三:遇毒,位不当也。

九四:利艰贞吉,未光也。

六五:贞厉无咎,得当也。

上九:何校灭耳,聪不明也。

\section{讲解}
卦辞说:

彖说:

象辞说:

\chapter{山火贲(bì)卦}
贲 {\Large ䷕}

\section{原文}
\subsection{经}
贲:亨。 小利有所往。

初九:贲其趾,舍车而徒。

六二:贲其须。

九三:贲如濡如,永贞吉。

六四:贲如皤如,白马翰如,匪寇婚媾。

六五:贲于丘园,束帛戋戋,吝,终吉。

上九:白贲,无咎。

\subsection{彖}
贲,亨;柔来而文刚,故亨。分刚上而文柔,故小利有攸往。刚柔交错,天文也;文明以止,人文也。观乎天文,以察时变;观乎人文,以化成天下。
\subsection{象}
山下有火,贲;君子以明庶政,无敢折狱。

初九:舍车而徒,义弗乘也。

六二:贲其须,与上兴也。

九三:永贞之吉,终莫之陵也。

六四:六四,当位疑也。 匪寇婚媾,终无尤也。

六五:六五之吉,有喜也。

上九:白贲无咎,上得志也。

\section{讲解}
卦辞说:

彖说:

象辞说:


\chapter{山地剥卦}
剥 {\Large ䷖}

\section{原文}
\subsection{经}
剥:不利有攸往。

初六:剥床以足,蔑贞凶。

六二:剥床以辨,蔑贞凶。

六三:剥之,无咎。

六四:剥床以肤,凶。

六五:贯鱼,以宫人宠,无不利。

上九:硕果不食,君子得舆,小人剥庐。

\subsection{彖}
剥,剥也,柔变刚也。不利有攸往,小人长也。 顺而止之,观象也。君子尚消息盈虚,天行也。
\subsection{象}
山附地上,剥;上以厚下,安宅。

初六:剥床以足,以灭下也。

六二:剥床以辨,未有与也。

六三:剥之无咎,失上下也。

六四:剥床以肤,切近灾也。

六五:以宫人宠,终无尤也。

上九:君子得舆,民所载也。 小人剥庐,终不可用也。

\section{讲解}
卦辞说:

彖说:

象辞说:

\chapter{地雷复卦}
复 {\Large ䷗}

\section{原文}
\subsection{经}
复:亨。 出入无疾,朋来无咎。 反复其道,七日来复,利有攸往。

初九:不远复,无祗悔,元吉。

六二:休复,吉。

六三:频复,厉无咎。

六四:中行独复。

六五:敦复,无悔。

上六:迷复,凶,有灾眚。用行师,终有大败,以其国君,凶;至于十年,不克征。

\subsection{彖}
复亨;刚反,动而以顺行,是以出入无疾,朋来无咎。反复其道,七日来复,天行也。利有攸往,刚长也。 复其见天地之心乎?

\subsection{象}
雷在地中,复;先王以至日闭关,商旅不行,后不省方。

初九:不远之复,以修身也。

六二:休复之吉,以下仁也。

六三:频复之厉,义无咎也。

六四:中行独复,以从道也。

六五:敦复无悔,中以自考也。

上六:迷复之凶,反君道也。

\section{讲解}
卦辞说:

彖说:

象辞说:

\chapter{天雷无妄卦}
无妄 {\Large ䷘}

\section{原文}
\subsection{经}
无妄:元,亨,利,贞。 其匪正有眚,不利有攸往。

初九:无妄,往吉。

六二:不耕获,不菑畲,则利有攸往。

六三:无妄之灾,或系之牛,行人之得,邑人之灾。

九四:可贞,无咎。

九五:无妄之疾,勿药有喜。

上九:无妄行,有眚,无攸利。

\subsection{彖}
无妄,刚自外来,而为主於内。动而健,刚中而应,大亨以正,天之命也。其匪正有眚,不利有攸往。无妄之往,何之矣?天命不祐,行矣哉?
\subsection{象}
天下雷行,物与无妄;先王以茂对时,育万物。

初九:无妄之往,得志也。

六二:不耕获,未富也。

六三:行人得牛,邑人灾也。

九四:可贞无咎,固有之也。

九五:无妄之药,不可试也。

上九:无妄之行,穷之灾也。

\section{讲解}
卦辞说:

彖说:

象辞说:


\chapter{山天大畜(xù)卦}
大畜 {\Large ䷙}

\section{原文}

\subsection{经}
大畜:利贞,不家食吉,利涉大川。

初九:有厉利已。

九二:舆说輹。

九三:良马逐,利艰贞。 曰闲舆卫,利有攸往。

六四:童牛之牿,元吉。

六五:豶豕之牙,吉。

上九:何天之衢,亨。

\subsection{彖}
大畜,刚健笃实,辉光日新其德,刚上而尚贤。能止健,大正也。不家食吉,养贤也。利涉大川,应乎天也。

\subsection{象}
天在山中,大畜;君子以多识前言往行,以畜其德。

初九:有厉利已,不犯灾也。

九二:舆说輹,中无尤也。

九三:利有攸往,上合志也。

六四:六四元吉,有喜也。

六五:六五之吉,有庆也。

上九:何天之衢,道大行也。

\section{讲解}
卦辞说:

彖说:

象辞说:


\chapter{山雷颐(yí)卦}
颐 {\Large ䷚}

\section{原文}

\subsection{经}
颐:贞吉。 观颐,自求口实。

初九:舍尔灵龟,观我朵颐,凶。

六二:颠颐,拂经,于丘颐,征凶。

六三:拂颐,贞凶,十年勿用,无攸利。

六四:颠颐吉,虎视眈眈,其欲逐逐,无咎。

六五:拂经,居贞吉,不可涉大川。

上九:由颐,厉吉,利涉大川。

\subsection{彖}
颐贞吉,养正则吉也。观颐,观其所养也;自求口实,观其自养也。天地养万物,圣人养贤,以及万民;颐之时大矣哉!
\subsection{象}
山下有雷,颐;君子以慎言语,节饮食。

初九:观我朵颐,亦不足贵也。

六二:六二征凶,行失类也。

六三:十年勿用,道大悖也。

六四:颠颐之吉,上施光也。

六五:居贞之吉,顺以从上也。

上九:由颐厉吉,大有庆也。

\section{讲解}
卦辞说:

彖说:

象辞说:


\chapter{泽风大过卦}
大过 {\Large ䷛}

\section{原文}

\subsection{经}
大过:栋桡,利有攸往,亨。

初六:藉用白茅,无咎。

九二:枯杨生稊,老夫得其女妻,无不利。

九三:栋桡,凶。

九四:栋隆,吉;有它吝。

九五:枯杨生华,老妇得士夫,无咎无誉。

上六:过涉灭顶,凶,无咎。

\subsection{彖}
大过,大者过也。栋桡,本末弱也。刚过而中,巽而说行,利有攸往,乃亨。大过之时义大矣哉!

\subsection{象}
泽灭木,大过;君子以独立不惧,遁世无闷。

初六:藉用白茅,柔在下也。

九二:老夫女妻,过以相与也。

九三:栋桡之凶,不可以有辅也。

九四:栋隆之吉,不桡乎下也。

九五:枯杨生华,何可久也。 老妇士夫,亦可丑也。

上六:过涉之凶,不可咎也。

\section{讲解}
卦辞说:屋梁脆弱曲折;利于有所前往,亨通。

彖说:大过,即是阳刚过盛的意思。栋梁弯曲,是由于栋梁两头太柔弱的缘故。阳刚过盛却处于中部,柔顺、喜悦地前往,所以前往有利并能亨通。大过卦的时势意义太大了!

象辞说:泽水淹没木舟,这是大过的卦象。君子观此卦象,应当独立无惧,遁世无闷。

九五:卦说,枯萎的杨树长了花,老妇人得到了少壮的男子为丈夫,无咎也无誉。象说,枯萎的杨树长了花,怎么能够长久呢。老妇人得了一个少壮的男子为丈夫,也是件丢人的事。国易堂说,不要追求短期的、虚假的繁荣。

上六:卦说,徒步过河被水淹没了头顶,凶险,无咎\footnote{没有过错}。象说,徒步过河遇到了凶险,没什么可责备的。国易堂说,在学习中,不要给自己设定过高的期望,应该制定更加合理的目标。

\section{国易堂占}
国易堂说:占得此卦的人,你可能是一个承担重任的人,或者是栋梁之材。你目前所承担的任务太重,就像被压弯的栋梁一样。这个时候适合前往某处去做事,结果是有利的。

在求名方面,最忌不务实际,追求虚名,以致盛名不符。唯以谦逊态度,谨慎行动,潜心努力,这样才能达到目的。

占得此卦之人,适宜外出,必要时不妨采取特殊行动。

在婚恋方面,自知之明最为重要,不可急于求成,应慎重考虑。在婚姻上,可能会出现老夫少妻或老妻少夫的情况。

占得此卦者,要注意居住房屋的质量。


\chapter{坎卦}
坎 {\Large ䷜}

\section{原文}

\subsection{经}
坎:习坎,有孚,维心亨,行有尚。

初六:习坎,入于坎窞,凶。

九二:坎有险,求小得。

六三:来之坎坎,险且枕,入于坎窞,勿用。

六四:樽酒簋贰,用缶,纳约自牖,终无咎。

九五:坎不盈,祗既平,无咎。

上六:系用徽纆,置于丛棘,三岁不得,凶。

\subsection{彖}
习坎,重险也。水流而不盈,行险而不失其信。维心亨,乃以刚中也。行有尚,往有功也。天险不可升也,地险山川丘陵也,王公设险以守其国,险之时用大矣哉!

\subsection{象}
水洊至,习坎;君子以常德行,习教事。

初六:习坎入坎,失道凶也。

九二:求小得,未出中也。

六三:来之坎坎,终无功也。

六四:樽酒簋贰,刚柔际也。

九五:坎不盈,中未大也。

上六:上六失道,凶三岁也。

\section{讲解}
卦辞说:

彖说:

象辞说:

\chapter{离卦}
离 {\Large ䷝}

\section{原文}


\subsection{经}
离:利贞,亨。 畜牝牛,吉。

初九:履错然,敬之无咎。

六二:黄离,元吉。

九三:日昃之离,不鼓缶而歌,则大耋之嗟,凶。

九四:突如其来如,焚如,死如,弃如。

六五:出涕沱若,戚嗟若,吉。

上九:王用出征,有嘉折首,获匪其丑,无咎。

\subsection{彖}
离,丽也;日月丽乎天,百谷草木丽乎土,重明以丽乎正,乃化成天下。柔丽乎中正,故亨;是以畜牝牛吉也。

\subsection{象}
明两作离,大人以继明照于四方。

初九:履错之敬,以辟咎也。

六二:黄离元吉,得中道也。

九三:日昃之离,何可久也。

九四:突如其来如,无所容也。

六五:六五之吉,离王公也。

上九:王用出征,以正邦也。

\section{讲解}
卦辞说:

彖说:

象辞说:

\chapter{泽山咸卦}
咸 {\Large ䷞}

\section{原文}

\subsection{经}
咸:亨,利贞,取女吉。

初六:咸其拇。

六二:咸其腓,凶,居吉。

九三:咸其股,执其随,往吝。

九四:贞吉悔亡,憧憧往来,朋从尔思。

九五:咸其脢,无悔。

上六:咸其辅,颊,舌。

\subsection{彖}
咸,感也。柔上而刚下,二气感应以相与,止而说,男下女,是以亨利贞,取女吉也。天地感而万物化生,圣人感人心而天下和平;观其所感,而天地万物之情可见矣!

\subsection{象}
山上有泽,咸;君子以虚受人。

初六:咸其拇,志在外也。

六二:虽凶,居吉,顺不害也。

九三:咸其股,亦不处也。 志在随人,所执下也。

九四:咸其股,亦不处也。 志在随人,所执下也。

九五:咸其脢,志末也。

上六:咸其辅,颊,舌,滕口说也。

\section{讲解}
卦辞说:

彖说:

象辞说:

\chapter{雷风恒卦}
恒 {\Large ䷟}

\section{原文}

\subsection{经}
恒:亨,无咎,利贞,利有攸往。

初六:浚恒,贞凶,无攸利。

九二:悔亡。

九三:不恒其德,或承之羞,贞吝。

九四:田无禽。

六五:恒其德,贞,妇人吉,夫子凶。

上六:振恒,凶。

\subsection{彖}
恒,久也。刚上而柔下,雷风相与,巽而动,刚柔皆应,恒。恒亨无咎,利贞;久於其道也,天地之道,恒久而不已也。利有攸往,终则有始也。日月得天,而能久照,四时变化,而能久成,圣人久於其道,而天下化成;观其所恒,而天地万物之情可见矣!

\subsection{象}
雷风,恒;君子以立不易方。

初六:浚恒之凶,始求深也。

九二:九二悔亡,能久中也。

九三:不恒其德,无所容也。

九四:久非其位,安得禽也。

六五:妇人贞吉,从一而终也。夫子制义,从妇凶也。

上六:振恒在上,大无功也。

\section{讲解}
卦辞说:

彖说:

象辞说:

\chapter{天山遁(dùn)卦}
遁 {\Large ䷠}

\section{原文}

\subsection{经}
遁:亨,小利贞。

初六:遁尾,厉,勿用有攸往。

六二:执之用黄牛之革,莫之胜说。

九三:系遁,有疾厉,畜臣妾吉。

九四:好遁君子吉,小人否。

九五:嘉遁,贞吉。

上九:肥遁,无不利。

\subsection{彖}
遁亨,遁而亨也。刚当位而应,与时行也。小利贞,浸而长也。遁之时义大矣哉!

\subsection{象}
天下有山,遁;君子以远小人,不恶而严。

初六:遁尾之厉,不往何灾也。

六二:执用黄牛,固志也。

九三:系遁之厉,有疾惫也。畜臣妾吉,不可大事也。

九四:君子好遁,小人否也。

九五:嘉遁贞吉,以正志也。

上九:肥遁,无不利;无所疑也。

\section{讲解}
卦辞说:

彖说:

象辞说:


\chapter{雷天大壮卦}
大壮 {\Large ䷡}

\section{原文}

\subsection{经}
大壮:利贞。

初九:壮于趾,征凶,有孚。

九二:贞吉。

九三:小人用壮,君子用罔,贞厉。 羝羊触藩,羸其角。

九四:贞吉悔亡,藩决不羸,壮于大舆之輹。

六五:丧羊于易,无悔。

上六:羝羊触藩,不能退,不能遂,无攸利,艰则吉。

\subsection{彖}
大壮,大者壮也。刚以动,故壮。大壮利贞;大者正也。正大而天地之情可见矣!

\subsection{象}
雷在天上,大壮;君子以非礼弗履。

初九:壮于趾,其孚穷也。

九二:九二贞吉,以中也。

九三:小人用壮,君子罔也

九四:藩决不羸,尚往也。

六五:丧羊于易,位不当也。

上六:不能退,不能遂,不祥也。 艰则吉,咎不长也。

\section{讲解}
卦辞说:

彖说:

象辞说:

\chapter{火地晋卦}
晋 {\Large ䷢}

\section{原文}

\subsection{经}
晋:康侯用锡马蕃庶,昼日三接。

初六:晋如,摧如,贞吉。 罔孚,裕无咎。

六二:晋如,愁如,贞吉。 受兹介福,于其王母。

六三:众允,悔亡。

九四:晋如鼫鼠,贞厉。

六五:悔亡,失得勿恤,往吉无不利。

上九:晋其角,维用伐邑,厉吉无咎,贞吝。

\subsection{彖}
晋,进也。明出地上,顺而丽乎大明,柔进而上行。是以康侯用锡马蕃庶,昼日三接也。

\subsection{象}
明出地上,晋;君子以自昭明德。

初六:晋如,摧如;独行正也。裕无咎;未受命也。

六二:受之介福,以中正也。

六三:众允之,志上行也。

九四:鼫鼠贞厉,位不当也。

六五:失得勿恤,往有庆也。

上九:维用伐邑,道未光也。

\section{讲解}
卦辞说:

彖说:

象辞说:

\chapter{地火明夷(yí)卦}
明夷 {\Large ䷣}

\section{原文}

\subsection{经}
明夷:利艰贞。

初九:明夷于飞,垂其翼。 君子于行,三日不食, 有攸往,主人有言。

六二:明夷,夷于左股,用拯马壮,吉。

九三:明夷于南狩,得其大首,不可疾贞。

六四:入于左腹,获明夷之心,于出门庭。

六五:箕子之明夷,利贞。

上六:不明晦,初登于天,后入于地。

\subsection{彖}
明入地中,明夷。内文明而外柔顺,以蒙大难,文王以之。利艰贞,晦其明也,内难而能正其志,箕子以之。

\subsection{象}
明入地中,明夷;君子以莅众,用晦而明。

初九:君子于行,义不食也。

六二:六二之吉,顺以则也。

九三:南狩之志,乃大得也。

六四:入于左腹,获心意也。

六五:箕子之贞,明不可息也。

上六:初登于天,照四国也。 后入于地,失则也。

\section{讲解}
卦辞说:

彖说:

象辞说:

\chapter{风火家人卦}
家人 {\Large ䷤}

\section{原文}

\subsection{经}
家人:利女贞。

初九:闲有家,悔亡。

六二:无攸遂,在中馈,贞吉。

九三:家人嗃嗃,悔厉吉;妇子嘻嘻,终吝。

六四:富家,大吉。

九五:王假有家,勿恤吉。

上九:有孚威如,终吉。

\subsection{彖}
家人,女正位乎内,男正位乎外,男女正,天地之大义也。家人有严君焉,父母之谓也。父父,子子,兄兄,弟弟,夫夫,妇妇,而家道正;正家而天下定矣。

\subsection{象}
风自火出,家人;君子以言有物,而行有恒。

初九:闲有家,志未变也。

六二:六二之吉,顺以巽也。

九三:家人嗃嗃,未失也;妇子嘻嘻,失家节也。

六四:富家大吉,顺在位也。

九五:王假有家,交相爱也。

上九:威如之吉,反身之谓也。

\section{讲解}
卦辞说:

彖说:

象辞说:

\chapter{火泽睽(kuí)卦}
睽 {\Large ䷥}

\section{原文}

\subsection{经}
睽:小事吉。

初九:悔亡,丧马勿逐,自复;见恶人无咎。

九二:遇主于巷,无咎。

六三:见舆曳,其牛掣,其人天且劓,无初有终。

九四:睽孤,遇元夫,交孚,厉无咎。

六五:悔亡,厥宗噬肤,往何咎。

上九:睽孤,见豕负涂,载鬼一车, 先张之弧,后说之弧,匪寇婚媾,往遇雨则吉。

\subsection{彖}
睽,火动而上,泽动而下;二女同居,其志不同行;说而丽乎明,柔进而上行,得中而应乎刚;是以小事吉。天地睽,而其事同也;男女睽,而其志通也;万物睽,而其事类也;睽之时用大矣哉!

\subsection{象}
上火下泽,睽;君子以同而异。

初九:见恶人,以辟咎也。

九二:遇主于巷,未失道也。

六三:见舆曳,位不当也。 无初有终,遇刚也。

九四:交孚无咎,志行也。

六五:厥宗噬肤,往有庆也。

上九:遇雨之吉,群疑亡也。

\section{讲解}
卦辞说:

彖说:

象辞说:


\chapter{水山蹇(jiǎn)卦}
蹇 {\Large ䷦}

\section{原文}

\subsection{经}
蹇:利西南,不利东北;利见大人,贞吉。

初六:往蹇,来誉。

六二:王臣蹇蹇,匪躬之故。

九三:往蹇来反。

六四:往蹇来连。

九五:大蹇朋来。

上六:往蹇来硕,吉;利见大人。

\subsection{彖}
蹇,难也,险在前也。 见险而能止,知矣哉!蹇利西南, 往得中也;不利东北,其道穷也。 利见大人,往有功也。 当位贞吉,以正邦也。 蹇之时用大矣哉!

\subsection{象}
山上有水,蹇;君子以反身修德。

初六:往蹇来誉,宜待也。

六二:王臣蹇蹇,终无尤也。

九三:往蹇来反,内喜之也。

六四:往蹇来连,当位实也。

九五:大蹇朋来,以中节也。

上六:往蹇来硕,志在内也。利见大人,以从贵也。

\section{讲解}
卦辞说:

彖说:

象辞说:

\chapter{雷水解(xiè)卦}
解 {\Large ䷧}

\section{原文}

\subsection{经}
解:利西南,无所往,其来复吉。 有攸往,夙吉。

初六:无咎。

九二:田获三狐,得黄矢,贞吉。

六三:负且乘,致寇至,贞吝。

九四:解而拇,朋至斯孚。

六五:君子维有解,吉;有孚于小人。

上六:公用射隼,于高墉之上,获之,无不利。

\subsection{彖}
解,险以动,动而免乎险,解。解利西南,往得众也。其来复吉,乃得中也。有攸往夙吉,往有功也。天地解,而雷雨作,雷雨作,而百果草木皆甲坼,解之时大矣哉!

\subsection{象}
雷雨作,解;君子以赦过宥罪。

初六:刚柔之际,义无咎也。

九二:九二贞吉,得中道也。

六三:负且乘,亦可丑也,自我致戎,又谁咎也。

九四:解而拇,未当位也。

六五:君子有解,小人退也。

上六:公用射隼,以解悖也。

\section{讲解}
卦辞说:

彖说:

象辞说:

\chapter{山泽损卦}
损 {\Large ䷨}
\section{原文}

\subsection{经}
损:有孚,元吉,无咎,可贞,利有攸往。曷之用?二簋可用享。

初九:已事遄往,无咎,酌损之。

九二:利贞,征凶,弗损益之。

六三:三人行,则损一人;一人行,则得其友。

六四:损其疾,使遄有喜,无咎。

六五:或益之,十朋之龟弗克违,元吉。

上九:弗损益之,无咎,贞吉,利有攸往,得臣无家。

\subsection{彖}
损,损下益上,其道上行。损而有孚,元吉,无咎,可贞,利有攸往。 曷之用? 二簋可用享;二簋应有时。损刚益柔有时,损益盈虚,与时偕行。

\subsection{象}
山下有泽,损;君子以惩忿窒欲。

初九:已事遄往,尚合志也。

九二:九二利贞,中以为志也。

六三:一人行,三则疑也。

六四:损其疾,亦可喜也。

六五:六五元吉,自上佑也。

上九:弗损益之,大得志也。

\section{讲解}
卦辞说:

彖说:

象辞说:

\chapter{风雷益卦}
益 {\Large ䷩}
\section{原文}

\subsection{经}
益:利有攸往,利涉大川。

初九:利用为大作,元吉,无咎。

六二:或益之,十朋之龟弗克违,永贞吉。 王用享于帝,吉。

六三:益之用凶事,无咎。 有孚中行,告公用圭。

六四:中行,告公从。 利用为依迁国。

九五:有孚惠心,勿问元吉。 有孚惠我德。

上九:莫益之,或击之,立心勿恒,凶。

\subsection{彖}
益,损上益下,民说无疆,自上下下,其道大光。利有攸往,中正有庆。利涉大川,木道乃行。 益动而巽,日进无疆。天施地生,其益无方。 凡益之道,与时偕行。

\subsection{象}
风雷,益;君子以见善则迁,有过则改。

初九:元吉无咎,下不厚事也。

六二:或益之,自外来也。

六三:益用凶事,固有之也。

六四:告公从,以益志也。

九五:有孚惠心,勿问之矣。惠我德,大得志也。

上九:莫益之,偏辞也。或击之,自外来也。

\section{讲解}
卦辞说:

彖说:

象辞说:

\chapter{泽天夬(guài)卦}
夬 {\Large ䷪}
\section{原文}

\subsection{经}
夬:扬于王庭,孚号,有厉,告自邑,不利即戎,利有攸往。

初九:壮于前趾,往不胜为咎。

九二:惕号,莫夜有戎,勿恤。

九三:壮于頄,有凶。 君子夬夬,独行遇雨,若濡有愠,无咎。

九四:臀无肤,其行次且。 牵羊悔亡,闻言不信。

九五:苋陆夬夬,中行无咎。

上六:无号,终有凶。

\subsection{彖}
夬,决也,刚决柔也。健而说,决而和,扬于王庭,柔乘五刚也。孚号有厉,其危乃光也。 告自邑,不利即戎,所尚乃穷也。利有攸往,刚长乃终也。

\subsection{象}
泽上于天,夬;君子以施禄及下,居德则忌。

初九:不胜而往,咎也。

九二:有戎勿恤,得中道也。

九三:君子夬夬,终无咎也。

九四:其行次且,位不当也。闻言不信,聪不明也。

九五:中行无咎,中未光也。

上六:无号之凶,终不可长也。

\section{讲解}
卦辞说:

彖说:

象辞说:

\chapter{天风姤(gòu)卦}
姤 {\Large ䷫}

\section{原文}

\subsection{经}
姤:女壮,勿用取女。

初六:系于金柅,贞吉,有攸往,见凶,羸豕孚蹢躅。

九二:包有鱼,无咎,不利宾。

九三:臀无肤,其行次且,厉,无大咎。

九四:包无鱼,起凶。

九五:以杞包瓜,含章,有陨自天。

上九:姤其角,吝,无咎。


\subsection{彖}
姤,遇也,柔遇刚也。勿用取女,不可与长也。天地相遇,品物咸章也。刚遇中正,天下大行也。姤之时义大矣哉!

\subsection{象}
天下有风,姤;后以施命诰四方。

初六:系于金柅,柔道牵也。

九二:包有鱼,义不及宾也。

九三:其行次且,行未牵也。

九四:无鱼之凶,远民也。

九五:九五含章,中正也。有陨自天,志不舍命也。

上九:姤其角,上穷吝也。


\section{讲解}
卦辞说:

彖说:

象辞说:


\chapter{泽地萃(cuì)卦}
萃 {\Large ䷬}
\section{原文}

\subsection{经}
萃:亨。 王假有庙,利见大人,亨,利贞。 用大牲吉,利有攸往。

初六:有孚不终,乃乱乃萃,若号一握为笑,勿恤,往无咎。

六二:引吉,无咎,孚乃利用禴。

六三:萃如,嗟如,无攸利,往无咎,小吝。

九四:大吉,无咎。

九五:萃有位,无咎。 匪孚,元永贞,悔亡。

上六:赍咨涕洟,无咎。

\subsection{彖}
萃,聚也;顺以说,刚中而应,故聚也。王假有庙,致孝享也。利见大人亨,聚以正也。用大牲吉,利有攸往,顺天命也。观其所聚,而天地万物之情可见矣。

\subsection{象}
泽上於地,萃;君子以除戎器,戒不虞。

初六:乃乱乃萃,其志乱也。

六二:引吉无咎,中未变也。

六三:往无咎,上巽也。

九四:大吉无咎,位不当也。

九五:萃有位,志未光也。

上六:赍咨涕洟,未安上也。

\section{讲解}
卦辞说:

彖说:

象辞说:

\chapter{地风升卦}
升 {\Large ䷭}

\section{原文}

\subsection{经}
升:元亨,用见大人,勿恤,南征吉。

初六:允升,大吉。

九二:孚乃利用禴,无咎。

九三:升虚邑。

六四:王用亨于岐山,吉无咎。

六五:贞吉,升阶。

上六:冥升,利于不息之贞。


\subsection{彖}
柔以时升,巽而顺,刚中而应,是以大亨。用见大人,勿恤;有庆也。 南征吉,志行也。

\subsection{象}
地中生木,升;君子以顺德,积小以高大。

初六:允升大吉,上合志也。

九二:九二之孚,有喜也。

九三:升虚邑,无所疑也。

六四:王用亨于岐山,顺事也。

六五:贞吉升阶,大得志也。

上六:冥升在上,消不富也。

\section{讲解}
卦辞说:

彖说:

象辞说:

\chapter{泽水困卦}
困 {\Large ䷮}

\section{原文}

\subsection{经}
困:亨,贞,大人吉,无咎,有言不信。

初六:臀困于株木,入于幽谷,三岁不觌。

九二:困于酒食,朱绂方来,利用亨祀,征凶,无咎。

六三:困于石,据于蒺藜,入于其宫,不见其妻,凶。

九四:来徐徐,困于金车,吝,有终。

九五:劓刖,困于赤绂,乃徐有说,利用祭祀。

上六:困于葛藟,于臲卼,曰动悔。 有悔,征吉。

\subsection{彖}
困,刚掩也。险以说,困而不失其所,亨;其唯君子乎?贞大人吉,以刚中也。有言不信,尚口乃穷也。

\subsection{象}
泽无水,困;君子以致命遂志。

初六:入于幽谷,幽不明也。

九二:困于酒食,中有庆也。

六三:据于蒺藜,乘刚也。入于其宫,不见其妻,不祥也。

九四:来徐徐,志在下也。虽不当位,有与也。

九五:劓刖,志未得也。乃徐有说,以中直也。利用祭祀,受福也。

上六:困于葛藟,未当也。动悔,有悔吉,行也。

\section{讲解}
卦辞说:

彖说:

象辞说:

\chapter{水风井卦}
井 {\Large ䷯}

\section{原文}

\subsection{经}
井:改邑不改井,无丧无得,往来井井。汔至亦未繘\footnote{繘(jú):井上汲水的绳子,繘井即用绳汲取井水。}井,羸其瓶,凶。

初六:井泥不食,旧井无禽。

九二:井谷射鲋,瓮敝漏。

九三:井渫不食,为我心恻,可用汲,王明,并受其福。

六四:井甃,无咎。

九五:井冽,寒泉食。

上六:井收勿幕,有孚元吉。

\subsection{彖}
巽乎水而上水,井;井养而不穷也。改邑不改井,乃以刚中也。汔至亦未繘井,未有功也。羸其瓶,是以凶也。

\subsection{象}
木上有水,井;君子以劳民劝相。

初六:井泥不食,下也。旧井无禽,时舍也。

九二:井谷射鲋,无与也。

九三:井渫不食,行恻也。求王明,受福也。

六四:井甃无咎,修井也。

九五:寒泉之食,中正也。

上六:元吉在上,大成也。

\section{讲解}
卦辞说:村邑会改迁但井制不会改变,人们来来往往打取井水,井自身没有什么损失也没什么获得。用绳打取井水几乎快要到了,但还没有打到井水,此时绳子还缠绕着水瓶,凶险。

彖说:巽作用于水然后把水弄上来,这就是井了;井养而不穷也。改邑不改井,乃以刚中也。用绳打取井水几乎快要到了,但还没有打到井水,所以称未有功也。此时绳子还缠绕着水瓶,而水瓶有触碰井壁的风险,是以凶也。

象辞说:木上面有水,这就是井的卦象了。井是民众共同劳作出来的,有了井水民众共同享用而得互相帮助,故君子观井之成和井之用而效仿之,这就是劳民劝相的道理了。

国易堂说:占得此卦者,在事业上处于平稳状态,当下不宜贸然前进,也不必后退,而应以积极的态度努力进修,提高自己,充实个人实力,待机而起。要注意与人的合作,相互协助。

在求名方面,应特别注意向贤德的人求教,以便被发现而受到推荐。同时要学习水井的精神,真诚奉献,不断丰富自己的才能,这样一定会受到社会的重若想外出,则要提前做好准备,若没有十分的必要和充分的把握不可随意出行。

在婚恋方面,不必因着急而结婚,会有般配的伴侣出现。

九三:井已经清理干净但是人们却不来饮用,为此我心甚是伤悲。可以来汲取井水了,盼望君王贤明,并受其福啊。此爻有怀才不遇之象,高岛断易说再过二爻时间\footnote{所谓二爻时间是按照你起卦时心中所想的时间刻度为基准,比如占之年,则再过二爻即再过两年;占之月,则再过两月。}可有转机。


\chapter{泽火革卦}
革 {\Large ䷰}

\section{原文}

\subsection{经}
革:巳日乃孚,元亨利贞,悔亡。

初九:巩用黄牛之革。

六二:巳日乃革之,征吉,无咎。

九三:征凶,贞厉,革言三就,有孚。

九四:悔亡,有孚改命,吉。

九五:大人虎变,未占有孚。

上六:君子豹变,小人革面,征凶,居贞吉。

\subsection{彖}
革,水火相息,二女同居,其志不相得,曰革。巳日乃孚;革而信之。文明以说,大亨以正,革而当,其悔乃亡。天地革而四时成,汤武革命,顺乎天而应乎人,革之时大矣哉!

\subsection{象}
泽中有火,革;君子以治历明时。

初九:巩用黄牛,不可以有为也。

六二:巳日革之,行有嘉也。

九三:革言三就,又何之矣。

九四:改命之吉,信志也。

九五:大人虎变,其文炳也。

上六:君子豹变,其文蔚也。小人革面,顺以从君也。

\section{讲解}
卦辞说:

彖说:

象辞说:

\chapter{火风鼎卦}
鼎 {\large ䷱}
\section{原文}

\subsection{经}
鼎:元吉,亨。

初六:鼎颠趾,利出否,得妾以其子,无咎。

九二:鼎有实,我仇有疾,不我能即,吉。

九三:鼎耳革,其行塞,雉膏不食,方雨亏悔,终吉。

九四:鼎折足,覆公餗,其形渥,凶。

六五:鼎黄耳金铉,利贞。

上九:鼎玉铉,大吉,无不利。

\subsection{彖}
鼎,象也。以木巽火,亨饪也。圣人亨以享上帝,而大亨以养圣贤。巽而耳目聪明,柔进而上行,得中而应乎刚,是以元亨。

\subsection{象}
木上有火,鼎;君子以正位凝命。

初六:鼎颠趾,未悖也。利出否,以从贵也。

九二:鼎有实,慎所之也。我仇有疾,终无尤也。

九三:鼎耳革,失其义也。

九四:覆公餗,信如何也。

六五:鼎黄耳,中以为实也。

上九:玉铉在上,刚柔节也。

\section{讲解}
卦辞说:鼎卦,大吉祥,亨通。

彖说:鼎卦的形状就像一个鼎,因为巽为木,离为火,以木生火烧鼎,可以进行烹饪之事。圣人用鼎烹饪用来祭拜上天,而大量的烹饪食物是用来供养圣贤。巽为风无孔不入表明此人耳目聪明,风很柔软表明此人柔顺上进,只要行动适中,就会受到刚强者的响应,因此卦辞上说元亨。

象辞说:木上有火,便是鼎卦的卦象。君子从这个卦象得到启发,应当端正自己的位置,重视上天赋予的使命。

国易堂说:占得此卦者,已经具备开拓事业的各种条件。个人条件很好,聪明冷静,但要注意应以端正的态度去为人处世,严于律已,无轻举妄动和邪思,刚中自守。要注意培养和吸收人才,为己所用。要与有才德的人合作,这样更利成功。

在求名上,首先应严于律已,不陷入与他人的怨仇之中,柔而上行,循序渐进,具备随时应变和随势应变的能力。如果得到知人者的善用,更是前途广大。即使暂时不受重视,无出路也无防,最终可实现抱负。

在外出方面,无重大事情不宜外出,如果为了工作和事业而出差,则会很顺

在婚恋方面,应该说个人条件比较不错,但是不要自视甚高,在选择另一半时要切合自己的实际。对于已婚人士来说,要注意不要出现三角恋爱、第三者插足的情况。

在身体健康方面,因为鼎除能烹煮食物外,还像煮药的砂锅,所以自己或家人可能会有因病而吃汤药的。




\chapter{震卦}
震 {\large ䷲}
\section{原文}

\subsection{经}
震:亨。 震来虩虩,笑言哑哑。 震惊百里,不丧匕鬯。

初九:震来虩虩,后笑言哑哑,吉。

六二:震来厉,亿丧贝,跻于九陵,勿逐,七日得。

六三:震苏苏,震行无眚。

九四:震遂泥。

六五:震往来厉,亿无丧,有事。

上六:震索索,视矍矍,征凶。 震不于其躬,于其邻,无咎。 婚媾有言。

\subsection{彖}
震,亨。震来虩虩,恐致福也。笑言哑哑,后有则也。震惊百里,惊远而惧迩也。 出可以守宗庙社稷,以为祭主也。

\subsection{象}
洊雷,震;君子以恐惧修省。

初九:震来虩虩,恐致福也。笑言哑哑,后有则也。

六二:震来厉,乘刚也。

六三:震苏苏,位不当也。

九四:震遂泥,未光也。

六五:震往来厉,危行也。其事在中,大无丧也。

上六:震索索,未得中也。虽凶无咎,畏邻戒也。


\section{讲解}
卦辞说:

彖说:

象辞说:



\chapter{艮卦}
艮 {\Large ䷳}


\section{原文}

\subsection{经}
艮:艮其背,不获其身,行其庭,不见其人,无咎。

初六:艮其趾,无咎,利永贞。

六二:艮其腓,不拯其随,其心不快。

九三:艮其限,列其夤,厉薰心。

六四:艮其身,无咎。

六五:艮其辅,言有序,悔亡。

上九:敦艮,吉。


\subsection{彖}
艮,止也。时止则止,时行则行,动静不失其时,其道光明。艮其止,止其所也。上下敌应,不相与也。 是以不获其身,行其庭不见其人,无咎也。

\subsection{象}
兼山,艮;君子以思不出其位。

初六:艮其趾,未失正也。

六二:不拯其随,未退听也。

九三:艮其限,危薰心也。

六四:艮其身,止诸躬也。

六五:艮其辅,以中正也。

上九:敦艮之吉,以厚终也。

\section{讲解}
卦辞说:

彖说:

象辞说:

\chapter{风山渐卦}
渐 {\Large ䷴}


\section{原文}

\subsection{经}
渐:女归吉,利贞。

初六:鸿渐于干,小子厉,有言,无咎。

六二:鸿渐于磐,饮食衎衎,吉。

九三:鸿渐于陆,夫征不复,妇孕不育,凶;利御寇。

六四:鸿渐于木,或得其桷,无咎。

九五:鸿渐于陵,妇三岁不孕,终莫之胜,吉。

上九:鸿渐于陆,其羽可用为仪,吉。

\subsection{彖}
渐之进也,女归吉也。进得位,往有功也。进以正,可以正邦也。其位刚,得中也。止而巽,动不穷也。

\subsection{象}
山上有木,渐;君子以居贤德,善俗。

初六:小子之厉,义无咎也。

六二:饮食衎衎,不素饱也。

九三:夫征不复,离群丑也。 妇孕不育,失其道也。利用御寇,顺相保也。

六四:或得其桷,顺以巽也。

九五:终莫之胜,吉;得所愿也。

上九:其羽可用为仪,吉;不可乱也。


\section{讲解}
卦辞说:

彖说:

象辞说:

\chapter{雷泽归妹卦}
归妹 {\Large ䷵}


\section{原文}

\subsection{经}
归妹:征凶,无攸利。

初九:归妹以娣,跛能履,征吉。

九二:眇能视,利幽人之贞。

六三:归妹以须,反归以娣。

九四:归妹愆期,迟归有时。

六五:帝乙归妹,其君之袂,不如其娣之袂良,月几望,吉。

上六:女承筐无实,士刲羊无血,无攸利。

\subsection{彖}
归妹,天地之大义也。天地不交,而万物不兴,归妹人之终始也。说以动,所归妹也。征凶,位不当也。 无攸利,柔乘刚也。

\subsection{象}
泽上有雷,归妹;君子以永终知敝。

初九:归妹以娣,以恒也。跛能履吉,相承也。

九二:利幽人之贞,未变常也。

六三:归妹以须,未当也。

九四:愆期之志,有待而行也。

六五:帝乙归妹,不如其娣之袂良也。其位在中,以贵行也。

上六:上六无实,承虚筐也。

\section{讲解}
卦辞说:

彖说:

象辞说:

\chapter{雷火丰卦}
丰 {\Large ䷶}

\section{原文}

\subsection{经}
丰:亨,王假之,勿忧,宜日中。

初九:遇其配主,虽旬无咎,往有尚。

六二:丰其蔀,日中见斗,往得疑疾,有孚发若,吉。

九三:丰其沛,日中见沫,折其右肱,无咎。

九四:丰其蔀,日中见斗,遇其夷主,吉。

六五:来章,有庆誉,吉。

上六:丰其屋,蔀其家,窥其户,阒其无人,三岁不觌,凶。

\subsection{彖}
丰,大也。明以动,故丰。王假之,尚大也。勿忧宜日中,宜照天下也。日中则昃,月盈则食,天地盈虚,与时消息,而况人於人乎?况於鬼神乎?

\subsection{象}
雷电皆至,丰;君子以折狱致刑。

初九:虽旬无咎,过旬灾也。

六二:虽旬无咎,过旬灾也。

九三:丰其沛,不可大事也。折其右肱,终不可用也。

九四:丰其蔀,位不当也。日中见斗,幽不明也。遇其夷主,吉;行也。

六五:六五之吉,有庆也。

上六:丰其屋,天际翔也。窥其户,阒其无人,自藏也。

\section{讲解}
卦辞说:

彖说:

象辞说:

\chapter{火山旅卦}
旅 {\Large ䷷}


\section{原文}

\subsection{经}
旅:小亨,旅贞吉。

初六:旅琐琐,斯其所取灾。

六二:旅即次,怀其资,得童仆贞。

九三:旅焚其次,丧其童仆,贞厉。

九四:旅于处,得其资斧,我心不快。

六五:射雉一矢亡,终以誉命。

上九:鸟焚其巢,旅人先笑后号啕。丧牛于易,凶。

\subsection{彖}
旅,小亨,柔得中乎外,而顺乎刚,止而丽乎明,是以小亨,旅贞吉也。旅之时义大矣哉!

\subsection{象}
山上有火,旅;君子以明慎用刑,而不留狱。

初六:旅琐琐,志穷灾也。

六二:得童仆贞,终无尤也。

九三:旅焚其次,亦以伤矣。以旅与下,其义丧也。

九四:旅于处,未得位也。得其资斧,心未快也。

六五:终以誉命,上逮也。

上九:以旅在上,其义焚也。丧牛于易,终莫之闻也。

\section{讲解}
卦辞说:

彖说:

象辞说:

\chapter{巽(xùn)卦}
巽 {\Large ䷸}

\section{原文}

\subsection{经}
巽:小亨,利攸往,利见大人。

初六:进退,利武人之贞。

九二:巽在床下,用史巫纷若,吉无咎。

九三:频巽,吝。

六四:悔亡,田获三品。

九五:贞吉悔亡,无不利。 无初有终,先庚三日,后庚三日,吉。

上九:巽在床下,丧其资斧,贞凶。

\subsection{彖}
重巽以申命,刚巽乎中正而志行。柔皆顺乎刚,是以小亨,利有攸往,利见大人。

\subsection{象}
随风,巽;君子以申命行事。

初六:进退,志疑也。利武人之贞,志治也。

九二:纷若之吉,得中也。

九三:频巽之吝,志穷也。

六四:田获三品,有功也。

九五:九五之吉,位正中也。

上九:巽在床下,上穷也。丧其资斧,正乎凶也。

\section{讲解}
卦辞说:

彖说:

象辞说:


\chapter{兑卦}
兑 {\Large ䷹}

\section{原文}

\subsection{经}
兑:亨,利贞。

初九:和兑,吉。

九二:孚兑,吉,悔亡。

六三:来兑,凶。

九四:商兑,未宁,介疾有喜。

九五:孚于剥,有厉。

上六:引兑。

\subsection{彖}
兑,说也。刚中而柔外,说以利贞,是以顺乎天,而应乎人。说以先民,民忘其劳;说以犯难,民忘其死;说之大,民劝矣哉!

\subsection{象}
丽泽,兑;君子以朋友讲习。

初九:和兑之吉,行未疑也。

九二:孚兑之吉,信志也。

六三:来兑之凶,位不当也。

九四:来兑之凶,位不当也。

九五:孚于剥,位正当也。

上六:上六引兑,未光也。

\section{讲解}
卦辞说:

彖说:

象辞说:

\chapter{风水涣卦}
涣 {\Large ䷺}

\section{原文}

\subsection{经}
涣:亨。王假有庙,利涉大川,利贞。

初六:用拯马壮,吉。

九二:涣奔其机,悔亡。

六三:涣其躬,无悔。

六四:涣其群,元吉。 涣有丘,匪夷所思。

九五:涣汗其大号,涣王居,无咎。

上九:涣其血,去逖出,无咎。

\subsection{彖}
涣,亨。 刚来而不穷,柔得位乎外而上同。王假有庙,王乃在中也。利涉大川,乘木有功也。

\subsection{象}
风行水上,涣;先王以享于帝立庙。

初六:初六之吉,顺也。

九二:涣奔其机,得愿也。

六三:涣其躬,志在外也。

六四:涣其群,元吉;光大也。

九五:王居无咎,正位也。

上九:涣其血,远害也。

\section{讲解}
卦辞说:

彖说:

象辞说:

\chapter{水泽节卦}
节 {\Large ䷻}


\section{原文}

\subsection{经}
节:亨。 苦节不可贞。

初九:不出户庭,无咎。

九二:不出门庭,凶。

六三:不节若,则嗟若,无咎。

六四:安节,亨。

九五:甘节,吉;往有尚。

上六:苦节,贞凶,悔亡。


\subsection{彖}
节,亨,刚柔分,而刚得中。苦节不可贞,其道穷也。说以行险,当位以节,中正以通。天地节而四时成,节以制度,不伤财,不害民。

\subsection{象}
泽上有水,节;君子以制数度,议德行。

初九:不出户庭,知通塞也。

九二:不出门庭,失时极也。

六三:不节之嗟,又谁咎也。

六四:安节之亨,承上道也。

九五:甘节之吉,居位中也。

上六:苦节贞凶,其道穷也。


\section{讲解}
卦辞说:

彖说:

象辞说:

\chapter{风泽中孚(fú)卦}
中孚 {\LARGE ䷼}


\section{原文}

\subsection{经}
中孚:豚鱼吉,利涉大川,利贞。

\subsection{彖}
中孚,柔在内而刚得中。说而巽,孚,乃化邦也。豚鱼吉,信及豚鱼也。利涉大川,乘木舟虚也。中孚以利贞,乃应乎天也。

\subsection{象}
泽上有风,中孚;君子以议狱缓死。

\chapter{雷山小过卦}
小过 {\LARGE ䷽}

\section{原文}
\subsection{经}
小过:亨,利贞,可小事,不可大事。飞鸟遗之音,不宜上,宜下,大吉。

初六:飞鸟以凶。

六二:过其祖,遇其妣;不及其君,遇其臣;无咎。

九三:弗过防之,从或戕之,凶。

九四:无咎,弗过遇之。 往厉必戒,勿用永贞。

六五:密云不雨,自我西郊,公弋取彼在穴。

上六:弗遇过之,飞鸟离之,凶,是谓灾眚。

\subsection{彖}
小过,小者过而亨也。过以利贞,与时行也。柔得中,是以小事吉也。刚失位而不中,是以不可大事也。有飞鸟之象焉,有飞鸟遗之音,不宜上宜下,大吉;上逆而下顺也。

\subsection{象}
山上有雷,小过;君子以行过乎恭,丧过乎哀,用过乎俭。

初六:飞鸟以凶,不可如何也。

六二:不及其君,臣不可过也。

九三:从或戕之,凶如何也。

九四:弗过遇之,位不当也。 往厉必戒,终不可长也。

六五:密云不雨,已上也。

上六:弗遇过之,已亢也。

\section{讲解}
卦辞说:

彖说:

象辞说:

序卦曰:“有其信者,必行之,故受之以小过。” 。有诚信的人必然会行动,因故而得小过卦。

象曰:“山上有雷,小过;”。这里指出小过的卦象就是山上有雷。山上有雷,雷会把山上的树木击坏,但没有雷怎有雨,没有雨山上的树木又怎得滋润和生长。

象曰:“君子以行过乎恭,丧过乎哀,用过乎俭。”。指的是所以君子参见小过的卦象之后应该行为上过于恭敬,丧事上过于悲哀,日用上过于节俭。此为小小过分之理也。

经谈“飞鸟”,大概因为小过卦从卦象看像一只展翅的飞鸟。

小过,亨通,利于坚守中正之道,可以做一些小事,不要去做大事。飞鸟飞过留下声音,不宜向上飞,宜于向下飞,如此则大为吉祥。因为上飞




\chapter{水火既济卦}
既济 {\Large ䷾}
\section{原文}

\subsection{经}
既济:亨,小利贞,初吉终乱。

初九:曳其轮,濡其尾,无咎。

六二:妇丧其髴,勿逐,七日得。

九三:高宗伐鬼方,三年克之,小人勿用。

六四:繻有衣袽,终日戒。

九五:东邻杀牛,不如西邻之禴祭,实受其福。

上六:濡其首,厉。

\subsection{彖}
既济,亨,小者亨也。利贞,刚柔正而位当也。初吉,柔得中也。终止则乱,其道穷也。

\subsection{象}
水在火上,既济;君子以思患而豫防之。

初九:曳其轮,义无咎也。

六二:七日得,以中道也。

九三:三年克之,惫也。

六四:终日戒,有所疑也。

九五:东邻杀牛,不如西邻之时也;实受其福,吉大来也。

上六:濡其首厉,何可久也。

\section{讲解}
卦辞说:

彖说:

象辞说:

\chapter{火水未济卦}
未济 {\Large ䷿}
\section{原文}

\subsection{经}
未济:亨,小狐汔济,濡(rú)其尾,无攸利。

初六:濡其尾,吝。

九二:曳其轮,贞吉。

六三:未济,征凶,利涉大川。

九四:贞吉,悔亡,震用伐鬼方,三年有赏于大国。

六五:贞吉,无悔,君子之光,有孚,吉。

上九:有孚于饮酒,无咎,濡其首,有孚失是。

\subsection{彖}
未济,亨;柔得中也。小狐汔济,未出中也。濡其尾,无攸利;不续终也。虽不当位,刚柔应也。

\subsection{象}
火在水上,未济;君子以慎辨物居方。

初六:濡其尾,亦不知极也。

九二:九二贞吉,中以行正也。

六三:未济征凶,位不当也。

九四:贞吉悔亡,志行也。

六五:君子之光,其晖吉也。

上九:饮酒濡首,亦不知节也。

\section{讲解}
卦辞说:亨通,小狐狸渡河快到了,尾巴濡湿了,没什么利益。

彖说:

象辞说:

九四:







\part{周易相关}
\chapter{基本术语}
\section{易经}
易经原有三,连山易,归藏易,周易,前两易已失传,

\section{爻的当位和不当位}
认为爻位从下往上数,奇数为阳,偶数为阴,于是奇数位为阳位,偶数位为阴位,若阳爻居阳位则为当位,若阴爻居阴位也为当位,反之为不当位。

\section{卦彖象}
严格意义上来说周易的作者不详,只能说是成书于周朝,最大的可能是周文王指定,其朝中文官收集史料编纂而成。其他彖、象、文言、系、序等是孔子所作的注解,当然可能会有其门人的部分贡献,但绝大部分应该都是孔子所作。

\section{变爻}
爻分为二,若为少阴少阳则该爻不动。若为老阴老阳则该爻为变爻,变动之爻。

\section{先天六十四卦顺序和后天六十四卦顺序}
周易一书的顺序或者说按照序卦传而来的顺序通常被人们称为后天六十四卦顺序,这个顺序更多的反应了作者认为事物发展的一种哲理性解释。

通常预测会按照先天六十四卦顺序来,先天六十四卦顺序就是在变爻到六爻的阶段,最终发展到不可逆转进而形成变卦。先天六十四卦更多的是揭示天理自然规律,而后天六十四严格意义上来说并没有顺序一说,只是方便大家阅读而解释出来的那个顺序。

\section{尊卑贵贱上下}
周易里面尊卑,贵贱,上下各自是分开的,虽然人们常谈尊贵,卑下,下贱,但正所谓上位也可能不尊,下位也有高贵之民,看看周易里面系辞部分说的很清楚:“天尊地卑,乾坤定矣。卑高以陈,贵贱位矣。”尊卑本只是无褒贬含义的高和低的意思,天高高在上,地低低在下,乾坤就这样确定下来了。然后注意下面的\emph{卑高}这个词的顺序 ,正是从人的角度去看,先看到地,再看到天,是言卑高。关于这块南怀瑾说的很好,人性就是如此,容易摸得着的东西就轻贱它,总是得不到的东西就觉得很珍贵。周易在这里说的很明白,本来天尊地卑,并没有贵贱概念,因为人性,所以出现了贵贱的概念了。

在周易里面上下有统治或管理上的上下位之分,再一次将上下和贵贱和尊卑混为一谈那是后来人的私心想法。周易里面谈上下更多是让人们注意到管理上的上下位之别,比如履卦的“君子辩上下”,其正对应孔子谈为政的核心原则就是:“君君,臣臣,父父,子子”。当然孔子的谈论更多的局限在古代社会的君臣父子这几大关系上做了比喻和延伸,就现代社会而言管理上的核心原则其实仍然是一致的,即上位要有上位的样子,下位要有下位的样子。各尽分工,各尽其职。

\section{大人}
周易谈论的大人肯定不是指官位上的大人,从周易常现的利见大人这几个字可以推断大人是指在本卦象上有能力帮助你的人。

\section{元亨利贞}
元更多的形容词含义,大的意思。比如“元亨”就是大亨通,“元吉”就是大吉祥。亨则是亨通的含义,并没有太大问题。利就是有利的意思。就是这个贞在解读上还是有点歧义的,比如有的地方是“小贞吉,大贞凶。”,有的地方是“利牝马之贞”。贞字在后的一般解读为贞正之德或贞正之道,这是没有问题的。然后我否定了认为贞就是占卜的意思这一说法,古人写文特别讲究言简意赅,本就在行占卜之事,若吉祥则最多用一个字,吉,而不会赘言贞吉。周易并没有一味要求人们去坚守贞正之道,在某些命运乖巧时运不济的时候,是不利于人们坚守贞正之道的,这是很实际的,同时在某些卦里面特别强调贞正之道的某些方面,比如“牝马之贞”或者“安贞”等。

那么贞字还有另外的含义吗,比如“小贞吉,大贞凶。”很多人都解释为小事吉祥,大事凶,又有人将贞字解释为占卜,我们现在假设小在古代还有小事的意思,那么他如果要表达小事吉祥,则只需要两个字小吉即可,而事实是小和大古今差异都不大,都是一个形容程度的词语。

要理解贞字最关键的是弄明白这个贞字原在人们中的头脑具象是何种情景,后面的名词动词形容词含义都是由这个头脑具象衍生出来的。我观察汉典上的字源字形,大体可以推测这更多的也是一个青铜器形象,比鼎小,考虑到真的形象也是类似的青铜器形象,而贞和蒸在读音上的接近,这些都不是偶然。贞字有些字形象会发现去掉了火,或者两个脚抬得很高,我可以推测这个青铜器后面渐渐成了一种祭祀相关用器,而且下面不再直接生火了。不管怎么说,我都不肯定贞字有占卜的含义在里面,唯一把贞字做占卜解释的那本书就是《周礼》,而这本书目前似乎人们已经断定它是成书于两汉之间,我们看到,在周易行文的时候,贞字基本上已经抽象为一种贞正的美德的含义了。所以我是赞同《周礼》这本书是伪托性质的,至少是成书在周易之后,然后作者因此误将贞字衍生出了占卜的意思。但我从贞字的形象是怎么看不到占卜一事的,至于说文解字将贞解作卜贝,又是希望将贞这个词的含义往占卜上靠,也可能是谬误的。

总的来说,我否认了贞字有占卜含义在里面。继而得出结论,贞作动词就是行贞正之道,作名词就是表示贞正之德。我不管贞这个容器在很久以前具体代表是什么礼仪或者祭祀事宜,这个实在难以考究了。但至少在周易成书之时,该字已经完全抽象化了,而至于认为贞为占卜意思的说法都不过是后人杜撰假想出来的。

那么什么是贞,为什么有的时候是利贞,有的时候贞凶呢?文天祥从容就义是完全了贞德,贞德简言之是坚持了人的正气。天地人以人为重为尊,然非常态。在某些形式下,天地太和之气不调,人坚守自己的正气不一定是吉利的。贞固然是值得夸奖的人的美德,但有时不合于天道,此亦非善也。

\section{利涉大川}
利涉大川,指当前形势利于克服艰难险阻。涉大川,指代如同跋涉大山大川一般困难的事情。利涉大川则当前靠自己的力量来做一件很困难的事是有希望成功的。


\section{皇极经世的时间刻度}
除去乾坤坎离四卦的先天六十四卦按照顺序每一卦管六运总共三百六十运。具体就是六十卦按照变爻从初爻变动到六爻分为六个阶段也就是这六个卦,这样三百六十运就分别对应了三百六十个卦,这个卦叫做值运之卦。一运360年。

在此值运之卦下,继续按照变爻从初爻变动到六爻分为六个阶段从而得到值世之卦,这个值世之卦管两世也就是六十年。

将这个值世之卦按照上面讨论的六十卦顺序从第一个值世之卦开始算起,一年一卦得值年之卦。



\part{附录}
\chapter{参考资料}
\begin{itemize}
\item \href{http://www.quanxue.cn/QT_XiaoYa/YiJingIndex.html}{劝学网小雅易经入门学习教程}
\item \href{http://www.guoyi360.com/zyqs/}{guoyi360国易堂网周易全解}
\item \href{http://www.xshiqi.com/category_zyzs/dgzs/gddy}{高岛断易}
\item \href{http://www.quanxue.cn/QT_MingXiang/ZhouYiZhuIndex.html}{王弼周易注}
\item \href{https://www.zdic.net/}{汉典}
\item \href{http://vsucai.cn/yizhuan/index.html}{国学经典网易传}
\item \href{https://ctext.org/book-of-changes/zhs}{中国哲学书电子化计划周易}
\end{itemize}








% 编者:万泽
\end{document}


