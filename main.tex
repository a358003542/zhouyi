% !Mode:: "TeX:UTF-8"

\documentclass[12pt,oneside]{book}

\usepackage[headchapter]{mybook} 
\usepackage{mybookcover}

% 六十四卦字体
\xeCJKDeclareSubCJKBlock{LIUSHISIGUA}{ "4DC0 -> "4DFF}
\setCJKmainfont[LIUSHISIGUA=DejaVu Sans]{Source Han Serif CN}


\title{周易初讲}
\author{Wander}
\hypersetup{
    pdftitle={周易初讲},
    pdfauthor={Wander},
    pdfcreator={Wander},
    pdfsubject={文学.哲学},
}
  

\begin{document}
\bookcover{book_cover}

\flypage{感谢上帝}


\frontmatter 
\addchtoc{序言}
\chapter*{序言}
本书对周易六十四卦进行了初步的讲解。


\addchtoc{目录}
\setcounter{tocdepth}{2}    
\tableofcontents




\mainmatter
\part{周易}
\chapter{乾卦}
\section{原文}
乾  {\Large ䷀}


\subsection{经}
乾:元亨,利贞。

初九:潜龙,勿用\footnote{唯宜潜藏,勿可施用。——康熙字典。}。

九二:见龙在田,利见大人。

九三:君子终日乾乾,夕惕若厉,无咎。

九四:或跃在渊,无咎。

九五:飞龙在天,利见大人。

上九:亢龙有悔。

用九:见群龙无首,吉。


\subsection{彖}
大哉乾元,万物资始,乃统天。云行雨施,品物\footnote{参看汉典“品物”词条,品物即万物。}流形。大明终始,六位时成,时乘六龙以御天。乾道变化,各正性命,保合大和,乃利贞。首出庶物,万国咸宁。

\subsection{象}
天行健,君子以自强不息。

初九:潜龙勿用,阳在下也。

九二:见龙在田,德施普也。

九三:终日乾乾,反复道也。

九四:或跃在渊,进无咎也。

九五:飞龙在天,大人造也。

上九:亢龙有悔,盈不可久也。

用九:用九,天德不可为首也。


\section{初讲}
卦辞说:乾卦,大亨通,利于贞正之道。

彖说:伟大啊,乾元,万物因此开始,于是统领天下。云气流行,雨水布施,万物运动变化而各具成形。从上到下对应万物的结束到开始,其都是阳爻,六爻应时而形成,时乘六龙来驾御天道。天道的变化,万物各个正定其本性和运命,保全太和之气,是故“利贞”。始出万物,万国皆得安宁。

象辞说:天行健,君子以自强不息。

初九:初九好比潜入地下之龙,不可施用。潜龙勿用,这是因为初九阳爻潜藏在下的缘故\footnote{此时你需要隐忍,还没有大展才干的时候。}。

九二:龙出现在了田野,适合去见到大人。龙在田不得位也。此利见大人是因为龙不得位,和九五的利见大人是有区别的,

九三:君子终日自强不息,到了晚上也不放松警惕如入危险之境,无咎。

九四:或腾跃而起,或退居深渊,无咎。

九五:飞龙在天,利见大人。九五的利见大人是因为你的飞龙在天是大人成就的【大人造也】。

上九:亢龙有悔。亢龙有悔,盈不可久也。

用九:用九是乾卦全部是阳爻,也即是乾卦本卦。一群龙却没有首领,吉祥。为什么吉祥呢?天生万物不居功不为首,是故吉祥。


\chapter{坤卦}
坤 {\Large ䷁}

\section{原文}
\subsection{经}
坤:元亨,利牝马之贞。君子有攸往,先迷后得主,利西南得朋,东北丧朋。安贞吉。

初六:履霜,坚冰至。

六二:直方大,不习无不利。

六三:含章可贞。或从王事,无成有终。

六四:括囊;无咎,无誉。

六五:黄裳,元吉。

上六:龙战于野,其血玄黄。

用六:利永贞。

\subsection{彖}
至哉坤元,万物资生,乃顺承天。坤厚载物,德合无疆。含弘光大,品物咸亨。牝马地类,行地无疆,柔顺利贞。君子攸行,先迷失道,后顺得常。西南得朋,乃与类行;东北丧朋,乃终有庆。安贞之吉,应地无疆。

\subsection{象}
地势坤,君子以厚德载物。

初六:履霜坚冰,阴始凝也。驯致其道,至坚冰也。

六二:六二之动,直以方也。不习无不利,地道光也。

六三:含章可贞;以时发也。或从王事,知光大也。

六四:括囊无咎,慎不害也。

六五:黄裳元吉,文在中也。

上六:龙战于野,其道穷也。

用六:用六永贞,以大终也。

\section{初讲}
卦辞说:坤卦,大亨通,利牝马之贞。君子有所前往,先迷而后得主。西南得朋有利,东北丧朋。安贞吉\footnote{安贞不可浅解为安于现状,牝马之贞指贞正之道偏坤德,安贞指贞正之道偏安守。}。

彖说:伟大啊,坤元。万物因你而生,顺承天道。坤用厚德载养万物,德性与天相合而无边无疆。包容博厚而广大,万物皆得亨通。牝马属于地上的生物,奔行于地而没有疆界,牝马性柔顺而利于贞正之道。君子有所前往,先迷失其道,后顺从于常道\footnote{即天道}。西南得到朋友,乃与之同行;东北失去朋友,最终有喜庆之事。安贞何以吉祥,应和地道的无边无疆。

象辞说:坤象征大地,君子应该效仿大地,胸怀宽广,包容万物。


初六:脚踩于霜而知气候变冷,坚冰将至。脚踩于霜,说明阴气开始凝结,照这个情况发展下去,必然迎来坚冰。本爻在提醒占者要见微知著,防微杜渐。

六二:直方大,此大地的形状。六二的行动如同大地的形状一样正直方正,即使不用修习也是无不利的。这是因为他将地道发扬光大了。

六三:胸怀美德和才华可以守正道,或许能够跟从君王做事,没有功名成就事情却有好的结果。

六四:将口袋扎紧,无咎亦无誉。此爻有守口如瓶之象,如此谨慎才得无害。

六五:黄色的下衣,吉祥。为何吉祥?温文之德在心中。

上六:与龙相战于旷野,龙流下了玄黄色的血。阴气极盛,地道也发展到头了。此爻提醒占者阴气过头复转为阳的道理。

用六:用六是坤卦全部是阴爻,也即是坤卦本卦。利于永远坚守正道,用六的永远坚守贞正之道,可用来得到大的善终。





\chapter{水雷屯(zhūn)卦}
屯 {\Large ䷂}

\section{原文}
\subsection{经}
屯:元亨,利贞,勿用有攸往,利建侯。

初九:磐桓;利居贞,利建侯。

六二:屯如邅(zhān)如,乘马班如。匪寇婚媾,女子贞不字,十年乃字。

六三:既鹿无虞,惟入于林中,君子几,不如舍,往吝。

六四:乘马班如,求婚媾,往吉,无不利。

九五:屯其膏,小贞吉,大贞凶。

上六:乘马班如,泣血涟如。

\subsection{彖}
屯,刚柔始交而难生,动乎险中,大亨贞。雷雨之动满盈,天造草昧,宜建侯而不宁。

\subsection{象}
云,雷,屯;君子以经纶。

初九:虽磐桓,志行正也。以贵下贱,大得民也。

六二:六二之难,乘刚也。十年乃字,反常也。

六三:既鹿无虞\footnote{虞:虞人,掌管山泽之官。},以从禽也。君子舍之,往吝穷也。

六四:求而往,明也。

九五:屯其膏,施未光也。

上六:泣血涟如,何可长也。

\section{初讲}
卦辞说:屯卦,大亨通,利于正道。不要有所前往,适宜建国立候。【屯卦还是很吉祥的,也利于君子去守正道,只是不要轻举妄动,适合做一些大事情的筹备工作。】

彖说:屯卦,阳刚之气和阴柔之气刚开始交汇,困难也随之而生,危险也在其中涌动,但终是大亨通,利贞的。雷雨之动满布大地,而上天的造化开物还处于蒙昧状态,适宜于建国立候,但不安宁。

象辞说:屯卦上面是云下面是雷,君子观此象而经纶天下。

【屯卦指事情才刚开始,寓意天地也是云雷互动而困难重重,此时不要轻举妄动,但也不要倦怠安宁,而应该效法此时云雷互动中上天经纬造化万物之象,立大志,定大计,定好大的方向和规划。经天纬地,如此安能不是大亨通和利贞的呢。】

初九:徘徊不前,利于安居的贞德,利于规划建侯般的伟业。虽然徘徊不前,但是其志得行其正。珍贵基层的平民百姓,会大大得到民心。

六二:困难啊难以前行啊,骑着马徘徊不前,不是匪寇是来商量嫁娶事宜的。女子坚守贞道不嫁,十年后才出嫁。六二之难,柔乘刚也。十年后才出嫁,回归正常罢了。

六三:追逐野鹿却没有山林之官作向导,只是跟着进入山林之中。君子明察之,不如舍弃吧,前往会有悔恨。即鹿无虞,只是跟着野兽跑。君子请舍弃吧,前往会有悔恨并陷入穷困。

六四:骑着马徘徊不前,去求婚,前往吉祥,无不利。为求婚而前往,是明智的选择。

九五:自我囤积财富,小贞吉,大贞凶。为什么会大贞凶呢?并没有将布施之德发扬光大。

上六:骑着马徘徊不前,悲泣不已,流泪如流血一样,涟涟不断。泣血涟如,这种状况怎么能长久呢。



\chapter{山水蒙卦}
蒙 {\Large ䷃}

\section{原文}
\subsection{经}
蒙:亨。 匪我求童蒙\footnote{童蒙:蒙昧的孩童。},童蒙求我。初筮告,再三渎,渎则不告。利贞。

初六:发蒙,利用刑人,用说\footnote{说,脱也。}桎梏,以往吝\footnote{参看重庆文理学院学报2006年1月出版的第五卷第1期田小中,“以往吝”衍文呢考,其认为这个以字只是一个衍文即多余的字,也就是这里应该解读为往吝。个人认为有一定的可信度。}。

九二:包蒙吉;纳妇吉;子克家。

六三:勿用娶女;见金夫,不有躬,无攸利。

六四:困蒙,吝。

六五:童蒙,吉。

上九:击蒙;不利为寇,利御寇。

\subsection{彖}
蒙,山下有险,险而止,蒙。蒙亨,以亨行时中也。匪我求童蒙,童蒙求我,志应也。初筮告,以刚中也。再三渎,渎则不告,渎蒙也。蒙以养正,圣功也。


\subsection{象}
山下出泉,蒙。君子以果行育德。

初六:利用刑人,以正法也。

九二:子克家,刚柔接也。

六三:勿用娶女,行不顺也。

六四:困蒙之吝,独远实也。

六五:童蒙之吉,顺以巽也。

上九:利用御寇,上下顺也。

\section{初讲}
卦辞说:蒙,亨通。不是我有求于童蒙,而是童蒙有求于我。【上天教育我们也好比教育童蒙一样】初次占筮就告诉,再三反复占筮就是亵渎了,亵渎则上天不会予以告知了。利贞。

彖说:蒙卦,上为艮为山为止,下为坎为水为险,故曰山下有险,险而止,这就是蒙卦了。蒙卦是亨通的,是因为他以亨道行动,得“时中”也。匪我求童蒙,童蒙求我,是因为童蒙有志于此。初筮告,是因为以刚正居于其中。再三渎,渎则不告,这个亵渎正是蒙昧的表现。将蒙昧培养入正道,这是圣人的功绩啊。

象辞说:山下冒出泉水,这就是蒙卦的卦象了。君子见此而知要以果敢的行动来培育良好的品德。

初六:启发蒙昧,利于用刑罚来约束人,让其将来免脱桎梏之苦。还在发蒙阶段贸然前往会有悔恨。利用刑人,是用来确立正确的法度。

九二:能够包容蒙昧,吉祥。娶媳妇吉祥,子女们能够持家了。子女们能够持家了,是因为刚爻与柔爻相接的缘故。

六三:不要娶这样的女人啊,她见到有钱的男人就委身于他,没什么好处。勿用娶女,是因为六三以柔乘九二之刚,没有顺从的美德\footnote{参考了\href{http://www.guoxuez.com/64gua/menggua/23431.html}{这个网页} 。}。

六四:困在蒙昧之中,悔恨啊。六四被困在蒙昧之中,是因为它这个柔爻独自远离于充实的阳爻。

六五:蒙昧的孩童,吉祥。童蒙之吉,是因为他能够对上面的阳爻也就是老师采用谦逊的态度来顺从他。

上九:上九击打蒙昧。不要把蒙昧当作匪寇一般击打,最好如艮山一般被动防御。利用御寇,是因为上下一心相互顺从的缘故。




\chapter{水天需卦}
需 {\Large ䷄}

\section{原文}

\subsection{经}
需:有孚,光亨,贞吉。利涉大川。

初九:需于郊。利用恒,无咎。

九二:需于沙。小有言,终吉。

九三:需于泥,致寇至。

六四:需于血,出自穴。

九五:需于酒食,贞吉。

上六:入于穴,有不速之客三人来,敬之终吉。

\subsection{彖}
需,须\footnote{此处须作等待之意。}也;险在前也。刚健而不陷,其义不困穷矣。需有孚,光亨,贞吉。位乎天位,以正中也。利涉大川,往有功也。

\subsection{象}
云上于天,需;君子以饮食宴乐。

初九:需于郊,不犯难行也。利用恒,无咎;未失常也。

九二:需于沙,衍\footnote{衍,宽衍,指内心宽宏大量。}在中也。虽小有言,以吉终也。

九三:需于泥,灾在外也。自我致寇,敬慎不败也。

六四:需于血,顺以听也。

九五:酒食贞吉,以中正也。

上六:不速之客来,敬之终吉。虽不当位,未大失也。

\section{初讲}
卦辞说:有诚信,光明亨通,贞吉。利涉大川。

彖说:需卦,等待的意思。需卦上为坎为险,下为乾为刚健,故危险在前方,但刚健而不陷,因天道正义遇险能通,而不会困穷。需有孚,光亨,贞吉。是因为九五爻位居天子位,居正而得中道。利涉大川是因为前往会有功绩。

象辞说:需卦上为坎为水,下为乾为天,水在天上称之为云,故曰需卦是云在天上之象。君子观此卦象,借饮食宴乐来积蓄精力,等待时机。


初九:等待于郊外,利于持之以恒,无咎。等待于郊外,没有犯难而行。利于恒无咎,未失天地常理也。

九二:等待于沙滩,小人有所言,终吉。等待于沙滩,内心宽宏大量,虽然小人有所言,但最终是吉祥的。

九三:等待于泥沼中,导致贼寇到来。等待于泥沼中,灾祸就在外面,是自我招来的贼寇,敬告人们要谨慎小心才不会陷入失败。

六四:等待于血泊\footnote{阴阳会战之地乃见血}中,刚刚从洞穴\footnote{坎为险为阴暗险陷之所。}中出来。此时六四身处杀伤之地,要柔顺听从九五的劝告方能脱险。

九五:等待于酒食中,贞吉。酒食贞吉是因为九五爻居中而得正的缘故。

上六:进入洞穴之中,有不速之客三人来,敬之,终吉。不速之客来,敬之终吉,三个不速之客,指内卦的三阳爻,象曰不当位,指上六入于穴,未大失也,仍是终吉之意。



\chapter{天水讼卦}
讼 {\Large ䷅}

\section{原文}
\subsection{经}
讼:有孚,窒惕\footnote{(zhì tì),参见汉典窒惕词条,为汉语词汇,意为恐惧。},中吉。终凶。利见大人,不利涉大川。

初六:不永所事,小有言,终吉。

九二:不克讼,归而逋\footnote{bū,逃亡。},其邑人三百户,无眚\footnote{灾祸}。

六三:食旧德,贞厉\footnote{厉,危也。},终吉,或从王事,无成。

九四:不克讼,复即命,渝\footnote{yú,改变,比如忠贞不渝。},安贞吉。

九五:讼元吉。

上九:或锡\footnote{xī,锡通赐}之鞶带\footnote{pán dài,古代官员的服饰},终朝三褫\footnote{chǐ,褫夺}之。

\subsection{彖}
讼,上刚下险,险而健,讼。讼有孚,窒惕中吉,刚来而得中也。终凶,讼不可成也。 利见大人,尚中正也。不利涉大川,入于渊也。

\subsection{象}
天与水违行,讼。君子以作事谋始。

初六:不永所事,讼不可长也。虽有小言,其辩明也。

九二:不克讼,归逋窜也。自下讼上,患至掇也。

六三:食旧德,从上吉也。

九四:复即命,渝,安贞,不失也。

九五:讼元吉,以中正也。

上九:以讼受服,亦不足敬也。


\section{初讲}
卦辞说:有诚信,内心恐惧,中吉,终凶。利见大人,不利涉大川。

彖说:讼卦上面为乾卦为刚,下面为坎卦为险。虽遇险但有刚健,这就是讼卦了。讼,有孚,窒惕,中吉,这是因为上面刚健的三爻来到坎险之地而居九二之中的缘故。终凶,争讼之事没有成功。利见大人,这是因为崇尚九五的中正之德。不利涉大川,这是因为入于坎险的深渊之地。

象辞说:坎卦上为天下为水,天从东向西转动\footnote{此处大概取太阳东升西落之意},水从西向东流,天水相背而行,这就是讼卦的卦象了。君子观此卦象而知道作事须谋始。

初六:不纠缠于争讼之事\footnote{此讼事很小,故曰事而不言讼。},虽然小人有所言论,终吉。不永所事,此争讼之事并不长久。虽有小小的责难之言,但通过辩解就能将是非曲直说明白了。

九二:不能胜讼,回来后又马上逃亡出去了。他所在村邑下有三百户人家,并没有受到牵连的灾祸。不克讼,是故回来后又马上逃窜中。以九二之下和上面有权有势的人争讼,灾祸是自己找的啊。

六三:享用旧有的德业,贞固自守是危险的,但最终会获得吉祥。可能会随君王作事,但并没有功名成就。食旧德,六三柔顺顺从上面的刚健三爻,还是能收获吉祥的。

九四:不能胜讼,继而复归到原来的命运状态,改变后,固守安贞而得吉祥。“复即命,渝,安贞”,并没有什么损失。

九五:争讼之事结果大吉祥。讼元吉,是因为九五的中正之德。

上九:可能会受到君王赐赏官服大带,一天之内却多次被褫夺。因为争讼而受到赏赐,这没什么值得尊敬的。



\chapter{地水师卦}
师 {\Large ䷆}

\section{原文}
\subsection{经}
师:贞,大人\footnote{有文作丈人,高岛断易认为此处应该按照子夏传所说的是大人吉,个人认为是正确的。假设是丈人则周易全文只有此处出现丈人一词,此其一也;再就是从师卦的含义来看,能够领众以做行军之事的,也必是一个大人了。}吉,无咎。

初六:师出以律,否臧凶。

九二:在师中,吉,无咎,王三锡命。

六三:师或舆尸\footnote{以车运尸},凶。

六四:师左次,无咎。

六五:田有禽,利执言,无咎。长子帅师,弟子舆尸,贞凶。

上六:大君有命,开国承家,小人勿用。

\subsection{彖}
师,众也,贞,正也,能以众正,可以王矣。刚中而应,行险而顺,以此毒\footnote{通“督”}天下,而民从之,吉又何咎矣。

\subsection{象}
地中有水,师;君子以容民畜众。

初六:师出以律,失律凶也。

九二:在师中吉,承天宠也。王三锡命,怀万邦也。

六三:师或舆尸,大无功也。

六四:左次无咎,未失常也。

六五:长子帅师,以中行也。弟子舆尸,使不当也。

上六:大君有命,以正功也。小人勿用,必乱邦也。

\section{初讲}
卦辞说:师,贞,大人吉,无咎。

彖说:师,众也,贞,正也。师卦下为坎为水为众,上为坤为地为众,故曰师为众;贞,守正道也。能够使众人皆守正道,可以王天下矣。九二刚爻居中而应六五,行于坎险之地而得坤顺。以此治理天下,而民众从之,吉又何咎矣。

象辞说:师卦上为坤为地,下为坎为水,地中有水,这就是师卦的卦象了。君子观此卦象而知容民畜众\footnote{畜养民众}。

初六:出师征战必须要有严明的纪律,否则就会蕴藏凶险。师出以律,失律凶也。

九二:九二在军中任统帅,吉,无咎,君王多次嘉奖。在师中吉,是因为承受六五君王的宠幸。王三锡命,是因为君王如坤胸怀万邦。

六三:出兵可能会有舆尸之败,凶险。师或舆尸,大败是也。

六四:率军撤退,无咎。左次无咎,不违背用兵有进有退的常理是也。

六五:田野里有野禽,利于发表言论,无咎。九二长子带兵出征,六三弟子以车载尸,贞凶。长子帅师,是因为九二以中正而行之,六三弟子舆尸,是六三这个弟子任用不当所致啊。

上六:六五君王有赏命于上六,开国承家\footnote{孔颖达疏:“若其功大,使之开国为诸侯;若其功小,使之承家为卿大夫。”},小人勿用。大君有命,以正功也。小人勿用,必乱邦也。



\chapter{水地比卦}
比 {\Large ䷇}

\section{原文}
\subsection{经}
比:吉。原筮,元\footnote{参见汉典元字解,元有始大的意思,此处应作始解。}永贞,无咎。不宁方来\footnote{参查汉典,此处方解为时间上的正或者将。},后夫凶。

初六:有孚比之,无咎。有孚盈缶\footnote{fǒu,古代一种盛酒的瓦器},终来有它吉。

六二:比之自内,贞吉。

六三:比之匪人。

六四:外比之,贞吉。

九五:显比,王用三驱,失前禽。 邑人不诫,吉。

上六:比之无首,凶。

\subsection{彖}
比,吉也,比,辅也,下顺从也。原筮,元永贞,无咎,以刚中也。不宁方来,上下应也。后夫凶,其道穷也。

\subsection{象}
地上有水,比。先王以建万国,亲诸侯。

初六:比之初六,有它吉也。

六二:比之自内,不自失也。

六三:比之匪人\footnote{个人认为王弼的注解是正确的,四自外比,二为五贞,近不相得,远则无应,所与比者,皆非已亲,故曰“比之匪人”。},不亦伤乎!

六四:外比于贤,以从上也。

九五:显比之吉,位正中也。舍逆取顺,失前禽也。邑人不诫,上使中也。

上六:比之无首,无所终也。

\section{初讲}
卦辞说:比卦,吉祥。原来就占卜过了的,从一开始就永守正道,无咎。因上为坎卦为险,故曰不安宁的事将要来到,后到的人有凶险啊。

彖说:比卦是吉祥的,比卦有辅佐的意思。下卦为坤为顺故曰下顺从也。原来就占卜过了的,从一开始就永守正道,无咎。这是因为九五以刚健而居中正之位,不安宁的事将要来到,因为上为坎为险所以不安宁的事将要到来,但有上下五阴爻和九五相应而亲比。后到的人有凶险啊,这是指上六爻亲比辅佐之道走到头了,竟然居于九五之上,以阴乘阳,傲慢而后到,自然凶险可知矣。

象辞说:比卦下为坤为地,上为坎为水,故曰地上有水,这就是比卦的卦象了。先王以封建万国来亲近诸侯。

初六:初六有诚信而亲比于九五,无咎。其诚信就好比美酒从容器中满盈而出,最终将有其他的吉祥来到。比之初六,有它吉也。

六二:比之自内,贞吉。六二爻乃内卦之主,故曰亲比来自内,比之自内,这是因为六二爻没有自我迷失正道啊。

六三:比之匪人\footnote{此处同意朱子语类,比之匪人,匪通非,无的意思,比之匪人即无人可比。},不亦伤乎。九五与六二内比,与六四外比,孤独六三却不是九五亲近的人,不是很令人悲伤吗。

六四:外比之,贞吉。下卦为内,上卦为外,六四爻向外亲比九五,贞吉。外比于九五贤君,来顺从尊上。

九五:九五彰显比卦之道。君王用三驱之礼狩猎,失去了前面的禽兽,封地上的人也不警戒,吉祥。显比之吉,是因为九五爻位居中正的缘故。不逆天而行而顺其自然,所以失掉前面的禽兽了。邑人不诫,这是因为九五君上以中正治国的缘故。

上六:上六爻比之无首,凶\footnote{六居上,比之终也。首谓始也,凡比之道,其始善则其终善矣。有其始而无其终者,或有矣,未有无其始而有终者也,故比之无首,至终则凶也。——程颐。上六爻即卦辞中所谈的后夫者也。}。比之无首,不得善终。



\chapter{风天小畜(xù)卦}
小畜\footnote{畜,止也,止则聚矣——程颐。} {\Large ䷈}

\section{原文}

\subsection{经}
小畜:亨。密云不雨,自\footnote{从或由}我西郊。

初九:复自道,何其咎,吉。

九二:牵\footnote{牵连的意思,参考了\href{http://baike.yidao5.com/jingzhuan/xiaoxugua/6428.shtml}{这个网页}。}复,吉。

九三:舆说\footnote{通脱}辐\footnote{轮中直木},夫妻反目。

六四:有孚,血\footnote{血者,恤也。可解释为忧愁远去,但血在周易中是有特殊含义的,此处即指九三和六四阴阳相斗。}去惕出,无咎。

九五:有孚挛\footnote{挛(luán),係(xì 通系,名为系物的绳子)也。此处指如同绳子系起来一样紧密相连的样子}如,富以其邻。

上九:既雨既处\footnote{既雨,和也。既处,止也。阴之畜阳,不和则不能止。既和而止,畜之道成矣。——程颐。},尚德载\footnote{载,积满也。——程颐。},妇贞厉。月几望,君子征凶

\subsection{彖}
小畜;柔得位,而上下应之,曰小畜。健而巽,刚中而志行,乃亨。密云不雨,尚\footnote{通上}往也。自我西郊,施未行也。

\subsection{象}
风行天上,小畜;君子以懿\footnote{君子所蕴畜者大,则道德经纶之业,小则文章才艺。君子观小畜之象,以懿美其文德。文德方之道义为小也。——程颐。}文德。

初九:复自道,其义吉也。

九二:牵复在中,亦不自失也。

九三:夫妻反目,不能正室也。

六四:有孚惕出,上合志也。

九五:有孚挛如,不独富也。

上九:既雨既处,德积载也。君子征凶,有所疑也。


\section{初讲}
卦辞说:小畜,亨通。浓云密布不下雨,从我西郊而来。

彖说:小畜,六四柔爻得位\footnote{柔爻居阴位},而上下诸阳爻与之相应,这就是小畜卦了。内卦为乾为刚健,外卦为巽为逊顺。九二九五阳刚之爻居中位而其志得行,是故亨通。密云不雨,此云向上行的缘故。自我西郊,雨施还未实行。

象辞说:小畜,上为巽为风,下为乾为天,风在天上行,这就是小畜的卦象了。君子应当懿美其文德\footnote{君子所蕴畜者大,则道德经纶之业,小则文章才艺。君子观小畜之象,以懿美其文德。文德方之道义为小也。——程颐。道德大而文德小,道德内而文德外。——执象易注。}。

初九:初九阳爻得位本欲进,但与六四相应而止,返回自己原先的道路,那里来的过错呢,吉祥。

九二:九二爻与初九爻\footnote{王弼解为和九五爻相牵连,个人赞同高岛断易的观点,是和初九爻相牵连。}相牵连而返回原处,得内卦之中位,[和初九爻一样]也不自失其道,吉祥。

九三:车脱其幅,夫妻反目\footnote{故不能前进,犹车舆说去轮辐,言不能行也。夫妻反目,阴制于阳者也。——程颐。}。这是因为九三阳爻被六四阴爻凌驾而止,丈夫不能端正家室的缘故。

六四:有诚信,[血斗的]忧愁远去,从惧怕中出来,无咎。这是因为九五君上和六四爻心志相合的缘故。

九五:九五爻[和六四爻]彼此有诚信,紧密相连的样子。九五爻让它的邻居[六四爻]更富有,这是因为九五爻不独富的缘故。

上九:雨已下也已经停了,这是尊尚小畜之德积满而成是也。妇若贞固自守则有危厉\footnote{ 妇贞厉,妇谓阴,以阴而畜阳,以柔而制刚,妇若贞固守,此危厉之道也。 安有妇制其夫,臣制其君,而能安者乎?——程颐。}\footnote{此似在谈论九三被六四凌驾而止,上九若不管不问则岂能自安。},月亮快到望日的时候,君子征动则有凶险。君子征凶,有所疑虑啊\footnote{阴将盛极,君子动则有凶也。阴敌阳则必消阳,小人抗君子则必害君子,安得不疑虑乎?——程颐。}\footnote{虽小畜之道已成,至阴极之时,君子尤当戒之勿动。}。


\chapter{天泽履(lǚ)卦}
履 {\Large ䷉}

\section{原文}

\subsection{经}
履:履虎尾,不咥\footnote{dié,咬}人,亨。

初九:素履往,无咎。

九二:履道坦坦\footnote{道之平也},幽人贞吉。

六三:眇\footnote{眇,一目小也——《说文》}能视,跛能履,履虎尾,咥人,凶。武人为于大君。

九四:履虎尾,愬愬\footnote{shuò,恐惧的样子},终吉。

九五:夬\footnote{guài,决也}履,贞厉。

上九:视履考祥,其旋\footnote{旋,回首,三省吾身之谓也。——《执象易注》}元吉。

\subsection{彖}
履,柔履刚也。说而应乎乾,是以履虎尾,不咥人,亨。刚中正,履帝位而不疚,光明也。

\subsection{象}
上天下泽,履;君子以辩上下,定民志。

初九:素履之往,独行愿也。

九二:幽人贞吉,中不自乱也。

六三:眇能视;不足以有明也。跛能履;不足以与行也。咥人之凶;位不当也。武人为于大君;志刚也。

九四:愬愬终吉,志行也。

九五:夬履贞厉,位正当也。

上九:元吉在上,大有庆也。

\section{初讲}
卦辞说:履卦,走路踩到老虎尾巴,老虎却不咬人,亨通。

彖说:履卦,下为兑为柔弱,上为乾为刚健,故曰以柔履刚。下卦为兑为悦,上卦为乾为天,心悦诚服地应对上天\footnote{王弼在此处作六三柔爻讲,不是很认同,因为从周易全文行文风格思路来看这里多以上下卦来分析的。},所以就算踩到老虎尾巴,老虎也不咬人。亨通。九五刚正而居中,登上帝位而不愧疚,这是因为内心光明的缘故。

象辞说:履卦上为乾为天,下为兑为泽,这就是履卦的卦象了。君子观此卦象来分辨上下之别,安定人民之志\footnote{夫上下之分明,然后民志有定,民志定,然后可以言治。——程颐。}。

初九:穿着朴素的鞋\footnote{此处素履直义是朴素的鞋,但也有衍生之义即素行。}前往,无咎。素履之往,独行己之愿是也。

九二:行进之路坦坦,幽居之人贞吉。幽人贞吉,这是九二爻居中位而不自乱其操行的缘故也。

六三:眼睛有病眇之勉强能看,腿脚有病跛之勉强能走路。踩到老虎尾巴,老虎咬人,凶。好武之人要做大君。眇能视;不足以有明也。跛能履;不足以与行也。咥人之凶;位不当也\footnote{此处指六三爻以阴爻居阳位,不得位。}。武人\footnote{此武人色厉内荏,内为阴爻,外为爻变的阳。}为于大君;志刚也\footnote{六三爻为履卦唯一的阴爻,本不得位,再爻变为阳爻,意欲刚强统领其他诸阳爻,这本是九五大君该做的。}。

九四:踩到老虎尾巴,很是恐惧的样子,最终会吉祥。愬愬终吉,这是九四爻其志得行的缘故。

九五:九五爻刚决而行,贞厉。夬履贞厉,这是因为九五爻得位而且尊位,如果刚决而行,则会失之专断,这是很危险的,故曰贞厉。

上九:审视履行的道路,考察祸福吉凶之预兆,这样的反身自省,大吉祥。上九的元吉,是因为履道大成,必有庆也。



\chapter{地天泰卦}
泰 {\Large ䷊}

\section{原文}

\subsection{经}
泰\footnote{泰者,通也。——《序卦》}:小往大来,吉,亨。

初九:拔茅茹\footnote{茅之为物,拔其根而相牵引者也。“茹”,相牵引之貌也。——王弼},以其汇\footnote{汇者,类也。},征吉。

九二:包荒,用冯河\footnote{不敢暴虎,不敢冯河——《诗·小雅》,暴虎冯河是一个成语,空手打虎,徒步渡河,喻冒险蛮干。},不遐遗。朋亡\footnote{无偏无私,不朋党。},得尚于中行\footnote{得尚于中行,言能配合中行之义也。尚,配也。——程颐。}。

九三:无平不陂\footnote{bēi,斜坡},无往不复,艰贞无咎。勿恤其孚,于食有福。

六四:翩翩不富,以其邻,不戒以孚\footnote{六四处泰之过中,以阴在上,志在下复,上二阴亦志在趋下。翩翩,疾飞之貌,四翩翩就下,与其邻同也。邻,其类也,谓五与上。夫人富而其类从者,为利也,不富而从者,其志同也。三阴皆在下之物,居上乃失其实,其志皆欲下行,故不富而相从,不待戒告而诚意相合也。——程颐}。

六五:帝乙归妹,以祉\footnote{zhǐ,福也。——《说文》}元吉。

上六:城复\footnote{通覆}于隍\footnote{隍者,城壕也,无水为隍,有水为池——高岛断易},勿用师。自邑告命,贞吝。

\subsection{彖}
泰,小往大来,吉,亨。则是天地交而万物通也;上下交而其志同也。内阳而外阴,内健而外顺,内君子而外小人,君子道长,小人道消也。

\subsection{象}
天地交,泰,后\footnote{后,继君体也——《说文》}以财\footnote{通裁}成天地之道,辅相天地之宜\footnote{财成,谓体天地交泰之道,而财制成其施为之方也。辅相天地之宜,天地通泰则万物茂遂。人君体之而为法制。使民用天时,因地利,辅助化育之功,成其丰美之利也。——程颐},以左右\footnote{治理也}民。

初九:拔茅征吉,志在外也。

九二:包荒,得尚于中行,以光大也。

九三:无往不复,天地际也。

六四:翩翩不富,皆失实也。不戒以孚,中心愿也。

六五:以祉元吉,中以行愿也。

上六:城复于隍,其命乱也。

\section{初讲}
卦辞说:泰卦,小往大来\footnote{阳爻代表阳气为大,阴爻代表阴气为小,阴气都离去到了外卦,阳气都来到了内卦,故曰小往大来。},吉祥,亨通。

彖说:泰卦,小往大来,吉祥,亨通。于是天地阴阳之气交感而万物通泰。君民上下结交然后心志相同。内卦为乾为阳,外卦为坤为阴,内刚健而外柔顺,内卦为君子而外卦为小人,君子之道盛长,小人之道消亡是也\footnote{是故吉祥而亨通。}。

象辞说:天地阴阳之气交感,这就是泰卦的卦象了。君王当效仿此天地通泰之道来成立体制,继而辅之以天地化育之功宜,以此来治理万民。

初九:就好像拔茅草一样根连根,初九爻这个贤士还连着他的同类九二九三贤士,征吉\footnote{泰卦征吉,否卦贞吉,征吉是主动出击行动吉祥,而贞吉是固守现状吉祥。}。拔茅征吉,是因为初九爻其志向是在外发展。

九二:包容八荒,以徒步渡河之勇,来无所遐弃偏远之才。无偏无私,来配合九二爻的中正而行之义。包荒,得尚于中行,来光明正大泰道【九二主泰】。

九三:没有那个平路走着走着不遇到斜坡的,没有谁一直前往前往而不返回的,艰贞无咎。不要担忧九三爻的诚信,他在饮食上是有福的。无往不复,是因为九三爻处天地交接之际,当明白往复屈伸之理。

六四:六四爻翩翩向下而飞,有六五爻和上六爻这两个阴爻与之为邻,他们不相戒备而诚信相待。翩翩不富是因为他们都是阴爻皆失实也,不戒以孚,是因为他们皆中心愿意。

六五:帝乙嫁出自己的妹妹【给九二】,因此得到福祉,大吉。以祉元吉,是因为六五爻居中而能行其愿是也。

上六:城墙倾覆于城壕中,不要用兵,上六爻自己在城邑中宣布君命,贞吝。城复于隍,泰道将灭,上下不交,其命将乱也。


\chapter{天地否(pǐ)卦}
否 {\Large ䷋}

\section{原文}

\subsection{经}
否\footnote{否,隔也——《广雅》闭塞不通之意。}:否之匪人\footnote{天地不交则不生万物,是无人道,故曰匪人,谓非人道也。——程颐},不利君子贞,大往小来。

初六:拔茅茹,以其汇,贞吉亨。

六二:包承。小人吉,大人否亨。

六三:包羞。

九四:有命无咎,畴\footnote{畴,通俦,同类。——执象易注}离祉\footnote{离,依附。祉,福祉。}。

九五:休否,大人吉。其亡其亡,系于苞桑。

上九:倾否,先否后喜。

\subsection{彖}
否之匪人,不利君子贞。大往小来,则是天地不交,而万物不通也;上下不交,而天下无邦也。内阴而外阳,内柔而外刚,内小人而外君子。小人道长,君子道消也。

\subsection{象}
天地不交,否;君子以俭德辟难,不可荣以禄。

初六:拔茅贞吉,志在君也。

六二:大人否亨,不乱群也。

六三:包羞,位不当也。

九四:有命无咎,志行也。

九五:大人之吉,位正当也。

上九:否终则倾,何可长也。


\section{初讲}
卦辞说:否卦非人道,不利君子贞。大往小来\footnote{可参考泰卦小往大来的解释。阳气都离去到了外卦,阴气都来到了内卦。}。

彖说:否卦非人道,不利君子贞。大往小来,于是天地阴阳之气不相交感而万物闭塞不通。君民上下不结交而天下无邦。内卦为坤为阴,外卦为乾为阳,内柔顺而外刚健,内卦为小人而外卦为君子,小人之道盛长,君子之道消亡是也。

象辞说:天地阴阳之气不相交感,这就是否卦的卦象了。君子当以节俭为美德,避免祸难,不可荣居厚禄之位。

初六:就好像拔茅草一样根连根,初六爻这个小人还连着他的同类六二和六三,初六爻固守现状则吉祥亨通。拔茅贞吉,是因为初六爻此时还是有忠君爱国之心的。

六二:六二小人包容初六顺承六三,小人吉祥,大人固守否道,则亨通。大人否亨,是因为大人没去扰乱内卦三爻的小人之群。

六三:六三小人包容六二初六的可羞之事,是因为六三爻自身居位不当的缘故。

九四:九四受九五[休否]之命,无咎,其和九五上九同类相互依附,共享福祉。有命无咎,是因为九四爻其志得行的缘故\footnote{乾行刚健}。

九五:九五爻休止否运,大人吉祥。否运将要亡了,否运将要亡了,却又好像系于苞桑之上,否运根深蒂固,怎么也亡不了。大人之吉,是因为九五爻得位而居中的缘故。

上九:否运倾覆,先否后喜。否终则倾,怎么可能长久不变呢。


\chapter{天火同人卦}
同人 {\Large ䷌}

\section{原文}

\subsection{经}
同人\footnote{即同仁,志同道合之人。}于野,亨。利涉大川,利君子贞。

初九:同人于门,无咎。

六二:同人于宗,吝。

九三:伏戎于莽,升其高陵,三岁不兴。

九四:乘其墉,弗克攻,吉。

九五:同人,先号咷\footnote{同“号啕大哭”的啕}而后笑。大师克相遇。

上九:同人于郊,无悔。

\subsection{彖}
同人,柔得位得中,而应乎乾,曰同人。同人曰:“同人于野,亨。利涉大川,”乾行也。文明以健,中正而应,君子正也。唯君子为能通天下之志。

\subsection{象}
天与火,同人;君子以类族辨物。

初九:出门同人,又谁咎也。

六二:同人于宗,吝道也。

九三:伏戎于莽,敌刚也。三岁不兴,安\footnote{安,辞也——王弼,语气辞。}行也。

九四:乘其墉,义弗克也,其吉,则困而反则也\footnote{以不克困苦而反归其法则,故得吉也。——孔颖达}。

九五:同人之先,以中直也。大师相遇,言相克也。

上九:同人于郊,志未得也。

\section{初讲}
卦辞说:在旷野与人亲同,亨通。利涉大川,利君子贞。

彖说:同人卦,六二柔爻得位又得中,而与外卦乾卦之主九五爻相应,称之为同人。同人卦说:“同人于野,亨。利涉大川,”。【为何呢】,因为乾卦刚健而善行。【为何利君子贞呢】因为内卦离卦文明而外卦乾卦刚健,二五爻皆中正\footnote{居中得位}而相应,君子之正道如是也。唯君子能通达天下人的心志。

象辞说:上为乾卦为天,下为离卦为火,上天下火,这就是同人的卦象了。君子观此卦象而知道族以类似之,物以辩别之的道理。【亦即对人求同存异,对物辨别分明;类族辨物,君子做人格物的道理就在其中了。】

初九:刚出门即与人亲同,无咎。出门同人,谁会去罪责他呢。【反观六二的过于吝啬只同人于宗族之中,谁又能指责他呢。此爻有近人友爱之象。】

六二:只与九五本宗族人亲同,吝。同人于宗,这是吝啬可鄙之道。

九三:埋伏兵戎于草莽之中,等上高陵,三年不敢兴兵打仗。伏戎于莽,[九三爻与九五爻敌对而欲强行亲同六二]是因为九五爻很是刚健。三岁不兴,这怎能行得通啊。

九四:九四爻登上九五爻的城墙欲攻打之,最终没有进攻,吉祥。乘其墉,按照道义是不能攻打的。其吉祥,是因为九四爻陷入困苦后而能反归其正当法则。

九五:九五亲同六二[六二,同人之主],【因有九三九四阻拦】,先号啕大哭而后欢笑。大军克敌制胜才得相遇啊。同人之先,以中直也。大师相遇,言相克也。九五同人之前所以号啕大哭者,是因为九五爻中正而直率。大师相遇,是说九五爻终能克敌制胜而和六二爻相遇。

上九:在郊野与人亲同,无悔\footnote{郊者,外之极也。处“同人”之时,最在于外,不获同志,而远于内争,故虽无悔吝,亦未得其志。——王弼}。同人于郊,志未得也\footnote{此处高岛断易结合卦辞对比郊野和旷野和通天下人之志来做解释,个人觉得解之过大。}。


\chapter{火天大有卦}
大有 {\Large ䷍}

\section{原文}

\subsection{经}
大有:元亨。

初九:无交害,匪咎,艰则无咎。

九二:大车以载,有攸往,无咎。

九三:公用亨\footnote{亨,享也。参考高岛断易。}于天子,小人弗克\footnote{克,能也。参考执象易注。}。

九四:匪其彭\footnote{“彭”者,盛多貌。参考高岛断易。},无咎。

六五:厥\footnote{厥,其也。参考执象易注。}孚交如,威如;吉。

上九:自天祐之,吉无不利。

\subsection{彖}
大有,柔得尊位,大中而上下应之,曰大有。其德刚健而文明,应乎天而时行,是以元亨。

\subsection{象}
火在天上,大有;君子以遏恶扬善,顺天休\footnote{休,美也。君子观大有之象,以遏绝众恶,扬明善类,以奉顺天休美之命。——程颐。}命。

初九:大有初九,无交害也。

九二:大车以载,积中不败也。

九三:公用亨于天子,小人害也。

九四:匪其彭,无咎;明辨晢\footnote{皙,明智也。参考执象易注。}也。

六五:厥孚交如,信以发志也。威如之吉,易而无备也。

上九:大有上吉,自天祐也。


\section{初讲}
卦辞说:大有,大亨通。

彖说:大有,六五阴柔爻得尊位,居中为大而上下诸阳爻与之相应,故曰大有【五刚之大,皆为尊位所有\footnote{参考高岛断易}。】。内卦乾卦刚健,外卦离卦文明,故曰其德性刚健而文明;德应于天道,行合乎时宜,时行也本就是顺应天道,六五人君有此德性,是以元亨。

象辞说:上为离为火,下为乾为天,火在天上,这就是大有卦的卦象了。君子观此卦象以遏恶扬善来顺天美命。

初九:大有初九,并未涉害。若要没有灾难,则艰难自守可无咎矣。大有初九,并未涉害是也。

九二:九二爻为六五所用,大车以载,其才足以重任。有所前往,无咎。总的情况是任重不危。大有九二所积用大车装载于其中,所积于中才不会败坏是也,故而无咎。

九三:九三王公爻享用六五天子爻的宴请款待,小人则不能这样。小人得到重用将会成为祸害。【此在告诫九三爻所居位置重要,要做正人君子,不要做小人。】

九四:不是九四爻的盛大多有,无咎。九四爻虽处盛地然非己之盛,只要明智地明辨这一点则无咎矣。

六五:六五爻诚信交往的样子,威严的样子,吉祥啊。厥孚交如,信以发志。【上能诚信待下而下有协助之志\footnote{参见高岛断易}】威严的样子是吉祥是六五若无威严则会为人所易慢,无戒备之心\footnote{参考《程传》}。

上九:上九爻居大有之世之极,自有天佑,吉无不利。



\chapter{地山谦卦}
谦 {\Large ䷎}

\section{原文}

\subsection{经}
谦:亨,君子有终。

初六:谦谦君子,用涉大川,吉。

六二:鸣\footnote{鸣者,声名闻之谓也。——王弼}谦,贞吉。

九三:劳谦,君子有终,吉。

六四:无不利,撝\footnote{huī,挥也。}谦。

六五:不富,以其邻,利用侵伐,无不利。

上六:鸣谦,利用行师,征邑国。

\subsection{彖}
谦,亨,天道下济\footnote{利泽下施,长养万物。——汉典}而光明,地道卑而上行。天道亏盈而益谦,地道变\footnote{倾变}盈而流\footnote{流聚}谦,鬼神害盈而福谦,人道恶盈而好谦。谦尊而光,卑而不可逾,君子之终也。

\subsection{象}
地中有山,谦;君子以裒\footnote{(póu)减少。“多者用谦以为裒,少者用谦以为益,随物而与,施不失平也。”——王弼}多益寡,称物平施。

初六:谦谦君子,卑以自牧\footnote{牧,养也。——王弼}也。

六二:鸣谦贞吉,中心得也。

九三:劳谦君子,万民服也。

六四:无不利,撝谦;不违则也。

六五:利用侵伐,征不服也。

上六:鸣谦,志未得也。可用行师,征邑国也。

\section{初讲}
卦辞说:谦,亨通,君子有善终\footnote{礼记曰:“君子曰终,小人曰死”,又有“老曰终,少曰死”。大体终字在古代除了结局终了之意思外还有褒义的意思,所以此处君子有终,大体就是君子有善终之意。}。

彖说:谦,亨通,天之道下施而万物光明,地之道卑下而上行。天道亏损盈者而增益谦者,地道倾变盈者而流聚谦者,鬼神祸害盈者而福佑谦者,人道厌恶盈者而喜好谦者。【天地鬼神人,有益谦,有流谦,有福谦,有好谦。简言之,天地鬼神人皆对谦者善。】故而谦虚能够让尊者更光荣,让卑者不可逾越,这就是卦辞所言君子有善终的原因。

象辞说:谦卦上为坤为地,下为艮为山,故曰地中有山,如山一般厚重而静静地待在大地之下,谦虚啊。君子观谦卦的卦象而知道减有余而补不足,称量财物平均施舍于人的道理。

初六:初六这个谦虚而又谦虚的君子,用此谦道来跋涉大川,吉祥。谦谦君子,初六这个谦谦君子于卑下之处自养谦德。

六二:六二谦德声名在外,贞吉。鸣谦贞吉,是因为六二爻[居中而得位],内心有谦德自然所得。

九三:九三爻[谦卦之主]劳苦功高而又谦虚,君子有善终,吉祥。劳谦君子,大家都会心悦诚服于他。

六四:无不利,挥扬谦德。六四爻挥扬自己的谦德吧,无不利,只要不违背法则。

六五:六五爻并不富足【阴爻】,因居于尊位而用以其邻【六四爻和上六爻】,利于用来侵伐,无不利。利用侵伐,是因为征伐的是骄逆不服之徒。

上六:上六爻谦德名声在外,利于用来兴兵打仗,征伐邑国。上六爻虽然鸣谦,但因为居谦卦之极,反而内心谦德之志并未得遂\footnote{不同于六二的中心得也。}。可用行师,征邑国\footnote{邑国,诸侯的封地,征邑国,自治也。}也。【当时之时可以兴兵打仗,来治理好自己的治下封地。】


\chapter{雷地豫卦}
豫 {\Large ䷏}

\section{原文}

\subsection{经}
豫\footnote{豫,乐也。——尔雅;豫,喜豫说乐之貌也。——周易郑注。}:利建侯行师。

初六:鸣豫,凶。

六二:介\footnote{介,独立操守,确乎其不可拔之谓,孟子所谓“不以三公易其介”者是也。——执象易注。}于石,不终日,贞吉。

六三:盱\footnote{(xū),张目上望之状。}豫,悔。迟有悔。

九四:由豫,大有得。勿疑。朋盍\footnote{盍,合也——王弼}簪\footnote{(zān)簪,疾也。——王弼。}。

六五:贞疾,恒不死。

上六:冥\footnote{冥,昏也——执象易注。}豫,成\footnote{成,终也——执象易注。}有渝\footnote{渝,变也——执象易注。},无咎。

\subsection{彖}
豫,刚应而志行,顺以动,豫。豫顺以动,故天地如\footnote{如,从也。}之,而况建侯行师乎?天地以顺动,故日月不过\footnote{过,失度。},而四时不忒\footnote{忒,差错。};圣人以顺动,则刑罚清而民服。豫之时义大矣哉!

\subsection{象}
雷出地奋\footnote{震动。奋,动也——广雅。},豫。先王以作乐崇德,殷\footnote{殷,盛也——执象易注。}荐\footnote{进献}之上帝,以配\footnote{配享}祖考\footnote{祖先}。

初六:初六鸣豫,志穷凶也。

六二:不终日,贞吉;以中正也。

六三:盱豫有悔,位不当也。

九四:由豫,大有得;志大行也。

六五:六五贞疾\footnote{贞疾,常病,痼疾。——高岛断易},乘刚也。恒不死,中未亡也。

上六:冥豫在上,何可长也。

\section{初讲}
卦辞说:豫卦,利建候行师。

彖说:豫卦,九四刚爻为众阴爻响应而其志得行,豫卦下为坤为顺,上为震为动,[众阴爻]顺而[九四爻]动之,这就是豫卦了。悦豫顺从地行动,所以连天地都会如此,则何况建候行师[有所不顺]乎?天地以之顺动之道,所以日月运行不失其度,而四时更替没有差错;圣人以之顺动之道,则刑罚清而万民服。豫卦运作之时意义重大啊。

象辞说:豫卦上为震卦为雷,下为坤卦为地,雷在地上震动,这就是豫卦的卦象了。先王通过制作音乐来崇尚道德,以盛大的仪式进献给上帝,以及配享祖先。

初六:初六爻安享悦豫之初,与九四相应而自鸣得意,凶险。初六鸣豫,其没什么志向必遭凶险。

六二:独立操守其介如石,见几而作不俟终日\footnote{言君子见事之几微,则须动作而应之,不得待终其日,言赴机之速也。——孔颖达},贞吉。不终日,贞吉;是因为六二爻居中而得位,有中正之德。

六三:六三爻张目仰视九四,想要豫悦九四,[但九四爻见其不得位而鄙弃之,是以]有悔,迟疑不决则必生悔恨。盱豫有悔,是因为六三爻以柔居阳位,位不得当的缘故。

九四:由于九四爻而来的豫悦\footnote{豫之所以为豫者,由九四也,为动之主,动而众阴悦顺,为豫之义。——程颐。},大有得益。不要疑虑,朋友会很快合聚。由豫,大有得;是因为九四爻其志得以大行的缘故。

六五:六五爻患有痼疾,久病而不死。六五贞疾,是因为六五爻以柔乘[九四]刚的缘故。恒不死,是因为六五爻居中而处尊位,未可得亡\footnote{四以刚动为豫之主,专权执制,非已所乘,故不敢与四争权,而又居中处尊,未可得亡,是以必常至于“贞疾,恒不死”而已。——王弼}。

上六:上六爻昏聩豫乐,终有变,无咎。冥豫在上,何可长也\footnote{豫极则变,人终也须变而应对之。}。



\chapter{泽雷随卦}
随 {\Large ䷐}

\section{原文}

\subsection{经}
随:元亨,利贞,无咎。

初九:官有渝,贞吉。出门交有功。

六二:系\footnote{系,牵挂。刚爻能自立曰随,柔爻不足自立曰系——高岛断易。}小子,失丈夫。

六三:系丈夫,失小子。随有求得,利居贞。

九四:随有获,贞凶。有孚在道,以明,何咎。

九五:孚于嘉,吉。

上六:拘系\footnote{拘禁,管束。——汉典。}之,乃从维\footnote{绳两系称维——虞翻。}之,王\footnote{周文王}用亨\footnote{通享,享祭。}于西山\footnote{岐山}。

\subsection{彖}
随,刚来而下柔,动而说,随。大亨贞,无咎,而天下随时,随时之义大矣哉!

\subsection{象}
泽中有雷,随;君子以向\footnote{向,接近。}晦\footnote{晦,夜晚。}入宴息\footnote{宴息,休息。}。

初九:官有渝,从正吉也。出门交有功,不失也。

六二:系小子,弗兼与\footnote{交往——汉典。}也。

六三:系丈夫,志舍下也。

九四:随有获,其义凶也。有孚在道,明功也。

九五:孚于嘉吉,位正中也。

上六:拘系之,上穷也。


\section{初讲}
卦辞说:随卦,大亨通,利于正道,无咎。

彖说:随卦,初九刚爻来到内卦,来到柔爻之下\footnote{此刚来而下柔众说纷纭,刚爻指初九这是确定的,柔有说是上六,但从爻辞来看上六并不是随卦之主,故我舍弃了这一说法。注意这里是下柔而不是柔下。}。下为震为动,上为兑为悦,行动并可喜悦,这就是随卦了。大亨通,[利于]正道,无咎,而天下万物皆随时而动,随时而动的含义很重大啊!

象辞说:随卦上为兑为泽,下为震为雷,泽中有雷,这就是随卦的卦象了。君子应当到了快晚上的时候就入室休息\footnote{君子观象以随时而动。随时之宜,万事皆然,取其最明且近者言之。君子以向晦入宴息:君子昼则自强不息,及向昏晦,则入居于内,宴息以安其身,起居随时,适其宜也。——程颐。}。

初九:初九爻主官之事有所变动,贞吉,出门与人交往有功。官有渝,其所从得正则吉祥。出门交有功,因为其所随不失正也。

六二:六二爻【联】系于初九这个小子就会失去九五这个大丈夫。系小子,是不能同时两个一起交往的。【随小随大当有所抉择。】

六三:六三爻【联】系于九四这个大丈夫,就会失去初九这个小子。随有所求则有所得,利居贞\footnote{虽然,固不可非理枉道以随于上,苟取爱说以遂所求。如此,乃小人邪谄趋利之为也,故云利居贞。自处于正,则所谓有求而必得者,乃正事君子之随也。}。系丈夫,六三爻其志舍下初九爻而不从也。

九四:九四爻被人随从,有所收获,贞凶。有诚信于正道,以明其功,何咎\footnote{既能著信在于正道,是明立其功,故无咎也。——孔颖达。}。随有获,其义凶也\footnote{九四以阳刚之才,处臣位之极,若于随有获,则虽正亦凶。有或,谓得天下之心隨于己。为臣之道,当使恩威一出于上,众心皆隨于君。若人心从己,危疑之道也,故凶。——程颐。}。有孚在道,以明其功。[则无咎矣。]

九五:九五有诚信随于嘉善,吉祥。孚于嘉吉,是因为九五爻位得正而居中\footnote{位正者德必正,时中者道必中。——执象易注。}。

上六:对上六拘系之,又从而维系之\footnote{拘系之,谓隨之极,如拘持縻系之。乃从维之,又从而维系之也,谓隨之固結如此。——程颐。},[但上六还是会如同周文王一样],用享祭祀于岐山之上\footnote{昔者太王用此道,亨王业于西山。太王避狄之难,去豳来岐,豳人老稚扶携以隨之如归市,——程颐。}。拘系之,是因为上六随道穷极将变。


\chapter{山风蛊(gǔ)卦}
蛊 {\Large ䷑}

\section{原文}

\subsection{经}
蛊\footnote{蛊,事也。……蛊之义,坏乱也。——序卦。},元亨\footnote{如卦之才以治蛊,则能致元亨。——程颐。},利涉大川。先甲三日,后甲三日\footnote{甲,数之首,事之始也,——程颐。古多言三日为多日的意思,不要强行解释先甲三日为甲日之前三日。}。

初六:干\footnote{干,干预,治理。}父之蛊,有子,考\footnote{生曰父,死曰考。——礼记。}无咎,厉终吉。

九二:干母之蛊,不可贞。

九三:干父之蛊,小有悔,无大咎。

六四:裕\footnote{宽容}父之蛊,往见吝。

六五:干父之蛊,用誉。

上九:不事\footnote{侍奉}王侯,高尚其事。

\subsection{彖}
蛊,刚上而柔下\footnote{蛊卦为刚上而柔下,咸卦为柔上而刚下,恒卦为刚上而柔下,分析得出结论,彖辞中这种上下的描述是专门针对三阳爻三阴爻的卦的,上指上卦,下指下卦。当说刚上的时候则必然要求上卦原为三阴爻,然后某个刚爻来到了上卦;当说柔下的时候则必然要求下卦原为三阳爻,然后某个柔爻来到下卦。比如这里是指上九刚爻来到上卦,初六柔爻来到下卦。},巽而止,蛊。蛊,元亨而天下治也。利涉大川,往有事也。先甲三日,后甲三日,终则有始,天行也。

\subsection{象}
山下有风,蛊。君子以振民育德\footnote{君子观有事之象,以振济于民,养育其德也。——程颐。}。

初六:干父之蛊,意承考也。

九二:干母之蛊,得中道也。

九三:干父之蛊,终无咎也。

六四:裕父之蛊,往未得也。

六五:干父用誉,承\footnote{蒙受。注意此处不要解释为继承,这里语境上并没有出现考这个字。}以德也。

上九:不事王侯,志可则也。

\section{初讲}
卦辞说:蛊卦,大亨通,利涉大川。先甲三日,后甲三日。

彖说:蛊卦,上九刚爻来到上卦,初六柔爻来到下卦,上卦为巽卦为巽顺,下卦为艮卦为停止,巽顺地停止,这就是蛊卦了。蛊卦,大亨通继而天下大治是也。利涉大川,前往有所事也。先甲三日成蛊之始,后甲三日治蛊之终\footnote{先甲三日后甲三日如何解释众说纷纭,这里大致取程颐的说法“甲者事之首”的意思。},终则必有始\footnote{夫有始则必有终,既终则必有始,天之道也。……先甲后甲而为之虑,所以能治蛊而致元亨也。——程颐。},天道运行如是也。

象辞说:蛊卦上为艮为山,下为巽为风,山下有风这就是蛊卦的卦象了。君子以振济于民来养育己德。

初六:初六治理父辈的蛊事,有子如此\footnote{蛊事之始,积弊未深,初六能堪其任。},则父辈无咎矣,有危险但最终会吉祥。干父之蛊,是因为初六的意思是继承父辈的事业。

九二:九二治理母辈的蛊事,不可固执守正。干母之蛊,得是刚柔适中之道是也\footnote{居于内中,宜干母事,故曰“干母之蛊”也。妇人之性难可全正,宜屈已刚。既干且顺,故曰“不可贞”也。干不失中,得中道也。——王弼。}。

九三:九三治理父辈的蛊事,小有悔,无大咎。干父之蛊,终无咎也\footnote{以刚干事,而无其应,故“有悔”也。履得其位,以正干父,虽“小有悔”,终无大咎。——王弼。}。

六四:六四宽容父辈的蛊事,前往则见吝\footnote{吝,恨惜也。——说文。}。裕父之蛊,前往没有什么所得\footnote{以四之才守常,居宽裕之时,则可矣。欲有所往,则未得也。加其所任,则不胜矣。——程颐。}。

六五:六五治理父辈的蛊事,用荣誉来激励九二这个刚阳之臣。干父用誉,九二贤臣以刚中之德承受之。

上九:上九不侍奉王侯,高尚其事。不事王侯,其志可为人们学习的法则也。



\chapter{地泽临卦}
临 {\Large ䷒}

\section{原文}

\subsection{经}
临:元亨,利贞。至于八月有凶。

初九:咸\footnote{咸,感也。}临,贞吉。

九二:咸临,吉,无不利。

六三:甘\footnote{甘者,佞邪说媚不正之名也。——王弼。}临,无攸利。既忧之,无咎。

六四:至临,无咎。

六五:知\footnote{知,智也。——执象易注。}临,大君之宜,吉。

上六:敦临,吉,无咎。

\subsection{彖}
临,刚浸\footnote{浸,渐也。——程颐。}而长,说而顺,刚中而应,大亨以正,天之道也。至于八月有凶,消不久也。

\subsection{象}
泽上有地,临。君子以教思无穷,容保民无疆。

初九:咸临贞吉,志行正也。

九二:咸临吉无不利,未顺命也。

六三:甘临,位不当也。既忧之,咎不长也。

六四:至临无咎,位当也。

六五:大君之宜,行中之谓也。

上六:敦临之吉,志在内也。

\section{初讲}
卦辞说:临卦,大亨通,利正道。到了八月会有凶险。

彖说:临卦,阳刚之气渐长,下为兑为悦上为坤为顺,喜悦而和顺,九二刚爻居中而与六五相应,大亨通以利于正道,天道如是也\footnote{此处对应乾卦的元亨利贞。}。到了八月会有凶险,因为阳气刚被阴气消尽不久\footnote{以一年十二月则一爻对应两月,故到八月底,阳气刚被阴气消尽。} \footnote{八月阳衰而阴长,小人道长,君子道消也,故曰有凶。——王弼。}。

象辞说:临卦下为兑为泽,上为坤为地,泽上有地,这就是临卦的卦象了。君子应当教思无穷,容保民无疆。

初九:初九与六四相感而临下,贞吉。咸临贞吉,是因为初九爻志在必行而行六四之正\footnote{初九阳爻志在必行,六四得正位,行正行六四之正者也。} \footnote{四履正位,而己应焉,志行正者也。——王弼。}。

九二:九二与六五相感而临下,吉祥,无不利。咸临吉无不利,是因为九二并未顺[六五君上之]命是也\footnote{若顺于五,则刚德不长,何由得“吉无不利”乎?全与相违,则失于感应,其得“咸临,吉无不利”,必未顺命也。——王弼。}。

六三:六三以佞媚邪悦之态来临下,无所利。既忧之,无咎。甘临,因为六三爻其位不得当是也\footnote{六三为兑卦之主,以悦示人,但不得位,又以柔乘二刚,故有甘临一说。}。既忧之,则咎不长矣\footnote{若能尽忧其危,改修其道,刚不害正,故咎不长。——王弼。}。

六四:六四亲至而临下,无咎。至临无咎,六四爻其位得当是也\footnote{居近君之位,为得其任,以阴处四,为得其正,与初相应,为下贤,所以无咎,盖由位之当也。——程颐。}。

六五:六五君上知人善任\footnote{知,智也。此处单解为知人善任,择其重点是也。}[九二]而临下,大君之道宜当如此,吉祥。大君之宜,行其中德说的也是这个\footnote{五有中德,故能倚任刚中之贤,成大君之宜,成知临之功,盖由行其中德也。——程颐。}。

上六:上六以敦厚临下\footnote{上六坤之极,顺之至也,而居临之终,敦厚于临也。与初二虽非正应,然大率阴求于阳,又其至顺,故志在从乎二阳,尊而应卑,高而从下,尊贤取善,敦厚之至也,故曰敦临。……六居临之终,而不取极义,临无过极,故止为厚义。——程颐。},吉祥,无咎。敦临之吉,是因为上六之志是要应乎内卦的初九和九二是也。



\chapter{风地观卦}
观 {\Large ䷓}

\section{原文}

\subsection{经}
观:盥\footnote{(guàn),洗手——汉典。此处指祭祀中的洗手仪式环节。}而不荐\footnote{荐,祭祀中的奉献酒食等祭品的仪式环节。},有孚颙\footnote{(yóng),温和肃敬的样子——汉典。}若。

初六:童观,小人无咎,君子吝。

六二:窥观,利女贞。

六三:观我生,进退。

六四:观国之光,利用宾于王。

九五:观我生,君子无咎。

上九:观其生\footnote{观我生,自观其道也。观其生,为民所观者也。——王弼。},君子无咎。

\subsection{彖}
大观在上,顺而巽,中正以观天下。观,盥而不荐,有孚颙若,下观而化也。观天之神道,而四时不忒,圣人以神道设教,而天下服矣。

\subsection{象}
风行地上,观。先王以省方,观民设教。

初六:初六童观,小人道也。

六二:窥观女贞,亦可丑也。

六三:观我生进退,未失道也。

六四:观国之光,尚\footnote{尚,谓志尚【词语,即志向】,其志意愿慕宾于王朝也。——程颐。}宾也。

九五:观我生,观民也。

上九:观其生,志未平也。

\section{初讲}
卦辞说:祭祀进行洗手仪式却没有奉献祭品,有诚心,温和肃敬的样子。【心诚最重要】

彖说:二阳爻为大,其观在上,观卦下为坤卦为柔顺,上为巽卦为巽顺。柔顺而又巽顺,九五爻居中正之位以观天下。观卦,祭祀进行洗手仪式却没有奉献祭品,有诚心,温和肃敬的样子,下面的人观之而受教化。观天之神道,四时运行没有差错,圣人以神道设教,天下之人莫不顺服。

象辞说:观卦上为巽为风,下为坤为地,风行在地上,这就是观卦的卦象。先王省察四方,观察民情并设立教化。

初六:初六观之稚嫩若童子观之\footnote{六以阴柔之质,居远于阳,是以观见者浅近如童稚然,故曰「童观」。——程颐。},小人无咎,君子吝\footnote{小人下民也,所见昏浅,不能识君子之道,乃常分也。不足谓之过咎,若君子而如是,则可鄙吝也。——程颐。}。初六童观,这是小人之道是也。【君子当戒远之】

六二:六二爻窥视观之\footnote{处在于内,无所鉴见。体性柔弱,从顺而已。犹有应焉【应于九五】,不为全蒙,所见者狭,故曰窥观。——王弼。},利女贞\footnote{利于女子的顺从贞正之道。}。窥观女贞,也是可羞丑的\footnote{处大观之时,居中得位,不能大观广鉴,窥观而已,诚可丑也。——王弼。}。【不能大观九五中正之道,仅仅窥视观之,观之颇为狭隘,虽有女子的顺从之德,亦可羞丑也。】

六三:六三爻观察自己一生的所作所为,来决定接下来的进退。未失道也\footnote{处进退之时,以观进退之几,未失道也。——王弼。}。【本爻更多是强调自我观察,自己要有自己的判断和主见。】

六四:六四爻观见国之盛德光辉\footnote{当观天下之政化,则人君之道德可见矣。——程颐。},利于利用起来作为九五君王的宾客\footnote{古者有贤德之人,则人君宾礼之,故仕进于王朝则谓之宾。——程颐。}。观国之光,六四的志向是愿意宾于王朝的\footnote{既观见国之盛德光华,古人所谓非常之遇也,所以志愿登进王朝,以行其道,故云「观国之光,尚宾也」。——程颐。}。

九五:九五爻观察自己一生的所作所为,君子之风盛行则九五君王无咎矣。观我生,观民也\footnote{我生出于己者,人君欲观已之施为善否,當观于民。民俗善则政化善也。王弼云:观民以察己之道是也。——程颐。}。

上九:上九爻为民所观其一生的所作所为,君子之风盛行则上九无咎矣。观其生,[上九虽已不在其位但其]志未得平息\footnote{[上九]不可以不在于位,故安然放意无所事也,是其至意未得安也。——程颐。}。


\chapter{火雷噬嗑(shìhé)卦}
噬嗑 {\Large ䷔}

\section{原文}

\subsection{经}
噬嗑\footnote{噬,啮也;嗑,合也。——王弼。}:亨。利用\footnote{治理,管理。比如荀子有:仁人之用国,将修志意,正身行。}狱。

初九:屦\footnote{(jù),通履。}校\footnote{校,木囚也。——说文。}灭趾,无咎。

六二:噬肤灭鼻,无咎。

六三:噬腊肉,遇毒。小吝,无咎。

九四:噬干胏\footnote{(zǐ),有骨的干肉。——汉典。},得金矢,利艰贞,吉。

六五:噬干肉,得黄金,贞厉,无咎。

上九:何\footnote{通荷,}校灭耳,凶。

\subsection{彖}
颐\footnote{颐,颌也。——方言十。颌:构成口腔上部和下部的骨头与肌肉等组织——汉典。}中有物,曰噬嗑,噬嗑而亨。刚柔分,动而明,雷电合而章。柔得中而上行,虽不当位,利用狱也。

\subsection{象}
雷电噬嗑。先王以明罚敕\footnote{通饬,告诫之意。现多用整饬法令一说。}法。

初九:屦校灭趾,不行也。

六二:噬肤灭鼻,乘刚也。

六三:遇毒,位不当也。

九四:利艰贞吉,未光也。

六五:贞厉无咎,得当也。

上九:何校灭耳,聪不明也。

\section{初讲}
卦辞说:噬嗑卦,亨通,利于治理刑狱之事\footnote{天下之间,非刑狱何以去之,不云利用刑,而云利用狱者。卦有明照之象,利于察狱也。狱者,所以究察情伪,得其情则知为间之道,然后可以设防与致刑也。——程颐。}。

彖说:颐中有物,曰噬嗑\footnote{颐中有物,啮而合之,噬嗑之义也。——王弼。}。噬嗑而亨通\footnote{有物间于颐中则为害,噬而嗑之则其害亡,乃亨通也,故云噬嗑而亨。——程颐。}。噬嗑卦刚爻和柔爻相间分开而不相杂\footnote{刚爻与柔爻相间,刚柔分而不相杂,为明辨之象,明辨察狱之本也。——程颐。},下为震为动,上为离为明,故曰动而明,下为震为雷,上为离为电,雷电相合而有章法\footnote{刚柔分动,不溷【混的异体字。溷,浊也。——广雅】乃明,雷电并合,不乱乃章,皆利用狱之义。——王弼。}。六五柔爻得中,居上尊位而行之\footnote{按照程颐的说法,上卦为离卦的噬嗑卦,晋卦,睽卦,鼎卦都有上行一词,上行言居尊位之意,王弼也谈到此处上行言所之在贵者也。},虽不当位,利于治理刑狱之事\footnote{治狱之道,全刚则伤于严暴,过柔则失于宽纵。五为用狱之主,以柔处刚而得中,得用狱之宜也。——程颐。}。

象辞说:噬嗑卦下为震为雷,上为离为电,雷电相合,这就是噬嗑卦的卦象了。先王以明察刑罚,整饬法令之。

初九:初九爻足戴枷械,遮没其脚趾\footnote{居无位之地以处刑初,受刑而非治刑者也。凡过之所始,必始于微,而后至于著。罚之所始,必始于薄,而后至于诛。过轻戮薄,故屦校灭趾,桎其行也。足惩而已,故不重也。——王弼。},无咎\footnote{过而不改,乃谓之过。小惩大诫,乃得其福,故无咎也。——王弼。}。屦校灭趾,使初九不得再行恶是也\footnote{古人制刑,有小罪则校其趾,盖取禁止其行,使不进于恶也。——程颐。}。

六二:六二爻治理刑狱之事用力过猛,如同噬肤\footnote{肤者,柔脆之物也。——王弼。二居中得正,是用刑得其中正也。用刑得其中正,则罪恶者易服,故取噬肤为象。——程颐。}却遮没了自己的鼻子,无咎。噬肤灭鼻,是因为六二爻乘[初九]刚爻的缘故\footnote{乘刚,乃用刑于刚强之人,不得不深严也,深严则得宜,乃所谓中也。——程颐。}。

六三:六三爻治理刑狱之事如同噬腊肉,遇毒\footnote{六居三处不当位,自处不得其当而刑于人,则不服而怨怼,悖犯之如噬啮干腊坚韧之物,而遇毒恶之味,反伤于口也。——程颐。腊以喻不服,毒以喻怨生——王弼。}。小有吝惜,无咎。遇毒,是因为六三爻位不得当的缘故。

九四:九四爻治理刑狱之事,如同噬干胏\footnote{胏,肉之有联骨者,干肉而兼骨,至坚难噬者也。——程颐。},得金矢,利艰贞,吉\footnote{噬至坚而得金矢,金取刚,矢取直。九四阳德刚直,为得刚直之道,虽用刚直之道,利在克艰其事,而贞固其守则吉也。——程颐。}。利艰贞吉,其道仍有未光大之处\footnote{虽体阳爻,为阴之主,履不获中,而居其非位,以斯噬物,物亦不服,故曰噬乾胏也。——王弼。}。

六五:六五爻治理刑狱之事,如同噬干肉,得黄金\footnote{干肉,坚也。黄,中也。金,刚也。以阴处阳,以柔乘刚,以噬于物,物亦不服,故曰:噬乾肉也。——王弼。五居中为得中道,处刚而[九]四辅以刚,得黄金也。——程颐。},贞厉,无咎\footnote{贞厉无咎,六五虽处中刚,然实柔体,故戒以必正固而怀危厉,则得无咎也。——程颐。}。贞厉无咎,是因为六五爻所为得当的缘故\footnote{用[九四]刚而能[自]守[中]正虑危也。——程颐。}。

上九:上九爻背负枷械遮没耳朵\footnote{处罚之极,恶积不改者也。——王弼。},凶险。何校灭耳,其聪不明是也\footnote{聪不明,故不虑恶积,至于不可解也。——王弼。}。



\chapter{山火贲(bì)卦}
贲 {\Large ䷕}

\section{原文}
\subsection{经}
贲\footnote{贲,饰也。——说文。}:亨,小利有攸往。

初九:贲其趾,舍车而徒。

六二:贲其须。

九三:贲如\footnote{如,辞助也。——程颐。}濡\footnote{(rú),润泽。}如,永贞吉。

六四:贲如皤\footnote{(pó),皤,老人白也。——说文。}如,白马翰\footnote{翰:高飞也。有一说是白色,但意思重复了,应不是此解。}如,匪寇婚媾。

六五:贲于丘园,束帛\footnote{束帛:古代用为聘问、馈赠的礼物。}戋戋\footnote{(jiān), 浅少之意。}。吝,终吉。

上九:白贲,无咎。

\subsection{彖}
贲亨。柔来而文\footnote{物相杂,故曰文。——系辞。}刚,故亨。分刚上而文柔,故小利有攸往。刚柔交错\footnote{刚柔交错一句王弼、郭京等人均认为应补之,同意。},天文也。文明以止,人文也。观乎天文,以察时变。观乎人文,以化成天下。

\subsection{象}
山下有火,贲。君子以明庶政\footnote{庶政:各种政务。},无敢折狱\footnote{折狱:即断狱。}。

初九:舍车而徒,义弗乘也。

六二:贲其须,与上兴也。

九三:永贞之吉,终莫之陵\footnote{通凌,侮也。}也。

六四:六四当位,疑也。匪寇婚媾,终无尤也。

六五:六五之吉,有喜也。

上九:白贲无咎,上得志也。


\section{初讲}
卦辞说:贲卦,亨通,小利于有所前往。

彖说:贲卦亨通,六二柔爻来到内卦而与初九和九三两刚爻相文错,是故亨通。上九刚爻分居上位来与六四六五两柔爻相文错,是故小利于有所前往\footnote{此处程颐谈到“八卦重而为六十四卦,皆由乾坤之变也”。}。刚柔相文错,天文是也。贲卦下为离为文明,上为艮为止,有文明而知止,人文是也\footnote{止物不以威武而以文明,人之文也。——王弼。}。观乎天文,以察四时之变。观乎人文,以教化天下。

象辞说:贲卦上为艮为山,下为离为火,山下有火,这就是贲卦的卦象了。君子应当让政务清明,不敢断狱\footnote{处贲之时,止物以文明,不可以威刑——王弼。}。【当明察秋毫如山火,断狱结案之时当见山而知止。】


初九:初九君子贲饰其脚趾\footnote{趾,取在下而所以行也。君子修饰之道,正其所行——程颐。},舍弃乘车而徒步行走。舍车而徒,是因为按照道义不应该乘车而行\footnote{在贲之始,以刚处下,居於无位——王弼。}。【初九无位,指位不高非不得位。其与六四爻相正应,故应是投奔六四而行走之。】

六二:六二爻贲饰其胡须\footnote{须之为物,上附者也。循其所履以附于上,故曰贲其须也。——王弼。}\footnote{卦之为贲,虽由两爻之变,而文明之义为重,二,实贲之主也,故主言贲之道,饰于物者,不能大变其质也。因其质而加饰耳,故取须义。须隨颐而动者也。——程颐。}。贲其须,六二爻与上面的九三爻同兴是也。【六二得位无应,九三得位亦无应,故六二九三彼此相比而同兴。】

九三:九三爻贲饰如同濡润一般\footnote{九三爻为六二六四相文而贲饰,若濡润其中一般。},永远坚守正道吉祥\footnote{三与二四非正应,相比而成相贲,故戒以常永贞正。贲者,饰也。贲饰之事,难乎常也,故永贞则吉。——程颐。}。永贞之吉,则必有好结果,谁也不能凌侮它。

六四:六四爻贲饰如同没有贲饰一般\footnote{四与初为正应,相贲者也,当贲如而为三所隔,故不获相贲,而皤如。皤,白也,未获贲也。——程颐。},白马要高飞的样子\footnote{其从正应之志如飞,故云翰如。——程颐。},不是强盗是要进行婚姻之事。六四当位,[为九三所]疑也。匪寇婚媾,终得相贲而无怨尤也。

六五:六五求贲饰于丘园\footnote{丘园,谓在外而近者,指上九也。——程颐。},礼微物薄。吝,终吉\footnote{六五虽居君位,而阴柔之才不足自守,与上之刚阳相比,而志从焉,获贲于外比之贤,贲于丘园也。若能受贲于上九,受其裁制,如求帛而戋戋,则虽其柔弱不能自为,为可吝少。然能从于人,成贲之功,终获其吉也。——程颐。}。六五之吉,有喜也。【程颐之解甚高,我关于同一事诚心卜卦而得比卦,实与上九相比而贲饰之意也。】

上九:上九白贲,无咎\footnote{上九贲之极也。贲饰之极,则失于华伪,惟能质白其贲,则无过失之咎。——程颐。}。白贲无咎,上九虽在外亦得其志也\footnote{上九为得志者,在上而文柔,成贲之功,六五之君又受其贲,故虽居无位之地,而实尸【此处应为担任,承担之意。】贲之功,为得志也,与他卦居极者异矣。既在上得志,处贲之极,将有华伪失实之咎,故戒以质素,则无咎。饰不可过也。——程颐。}。



\chapter{山地剥卦}
剥 {\Large ䷖}

\section{原文}
\subsection{经}
剥,不利有攸往。

初六:剥床以足,蔑\footnote{通灭,古文蔑灭通用。——yidao5。}贞凶。

六二:剥床以辨\footnote{床足之上,床身之下,分辨处也。——康熙字典。},蔑贞凶。

六三:剥之,无咎。

六四:剥床以肤,凶。

六五:贯鱼\footnote{贯鱼谓此众阴也,骈头相次,似贯鱼也。——王弼。},以宫人宠,无不利。

上九:硕果不食,君子得舆\footnote{车中装载东西的部分,后泛指车。},小人剥庐\footnote{庐,舍也。——广雅。}。

\subsection{彖}
剥,剥也,柔变刚也。不利有攸往,小人长也。 顺而止之,观象也。君子尚消息盈虚\footnote{指事物的盛衰变化或行为的出处进退。——汉典。},天行也。

\subsection{象}
山附地上,剥。上以厚下安宅。

初六:剥床以足,以灭下也。

六二:剥床以辨,未有与也。

六三:剥之无咎,失上下也。

六四:剥床以肤,切近灾也。

六五:以宫人宠,终无尤也。

上九:君子得舆,民所载也。小人剥庐,终不可用也。

\section{初讲}
卦辞说:剥卦,不利于有所前往。

彖说:剥卦,剥落的意思,阴柔变更阳刚是也\footnote{夏至一阴生而渐长,一阴长则一阳消,至于建戌则极而成剥,是阴柔变刚阳也。——程颐。}。不利于有所前往,是因此小人之道正盛长。顺时而止,乃能观剥之象也。君子应当崇尚消息盈虚之理,顺天而行之\footnote{君子存心消息盈虚之理,而能顺之,乃合乎天行也。理有消衰,有息长,有盈满,有虚损,顺之则吉,逆之则凶,君子随时敦尚,所以事天也。——程颐。}。

象辞说:剥卦上为艮为山,下为坤为地,山附于地,这就是剥卦的卦象了。上位者应当知道厚下安宅的道理\footnote{上谓人君与居人上者,观剥之象而厚固其下,以安其居也。下者上之本,未有基本固而能剥者也,故上之剥必自下,下剥则上危矣。为人上者知理之如是,则安养人民,以厚其本,乃所以安其居也。《书》曰:民惟邦本,本固邦宁。——程颐。}。


初六:初六剥之始若剥床之足,灭正而致凶\footnote{阴之剥阳,自下而上,以床为象者,取身之所处也。自下而剥,渐至于身也。剥床以足,剥床之足也。剥始自下,故为剥足。阴自下进,渐消蔑于贞正,凶之道也。——程颐。}。剥床以足,以之消灭下本是也。

六二:六二剥若剥床以辨,灭正而致凶\footnote{辨分隔上下者,牀之干也。阴渐进而上剥至于辨,愈蔑于正也,凶益甚矣。——程颐。}。剥床以辨,未有相与者也\footnote{阴之侵剥于阳,得以益盛,至于剥辨者,以阳未有应与故也。小人侵剥君子,若君子有与,则可以胜小人,不能为害矣。唯其无与,所以被蔑而凶,当消剥之时,而无徒与岂能自存也。言未有与,剥之未盛,有与犹可胜也,示人之意深矣。——程颐。}。

六三:六三剥之,无咎。剥之无咎,是因为六三与上下四阴爻相失于剥之道\footnote{三居剥而无咎者,其所处与上下诸阴不同,是与其同类相失于处剥之道,为无咎。}。【各阴爻剥落阳爻,独六三与上九相应,相应而彼此相互成就,但六三不得位,又处于群阴中,其势孤弱,不言吉,只得无咎。】

六四:六四剥床以肤,凶\footnote{始剥于床足,渐至于肤。肤,身之外也。将灭其身矣,其凶可知。阴长已盛,阳剥已甚,贞道已消,故更不言蔑贞,直言凶也。——程颐。}。剥床以肤,切近灾也\footnote{五为君位,剥已及四。在人则剥其肤矣。剥及其肤,身垂于亡矣,切近于灾祸也。——程颐。}。

六五:六五为剥之主,引领众阴若贯鱼,若能如宫人获宠[于上九],无不利。以宫人宠,终无过尤也\footnote{五能使群阴顺序如贯鱼,然反获宠爱于在上之阳,如宫人则无所不利也。宫人,宫中之人,妻妾侍使也。以阴言,且取获宠爱之义。以一阳在上,众阴有顺从之道,故发此义。——程颐。}。

上九:上九独阳若硕果不食,行君子之道则得车舆,行小人之道则将剥庐。君子得舆,车舆之上有民众所载。小人剥庐,小人之道终不可用也。【剥极则复,小人之道剥庐,无所容身之处,终不可用小人之道矣。】


\chapter{地雷复卦}
复 {\Large ䷗}

\section{原文}
\subsection{经}
复:亨。出入无疾,朋来无咎。反复其道,七日来复,利有攸往。

初九:不远复,无祗悔\footnote{(dī),此处同意程颐的解释:祗,宜音袛,抵也。《玉篇》云:适也。义亦同。无祗悔,不至于悔也。 },元吉。

六二:休复,吉。

六三:频复,厉,无咎。

六四:中行独复。

六五:敦复,无悔。

上六:迷复,凶。有灾眚,用行师,终有大败,以其国君,凶。至于十年不克征。

\subsection{彖}
复亨。刚反,动而以顺行,是以出入无疾,朋来无咎。反复其道,七日来复,天行也。利有攸往,刚长也。复其见天地之心乎?

\subsection{象}
雷在地中,复。先王以至日闭关,商旅不行,后\footnote{后,继君体也。——说文。此处后即君王之意。}不省方。

初九:不远之复,以修身也。

六二:休复之吉,以下仁\footnote{上下相亲谓之仁。——礼记。}也。

六三:频复之厉,义无咎也。

六四:中行独复,以从道也。

六五:敦复无悔,中以自考也。

上六:迷复之凶,反君道也。

\section{初讲}
卦辞说:复卦,亨通。出入无疾病,朋友来了言行也无过失。阴阳消长反复之道,七日为一来复,利于有所前往。

彖说:复卦亨通,初九刚爻复返于初,下为震为动,上为坤为顺,震动而以坤顺之德行之,是以出入无疾,朋来无咎\footnote{复生于内,入也。长进于外,出也。先云出,语顺耳。阳生非自外也,来于内,故谓之入。——程颐。}\footnote{坤得朋也,又有坤顺之德,故朋来无咎也。}。反复其道,七日来复,天地运行如是也。利于有所前往,阳刚君子之道盛长,复卦其内君可见天地之心乎\footnote{其道反复往来,迭消迭息,七日而来复者,天地之运行如是也。消长相因,天之理也,阳刚君子之道长,故利有攸往,一阳复于下,乃天地生物之心也。先儒皆以静为见天地之心,盖不知动之端乃天地之心也,非知道者孰能识之。——程颐。}?

象辞说:复卦下为震为雷,上为坤为地,雷在地中,这就是复卦的卦象了。先王于冬至日闭关,商旅不得通行,君王也不省察四方\footnote{先王顺天道,当至日,阳之始生,安静以养之,故闭关使商旅不得行,人君不省视四方,观复之象,而顺天道也。在一人之身亦然,当安静以养其阳。——程颐。}。

初九:初九爻不远而复返,不至于有悔,大吉。不远之复,以修身也\footnote{不远而复者,君子所以修其身之道也。学问之道无他也,唯知不善则速改以从善而已。——程颐。}。

六二:六二爻休美之复,吉祥。休复之吉,是因为其能下亲于初九\footnote{为复之休美而吉者,以其能下仁也。仁者,天下之公,善之本也。初复于仁,二能亲而下之,是以吉也。——程颐。}。

六三:六三爻频复频失,危险,无咎。虽有频复频失之厉,但复善则义无咎也\footnote{三以阴躁动,处动之极。复之频数而不能固者也,复贵安固,频复频失,不安于复也。复善而屡失,危之道也。圣人开迁善之道与其复,而危其屡失,故云「厉,无咎」。不可以频失,而戒其复也。频失则为危。屡复何咎过?在失而不在复也。——程颐。}。

六四:六四与初九正应,行于群阴之中而独自复于正道。中行独复,以从于阳刚君子之善道也\footnote{群阴之中而独能复,自处于正,下应于阳刚,其志可谓善矣。不言吉凶者,盖四以柔居群阴之间,初方甚微,不足以相援,无可济之理,故圣人但称其能独复,而不欲言其独从道而必凶也。曰:然则不言无咎,何也?曰:以阴居阴,柔弱之甚,虽有从阳之志,终不克济,非无咎也。——程颐。}。

六五:六五爻敦厚地复于正道,无悔。敦复无悔,居中而自我考查是也\footnote{六五以中顺之德处君位,能敦笃于复善者也,故无悔。虽本善,戒亦在其中矣。阳复方微之时,以柔居尊,下复无助,未能致亨吉也,能无悔而已。——程颐。}\footnote{此处自考程颐似乎解为自成,大体可以解为自考而成之意,故此处简单解为自我考查。}。

上六:上六爻迷而不复,凶险,有灾眚\footnote{有灾眚,灾,天灾自外来;眚,己过由自作。既迷不复善,在己则动皆过失,灾祸亦自外而至,盖所招也。——程颐。},用之行师出兵,终有大败,以它为国君,凶险。以至于十年都不能征战胜利。迷复之凶,以其违反为君之道是也。


\chapter{天雷无妄卦}
无妄 {\Large ䷘}

\section{原文}
\subsection{经}
无妄,元亨,利贞。其匪正,有眚\footnote{眚,过也,灾也。——广韵。},不利有攸往。

初九:无妄往,吉。

六二:不耕获,不菑畲\footnote{菑畲(zī shē),菑,初耕的田地。畲,三岁治田也——说文。田一岁曰菑,三岁曰畬。——程颐。},则利有攸往。

六三:无妄之灾,或系之牛,行人之得,邑人之灾。

九四:可贞,无咎。

九五:无妄之疾,勿药,有喜。

上九:无妄行,有眚,无攸利。

\subsection{彖}
无妄,刚自外来而为主于内。动而健,刚中而应,大亨以正,天之命也。其匪正有眚,不利有攸往。无妄之往,何之矣。天命不祐,行矣哉。

\subsection{象}
天下雷行,物与\footnote{此处物与当为一词,程颐解与为赋予之意,有宋张载之西铭曰:民吾同胞,物吾与也。后有民胞物与一成语,意思是泛爱一切人和物。但这些都是后面的,王弼将与字解为辞,大概皆的意思。此处我决定简单解为万物之意。}无妄,先王以茂对时育万物。

初九:无妄之往,得志也。

六二:不耕获,未富也。

六三:行人得牛,邑人灾也。

九四:可贞无咎,固有之也。

九五:无妄之药,不可试也。

上九:无妄之行,穷之灾也。

\section{初讲}
卦辞说:无妄卦,大吉祥,利正道。若其所行不正,则有灾眚,不利于有所前往\footnote{无妄者至诚也,至诚者天之道也。天之化育万物,生生不穷,各正其性命,乃无妄也。人能合无妄之道,则所谓与天地合其德也。无妄有大亨之理,君子行无妄之道,则可以致大亨矣。无妄,天之道也,卦言人由无妄之道也。利贞,法无妄之道,利在贞正,失贞正则妄也。虽无邪心,苟不合正理,则妄也,乃邪心也,故有匪正则为过眚。既已无妄,不宜有往,往则妄也。——程颐。}。

彖说:初九刚爻自外来到内卦而为主于内\footnote{震以初爻为主,成卦由之,故初为无妄之主,动以天为无妄。——程颐。}。无妄卦下为震为动,上为乾为刚健,故动而刚健。九五刚爻居中而与六二相应,大亨通于正道,天命如是也\footnote{刚自外来,而为主于内,则柔邪之道消矣。动而愈健,则刚直之道通矣。刚中而应,则齐明之德著矣。故大亨以正也。——王弼。}。其匪正有眚,不利有攸往。无妄之往,那里去呢?天命不佑,真的要行动\footnote{所谓无妄,正而已。小失于正,则为有过,乃妄也。所谓匪正,盖由有往,若无妄而不往,何由有匪正乎。无妄者,理之正也,更有往将何之矣,乃入于妄也。往则悖于天理,天道所不祐,可行乎哉?——程颐。}?

象辞说:无妄卦上为乾为天,下为震为雷,天下雷行,万物皆无妄,先王以茂盛之道来顺时养育万物\footnote{天道生万物,各得其性命而不妄。王者体天之道,养育人民,以至昆虫草木,使各得其宜,乃对时育物之道也。——程颐。}。

初九:初九爻以无妄之道前往,吉祥\footnote{卦辞言不利有攸往,谓既无妄,不可复有往也。过则妄矣。爻言往吉,谓以无妄之道而行则吉也。——程颐。}。【初九爻无妄之主,得位,既得无妄之道,则前往也是吉祥的。】无妄之往,无不得其志也\footnote{以无妄而往,无不得其志也。盖诚之于物,无不能动,以之修身则身正,以之治事则事得,其理以之临人则人感而化,无所往而不得其志也。——程颐。}。

六二:六二不耕而获,不菑而畲,如此则利于有所前往\footnote{不耕而穫,不菑而畲,谓不首造其事,因其事理所当然也。首造其事,则是人心所作为,乃妄也。因事之当然,则是顺理应物,非妄也。——程颐。}。不耕获,这是因为六二中空未富得位而九五代行其成。

六三:六三不得位,不得无妄之道欲应上九而妄动而得灾眚,此等无妄之灾\footnote{三以阴柔而不中正,是为妄者也。又志应于上,欲也,亦妄也。在无妄之道,为灾害也。人之妄动由有欲也,妄动而得,亦必有失,虽使得其所利,其动而妄,失已大矣,况复凶悔随之乎。知者见妄之得,则知其失,必与称也,故圣人因六三有妄之象,而发明其理,云无妄之灾。——程颐。},这就好比有人牵头牛,路过的行人得到了,村邑里的人就会彼此猜疑之灾。行人得牛,邑人灾也\footnote{行人得牛,乃邑人之灾也。有得则有失,何足以为得乎。——程颐。}。

九四:九四不得位,不得无妄之道,但因其并无相应,可自我贞固守之,无咎\footnote{可贞固守此,自无咎也。——程颐。}。可贞无咎,固有之也\footnote{仁义礼智,非由外铄我也,我固有之也——孟子。}。

九五:九五居中得位得无妄之道,下相应于六二亦得位,无妄之至仍得疾,此等无妄之疾,勿以药治,有喜\footnote{九五以中正当尊位,下复以中正顺应之,可谓无妄之至者也。其道无以加矣。疾为之病者也。以九五之无妄,如其有疾,勿以药治,则有喜也。人之有疾,则以药石攻去其邪,以养其正。若气体平和,本无疾病,而攻治之,则反害其正矣,故勿药则有喜也。有喜谓疾自亡也。——程颐。}。无妄之药,试都不可试一下也\footnote{人之有妄,理必修改,既无妄矣,复药以治之,是反为妄也。其可用乎,故云不可试也。试,暂用也,犹曰少尝之也。——程颐。}。

上九:上九不得位不得无妄之道,下相应六三亦不得位,此等不得无妄之道而仍要行动,有灾眚,无所利也。无妄之行,无妄之道穷而为灾眚是也。


\chapter{山天大畜(xù)卦}
大畜\footnote{小畜是小有畜止小有畜积,而大畜因上为艮卦为止,畜止已经明确,故大畜多言畜养之意,取其重者也。} {\Large ䷙}

\section{原文}

\subsection{经}
大畜:利贞,不家食吉,利涉大川。

初九:有厉利已\footnote{(yǐ),已,停止,艮也。——执象易注}。

九二:舆说輹\footnote{(fù),古代在车轴下面束缚车轴的东西——汉典;輹,车轴缚也。——说文。}。

九三:良马逐,利艰贞。日闲\footnote{闲,通娴,有词闲习,熟悉演练练习之意。}舆卫,利有攸往。

六四:童牛之牿\footnote{(gù),绑在牛角上使其不能抵入的横木——汉典},元吉。

六五:豶豕\footnote{豶(fén)豕,阉割过的公猪}之牙,吉。

上九:何\footnote{(hè)古义可通荷,比如何天之宠,何天之休,但这里王弼和程颐皆认为这是一个语气辞。}天之衢,亨。

\subsection{彖}
大畜,刚健,笃实,辉光,日新其德。刚上而尚贤\footnote{谓上九也。处上而大通,刚来而不距,尚贤之谓也。——王弼},能止健,大正也。不家食吉\footnote{有大畜之实,以之养贤,令贤者不家食,乃吉也。——王弼},养贤也。利涉大川,应乎天也\footnote{大正应天,不忧险难,故利涉大川——王弼}。

\subsection{象}
天在山中,大畜;君子以多识前言往行,以畜其德。

初九:有厉利已,不犯灾也。

九二:舆说輹,中无尤也。

九三:利有攸往,上合志也。

六四:六四元吉,有喜也。

六五:六五之吉,有庆也。

上九:何天之衢,道大行也。

\section{初讲}
卦辞说:大畜,利于正道,不食于家【贤士食于外而得尊养之。】吉祥,利涉大川。

彖说:乾体刚健,艮山笃实,其内有才若光辉,而能日日更新己德。上九刚爻居上而尊尚贤士,上卦艮卦能止乾健,大正道是也。不家食吉,尊养贤士是也。利涉大川,顺应天道是也。

象辞说:下卦为乾卦为天,上卦为艮卦为山,天在山中,这就是大畜卦的卦象了。君子应当多识前言往行,以畜其德。

初九:有危险,利于停止。有厉利已,不去触犯灾难是也【初九为六四所止】。

九二:车子的輹脱落。舆说輹,九二居中不行也并没有什么好怨恨的。【九二为六五所止】。

九三:良马驰逐,利艰贞。日常训练车马防卫,利于有所前往。利有攸往,是因为九三爻与上九爻心志相合【上九畜极则通故利于有所前往】。

六四:六四爻畜刚健之初爻者初九爻若童牛之牯,大吉祥。六四元吉,有喜也\footnote{故畜止于微小之前,则大善而吉。不劳而无伤,故可喜也。——程颐}。

六五:六五爻畜刚健之中爻者九二爻若豶豕之牙,吉祥。六五之吉,有庆也\footnote{豕,刚躁之物,而牙为猛利,若强制其牙,则用力劳而不能止其躁猛,虽絷之维之,不能使之变也。若豶去其势,则牙虽存而刚躁自止,其用如此,所以吉也。……若知其本,制之有道,则不劳无伤而俗革,天下之福庆也。——程颐}。

上九:若天路一般四通八达,亨通。何天之衢,是因为畜极则通,道路大通行是也。


\chapter{山雷颐(yí)卦}
颐 {\Large ䷚}

\section{原文}

\subsection{经}
颐\footnote{颐者,养也。——序卦。}:贞吉。观颐,自求口实。

初九:舍尔灵龟,观我朵颐\footnote{指鼓动腮颊嚼东西的样子。——汉典。},凶。

六二:颠颐\footnote{获养于下或求养于下曰颠颐。},拂经\footnote{拂,违也。经,经常也。拂经,违反常理之意。——汉典。}。于丘颐,征凶。【此处断句各家有所差异,此为我之解。拂经的意思是一般为上养下。】

六三:拂颐,贞凶。十年勿用,无攸利。

六四:颠颐,吉。虎视眈眈,其欲逐逐\footnote{急于攫取之貌。——汉典。},无咎。

六五:拂经,居贞吉,不可涉大川。

上九:由颐,厉吉,利涉大川。

\subsection{彖}
颐贞吉,养正则吉也。观颐,观其所养也。自求口实,观其自养也。天地养万物,圣人养贤以及万民,颐之时大矣哉。

\subsection{象}
山下有雷,颐。君子以慎言语,节饮食。

初九:观我朵颐,亦不足贵也。

六二:六二征凶,行失类也。

六三:十年勿用,道大悖\footnote{悖,违也。}也。

六四:颠颐之吉,上施光也。

六五:居贞之吉,顺以从上也。

上九:由颐厉吉,大有庆也。

\section{初讲}
卦辞说:颐卦,守正吉祥。观颐,自求口实。

彖说:颐贞吉,养正则吉也。观颐,观其所养也。自求口实,观其自养也\footnote{贞吉,所养者正则吉也。所养谓所养之人与养之之道。自求口实谓其自求养身之道,皆以正则吉也。——程颐。}。天地养万物,圣人养贤以及万民,颐卦运作之时很重大啊。

象辞说:颐卦上为艮为山,下为震为雷,山下有雷,这就是颐卦的卦象了。君子应当谨慎言语,节制饮食。

初九:初九舍弃你的灵龟\footnote{龟能咽息不食,灵龟喻其明智而可以不求养于外也。——程颐。},观我[六四]的朵颐之态,凶险\footnote{上应于四,不能自守,志在上行,说所欲而朵颐者也。心既动,则其自失必矣。迷欲而失己,以阳而从阴,则何所不至,是以凶也。——程颐。}。观我朵颐,亦不足贵也。

六二:六二颠倒向下求颐养于初九,反常。求颐养于山丘之上的上九,征凶。六二征凶,行失类也\footnote{与上九不是正应,前行并没有同类。}。

六三:六三拂逆了颐道,贞凶\footnote{颐之道,唯正则吉。三以阴柔之质而处不中正,又在动之极,是柔邪不正而动者也。其养如此,拂违于颐之正道,是以凶也。——程颐。}。十年勿用,无攸利\footnote{三乃拂违正道,故戒以十年勿用。十,数之终,谓终不可用,无所往而利也——程颐}。十年勿用,这是因为其与颐道大悖的缘故。【下卦三爻为震为动,故多言动。六三所处非正,而颐卦贞吉,守正则吉,故尤六三戒之曰十年勿用。】

六四:六四获颐养于初九,吉祥。其贪婪注视,急于攫取的样子,无咎\footnote{六四虽能顺从刚阳,不废厥职,然质本阴柔,赖人以济,人之所轻,故必养其威严,耽耽然如虎视,则能重其体貌,下不敢易。又从于人者必有常,若间或无继,则其政败矣。其欲谓所须用者,必逐逐相继而不乏,则其事可济。若取于人而无继,则困穷矣。既有威严,又所施不穷,故能无咎也。——程颐。}。颠颐之吉,是因为六四居上而能下施光明的缘故。

六五:六五拂经反常获颐养于上九\footnote{六五,颐之时,居君位,养天下者也。然其阴柔之质,才不足以养天下,上有刚阳之贤,故顺从之,赖其养己以济天下。君者养人者也,反赖人之养,是违拂于经常。——程颐。},居贞吉,不可涉大川。居贞之吉,是因为六五能够顺从于上九是也。

上九:天下因由上九而得颐养,厉吉,利涉大川\footnote{上九以刚阳之德居师傅之任,六五之君柔顺而从于己,赖己之养,是当天下之任,天下由之以养也。以人臣而当是任,必常怀危厉则吉也。如伊尹周公,何尝不忧勤竞畏,故得终吉。夫以君之才不足而倚赖于己身,当天下大任,宜竭其才力,济天下之艰危,成天下之治安,故曰利涉大川。得君如此之专,受任如此之重,苟不济天下艰危,何足称委遇而谓之贤乎。当尽诚竭力而不顾虑,然惕厉则不可忘也。——程颐。}。由颐厉吉,大有庆也\footnote{若上九之当大任如是,能竞畏如是,天下被其德泽,时大有福庆也。——程颐。}。



\chapter{泽风大过卦}
大过 {\Large ䷛}

\section{原文}

\subsection{经}
大过:栋桡\footnote{栋桡,意思是屋梁脆弱曲折——汉典。桡指木头弯曲。},利有攸往,亨。

初六:藉\footnote{藉(jiè)藉,祭藉也。——说文。即祭祀时用来垫祭品的衬垫。}用白茅\footnote{白茅,取其洁也——高岛断易},无咎。

九二:枯杨生稊\footnote{tí,杨柳新长出的嫩芽。},老夫得其女妻,无不利。

九三:栋桡,凶。

九四:栋隆,吉,有它吝。

九五:枯杨生华,老妇得士夫\footnote{年轻男子},无咎无誉。

上六:过涉灭顶,凶,无咎。

\subsection{彖}
大过,大者过也。栋桡,本末弱也。刚过而中,巽而说行,利有攸往,乃亨。大过之时大矣哉!

\subsection{象}
泽灭木,大过;君子以独立不惧,遁世无闷。

初六:藉用白茅,柔在下也。

九二:老夫女妻,过以相与\footnote{相处,相交往。他近日和衔玉的那位令郎相与甚厚。——红楼梦。}也。

九三:栋桡之凶,不可以有辅也。

九四:栋隆之吉,不桡乎下也。

九五:枯杨生华,何可久也。老妇士夫,亦可丑也。

上六:过涉之凶,不可咎也。

\section{初讲}
卦辞说:房屋栋梁弯曲,利于有所前往,亨通。

彖说:大过,大阳过于阴也。栋桡,本末两阴爻过于柔弱也。四刚爻虽过而居中\footnote{这里我同意高岛断易的观点,这里应该是从整个卦象来谈论的。},下卦为巽上卦为兑为悦,巽顺和悦地前行,故能利于有所前往,乃亨通。大过卦运作之时很重大啊\footnote{是君子有为之时也——王弼;如立非常之大事,兴百事之大功,成绝俗之大德,皆大过之事也。——程颐}!

象辞说:大过卦上为兑为泽,下为巽为木,泽水淹没树木\footnote{泽水本当润养树木,乃至灭没树木,则过甚矣——程颐},这就是大过的卦象了。君子观此卦象,应当做到独立无惧,遁世无闷。

初六:祭藉用的是白茅草,无咎。藉用白茅,之所以无咎是因为初六柔爻居于下卦巽卦之下,卑以处下,过于敬慎。

九二:枯萎的杨柳树又长出新的嫩芽,九二爻这个老男人得到了一个少女\footnote{九二爻为下卦之主,九二阳爻于大过中阳过则为老男人,下卦巽卦为处女。},无不利。老夫女妻,他们相互交往还是过分了些\footnote{谓九二初六阴阳相与之和过于常也。——程颐}。

九三:房屋栋梁弯曲,凶险。栋桡之凶,是因为九三爻这个位置[好比房屋栋梁的中间部位]是不可以有辅助的\footnote{栋当室之中,不可加助,是不可以有辅也。——程颐}。

九四:房屋栋梁隆起,吉祥,不过可惜有其他憾事\footnote{大家皆言此处的它吝是因为九四爻应于初六爻所致。}。栋隆之吉,是因为房屋栋梁没有向下弯曲\footnote{栋隆,九四爻能自胜其任也。}。

九五:枯萎的杨树长了花,上六这个老妇人得到九五这个年轻男子\footnote{这里的年轻是指九五相较上六很年轻}为丈夫,无咎也无誉。枯杨生华,怎么能够长久呢。老妇士夫,亦可丑也\footnote{老妇而得士夫,岂能成生育之功,为可丑矣。——程颐}。

上六:徒步过河被水淹没了头顶,凶险,无咎\footnote{处太过之极,过之甚也。涉难过甚,故至于灭顶凶。——王弼}。上六过涉之凶,不应该责备它的\footnote{虽凶无咎,不害义也。——王弼;乃自为之,不可以有咎也。——程颐}。


\chapter{坎卦}
坎 {\Large ䷜}

\section{原文}

\subsection{经}
坎:习坎,有孚,维心亨,行有尚。

初六:习坎,入于坎窞\footnote{窞,(dàn )。坎中小坎也——说文。},凶。

九二:坎有险,求小得。

六三:来之\footnote{此处为往,到之意。}坎坎,险且枕,入于坎窞,勿用\footnote{不可有所作为——汉典。}。

六四:樽酒,簋\footnote{簋,(guǐ)。古代盛食物器具,圆口,双耳。——汉典。}贰,用缶\footnote{缶,(fǒu)。古代一种大肚子小口儿的盛酒瓦器。——汉典。},纳约自牖\footnote{牖,(yǒu)。窗户——汉典。},终无咎。

九五:坎不盈,祗既平\footnote{祗,(zhī)。此处祗字意思众说纷纭,个人意见取只字意,也就是通只。既作语辞。},无咎。

上六:系用徽纆\footnote{徽纆,(huī mò)。捆绑犯人的绳索。——汉典。},置于丛棘,三岁不得,凶。

\subsection{彖}
习坎,重险也。水流而不盈,行险而不失其信。维心亨,乃以刚中也。行有尚,往有功也。天险不可升\footnote{升者,登也。——汉典解升卦之升字意。}也,地险山川丘陵也,王公设险\footnote{防御工事——汉典。}以守其国,险之时用大矣哉!

\subsection{象}
水洊\footnote{洊,(jiàn)。再次,屡次——汉典。}至,习坎。君子以常德行,习教事。

初六:习坎入坎,失道凶也。

九二:求小得,未出中也。

六三:来之坎坎,终无功也。

六四:樽酒簋贰,刚柔际也。

九五:坎不盈,中未大也。

上六:上六失道,凶三岁也。

\section{初讲}
卦辞说:熟习困难,有诚信,维系心中这份诚信则能亨通,行动则有嘉尚\footnote{阳实在中,为中有孚信。维心亨,维其心诚一,故能亨通。至诚可以通金石,蹈水火,何险难之不可亨也。行有尚,谓以诚一而行,则能出险,有可嘉尚,谓有功也。不行则常在险中矣。——程颐。}。

彖说:需要熟习困难,是因为有重重坎险。水流于坎中而不盈出,行于险地而不失其诚信\footnote{水流而不盈,阳动于险中而未出于险,乃水性之流行而未盈于坎,既盈则出乎坎矣。行险而不失其信,阳刚中实,居险之中,行险而不失其信者也。——程颐。}。维系心中这份诚信则能亨通,这是因为九二九五刚爻有中正之德\footnote{维心可以亨通者,乃以其刚中也。中实为有孚之象,至诚之道,何所不通。以刚中之道而行,则可以济险难而亨通也。——程颐。}。行动则有嘉尚,是因为前往将有其功。天险不可登也,地险山川丘陵也,王公设防御工事以守其国,险卦运作之时用处很大啊\footnote{高不可升者,天之险也。山川丘陵,地之险也。王公君人者,观坎之象知险之不可陵也,故设为城郭沟池之险,以守其国,保其民人,是有用险之时,其用甚大,故赞其大矣哉。——程颐。}。

象辞说:坎卦上下卦皆为坎为水,像是水流接连而至,熟习坎险。君子应当常久其德行,熟习所教之事。

初六:初六熟习坎险,入于坎险之时的坎险之地,凶险。习坎入坎,初六失道,上又无应援,凶险之至。

九二:九二处坎卦之时又居此有险之地,求之能有小得。\footnote{此处所求者程颐解之为九二刚中以自求,这是一解。此外与九五虽不是正应但只是所求能有小得也是说得通的。}。求小得,但九二仍未能出坎险之中\footnote{方为二阴所陷,在险之地,以刚中之才,不至陷于深险,是所求小得。然未能出坎中之险也。——程颐。}。

六三:六三来来往往都是坎\footnote{六三在坎陷之时,以阴柔而居不中正,其处不善,进退与居皆不可者也。来下则入于险之中,之上则重险也。退来与进之皆险,故云来之坎坎。——程颐。},坎险之中姑且高枕,已经入于坎险之时的坎险之地,不可有所作为。来之坎坎,终无功也\footnote{来之皆坎,无所用之,徒劳而已。——王弼。}。

六四:一樽之酒,二簋之食,用瓦缶之器盛酒,缴纳简约就在自家窗户下进行,终无咎。樽酒簋贰,[如此简约之礼为何终无咎],因为六四与九五刚柔相比而相亲\footnote{刚柔相比而相亲焉,际之谓也。——王弼。}。

九五:九五坎水流而不盈,但也只是平稳,无咎。坎不盈,九五刚中之道未能光大\footnote{九五刚中之才而得尊位,当济天下之险难,而坎尚不盈,乃未能平乎险难,是其刚中之道未光大也。——程颐。}。

上六:上六用绳索拘系之,置于丛棘\footnote{因为徽纆是捆绑犯人的绳索,此处丛棘有解为牢房的,这是可以的,但这里解释就取直意了,就这直意读者再可具体指代某坎险受困之地。}之中,多年不得出,凶险。上六失道,凶三岁也\footnote{上六失道,主要是指其居坎险受困之极地,有说以阴乘阳的,这应该关系不大。卦辞中有行有尚一句,上六当听之。}。


\chapter{离卦}
离 {\Large ䷝}

\section{原文}


\subsection{经}
离:利贞,亨,畜牝牛\footnote{畜牝牛,谓养其顺德。人之顺德,由养以成,既丽于正,当养习以成其顺德也。——程颐。}吉。

初九:履错然,敬之,无咎。

六二:黄离,元吉。

九三:日昃\footnote{昃,(zè)。日在西方时侧也——说文。}之离,不鼓缶而歌,则大耋\footnote{耋,(dié)。年八十曰耋。——说文。}之嗟,凶。

九四:突如其来如,焚如,死如,弃如。

六五:出涕沱若,戚嗟若,吉。

上九:王用出征,有嘉折首,获匪其丑\footnote{嘉,嘉美。匪,不是。丑,丑类,其胁从。——yidao5.com},无咎。

\subsection{彖}
离,丽也\footnote{古代丽还有依附,附着的意思。}。日月丽乎天,百谷草木丽乎土,重明以丽乎正,乃化成天下。柔丽乎中正,故亨,是以畜牝牛吉也。

\subsection{象}
明两作,离。大人以继明\footnote{继谓不绝也,明照相继,不绝旷也。——王弼。}照于四方。

初九:履错之敬,以辟\footnote{辟,(bì)。古同避。——汉典。}咎也。

六二:黄离元吉,得中道也。

九三:日昃之离,何可久也。

九四:突如其来如,无所容也。

六五:六五之吉,离王公也。

上九:王用出征,以正邦也。

\section{初讲}
卦辞说:离卦,利于守正道,亨通,畜养顺德吉祥。

彖说:离,依附的意思。日月依附于天,百谷草木依附于地\footnote{此处多为断句。但若只是离卦还不足以化成天地万物,离卦是依附于正道的光,照亮了这天地,才让这天地化成万物。},离卦上下重明依附于正道,于是演化而成天下万物。六二柔爻依附居中而得正,故亨通,是故畜养顺德吉祥\footnote{此处程颐认为柔爻指的是六二和六五,其还特别解释到:二则中正矣,五以阴居阳,得为正乎?曰:离主于所丽,五中正之位,六丽于正位,乃为正也。虽如此,但这里我认为此处只是在单独谈论六二柔爻,六二内卦之主,亦即内心,所以后面又跟上一句畜养顺德吉祥。人之所为,顺天之卦象所示而与之协同,自然能吉祥。}。

象辞说:上下光明两两相继而作,这就是离卦的卦象了。大人物应当不断用光明来照耀四方。


初九:初九步履交错的样子\footnote{阳固好动,又居下,而离体阳居下,则欲进。离性炎上,志在上,丽几于躁动。其履错然,谓交错也,虽未进,而迹已动矣。——程颐。},敬慎之,则无咎。履错之敬,以避开过错也\footnote{初九步履交错急躁欲行,但其上无正应而自得其位,最好还是谨慎行事,才能得无咎。}。

六二:六二黄色之明\footnote{二居中得正,丽于中正也。黄,中之色,文之美也。文明中正,美之盛也。故云“黄离”,以文明中正之德,上同于文明中顺之君,其明如是,所丽如是,大善之吉也。——程颐。},大吉祥。黄离元吉,得中道也。

九三:九三太阳偏西之明\footnote{九三居下体之终,是前明将尽,后明当继之时,人之始终,时之革易也。故为日昃之离,日下昃之明也,昃则将没矣。——程颐。},不敲着瓦盆唱歌,却到七八十岁的年纪大发嗟叹,如此则凶险。日昃之离,何可久也\footnote{日既倾昃,明能久乎?明者知其然也,故求人以继其事,退处以休其身。安常处顺,何足以为凶也。——程颐。}。

九四:九四突如其来的样子,大肆焚烧的样子,将死的样子,被遗弃的样子\footnote{处于明道始变之际,昏而始晓,没而始出,故曰“突如其来如”。其明始进,其炎始盛,故曰“焚如”。逼近至尊,履非其位,欲进其盛,以炎其上,命必不终,故曰“死如”。违“离之义,无应无承,无所不容,故曰“弃如”也。——王弼。}。突如其来如,为天下所不能容也\footnote{今四突如其来,失善继之道也。又承六五阴柔之君,其刚盛陵烁之势,气焰如焚然,故云焚如。四之所行不善如此,必被祸害,故曰死如。失继绍之义,承上之道,皆逆德也,众所弃绝,故云弃如。至于死弃,祸之极矣,故不假言凶也。——程颐。}。

六五:六五流出眼泪滂沱的样子,忧愁嗟叹的样子\footnote{履非其位,不胜所履。以柔乘刚,不能制下,下刚而进,将来害已,忧伤之深,至于沱嗟也。——王弼。},吉祥。六五之吉,是因为其依附的是王公之正位。

上九:[六五]君王可用上九来出兵征伐\footnote{九以阳居上,在离之终,刚明之极者也。明则能照,刚则能断,能照足以察邪恶,能断足以行威刑,故王者宜用如是。刚明以辨天下之邪恶,而行其征伐,则有嘉美之功也。——程颐。},有嘉美之事折获其魁首,折获的不是其丑类胁从,无咎\footnote{夫明极则无微不照,断极则无所宽宥,不约之以中,则伤于严察矣。去天下之恶,若尽究其渐涤诖误,则何可胜诛?所伤残亦甚矣。故但当折取其魁首,所执获者,非其丑类,则无残暴之咎也。——程颐。}。王用出征,以正治其邦国也。


\chapter{泽山咸卦}
咸 {\Large ䷞}

\section{原文}

\subsection{经}
咸\footnote{咸卦的咸虽读作xián,但含义是通假为感应的感字。程颐注道:“咸,感也。不曰感者,咸有皆义,男女交相感也。”可作参考。}:亨,利贞,取\footnote{取,娶也。}女吉。

初六:咸其拇。

六二:咸其腓,凶,居吉。

九三:咸其股,执其随,往吝。

九四:贞吉,悔亡。憧憧\footnote{往来不停的样子。——汉典。}往来,朋从尔思。

九五:咸其脢,无悔。

上六:咸其辅、颊、舌。

\subsection{彖}
咸,感也。柔上而刚下\footnote{此处柔上而刚下解法多样,我的解法是上天下地,上六柔爻上居之,九三刚爻下安之,如此得咸卦。},二气感应以相与,止而说,男下女,是以亨利贞,取女吉也。天地感而万物化生,圣人感人心而天下和平。观其所感,而天地万物之情可见矣。

\subsection{象}
山上有泽,咸。君子以虚受人。

初六:咸其拇,志在外也。

六二:虽凶居吉,顺不害也。

九三:咸其股,亦不处\footnote{処,止也。得几而止。——说文。}也。志在随人,所执下也。

九四:贞吉悔亡,未感害也。憧憧往来,未光大也。

九五:咸其脢,志末也。

上六:咸其辅颊舌,滕口说也\footnote{滕(téng),水超涌也——说文。滕口说也汉典后面解释有张口说话,但因有这个滕字,最后改为夸夸其谈是也。}。

\section{初讲}
卦辞说:咸卦,亨通,利于正道,娶女吉祥\footnote{取女吉,以卦才言也。卦有柔上刚下,二气感应,相与止而说,男下女之义。以此义取女,则得正而吉也。——程颐。}。

彖说:咸卦的咸字,感应的意思。[上天下地]上六柔爻上居之,九三刚爻下安之,全卦上下阴阳之气皆相感而正应,彼此相亲与。咸卦下为艮为止,上为兑为悦,如此止于愉悦。咸卦下为艮为少男,上为兑为少女,少男下求少女,所以卦辞说亨通,利于正道,娶女吉祥是也\footnote{王弼细批道:二气感应以相与故亨通;止而说,故利贞;男下女故取女吉。}。天地阴阳二气交感而化生万物,圣人感人民之心而天下和平。观察事物之所感,则天地万物之情理可见矣\footnote{既言男女相感之义,复推极感道,以尽天地之理、圣人之用。天地二气交感而化生万物,圣人至诚以感亿兆之心而天下和平。天下之心所以和平,由圣人感之也。观天地交感化生万物之理,与圣人感人心致和平之道,则天地万物之情可见矣。感通之理,知道者默而观之可也。——程颐。}。

象辞说:咸卦下为艮为山,上为兑为泽,故曰山上有泽,这就是咸卦的卦象了。君子应当以虚怀若谷的心胸接受他人。

初六:初六感应在足大指\footnote{初六在下卦之下,与四相感。以微处初,其感未深,岂能动于人?故如人拇之动,未足以进也。拇,足大指。人之相感,有浅深轻重之异,识其时势,则所处不失其宜矣。——程颐。},咸其拇,其志在外卦的九四是也\footnote{初志之动,感于四也,故曰在外。志虽动而感未深,如拇之动,未足以进也。——程颐。}。

六二:六二感应在腿肚子,凶险,安居则吉祥\footnote{二以阴居下,与五为应,故设咸腓之戒。腓,足肚,行则先动,足乃举之,非如腓之自动也。二若不守道,待上之求,而如腓自动,则躁妄自失,所以凶也。安其居而不动,以待上之求,则得进退之道而吉也。——程颐。}。虽凶居吉,但若其能顺于事理则并无所害\footnote{二居中得正,所应又中正,其才本善,以其在咸之时,质柔而上应,故戒以先动求君则凶,居以自守则吉。象复明之云:非戒之不得相感,唯顺理则不害,谓守道不先动也。——程颐。}。

九三:九三感应在大腿,执意于随物而动,如此而往则可羞吝。咸其股,其也不是静处\footnote{此处不处更是在强调其也没有遵从艮卦的止道。}。志在随人,其所操持很是卑下\footnote{股之为物,随足者也。进不能制动,退不能静处,所感在股,志在随人者也。志在随人,所执亦以贱矣。用斯以往,吝其宜也。——王弼。}。

九四:九四守正道吉祥,悔恨消亡\footnote{程颐此处谈论甚是,其他诸爻皆感应于身体的某个部位,而独九四没用这样的类比,因为九四咸在心,感应当然由心,故不再重复表述了。九四是感之主,贞正则吉而悔亡是也。}。人群往来不停,其朋类能从其所感所思。贞吉悔亡,如此则并未感应到灾害。憧憧往来,九四感之道仍未光大啊\footnote{始在于感,未尽感极,不能至于无思以得其党,故有憧憧往来,然后朋从其思也。——王弼。}。

九五:九五感应在背脊,无悔。咸其脢,那点心志是很浅末的\footnote{此爻有点众说纷纭,我认为程颐的解法较为正确。九五得位最好不因感而往,九五所感者,主要是六二,九五人君居尊位又得位,最好不要因为和六二的一点背脊上的交感私欲而坏了天下之正位。}。

上六:上六感应在上颌面颊舌头\footnote{上阴柔而说体,为说之主,又居感之极,是其欲感物之极也,故不能以至诚感物,而发见于口舌之间,小人女子之常态也,岂能动于人乎?——程颐。}。咸其辅颊舌,上六夸夸其谈是也。

\chapter{雷风恒卦}
恒 {\Large ䷟}

\section{原文}

\subsection{经}
恒:亨,无咎,利贞,利有攸往。

初六:浚\footnote{(jùn),浚,深也。——孔颖达。}恒,贞凶,无攸利。

九二:悔亡。

九三:不恒其德,或承之羞,贞吝。

九四:田无禽。

六五:恒其德贞,妇人吉,夫子凶。

上六:振恒,凶。

\subsection{彖}
恒,久也。刚上而柔下,雷风相与,巽而动,刚柔皆应,恒。恒亨无咎利贞,久于其道也。天地之道,恒久而不已也。利有攸往,终则有始也。日月得天而能久照,四时变化而能久成,圣人久于其道而天下化成。观其所恒,而天地万物之情可见矣。

\subsection{象}
雷风,恒。君子以立不易方。

初六:浚恒之凶,始求深也。

九二:九二悔亡,能久中也。

九三:不恒其德,无所容也。

九四:久非其位,安得禽也。

六五:妇人贞吉,从一而终也。夫子制义,从妇凶也。

上六:振恒在上,大无功也。

\section{初讲}
卦辞说:恒卦,亨通,无咎,利于守正道,利于有所前往。

彖说:恒卦,长久的意思。震刚在上而巽柔在下,上为震为雷,下为巽为风,雷风相与\footnote{相与,相交之意。},巽顺地行动,【整卦】刚爻与柔爻皆为正应,这就是恒卦了。【卦辞说】恒亨无咎利贞,这是其长久于可恒之正道的缘故\footnote{恒之道,可致亨而无过咎,但所恒宜得其正,失正则非可恒之道也,故曰久于其道,其道可恒之正道也。不恒其德,与恒于不正,皆不能亨而有咎也。——程颐。}。天地之道,恒久而不已。利有攸往,动则终而复始以至无穷也\footnote{天下之理,未有不动而能恒者也。动则终而复始,所以恒而不穷,凡天地所生之物,虽山岳之坚厚,未有能不变者也,故恒非一定之谓也,一定则不能恒矣。唯随时变易,乃常道也,故云利有攸往。明理之如是,惧人之泥于常也。——程颐。}。日月得天地之道而能长久照耀,四时变化终而复始而能长久运行,圣人长久于可恒之正道则天下教化而成。观察他们的恒久,而天地万物之情理可见矣。

象辞说:恒卦上为震为雷,下为巽为风,雷风相与,这就是恒卦的卦象了。君子应当立下恒久不易之方。

初六:初六浚恒,贞凶\footnote{初爻当恒之始,以始求终,所当循序渐进,方能几及,所谓登高必自卑,行远必自迩,由此以往,无不利也。若乃辍等以求,如撮土而期为山,勺水而欲成海,初基乍立,后效殊奢,事虽不失其正,要必难免于凶也,故曰浚恒贞凶。——高岛断易。},无所利。浚恒之凶,是因为其居恒之始而所求颇深。

九二:九二悔恨消亡。九二悔亡,以其能恒久于中是也\footnote{处非其常,本当有悔,而九二以中德而应于五,五复居中,以中而应中,其处与动,皆得中也,是能恒久于中也。能恒久于中,则不失正矣。——程颐。}。

九三:九三不以恒为其德,或承受羞辱,贞吝。不恒其德,无所容处其身是也\footnote{【九三】于恒处而不处,不恒之人也。其德不恒,则羞辱或承之矣。或承之,谓有时而至也。贞吝,固守不恒以为恒,岂不可羞吝乎?——程颐。}。

九四:九四田猎无禽兽之获。久非其位,安得禽也\footnote{以阳居阴,处非其位,处非其所,虽常何益?人之所为,得其道则久而成功,不得其道则虽久何益?故以田为喻,言九之居四,虽使恒久,如田猎而无禽兽之获,谓徒用力而无功也。——程颐。}。

六五:六五以恒为德贞固自守,妇人吉,夫子凶\footnote{妇人恒守义理是吉祥的,但夫子是制定义理的,怎能以恒守义理为德呢,如此则很是凶险。}。妇人贞吉,从一而终也。夫子制义,从妇凶也。【此处我决定弱化妇人和夫子之原有的男尊女卑的思想,而仅仅从妇人守义理和夫子制义理的角度出发来解之。】【恒卦上卦为震卦,虽在谈恒,亦在谈动也。九四恒之无位当动,六五居尊位恒守旧理也不合适当动,上六动之过多无恒。恒卦非恒守不变,终而复始变化方得恒道。】

上六:上六振动无恒,凶险。振恒在上,雷声大作而并无寸功\footnote{居上之道,必有恒德,乃能有功。若躁动不常,岂能有所成乎?——程颐。}。


\chapter{天山遁(dùn)卦}
遁 {\Large ䷠}

\section{原文}

\subsection{经}
遁\footnote{遁卦原字为遯,因含义与现在的遁字基本相同,故干脆直接写为遁。}:亨,小利贞。

初六:遁尾,厉,勿用有攸往。

六二:执之用黄牛之革\footnote{黄,中色;牛,顺物。黄牛之革在革卦也有一处,皆指代中顺之道。},莫之胜说。

九三:系遁,有疾厉,畜臣妾吉。

九四:好遁,君子吉,小人否。

九五:嘉遁,贞吉。

上九:肥遁,无不利。

\subsection{彖}
遁亨,遁而亨也。刚当位而应,与时行也。小利贞,浸而长也。遁之时义大矣哉!

\subsection{象}
天下有山,遁。君子以远小人,不恶而严。

初六:遁尾之厉,不往何灾也。

六二:执用黄牛,固志也。

九三:系遁之厉,有疾惫也。畜臣妾吉,不可大事也。

九四:君子好遁,小人否也。

九五:嘉遁贞吉,以正志也。

上九:肥遁无不利,无所疑也。

\section{初讲}
卦辞说:遁卦,亨通,小小利于正道\footnote{遁者,阴长阳消,君子遁藏之时也。君子退藏以伸其道,道不屈则为亨,故遁所以有亨也。在事亦有由遁避而亨者。虽小人道长之时,君子知几退避,固善也。然事有不齐,与时消息【成语,形容事物无常,随着时间的推移而兴盛衰亡。消,消亡。息,生息。】,无必同也。阴柔方长,而未至于甚盛,君子尚有迟迟致力之道,不可大贞,而尚利小贞也。——程颐。}。

彖说:遁卦亨通是因为遁退而有亨通。九五刚爻当得中正之位而与六二相正应,随时而行动之\footnote{五以刚阳之德,处中正之位,又下与六二以中正相应,虽阴长之时,如卦之才,尚当随时消息,苟可以致其力,无不至诚自尽以扶持其道,未必于遁藏而不为,故曰与时行也。——程颐。}。小小利于正道,这是因为浸润其下而渐长是也\footnote{阴道欲浸而长,正道亦未全灭,故小利贞也。——王弼。}。遁卦运作之时意义重大啊\footnote{最后一句是在强调遁卦运作之时意义是很大的,有可为之时,有利贞之处,不要一味消极遁退而误了遁卦之时。}!

象辞说:遁卦上为乾卦为天,下为艮卦为山,天下有山,这就是遁卦的卦象了。君子应当远离小人,不必恶声呵斥,威严应对即可\footnote{君子观其象,以避远乎小人,远小人之道,若以恶声厉色,适足以致其怨忿,唯在乎矜庄威严,使知敬畏,则自然远矣。——程颐。}。

初六:初六遁尾\footnote{他卦以下为初。遁者往遁也,在前者先进,故初乃为尾。尾,在后之物也,遁而在后不及者也,是以危也。初以柔处微,既已后矣,不可往也,往则危矣。微者易于晦藏,往既有危,不若不往之无灾也。——程颐。},危险,不要施用有所前往。遁尾之厉,不前往则能有什么灾祸呢\footnote{见几先遁,固为善也;遁而为尾,危之道也。往既有危,不若不往而晦藏,可免于灾,处微故也。——程颐。}。

六二:六二用黄牛之革[中顺之道]执系之\footnote{二与五为正应,虽在相违遁之时,二以中正顺应于五,五以中正亲合于二,其交自固。黄,中色,牛,顺物,革,坚固之物。二五以中正顺道相与,其固如执系之以牛革也。——程颐。},没有什么能让其挣脱\footnote{此处说作脱解会让前后文意思更连贯,程颐的其固不可胜说之解可作参考。}。执用黄牛,六二以中顺之道巩固其志是也\footnote{九五嘉遁,六二顺之自也会嘉遁,不多言矣。}。

九三:九三系遁\footnote{九三上无正应,在遁之时还牵系着六二,故曰系遁。},有疾缠身危险,畜养臣妾吉祥。系遁之厉,有疾病缠身使其疲惫。畜臣妾吉,不可当任大事\footnote{程颐此处谈及刘备携民渡江的故事,其于逃遁之时却牵系着百姓,实在危险。但九三爻辞却不是单单谈论危险,也承认了畜养臣妾的吉祥,刘备的屡败而屡次东山再起,都得益于他的畜养臣妾带来的吉祥,但象其后却又加上一句泼了一瓢凉水,不可大事也。熟悉三国的明白这说的就是刘备其人因这个牵系私情的毛病,实在当不得一国之主这个重任。}。

九四:九四好遁\footnote{四与初为正应,是所好爱者也。君子虽有所好爱,义苟当遁,则去而不疑,所谓克己复礼,以道制欲,是以吉也。小人则不能以义处,暱于所好,牵于所私,至于陷辱其身而不能已,故在小人则否也。——程颐。},君子吉祥,小人不善\footnote{九四不得位不得遁道,不知何时而遁,而又有初六之正应所好影响,故爻辞戒之曰君子好遁。}。君子好遁,小人否也\footnote{君子虽有好而能遁,不失于义;小人则不能胜其私意,而至于不善也。——程颐。}。

九五:九五嘉美地遁退\footnote{九五中正,遁之嘉美者也。处得中正之道,时止时行,乃所谓嘉美也,故为贞正而吉。九五非无系应,然与二皆以中正自处,是其心志及乎动止,莫非中正,而无私系之失,所以为嘉也。——程颐。},守正道吉祥\footnote{九五得位,故得遁道,能做到卦辞所言的与时行也,而又有六二的中顺之道,如此嘉遁,自是贞吉。}。嘉遁贞吉,以正其志也\footnote{志正则动必由正,所以为遁之嘉也。居中得正,而应中正,是其志正也,所以为吉。人之遁也,止也,唯在正其志而已矣。——程颐。}。

上九:上九肥遁\footnote{肥,饶裕也——康熙字典。上九遁之极也,因上九不得位,遁得有点过头了,故曰肥遁。也就是遁之绰绰有余。},无不利\footnote{最处外极,无应于内,超然绝志,心无疑顾,忧患不能累,矰缴不能及,是以肥遁无不利也。——王弼。}。肥遁无不利,其于遁退无所疑滞是也\footnote{其遁之远,无所疑滞也。盖在外则已远,无应则无累,故为刚决无疑也。——程颐。}。



\chapter{雷天大壮卦}
大壮 {\Large ䷡}

\section{原文}

\subsection{经}
大壮:利贞。

初九:壮于趾,征凶有孚。

九二:贞吉。

九三:小人用壮,君子用罔\footnote{罔,wǎng,无也。——尔雅。},贞厉。羝\footnote{羝,dī,公羊。——汉典。}羊触藩\footnote{藩,fān,篱笆。——汉典。},羸\footnote{同累,缠绕,困住——汉典。}其角。

九四:贞吉,悔亡,藩决不羸,壮于大舆之輹。

六五:丧羊于易\footnote{通场,边界——汉典。},无悔。

上六:羝羊触藩,不能退,不能遂,无攸利,艰则吉。

\subsection{彖}
大壮,大者\footnote{大者谓阳爻——王弼。}壮也。刚以动,故壮。大壮利贞,大者正也。正大而天地之情可见矣。

\subsection{象}
雷在天上,大壮。君子以非礼弗履。

初九:壮于趾,其孚穷也。

九二:九二贞吉,以中也。

九三:小人用壮,君子罔也

九四:藩决不羸,尚\footnote{尚,上也。——广雅。}往也。

六五:丧羊于易,位不当也。

上六:不能退,不能遂,不祥也。艰则吉,咎\footnote{咎,灾也。——说文。}不长也。

\section{初讲}
卦辞说:大壮卦,利于贞正之道\footnote{大壮之道,利于贞正也。大壮而不得其正,强猛之为耳,非君子之道壮盛也。——程颐。}。

彖说:大壮卦,阳气正在壮大之意\footnote{所以名大壮者,谓大者壮也。阴为小,阳为大。阳长以盛,是大者壮也。——程颐。}。大壮卦下为乾为刚健,上为震为行动,刚健而行动故壮大之\footnote{下刚而上动,以乾之至刚而动,故为大壮。为大者壮,与壮之大也。——程颐。}。卦辞说大壮利贞,是因为阳气本是正气的缘故\footnote{大者获正,故利贞也。——王弼。}。至正至大而天地[无边]之情形可见矣。

象辞说:大壮卦上为震为雷,下为乾为天,雷在天上,这就是大壮卦的卦象了。君子应当敬畏天地,不要涉足非礼之域。

初九:初九爻状于脚趾而欲行进,征行凶险虽有信心。初九爻壮于趾,征行而致凶险,这样他的信心也将穷尽矣。

九二:九二爻贞正吉祥。九二贞吉,是因为他居中的缘故\footnote{刚柔得中不过于壮,得贞正而吉也。——程颐。}。

九三:能力小的人还可以行壮大之事,君子不要做了,继续行大壮之道危险\footnote{九三以刚居阳而处壮,又当乾体之终,壮之极者也。极壮如此,在小人则为用壮,在君子则为用罔。——程颐。}。否则就会像一只公羊抵触篱笆,被缠住了它的角。能力小的人用壮遇到阻碍就知道返回了,君子告诫要用罔,否则就会被缠住角而陷入危险的境地了。

九四:九四贞吉,悔恨消失,决开篱笆而不被缠住,比大车的轮辐还强壮呵\footnote{四阳刚长,盛壮已过中壮之甚也。......藩所以限隔也,藩篱决开,不复羸困其壮也,高大之车,轮輹强壮,其行之利可知。故云:壮于大舆之輹。輹,轮之要处也。车之败当在折輹,輹壮则车强矣。云壮于輹,谓壮于进也,輹与辐同。——程颐。}。藩决不羸,往上去矣。

六五:六五爻于边界丢失了九四这只羊,无悔。丧羊于易,这是因为六五不当位的缘故。

上六:九三公羊抵触上六篱笆,不能后退,也不能成功逃脱,没什么好处,虽很艰难终能吉祥。九三公羊不能退,也不能遂,这种情况不是很好。艰则吉,是因为灾祸不会很长的了。


\chapter{火地晋卦}
晋 {\Large ䷢}

\section{原文}

\subsection{经}
晋\footnote{晋,进也。日出,万物进。——说文。}:康\footnote{褒扬。康周公,故以赐鲁也。——礼记。}侯用锡\footnote{xī,赐也——尔雅。}马蕃庶\footnote{fán shù,繁盛、众多——汉典。},昼日三接\footnote{一日之内多次接见,形容深受宠爱。}。

初六:晋如,摧如,贞吉。罔孚,裕,无咎。

六二:晋如,愁如,贞吉。受兹介\footnote{介,大也。——程颐。}福,于其王母。

六三:众允\footnote{允,信也。——说文。},悔亡。

九四:晋如鼫\footnote{shí,五技鼠也,这里记作硕鼠的都是错误的。}鼠,贞厉。

六五:悔亡,失得勿恤。往吉,无不利。

上九:晋其角,维\footnote{通唯。维予与女。——小雅。}用伐邑\footnote{邑,国也。——说文。},厉吉。无咎,贞吝。

\subsection{彖}
晋,进也。明出地上,顺而丽乎大明,柔进而上行。是以康侯用锡马蕃庶,昼日三接也。

\subsection{象}
明出地上,晋。君子以自昭明德。

初六:晋如摧如,独行正也。裕无咎,未受命也。

六二:受兹介福,以中正也。

六三:众允之,志上行也。

九四:鼫鼠贞厉,位不当也。

六五:失得勿恤,往有庆也。

上九:维用伐邑,道未光也。

\section{初讲}
卦辞说:晋卦,[六五]褒扬[六二]王侯用众多车马作为赏赐,一日之内多次接见\footnote{从诸爻来看,当位得晋道者唯有六二,六五昼日三接,暗含赶忙补救之意。}。

彖说:晋卦,上进的意思。晋卦上为离为明,下为坤为地,晋卦有明出地上之象。坤顺离丽乎天下大明,[六五]柔爻进而居上尊位而行之\footnote{柔进而上行,凡卦离在上者,柔居君位多云“柔进而上行”:噬嗑、睽、鼎是也。——程颐。}。所以褒扬[六二]王侯,用众多车马作为赏赐,一日之内多次接见。

象辞说:晋卦上为离为明,下为坤为地,明出地上,这就是晋卦的卦象了。君子应当自我[内心]昭明乃明德行\footnote{昭,明之也。传曰:昭德塞违,昭其度也。君子观明出地上,而益明盛之象,而以自昭其明德,去蔽致知,昭明德于己也。明明德于天下,昭明德于外也;明明德在己,故云自昭。——程颐。}。

初六:初六上进的样子,被挫败的样子,贞正吉祥\footnote{初居晋之下,进之始也。晋如,升进也。摧如,抑退也。于始进而言,遂其进不遂其进,唯得正则吉也。——程颐。}。没有信心,把心放宽,无咎\footnote{罔孚者,在下而始进,岂遽能深见信于上。苟上未见信,则当安中自守,雍容宽裕,无急于求上之信也。苟欲信之心切,非汲汲以失其守,则悻悻以伤于义矣,皆有咎也。故裕,则无咎。君子处进退之道也。——程颐。}。晋如摧如,初六独行正道是也\footnote{无进无抑,唯独行正道也。——程颐。}。裕无咎,是因为初六未受任命是也。

六二:六二上进的样子,忧愁的样子,贞正吉祥\footnote{六二在下,上无应援,以中正柔和之德,非强于进者也,故于进为可忧愁,谓其进之难也。然守其贞正,则当得吉,——程颐。}。受此大福,于六五王母。受兹介福,是因为六二当位而得中正之德的缘故。

六三:初六初二都相信六三\footnote{坤为众。},悔恨消失\footnote{六居三,不得中正,宜有悔咎。而三在顺体之上,顺之极者也。三阴皆顺上者是也。是三之顺上与众同志,众所允从,其悔所以亡也。——程颐。}。众允之,他们的志向都是向上进行是也\footnote{上行上顺,丽于大明也。上从大明之君,众志之所同也。——程颐。}。

九四:九四上进的样子好像鼫鼠\footnote{许慎《说文》云:“鼠,穴虫之总名也。鼫,五伎鼠也,能飞不能上屋,能缘不能穷木,能浮不能渡谷,能穴不能掩身,能走不能先人。”鼫鼠者,求得而未必能者也。晋如鼫鼠者,小人贪鄙,狐媚猿攀,窃居高位而畏于人也。——执象易注。},继续求晋进之道会很危险。鼫鼠贞厉,是因为九四不当位的缘故。

六五:悔恨消失,不要患得患失。大胆前往吧,没什么不好的事情\footnote{下既同德顺附,当推诚委任,尽众人之才,通天下之志,勿复自任其明、恤其失得,如此而往,则吉而无不利也。——程颐。}。失得勿恤,是因为前往将有福庆是也\footnote{以大明之德,得下之附,推诚委任,则可以成天下之大功,往而有福庆也。——程颐。}。

上九:上九晋进如用角顶\footnote{角刚而居上之物。上九以刚居卦之极,故取角为象,以阳居上刚之极也。在晋之上,进之极也。刚极则有强猛之过,进极则有躁急之失。以刚而极于进,失中之甚也。——程颐。},唯独可用来攻伐国邑,虽然危险但还是可得吉祥。无咎,继续晋进之道会很难行的。维用伐邑,但上九晋进之道并未光大是也\footnote{维用伐邑,既得吉而无咎。复云:贞吝者,贞道未光大也。以正理言之,犹可吝也。夫道既光大,则无不中正,大安有过也。今以过刚自治,虽有功矣,然其道未光大,故亦可吝。圣人言尽善之道。——程颐。}。




\chapter{地火明夷(yí)卦}
明夷 {\Large ䷣}

\section{原文}

\subsection{经}
明夷\footnote{夷者,伤也。——程颐。明夷,明受夷也。——高岛断易。}:利艰贞。

初九:明夷于飞,垂其翼。君子于行,三日不食。有攸往,主人有言。

六二:明夷,夷于左股,用拯\footnote{拯,救也——高岛断易。}马壮,吉。

九三:明夷于南狩,得其大首,不可疾贞。

六四:入于左腹,获明夷之心,于出门庭。

六五:箕子之明夷,利贞。

上六:不明晦,初登于天,后入于地。

\subsection{彖}
明入地中,明夷。内文明而外柔顺,以蒙大难,文王以\footnote{此处以有解为似字的,个人有点赞同,但仍决定采用以的最基本的意思而解之,即:以,用也——说文。}之。利艰贞,晦其明也,内难而能正其志,箕子以之。

\subsection{象}
明入地中,明夷。君子以莅\footnote{(lì)此处莅者,临也。——康熙字典。临者,从上往下看是也,所以此处解为治理管理也并无不妥。}众,用晦而明。

初九:君子于行,义不食也。

六二:六二之吉,顺以则也。

九三:南狩之志,乃大得也。

六四:入于左腹,获心意也。

六五:箕子之贞,明不可息也。

上六:初登于天,照四国\footnote{四国,犹言四方也。——郑玄。}也。后入于地,失则也。

\section{初讲}
卦辞说:明夷卦,利于艰难守贞正之道\footnote{君子当明夷之时,利在知艰难,而不失其贞正也。在昏暗艰难之时,而能不失其正,所以为君子也。——程颐。}。

彖说:明入于地中,明夷卦\footnote{明入于地,其明灭也,故为明夷。——程颐。}。内卦为离卦为文明,外卦为坤卦为柔顺,得以蒙犯大难,周文王以用此也。利于艰难守贞正之道,晦藏己之光明\footnote{不晦其明则被祸患,不守其正则非贤明。——程颐。}。内有难\footnote{箕子当纣之时,身处其国內,切近其难,故云:內难。——程颐。}而能自守其正志,箕子以用此也。

象辞说:明夷卦下为离卦为明,上为坤卦为地,明入于地中,这就是明夷卦的卦象了。君子莅临民众,应当知道使用晦藏之道才是明智的\footnote{坤为众,晦藏自己的才智聪明而让民众的才智聪明得以显现}。

初九:明夷初九于飞行时,垂下其羽翼\footnote{并不想上飞,因初九知道上将有伤于己。}。初九君子于远行之时,多日不食\footnote{君子于行,谓去其禄位而退藏也。三日不食,谓困穷之极也。——程颐。}。有所往适[以远害],[六四]主人有责言。君子于行,理应不再食六四主人的俸禄了。【程颐解此爻提到了穆生之去楚申公的典故,可参考之\footnote{楚元王敬礼穆生,每置酒,常为穆生设醴。及王戊即位,常设,后忘设焉。穆生退,曰:“可以逝矣。醴酒不设,王之意怠。楚人将钳我于市。”称疾卧。申公与白生强起之,曰:“独不念先王之德欤?今王一旦失小礼,何足至此。”穆生曰:“君子见几而作,不俟终日。先王所以礼吾三人者,为道之存故也。今而忽之,是忘道也。忘道之人,胡可与久处?岂为区区之礼哉!”遂谢病去。申公、白生独留。王戊稍淫暴,与吴通谋,二人谏不听,衣之赭衣,使杵臼舂于市。申公愧之,归鲁教授,不出门。已而赵绾、王臧言于武帝,复以安车蒲轮召,卒坐臧事,病免。死。穆生远引于未萌之前,而申公眷恋于既悔之后。谓祸福皆天不可避就者,未必然也。可书之座右,为士君子终身之戒。——穆生去楚王戊,苏轼。}。】

六二:明夷六二爻有伤于左股\footnote{夷于左股,谓伤害其行而不甚切也。——程颐。},[后伤害来到之时]用来拯救的马很强壮,吉祥。六二之吉,是因为他顺处而有法则是也\footnote{六二之得吉者,以其顺处而有法则也,则谓中正之道能顺而得中正,所以处明伤之时,而能保其吉也。——程颐。}。

九三:明夷的九三爻在南方狩猎,得其大首领\footnote{大首,谓暗之魁首,上六也。——程颐。},不可疾贞\footnote{民之迷也,其日固已久矣。化宜以渐,不可速正,故曰不可疾贞。——王弼。}。南狩之志,乃能得其大首是也\footnote{夫以下之明除上之暗,其志在去害而已。如商周之汤武,岂有意于利天下乎。得其大首,是能去害而大得其志矣。——程颐。}\footnote{此处似乎有的地方是乃大得也,从上一句南狩之志来看,后一句乃得大也,即王弼注解的大首也,更贴切一点。}。

六四:六四爻入于左腹,获得上六明夷之主\footnote{这里我解的和程颐略有不同,程颐认为此处为六五明夷之君,我这里解为明夷之主上六,因六五有箕子之贞。}的心意\footnote{夫小人之事君,未有由显明以道合者也,必以隐僻之道,自结于上。右当用,故为明显之所。左不当用,故为隐僻之所。人之手足,皆以右为用。世谓僻所,谓僻左,是左者隐僻之所也。四由隐僻之道,深入其君,故云入于左腹,入腹谓其交深也。——程颐。},于是无所顾忌地从门庭出来。入于左腹,六四爻获得明夷之主的心意是也。

六五:明夷卦的六五若箕子,利贞。箕子之贞\footnote{箕子,商之旧臣,而同姓之亲,可谓切近于纣矣,若不自晦其明,被祸可必也,故佯狂为奴,以免于害。虽晦藏其明而內守,守其正所谓內难而能正其志,所以谓之仁于明也。若箕子,可谓贞矣。——程颐。},其明不可息也\footnote{箕子晦藏不失其贞固,虽遭患难其明自存,不可灭息也。——程颐。}。

上六:上六爻不明而昏晦,初登于天,后入于地下\footnote{处明夷之极,是至晦者也。本其初也,在乎光照,转至于晦,遂入于地。——王弼。}。初登于天,光照四方。后入于地,失去法则是也。【上六爻可比作纣王初是权势光照于四方的天子,而后杀比干,囚箕子,自毁其明,终至亡国是也。】


\chapter{风火家人卦}
家人 {\Large ䷤}

\section{原文}

\subsection{经}
家人:利女贞。

初九:闲\footnote{闲,阑也。——说文。此处引申为防闲之意,故康熙字典此处疏曰:治家之道在初,即须严正立法防闲也。}有家,悔亡。

六二:无攸遂\footnote{遂,往也。——广雅。因大壮卦有不能退,不能遂。此处可解释为前往。},在中馈\footnote{馈,进食。——汉典。},贞吉。

九三:家人嗃嗃\footnote{(hè hè),严酷的样子——汉典。},悔厉吉。妇子嘻嘻,终吝。

六四:富家,大吉。

九五:王假\footnote{假,至也。}有家,勿恤,吉。

上九:有孚,威如,终吉。

\subsection{彖}
家人,女正位乎内,男正位乎外,男女正,天地之大义也。家人有严君焉,父母之谓也。父父,子子,兄兄,弟弟,夫夫,妇妇,而家道正。正家而天下定矣。

\subsection{象}
风自火出,家人。君子以言有物,而行有恒。

初九:闲有家,志未变也。

六二:六二之吉,顺以巽也。

九三:家人嗃嗃,未失也。妇子嘻嘻,失家节也。

六四:富家大吉,顺在位也。

九五:王假有家,交相爱也。

上九:威如之吉,反身之谓也。

\section{初讲}
卦辞说:家人卦,利于女子正道\footnote{家人之道利在女正,女正则家道正矣。夫夫妇妇而家道正,独云利女贞者,夫正者身正也,女正者家正也,女正则男正,可知矣。——程颐。}。

彖说:家人卦,六二爻得位居内卦是女正位乎内,九五爻得位居外卦是男正位乎外,男女各正其位,天地阴阳之大义也\footnote{彖以卦才而言,阳居五在外也,阴居二处内也,男女各得其正位也。尊卑内外之道,正合天地阴阳之大义也。——程颐。}。家人有威严的君上,说的就是父母\footnote{家人之道必有所尊严,而君长者谓父母也,虽一家之小,无尊严则孝敬衰,无君长则法度废。有严君而后家道正,家者国之者也。——程颐。}。父行父道,子行子道,兄行兄道,弟行弟道,夫行夫道,妇行妇道,如此则家道正。推此正家之道于天下,则天下定矣\footnote{父子、兄弟、夫妇各得其道,则家道正矣。推一家之道,可以及天下,故家正则天下定矣。——程颐。}。【此处说的观点和孔子的君君臣臣父父子子的政治观点是一致的。】

象辞说:家人卦上为巽为风,下为离为火,火于内而风于外,故曰风自火出,这就是家人卦的卦象了。君子应当言之有物,行之有恒。


初九:初九爻防闲法度有家道,悔亡。闲有家,家人志意并未变动是也\footnote{闲之于始,家人志意未变动之前也,正志未流散变动而闲之,则不伤恩、不失义,处家之善也。是以悔亡。志变而后治,则所伤多矣,乃有悔也。——程颐。}。

六二:六二爻无所前往,居中进食,贞吉。六二之吉,顺以巽也\footnote{居内处中,履得其位,以阴应阳,尽妇人之正,义无所必,遂职乎中馈,巽顺而已,是以贞吉也。——王弼。}。

九三:家人彼此严酷的样子,可能会悔于治家过于严厉但终是吉祥的,妇子彼此嬉笑的样子,治家过于宽松,终将有吝惜悔恨的事发生。家人嗃嗃,未失也。妇子嘻嘻,失家节也\footnote{虽嗃嗃于治家之道,未为甚失,若妇子嘻嘻,是无礼法,失家之节,家必乱矣。——程颐。}。

六四:六四爻能富其家,大吉。富家大吉,是因为六四爻巽顺而得其正位\footnote{以巽顺而居正位,正而巽顺,能保有其富者也。富家之大吉也。——程颐。}。

九五:九五爻君王至而有家道,不用担忧,吉祥。王假有家,交相爱也\footnote{假,至也。履正而应,处尊体巽,王至斯道,以有其家者也。居於尊位,而明于家道,则下莫不化矣。父父、子子、兄兄、弟弟、夫夫、妇妇,六亲和睦交相爱乐而家道正,正家而天下定矣。故王假有家,则勿恤而吉。——王弼。}。

上九:上九爻有诚信,威严的样子,终能吉祥\footnote{上卦之终,家道之成也,故极言治家之本。治家之道,非至诚不能也,故必中有孚信,则能常久。而众人自化为善,不由至诚,已且不能常守也,况欲使人乎。故治家以有孚为本,治家者在妻孥情爱之间,慈过则无严,恩胜则掩义,故家之患,常在礼法不足而渎慢生也。长失尊严,少忘恭顺,而家不乱者未之有也,故必有威严则能终吉。保家之终,在有孚威,如二者而已,故于卦终言之。——程颐。}。威如之吉,是因为上九爻能反过来要求自身\footnote{治家之道以正身为本,故云反身之谓。爻辞谓治家当有威严,而夫子又复戒云:当先严其身也。威严不先行于己,则人怨而不服。故云:威如而吉者,能自反于身也。——程颐。}。


\chapter{火泽睽(kuí)卦}
睽 {\Large ䷥}

\section{原文}

\subsection{经}
睽:小事吉。

初九:悔亡,丧马勿逐,自复。见恶人,无咎。

九二:遇主于巷,无咎。

六三:见舆曳,其牛掣,其人天且劓\footnote{天髠首也,劓截鼻也。三从正应而四隔止之,三虽阴柔处刚而志行,故力进以犯之,是以伤也。——程颐},无初有终。

九四:睽孤,遇元夫\footnote{指初九},交孚,厉无咎。

六五:悔亡,厥\footnote{其}宗噬肤,往何咎。

上九:睽孤,见豕负涂,载鬼一车, 先张之弧,后说\footnote{通脱}之弧,匪寇婚媾,往遇雨则吉。

\subsection{彖}
睽,火动而上,泽动而下,二女同居,其志不同行。说\footnote{通悦}而丽\footnote{依附}乎明,柔进而上行,得中而应乎刚,是以小事吉。天地睽而其事同也,男女睽而其志通也,万物睽而其事类\footnote{指事物的类似性}也。睽之时用\footnote{王弼注:非用之常,用有时也。}大矣哉!

\subsection{象}
上火下泽,睽。君子以同而异。

初九:见恶人,以辟\footnote{通避}咎也。

九二:遇主于巷,未失道也。

六三:见舆曳,位不当也。无初有终,遇刚也。

九四:交孚无咎,志行也。

六五:厥宗噬肤,往有庆也。

上九:遇雨之吉,群疑亡也。

\section{初讲}
卦辞说:睽卦,小事吉祥。

彖说:睽卦上为离为火,火动向上,下为兑为泽,泽动向下,又离为中女,兑为少女,是以二女同居,志向不同,很难一起行动。内心愉悦而依附于光明,六五柔爻进而居上尊位\footnote{六五以柔居尊位——程颐。}而行,得中位而与九二刚爻相应\footnote{以有此三德也。——王弼},是以小事吉。天地分离但它们养育万物之事是相同的,男女分离但它们交感之志是相通的,万物分离但它们之间还是很类似的。睽卦运作之时用处很大啊!

象辞说:睽卦的卦象是上面是火下面是泽,君子观此卦象而知道求同存异的道理。


初九:悔恨消失,丢失的马不要去追寻,它自己会回来。见到交恶之人,无咎。见恶人,不避咎故而无咎。

九二:九二在小巷遇到六五,无咎。遇主于巷,未失道也。

六三:上互卦\footnote{互卦,去掉初爻和六爻,中间四爻上三爻称为上互卦,下三爻称为下互卦。}为坎为舆,车子有两爻向上而一爻向下,下互卦为离为牛,牛两爻向下一爻向上,故有车子拖曳受困,牛拉车却反而掣阻车子。六三其人受到了剃光头和割鼻子的刑罚。起初不好但最终有好的结局的。见舆曳,是因为六三爻位置不好。无初有终,因为六三与上九相应和故终有好结局。

九四:九四这个睽爻很孤独,但和初九这个阳爻是相应的,遇之,以诚信结交之,虽然很危险但不会有灾祸的。交孚无咎,因为九四爻其志得行的缘故。

六五:悔恨消失,九二厥宗设宴吃肉,六五爻前往会之,何咎?厥宗噬肤,六五爻前往会有喜庆之事。

上九:上九这个睽爻很孤独,看见猪背上涂满污泥,又看见车上满载着打扮得像鬼一样的人,先张弓欲射,后又放下弓箭,不是强盗来了,是迎亲的队伍,前往遇到下雨天会吉祥。遇雨之吉,是因为之前的满腹疑虑都烟消云散了。


\chapter{水山蹇(jiǎn)卦}
蹇 {\Large ䷦}

\section{原文}

\subsection{经}
蹇\footnote{蹇,跛也。——说文。}:利西南,不利东北。利见大人,贞吉。

初六:往蹇,来誉。

六二:王臣蹇蹇,匪躬之故。

九三:往蹇,来反。

六四:往蹇,来连\footnote{此处有成语往蹇来连,并将连仍然解释为艰难,这实在不通。给出的例子“孟轲虽连蹇”,其中连字也当作接连解,从象形通义等等都联系不到难义。}。

九五:大蹇,朋来。

上六:往蹇,来硕\footnote{硕,头大也。——说文。},吉。利见大人。

\subsection{彖}
蹇,难也,险在前也。见险而能止,知矣哉。蹇利西南,往得中\footnote{此处应为众通坤义。}也。不利东北,其道穷也。利见大人,往有功也。当位贞吉,以正邦也。蹇之时用大矣哉。

\subsection{象}
山上有水,蹇。君子以反身修德。

初六:往蹇来誉,宜待也。

六二:王臣蹇蹇,终无尤也。

九三:往蹇来反,内喜之也。

六四:往蹇来连,当位实也。

九五:大蹇朋来,以中节也。

上六:往蹇来硕,志在内也。利见大人,以从贵也。

\section{初讲}
卦辞说:蹇卦,利好于西南方,不利于东北方。利于去见能力大的人,贞正吉祥。

彖说:蹇,难也,因为蹇卦上为坎为险,所以险在前是也\footnote{险在前也,坎险在前,下止而不得进,故为蹇。——程颐。}。蹇卦下为艮为止,见险而能止,这就是有智慧啊。蹇卦利于西南方,是因为坤在西南方,坤为众,前往西南将得到众人的帮助是也。不利于东北方,是因为艮在东北方,艮为止,止于蹇难之境则会陷入穷途是也。利于去见能力大的人,是因为前往将成济蹇之功。[除初六外各爻]皆当位所以贞正吉祥\footnote{蹇之诸爻除初外,馀皆当正位,故为贞正而吉也。初六虽以阴居阳而处下,亦阴之正也,以如此正道正其邦,可以济于蹇矣。——程颐。},可以用来正治其邦。蹇卦运作之时用处很大啊。

象辞说:蹇卦下为艮为山,上为坎为水,山上有水,这就是蹇卦的卦象了。君子应当反省自身,以修德行。

初六:初六爻前往有蹇难,回来有赞誉\footnote{处难之始,居止之初,独见前识,睹险而止,以待其时,知矣哉!故往则遇蹇,来则得誉。——王弼。}。往蹇来誉,这是因为初六此时适宜静待是也。

六二:六二大臣和九五君王困难重重,不是他们自身的缘故\footnote{二以中正之德居艮,体止于中正者也,与五相应,是中正之人为中正之君所信任,故谓之王臣。虽上下同德,而五方在大蹇之中,致力于蹇难之时,其艰蹇至甚,故为蹇于蹇也。二虽中正,以阴柔之才岂易胜其任,所以蹇于蹇也。志在济君于蹇难之中,其蹇蹇者,非为身之故也。——程颐。}。王臣蹇蹇,最终不会有过失的\footnote{虽艰戹于蹇时,然其志在济君难,虽未成功,然终无过尤也。圣人取其志义,而谓其无尤,所以劝忠尽也。——程颐。}。

九三:九三前往有蹇难,回来反省自身\footnote{此处反字程颐解为返回,但来已有回来返回之意,不符合古文文字简之又简的习惯。内卦为主观为自己的心,九三又是内卦之主,正合乎象所言君子应当反身修德之理。}。往蹇来反,九三内心欢喜之是也。

六四:六四前往有蹇难,回来联合九三和九五\footnote{连有辇义,六四夹在九三和九五之间看起来有辇车之象。初六因为不和六四正应,所以谈论六四只谈论九三和九五相比而来的助力。}。往蹇来连,九三和九五当位会实心帮忙的。

九五:九五大大蹇难\footnote{处难之时,独在险中,难之大者也,故曰大蹇。——王弼。},有朋友会来帮忙的\footnote{六二,六四,上六这些朋友都会过来帮忙的。}。大蹇朋来,是因为九五当位居中守节的缘故。

上六:上六前往有蹇难,回来会得到九三\footnote{大,阳爻也,九三为内卦之主,有大头之象。}的帮助,吉祥。利于去见九五大人。往蹇来硕,是因为九三其志在内心是也。利见大人,上六可以跟从九五贵人是也\footnote{上六应三,而从五,志在内也。蹇既极而有助,是以硕而吉也。六以阴柔当蹇之极,密近刚阳中正之君,自然其志从附,以求自济,故利见大人,谓从九五之贵也。所以云:从贵,恐人不知大人为指五也。——程颐。}。



\chapter{雷水解(xiè)卦}
解\footnote{查阅之后确认解卦读音为xiè,因为《易·杂卦传》里有:谢,缓也。即谢,通松懈的懈。} {\Large ䷧}

\section{原文}

\subsection{经}
解:利西南,无所往,其来复吉。有攸往,夙\footnote{夙,早也。}吉。

初六:无咎。

九二:田\footnote{田,猎也。}获三狐,得黄矢,贞吉。

六三:负且乘,致寇至,贞吝。

九四:解而拇\footnote{震为足也,此处拇指大脚趾。},朋至斯孚。

六五:君子维\footnote{维犹系也——闻一多。}有解,吉;有孚于小人。

上六:公用射隼\footnote{sǔn,一种凶猛的鸟。},于高墉\footnote{yōng,城墙。}之上,获之,无不利。

\subsection{彖}
解,险以动,动而免乎险,解。解利西南,往得众也,其来复吉,乃得中也。有攸往夙吉,往有功也。天地解,而雷雨作,雷雨作,而百果草木皆甲坼\footnote{chè,甲坼,指草木发芽时种子外皮裂开。},解之时大矣哉!

\subsection{象}
雷雨作,解;君子以赦过宥罪。

初六:刚柔之际,义无咎也。

九二:九二贞吉,得中道也。

六三:负且乘,亦可丑也,自我致戎,又谁咎也。

九四:解而拇,未当位也。

六五:君子有解,小人退也。

上六:公用射隼,以解悖也。

\section{初讲}
卦辞说:解卦,利西南,无所往,返回原来的地方就吉祥【蹇解而难已平,无难则无所往,休养生息之\footnote{参见高岛断易}】。有所往早早行动吉祥【蹇解而难犹在,有难则早早前往之\footnote{参见高岛断易}】。

彖说:解卦,下为坎为险,上为震为动。是故遇到危险而行动,行动而免于危险,这就是解卦了。解卦利西南,往得众也,其来复吉,乃得中也。【坎险已解,九二爻仍返回原处,故曰其来复,爻变为坤为西南,故曰利西南,坤为众也,故曰往得众。九二爻安居于内卦之中,故曰乃得中道\footnote{参考了\href{https://www.eee-learning.com/book/neweee40}{这个网页}}。】有所往早早行动吉祥,前往可获成功。天地阴阳之气缓和而雷雨大作,雷雨大作然后百果草木种子皆发芽,解卦运作之时很重大的啊。

象辞说:解卦上为震为雷,下为坎为雨,上雷下雨,雷雨大作,这就是解卦的卦象了。君子观此卦象而知道天地疏解的道理,故而赦免那些有过错的人,宽恕那些有罪的人。


初六:初六柔爻与九四刚爻相交,理应无咎。

九二:九二爻猎获了很多狐狸\footnote{狐者,隐伏之物也,隐喻隐藏的某些奸诈小人或者麻烦。},得到了六五君上的黄色箭矢的赏赐,贞吉。九二贞吉,是因为它守中正之道的缘故。

六三:六三爻作小人之事却乘君子之器\footnote{子曰:作易者,其知盜乎!易曰:「負且乘,致寇至。」負也者,小人之事也。乘也者,君子之器也。},招来贼寇的到来,贞吝。负且乘,亦可丑也,自我致戎,又谁咎也。

九四:九四爻疏解自己的大脚趾,因为它位置不当的缘故,被六三爻小人弄得疲惫不堪,六五朋友来了才是可以相信的。

六五:六五君子虽被约束但有解,吉祥,有诚信于小人\footnote{居尊履中而应乎刚,可以有解而获吉矣。以君子之道解难释险,小人虽间,犹知服之而无怨矣。故曰“有孚于小人”也。——王弼}。君子有解,小人自退。

上六:上六王公爻射隼于高高的城墙之上,获之,无不利。公用射隼,以解悖也。上六疏解之极也,尤有悖逆者,非射杀之不能解也。


\chapter{山泽损卦}
损 {\Large ䷨}
\section{原文}

\subsection{经}
损:有孚,元吉,无咎,可贞,利有攸往。曷\footnote{曷(hé)。曷,何也。——说文。}之用?二簋\footnote{簋(guǐ )。簋,黍稷方器也。——说文。二簋在祭祀中是至简的祭品。}可用享\footnote{享(xiǎng。)享祀,祭祀之意。}。

初九:已\footnote{已(yǐ)。已,成也。——广雅。}事遄\footnote{遄(chuán)。遄,速也。——尔雅。}往,无咎,酌损之。

九二:利贞,征凶,弗损,益之。

六三:三人行则损一人,一人行则得其友。

六四:损其疾,使遄有喜,无咎。

六五:或益之十朋之龟\footnote{成语,古代以贝为货币,五贝一串,两串一朋,十朋就是一百贝,十朋指的是价值不菲。而十朋之龟还有说法:十朋之龟者,一曰神龟,二曰灵龟,三曰摄龟,四曰宝龟,五曰文龟,六曰筮龟,七曰山龟,八曰泽龟,九曰水龟,十曰火龟。——尔雅。这些是占吉凶的十类龟,古代视之为大宝。总之现在十朋之龟这个成语意思就是价值不菲的宝物。},弗克违,元吉。

上九:弗损,益之,无咎,贞吉,利有攸往,得臣无家。

\subsection{彖}
损,损下益上,其道上行。损而有孚,元吉,无咎,可贞,利有攸往。曷之用? 二簋可用享。二簋应有时。损刚益柔有时,损益盈虚,与时偕行\footnote{成语,意思是所作所为当变通趋时之。}。

\subsection{象}
山下有泽,损。君子以惩忿窒欲\footnote{成语,意思是克制住愤怒,抑制住欲望。}。

初九:已事遄往,尚\footnote{尚,上也。——广雅。}合志也。

九二:九二利贞,中以为志也。

六三:一人行,三则疑也。

六四:损其疾,亦可喜也。

六五:六五元吉,自上佑也。

上九:弗损益之,大得志也。

\section{初讲}
卦辞说:损卦,有诚信,大吉祥,无咎,可守正道,利于有所前往。何用呢?二簋就可以用来享祀\footnote{二簋之约可用享祀,言在乎诚而已,诚为本也,天下之害无不由末之胜也。峻宇雕墙本于宫室,酒池肉林本于饮食,淫酷残忍本于刑罚,穷兵黩武本于征讨,凡人欲之过者,皆本于奉养,其流之远则为害矣。先王制其本者,天理也。后人流于末者,人欲也。损之义,损人欲以复天理而已。——程颐。}。

彖说:损卦,减损下而增益上,其道上行\footnote{损之所以为损者,以损于下而益于上也。取下以益上,故云其道上行,夫损上而益下,则为益。损下而益上,则为损。损基本以为高者,岂可谓之益乎。——程颐。}。损卦的有孚,元吉,无咎,可贞,利有攸往。曷之用? 二簋可用享。二簋这样至简的祭品应当有时而为之,损刚益柔这样的行为应当有时而为之\footnote{刚为德长,损之不可以为常也。——王弼。}。是增益还是减损,是盈是虚,这是要与时偕行的\footnote{损下而益上,损刚而益柔,至简之祭,当损得其时,什么时候呢?损卦当运作之时也。}。

象辞说:损卦上卦为艮为山,下卦为兑为泽,山下有泽,这就是损卦的卦象了。君子应当惩忿窒欲\footnote{君子观损之象,以损于己,在修己之道,所当损者唯忿与欲,故以惩戒其忿怒,窒塞其意欲也。——程颐。}。

初九:初九爻将手头上的事做完就速速前往\footnote{此处已字可解为止也可解为成,这里我更偏向于成的意思。因为六四阴爻听从与初九阳爻,主动权在初九手上,况且后又有酌情损之,所以此处解为成,手头上的事做完再去,不那么紧迫更合理一些。},无咎,酌情损己而益六四之。已事遄往,这是因为上面的六四与初九心志相合。

九二:九二爻利于守正道,征行则凶险。不要自损其刚贞以悦六五,反而是对六五有益的\footnote{王弼谈到了刚不可全削,正对应了损卦卦辞的损益盈虚,与时谐行。即使是初九亦要酌情损之。程颐有:守其中乃贞也,弗损益之,不自损其刚贞,则能益其上,乃益之也。大概说明了为何初九损之而九二不能损,因为九二居中正之位,内卦之主,守其中正之德才是最重要的,故象继续解释到中以为志也。}。九二利贞,当以守中正之德以为志也\footnote{九居二非正也,处说非刚也,而得中为善,若守其中德何有不善,岂有中而不正者,岂有中而有过者,二所谓利贞谓以中为志也。志存乎中则自正矣,大率中重于正,中则正矣。正不必中也,能守中则有益于上矣。——程颐。}。

六三:六三与多人同行则将损其中一人,一人独行则将得到[上九]朋友\footnote{若是六三独行则会得到上九这个朋友,因六三不当位,最好不要行损道,自会有上九这个朋友来帮助自己。}。一人行,三则疑也\footnote{这里众说纷纭,也许最简单的解释就是字面意思,即六三与多人同行的情况是存在疑问的,没必要再去强行解释添加细节。}。

六四:六四减损自身疾病,从速则有喜,无咎。损其疾,亦可喜也\footnote{疾谓疾病,不善也。损于不善,唯使之遄(chuán,遄,速也。)速,则有喜而无咎。人之损过,唯患不速,速则不至于深过,为可喜也。——程颐。六四当位当行损道。}。

六五:六五或能增益【获赠】十朋之龟这样的宝物,不能违背了人家的好意,大吉祥。六五元吉,这是来自上天的福佑\footnote{所以得元吉者,以其能尽众人之见,合天地之理,故自上天降之福祐也。——程颐。}。

上九:上九不要损则反而是有益的\footnote{上九与九二皆不当位故都不可行损道,故都有弗损益之这句,但两者是有区别。损道是损下而益上,于九二是不自损,于上九是不损下。}。无咎,守正道吉祥,利于有所前往,得到臣民而无家人\footnote{上九居上不损下是故得到臣民,不损下而不得私利,是故无家人。}。弗损益之,大得志也\footnote{居上不损下而反益之,是君子大得行其志也。君子之志,唯在益于人而已。 ——程颐。}。

\chapter{风雷益卦}
益 {\Large ䷩}
\section{原文}

\subsection{经}
益:利有攸往,利涉大川。

初九:利用为大作,元吉,无咎。

六二:或益之十朋之龟,弗克违,永贞吉。王用享于帝,吉。

六三:益之用\footnote{用,可施行也。——说文。}凶事\footnote{凶事谓患难非常之事。——程颐。},无咎。有孚中行,告公用圭。

六四:中行,告公从,利用为依迁国\footnote{为依,依附于上也。迁国,顺下而动也。...自古国邑民不安,其居则迁,迁国者顺下而动也。 ——程颐。}。

九五:有孚惠心,勿问元吉。有孚惠我德。

上九:莫益之,或击之,立心勿恒,凶。

\subsection{彖}
益,损上益下,民说无疆\footnote{无穷极也——程颐。}。自上下下,其道大光。利有攸往,中正有庆。利涉大川,木道乃行。益动而巽\footnote{巽,具也。——说文。},日进无疆。天施地生,其益无方\footnote{无方所,即无所不在的意思。}。凡益之道,与时偕行。

\subsection{象}
风雷,益。君子以见善则迁,有过则改。

初九:元吉无咎,下不厚事\footnote{事,职也。——说文。}也。

六二:或益之,自外来也。

六三:益用凶事,固有之也。

六四:告公从,以益志也。

九五:有孚惠心,勿问之矣。惠我德,大得志也。

上九:莫益之,偏辞\footnote{此有求而彼不应,是偏辞也。——孔颖达。}也。或击之,自外来也。

\section{初讲}
卦辞说:益卦,利于所有前往,利于克服艰难险阻。

彖说:益卦,减损上而增益下是也,民众悦之无穷极是也。将益处自上降下到下面,益道得以大光显。卦辞上说利于有所前往,是因为六二九五居中得正有福庆\footnote{五以刚阳中正居尊位,二复以中正应之,是以中正之道益天下,天下受其福庆也。——程颐。}。卦辞说利涉大川,是因为益卦上为巽为木,下为震为动,以木道而行动\footnote{木者,以涉大川为常而不溺者也。以益涉难,同乎木也。——王弼。}是以利涉大川。益卦下为震为动,上为巽为具备,行动然后具备,每日进益无穷极也。天施益于地而生万物,这样的益处是无所不在的。益卦的道理,就是要与时偕行。

象辞说:益卦上为巽为风,下为震为雷,风雷相加,这就是益卦的卦象了。君子应当见善则迁,有过则改。

初九:利于用初九爻来进行大作为\footnote{初九,震动之主,刚阳之盛也。居益之时,其才足以益物,虽居至下,而上有六四之大臣应于己。四巽顺之主,上能巽于君,下能顺于贤才也。在下者不能有为也,得在上者应从之,则宜以其道辅于上,作大益天下之事,利用为大作也。——程颐。},大吉祥,无咎。元吉无咎,是因为初九居下并未担任要职。

六二:六二或能收益十朋之龟,不能违背了人家的好意,长久贞正吉祥。[六二此时当效仿周文王],用亨祭祀上帝,吉祥。或益之,益处自外卦各爻而来是也\footnote{既中正虚中,能受天下之善而固守,则有有益之事,众人自外来,益之矣。或曰:自外来,岂非谓五乎。曰:如二之中正虚中,天下孰不顾益之。五为正应,固在其中矣。 ——程颐。}。

六三:六三若于患难非常之事可增益之,无咎\footnote{六三不当位不得益道本不当增益之的。}。有诚信,合于中道而行。告诉六三别忘了用圭之礼\footnote{六三为震之极,有躁动之象,故劝之别忘了用圭之礼节。}。益用凶事,本当如此啊\footnote{六三益之,独可用于凶事者,以其固有之也,谓专固自任其事也。居下当禀承于上,乃专任其事。唯救民之凶灾,拯时之艰急则可也,乃处急难变故之权宜,故得无咎,若平时则不可也。——程颐。}。

六四:六四合于中道而行,告诉六四听从,此时利于为[初九]而顺下而动。告公从,这样来成全益下之志。

九五:九五有至诚施惠之心,不要问自是大吉祥\footnote{五刚阳中正居尊位,又得六二之中正相应,以行其益,何所不利!以阳实在中,有孚之象也。以九五之德之才之位而中心至诚,在惠益于物,其至善大吉,不问可知,故云勿问元吉。——程颐。}。人们至诚感怀九五君王的施惠之德\footnote{有孚惠我德,人君至诚益于天下,天下之人,无不至诚爱戴,以君之德泽为恩惠也。 ——程颐。}。有孚惠心,大吉祥问都不用问了。惠我德,大得其志是也\footnote{人君有至诚惠益天下之心,其元吉不假言也,故云勿问之矣。天下至诚,怀吾德以为惠,是其道大行,人君之志得矣。 ——程颐。}。

上九:不要增益他\footnote{上居无位之地,非行益于人者也。以刚处益之极,求益之甚者也,所应者阴,非取善自益者也。——程颐。},或者还要打击他们,立下心愿不要恒久,凶险\footnote{上九益卦之极,过益是也,然这个凶字还是让人想不到,看来此爻有农夫与蛇之象。}。莫益之,只是上九一厢情愿地要去增益他\footnote{上九不当位,不要去增益他人,这里不要增益主要指的是六三。六三不当位也最好不要受益。}。或击之,或者外卦各爻有人要去增益[六三],甚至还要打击他们。



\chapter{泽天夬(guài)卦}
夬 {\Large ䷪}
\section{原文}

\subsection{经}
夬:扬\footnote{扬,宣扬。——执象易注。}于王庭,孚号有厉。告自邑,不利即戎\footnote{此处很多将即解释为立即,而程颐此处解为:即,从也。从戎,尚武也。这里同意程颐的解法,又论语中有:善人教民七年,亦可以即戎矣。将即戎解释为一个词语也是可以的,即用兵动武之意。},利有攸往。

初九:壮\footnote{此处执象易注解释为通戕,包括姤卦的女壮,我认为解释不通,壮此处当取壮字的原意,即健壮之意,但在周易里面壮还有更多的含义,用来描述一种阳气过盛的现象。}于前趾,往不胜,为咎。

九二:惕号,莫夜\footnote{莫夜,黑夜——执象易注。}有戎,勿恤。

九三:壮于頄\footnote{(qiú),颧骨。},有凶。君子夬夬,独行遇雨,若濡有愠,无咎。

九四:臀无肤,其行次且\footnote{即趑趄(zī jū),想前进又不敢前进的样子。行次且,进不前也;——程颐。}。牵羊悔亡,闻言不信。

九五:苋陆\footnote{(xiàn lù),即商陆。多年生草本,春初发苗,叶卵形而大。——汉典。}夬夬,中行无咎。

上六:无号,终有凶。

\subsection{彖}
夬,决\footnote{决,以泽天卦象来说是溃决之意;继而有:夬,分決[决]也——说文,即决去之意。}也,刚决柔也。健而说,决而和。扬于王庭,柔乘五刚也。孚号有厉,其危乃光也。告自邑,不利即戎,所尚乃穷也。利有攸往,刚长乃终也。

\subsection{象}
泽上于天,夬。君子以施禄及下,居德\footnote{此处居取奇货可居之意,即积存。德,功德,——执象易注。此处居德解释和程颐说法有所不同,不取安处其德之意,而多取积存囤积之意。}则忌。

初九:不胜而往,咎也。

九二:有戎勿恤,得中道也。

九三:君子夬夬,终无咎也。

九四:其行次且,位不当也。闻言不信,聪不明也。

九五:中行无咎,中未光也。

上六:无号之凶,终不可长也。

\section{初讲}
卦辞说:宣扬[上六小人之恶]于王庭之上,真诚呼号[上六小人]很有危险。告知自我封邑\footnote{邑,私邑。告自邑,先自治也。——程颐。},不利于动用武力\footnote{宣扬,呼号,告知皆应了兑卦的口,只动口不动武。},利于有所前往。

彖说:夬,决也,五刚爻决去上六柔爻是也。夬卦下为乾为刚健,上为兑为和悦,是故刚健而喜悦,决去而后能和。宣扬[上六小人之恶]于王庭之上,是因为上六这个柔爻乘五刚爻。真诚呼号[上六小人]很有危险,这样[上六小人的]危险得以光之于众。告知自我封邑,不利于动用武力,因为尚武乃是穷途了\footnote{阳长阴消已是大势所趋,何须穷兵黩武。}。利于有所前往,刚阳盛长决去一阴乃是其终也。

象辞说:夬卦上为兑为泽,下为乾为天,泽上于天,这就是夬卦的卦象了。君子应当施禄泽于下,自己居积功德为忌讳。

初九:初九爻急于行动就像前脚趾很健壮一样,前往并不能收获胜利,为咎也。不胜而往,为咎也\footnote{前趾,谓进行,人之决于行也。行而宜,则其决为是,往而不宜,则决之过也。故往而不胜,则为咎也。——程颐。}。

九二:九二爻心怀畏惧地呼号,夜晚有兵戎,不要忧虑。有戎勿恤,因为九二爻得中道而行\footnote{莫夜有兵戎,可惧之甚也,然可勿恤者,以自处之善也。既得中道又知惕惧,且有戒备,何事之足恤也。——程颐。}。

九三:九三爻颧骨很健壮,有凶。九三君子夬夬\footnote{夬夬,谓夬其夬,果决其断也。——程颐。},其独行遇到了上六小人,下起了雨\footnote{己若以私应之,故不与众同而独行,则与上六阴阳和合,故云遇雨。易中言雨者,皆谓阴阳和也。——程颐。},其他阳爻若见到九三爻衣服沾湿了会很是愠怒,无咎。君子夬夬,终无咎也【九三君子不牵绊于与上六爻的私好,果决其断,终无咎也。】。

九四:九四爻居不安若臀部没有皮肤一般,其行趑趄不前。若和九五相互牵连而行则悔恨渐消\footnote{兑为羊,和王弼观点略有不同,我认为九四九五皆是这里谈论的牵羊中的羊。},只是听到九五的言谈多少有些不相信。其行次且,是因为九四以阳爻居柔位,位不当也。闻言不信,其聪不明也\footnote{其行次且,刚然后能明。处柔则迁失其正性,岂复有明也。——程颐。}。

九五:九五爻[决去上六小人]若苋陆夬夬\footnote{苋陆,柔脆草木,似决去甚易,但其根和种子皆可繁殖,以此比喻小人势力很难根除。九五居尊位决去小人,看上去很容易,却因为和上六相比,需要夬夬,果决其断,才能决去之。},秉中正之道而行则无咎矣。中行无咎,然中正之道还是没有光大啊\footnote{夫人心正意诚,乃能极中正之道,而充实光辉。五心有所比,以义之不可而决之,虽行于外,不失中正之义,可以无咎,然于中道未得为光大也,盖人心一有所欲,则离道矣。——程颐。}。

上六:上六小人呼号也没有用,终必有凶也。无号之凶,终不可长也\footnote{阳刚君子之道,进而益盛。小人之道,既已穷极,自然消亡,岂复能长久乎。虽号咷,无以为也,故云终不可长也。——程颐。}。



\chapter{天风姤(gòu)卦}
姤 {\Large ䷫}

\section{原文}

\subsection{经}
姤:女壮,勿用取女。

初六:系于金柅\footnote{柅:塞于车轮下的制动之木——汉典。},贞吉。有攸往,见凶。羸豕\footnote{羸豕: (léi shǐ)羸弱的猪,汉典中有整词意思为母猪一说,亦可。}孚\footnote{说文解字中有浮,从水。孚聲。此处疑似是文字记录有误,应该为浮躁的浮,而不是诚信的孚。}蹢躅\footnote{蹢躅:(zhí zhú)徘徊不前的样子——汉典。}。

九二:包有鱼,无咎,不利宾。

九三:臀无肤,其行次且\footnote{次且:犹豫不进貌——汉典。},厉,无大咎。

九四:包无鱼,起凶。

九五:以杞包瓜,含章\footnote{同彰。而尧舜之所以章也——吕氏春秋。},有陨自天。

上九:姤其角\footnote{参见晋其角,所以周易里面的角字除了动物角之意外已经有了角落之意。},吝,无咎。


\subsection{彖}
姤,遇也,柔遇刚也。勿用取女,不可与长也。天地相遇,品物咸章也。刚遇中正,天下大行也。姤之时义大矣哉。

\subsection{象}
天下有风,姤。后\footnote{后,继君体也——《说文》}以施命诰四方。

初六:系于金柅,柔道牵也。

九二:包有鱼,义不及\footnote{此处当取给意。}宾也。

九三:其行次且,行未牵也。

九四:无鱼之凶,远民也。

九五:九五含章,中正也。有陨自天,志不舍命也。

上九:姤其角,上穷吝也。


\section{初讲}
卦辞说:姤卦内为巽为长女,外为乾为健壮,年长的女子对外很强壮,不要娶这样的女子啊。

彖说:姤,遇见之意\footnote{姤,遇也。——广雅},初六柔爻遇见五刚爻是也。卦辞上说勿用取女,因为【和这样强势的女子】是不可能与之长久的。天地相遇,万物皆章明。九二刚爻遇到居中而得正的九五爻,其道可大行于天下矣\footnote{以卦才言也,五与二皆以阳刚居中与正,中正相遇也。君得刚中之臣,臣遇中正之君,君臣以刚阳遇中正,其道可以大行于天下矣。——程颐。}。姤卦运作之时意义重大啊\footnote{赞姤之时,与姤之义,至大也。天地不相遇则万物不生,君臣不相遇则政治不兴,圣贤不相遇则道德不亨,事物不相遇则功用不成,姤之时与义皆甚大也。——程颐。}。

象辞说:姤卦上为乾为天,下为巽为风,天下有风,这就是姤卦的卦象了。君王当施发命令并昭告四方。

初六:初六联系九四金柅,【而被坚决制止】,贞吉。有所前往,见凶\footnote{姤,阴始生而将长之卦,一阴生则长而渐盛,阴长则阳消,小人道长也,制之当于其微而未盛之时。柅,止车之物。金,为之坚强之至也。止之以金柅而又繫之,止之固也。固止使不得进,则阳刚贞正之道吉也。使之进往,则渐盛而害于阳,是见凶也。——程颐。}。羸弱的猪在那里浮躁徘徊的样子\footnote{羸豕孚蹢躅,圣人重为之戒,言阴虽甚微,不可忽也。——程颐。}。系于金柅,【为何贞吉】柔道被牵制的缘故是也\footnote{牵者,引而进也。阴始生而渐进,柔道方牵也。系之于金柅,所以止其进也。不使进,则不能消正道,乃贞吉也。——程颐。}。

九二:九二包裹着初六这条鱼,无咎\footnote{二于初,若能固蓄之,如包苴之有鱼,则于遇为无咎矣。宾,外来者也。不利宾,包苴之鱼岂能及宾,谓不可更及外人也。——程颐。},不利于宴请宾客\footnote{言有他人之物,于义不可及宾也。——孔颖达。参考成语及宾有鱼之意:拿他人的鱼请客。比喻利用他人的力量来加强自己势力。}。包有鱼,按照道义是不应该【拿他人之物】来宴请宾客的\footnote{擅人之物,以为已惠,义所不为,故“不利宾”也。——王弼。}。

九三:九三臀部没有皮肤,其行犹豫不前的样子,危险,无大咎\footnote{二与初既相遇,三说初而密比于二,非所安也。又为二所忌,恶其居不安,若臀之无肤也。处既不安则当去之,而居姤之时,志求乎遇,一阴在下,是所欲也,故处虽不安,而其行又次且也。次且,进难之状,谓不能遽舍也。然三刚正而处巽,有终不迷之义,若知其不正,而怀危惧不敢妄动,则可以无大咎也。非义求遇,固巳有咎矣;知危而止,则不至于大也。——程颐。}。其行次且,其行未【受到初六】牵连是也。

九四:九四包裹里没有初六这条鱼,凶变将起也\footnote{四当姤遇之时,居上位而失其下,下之离由己之失德也。四之失者,不中正也,以不中正而失其民,所以凶也。......曰在四而言,义当有咎,不能保其下,由失道也,岂有上不失道而下离者乎。......四以下睽,故主民而言为上而下离,必有凶变起者,将生之谓。民心既离,难将作矣。}。无鱼之凶,是因为九四大臣远离民众【凶将至矣】。

九五:以杞叶包瓜\footnote{杞,高木而叶大。处高体大而可以包物者,杞也。——程颐。似乎古代枸杞和现代品种略有不同,这里认同程颐所言,认为是杞叶,并这个叶子很大可以包瓜。},内心彰明\footnote{九五尊居君位,而下求贤才,以至高而求至下,犹以杞叶而包瓜,能自降屈如此,又其内蕴中正之德充实,章美人君如是,则无有不遇所求者也。——程颐。},必有【贤者】自天落下而和九五君上相遇的\footnote{自古人君至诚降屈,以中正之道求天下之贤,未有不遇者也。高宗感于梦寐,文王遇于渔钓,皆由是道也。 ——程颐。}。九五含章,其内蕴中正之德。有陨自天,是因为其内心之志并不舍弃自身天命。

上九:上九相遇之道穷极若到了死角,这是可吝惜的,无咎\footnote{姤之极也,无遇亦无害,可得无咎二字。}。姤其角,居上之穷极而致吝也\footnote{上九和九五皆相遇甚难,不同的是九五内心蕴含中正之德,故能有陨自天。}。



\chapter{泽地萃(cuì)卦}
萃 {\Large ䷬}
\section{原文}

\subsection{经}
萃,亨。王假\footnote{假,至也,王以聚至有庙也。——王弼。}有庙,利见大人,亨,利贞。用大牲,吉。利有攸往。

初六:有孚不终,乃乱乃萃。若号,一握为笑\footnote{一握,俗语一团也,谓众以为笑也——程颐。},勿恤,往无咎。

六二:引吉,无咎,孚乃利用禴\footnote{(yuè),古同礿。天子四时之祭,春曰礿,夏曰禘,秋曰尝,冬曰烝。——礼记。}。

六三:萃如嗟如,无攸利。往无咎,小吝。

九四:大吉,无咎。

九五:萃有位,无咎。匪孚,元永贞,悔亡。

上六:赍咨涕洟\footnote{赍,(jī)。咨,叹息。赍字汉典里面没有嗟叹之意,只有持有之意,但此处王弼和程颐二家皆解为嗟叹之辞,此解较为合适。洟,(yí)鼻涕。涕,眼泪。},无咎。

\subsection{彖}
萃,聚也。顺以说,刚中而应,故聚也。王假有庙,致孝享\footnote{孝享,祭祀。}也。利见大人亨,聚以正也。用大牲吉,利有攸往,顺天命也。观其所聚,而天地万物之情可见矣。

\subsection{象}
泽上于地,萃。君子以除\footnote{此处除为修治、修整之意。}戎器,戒不虞。

初六:乃乱乃萃,其志乱也。

六二:引吉无咎,中未变也。

六三:往无咎,上巽也。

九四:大吉无咎,位不当也。

九五:萃有位,志未光也。

上六:赍咨涕洟,未安上也。

\section{初讲}
卦辞说:萃卦,亨通。王道至而有庙\footnote{王者萃聚天下之道,至于有庙,极也。——程颐。君王萃聚天下人心之极莫过于入宗庙行祭祀之礼。}。利于见大人,亨通,利贞\footnote{聚得大人,乃得通而利正也。——王弼。}。用盛大的牺牲祭祀,吉祥。利于有所往。

彖说:萃卦,聚集的意思。下为坤为顺,上为兑为悦,下顺而上悦,九五刚爻居中而与六二相正应,故萃聚是也。君王至宗庙,行祭祀之事。利见大人亨,是因为这是聚之以正道。用大牲吉,利有攸往,这是因为顺从了天命\footnote{萃者丰亨之时也,其用宜称,故用大牲。吉事莫重于祭,故以祭享而言,上交鬼神,下接民物,百用莫不皆然。当萃之时,而交物以厚,则是享丰富之吉也。天下莫不同其富乐矣。若时之厚而交物以薄,乃不享其丰美,天下莫之与而悔吝生矣。盖随时之宜,顺理而行,故《彖》云:顺天命也。夫不能有为者,力之不足也。当萃之时,故利有攸往。大凡兴工立事,贵得可为之时,萃而后用,是以动而有裕,天理然也。——程颐。}。观天地万物之所聚,则其情理可见矣\footnote{观萃之理可以见天地万物之情也。天地之化育,万物之生成,凡有者皆聚也。有无动静终始之理,聚散而已,故观其所以聚,则天地万物之情可见矣。——程颐。天地即万物之所聚也。}。

象辞说:萃卦上为兑为泽,下为坤为地,泽于地上,这就是萃卦的卦象了。君子应当除治戎器,以戒备不虞之事。

初六:初六爻有信正应于九四却没有善始善终,乃是上两阴爻惑乱其心乃是与其同类相萃聚是也。若号呼以求正应,上两阴爻皆以为嘲笑\footnote{此处一握之解各家略有不同,不过基本意思指代上两阴爻是一致的。},不要忧恤,前往无咎。乃乱乃萃,其心志为同类所惑乱是也。

六二:六二得九五援引吉祥,无咎,[六二]有诚信,利于用来进行禴之薄祭。引吉无咎,其中正之德未变也。

六三:六三爻萃聚嗟叹的样子\footnote{三阴柔不中正之人也,求萃于人,而人莫与,求四则非其正应,又非其类,是以不正,为四所弃也。与二,则二自以中正应五,是以不正为二所不与也。故欲萃如,则为人弃绝,而嗟如,不获萃而嗟恨也。上下皆不与,无所利也,惟往而从上六,则得其萃,为无咎也。——程颐。},无所利。前往[萃于上六]无咎,小吝\footnote{二阴相合,犹不若一阴一阳之应,故有小吝也。——王弼。}。往无咎,这是因为上[六]巽顺的缘故。

九四:九四当萃之时,得上下之聚\footnote{四当萃之时。上比九五之君,得君臣之聚也。下比下体群阴,得下民之聚也。得上下之聚,可谓善矣。然四以阳居阴,非正也,虽得上下之聚,必得大吉,然后为无咎也。——程颐。},大吉祥,无咎。大吉无咎,然其所在位仍有不当。

九五:九五爻于萃时有此尊位,无咎。若仍有不信服而萃聚者,则修元永贞之德,悔恨自会消亡\footnote{如是而有不信而未归者,则当自反以修其元永贞之德,则无思不服而悔亡矣。元永贞者,君之德民所归也,故比天下之道与萃天下道,皆在此三者。王者既有其位,又有其德,中正无过咎,而天下尚有未信服归附者,盖其道未光大也,元永贞之道未至也。——程颐。}。萃有位,然其[萃聚天下之]志仍未光大\footnote{此处王弼谈到了九四自守而匪孚。}。

上六:上六爻嗟叹而涕洟,无咎\footnote{处聚之时,居於上极,五非所乘,内无应援。处上独立,近远无助,危莫甚焉。赍咨,嗟叹之辞也。若能知危之至,惧祸之深,忧病之甚,至于涕洟,不敢自安,亦无所不害,故得无咎也。——王弼。}。赍咨涕洟,[上六]未安处于上是也。


\chapter{地风升卦}
升 {\Large ䷭}

\section{原文}

\subsection{经}
升:元亨,用见大人,勿恤,南征吉。

初六:允升,大吉。

九二:孚乃利用禴\footnote{(yuè),古同礿。天子四时之祭,春曰礿,夏曰禘,秋曰尝,冬曰烝。——礼记。}\footnote{礿,薄也,春物未成,祭品鲜薄。——执象易注。},无咎。

九三:升虚邑。

六四:王用亨于岐山,吉无咎。

六五:贞吉,升阶。

上六:冥升,利于不息之贞。


\subsection{彖}
柔以时升,巽而顺,刚中而应,是以大亨。用见大人,勿恤,有庆也。 南征吉,志行也。

\subsection{象}
地中生木,升。君子以顺德,积小以高大。

初六:允升大吉,上合志也。

九二:九二之孚,有喜也。

九三:升虚邑,无所疑也。

六四:王用亨于岐山,顺事也。

六五:贞吉升阶,大得志也。

上六:冥升在上,消不富也。

\section{初讲}
卦辞说:升卦,大亨通,用来见大人,勿忧恤,南征\footnote{南,人之所向。南征,谓前进也。——程颐。}\footnote{坤居西南为得众,巽居东南,离居南为文明,南征,往文明征进,王弼此处谈丽乎大明,亦即指离卦也。}吉祥。

彖说:柔爻以时而升\footnote{以二体言。柔升,谓坤上行也。巽既体卑而就下,坤乃顺时而上,升以时也,谓时当升也。——程颐。},升卦下为巽,上为坤为顺,巽而顺,九二刚爻居中而与六五相应,是故大亨通\footnote{纯柔则不能自升,刚亢则物不从。既以时升,又巽而顺,刚中而应,以此而升,故得大亨。——王弼。}。用见大人,勿恤,有庆贺之事\footnote{凡升之道,必由大人。升于位则由王公,升于道则由圣贤。用巽顺刚中之道,以见大人,必遂其升。勿恤,不忧其不遂也。遂其升则己之福庆,而福庆及物也。——程颐。 }。 南征吉,其志得行也。

象辞说:升卦上为坤为地,下为巽为木,地中生木,这就是升卦的卦象了。君子应当以巽顺之德,积小以成高大\footnote{木生地中,长而上升,为升之象。君子观生之象,以顺修其德,积累微小以至高大也,顺则可进,逆乃退也。万物之进皆以顺道也,善不积不足以成名,学业之充实,道德之崇高,皆由积累而至。积小所以成高大,升之义也。——程颐。}。

初六:初六应允九二而升,大吉祥\footnote{初之柔巽,唯信从于二,信二而从之同升,乃大吉也。——程颐。}。允升大吉,与上者[九二]合志也。

九二:九二有诚信,利于用来进行禴之薄祭,无咎\footnote{二阳刚而在下,五阴柔而居上。夫以刚而事柔,以阳而从阴,虽有时而然,非顺道也。以暗而临明,以刚而事弱,若黾【(mĭn )勉也。】勉于事势,非诚服也。上下之交不以诚,其可久乎?其可以有为乎?五虽阴柔,然居尊位。二虽刚阳,事上者也,当内存至诚,不假文饰于外,诚积于中,则自不事外饰,故曰「利用禴」,谓尚诚敬也。——程颐。}。九二之孚,有可喜之事也。

九三:九三上升如同入无人之邑\footnote{三以阳刚之才,正而且巽,上皆顺之,复有援应,以是而升,如入无人之邑,孰御哉。——程颐。}。升虚邑,其进无疑阻也。

六四:文王用亨祭于岐山,吉祥,无咎\footnote{四柔顺之才,上顺君之升,下顺下之进,己则止其所焉。以阴居柔,阴而在下,止其所也。昔者,文王之居岐山之下,上顺天子而欲致之有道,下顺天下之贤而使之升进,己则柔顺谦恭不出其位。至德如此,周之王业用是而亨也。四能如是,则亨而吉且无咎矣。四之才固自善矣,复有无咎之辞,何也?曰:四之才虽善,而其位当戒也。居近君之位,在升之时,不可复升,升则凶咎可知,故云,如文王则吉而无咎也。然处大臣之位,不得无事。于升当上升其君之道,下升天下之贤,己则止其分焉。分虽当止而德则当升也,道则当亨也。尽斯道者,其唯文王乎。——程颐。}。王用亨于岐山,顺事也\footnote{四居近君之位而当升时,得吉而无咎者,以其有顺德也。以柔居坤,顺之至也。文王之亨于岐山,亦以顺时而已。上顺于上,下顺乎下,己顺处其义,故云:顺事也。——程颐。}。

六五:六五守正道吉祥,于天子之阶\footnote{阶,陛也。——说文。}上升\footnote{五以下有刚中之应,故能居尊位而吉。然质本阴柔,必守贞固乃得其吉也。若不能贞固,则信贤不笃,任贤不终,安能吉也。——程颐。}。贞吉升阶,大得其志也\footnote{倚任贤才而能贞固如是而升,可以致天下之大治,其志可大得也。君道之升,患无贤才之助尔,有助则犹自阶而升也。——程颐。}。

上六:上六昏冥于升,利于不停息地守正道\footnote{六以阴居升之极,昏冥于升,知进而不知止者也,其为不明甚矣。然求升不已之心,有时而用,于贞正而当不息之事则为宜矣。君子于贞正之德,终日乾乾,自强不息,如上六不已之心,用之于此则利也。——程颐。}。冥升在上,其将消亡而不富也\footnote{昏冥于升,极上而不知已,唯有消亡,岂复有加益也。不富,无复增益也。升既极,则有退而无进也。——程颐。}。


\chapter{泽水困卦}
困 {\Large ䷮}

\section{原文}

\subsection{经}
困:亨,贞大人吉\footnote{此处有把贞字分开的,参考彖的贞大人吉,此处决定不分开。},无咎。有言不信。

初六:臀困于株木\footnote{守株待兔之株意。株,木根也——说文。在土曰根,在土上曰株。株木,无枝叶之木也。——程颐。},入于幽谷,三岁不觌\footnote{(dí),相见。}。

九二:困于酒食,朱绂\footnote{(fú),朱绂,古代礼服上的红色蔽膝。}\footnote{朱绂,王者之服,蔽膝也,以行来为义,故以蔽膝言之。——程颐。}方来,利用亨祀,征凶,无咎。

六三:困于石,据于蒺藜\footnote{( jí lí),一种匍匐于地的带刺植物。据,根据地之据意。},入于其宫,不见其妻,凶。

九四:来徐徐\footnote{徐徐而来。},困于金车,吝,有终。

九五:劓\footnote{(yì),古代割鼻的酷刑。}刖\footnote{(yuè),古代砍腿的酷刑。},困于赤绂\footnote{天子朱绂,诸侯赤绂——说文。},乃徐有说,利用祭祀。

上六:困于葛藟\footnote{(gě lěi),一种葡萄科植物。},于臲卼\footnote{(niè wù),动摇不安的样子。},曰动悔。有悔,征吉。

\subsection{彖}
困,刚掩也。险以说,困而不失其所亨\footnote{此处有将亨字分开的,多方考证,王弼,程颐,东坡易传等皆是不分开。},其唯君子乎?贞大人吉,以刚中也。有言不信,尚口乃穷也。

\subsection{象}
泽无水,困。君子以致命遂志。

初六:入于幽谷,幽不明也。

九二:困于酒食,中有庆也。

六三:据于蒺藜,乘刚也。入于其宫,不见其妻,不祥也。

九四:来徐徐,志在下也。虽不当位,有与也。

九五:劓刖,志未得也。乃徐有说,以中直也。利用祭祀,受福也。

上六:困于葛藟,未当也。动悔有悔,吉行也。

\section{初讲}
卦辞说:困卦,亨通,守正的大人吉祥,无咎。有言而人不信\footnote{当困而言,人谁信之——程颐。}。

彖说:困卦,刚爻被柔爻所掩蔽\footnote{刚阳君子而为阴柔小人所掩蔽,君子之道窒之时也。——程颐。}。下为坎为险,上为兑为说,处坎险之地而悦,处困之时而不失所亨,唯有君子才能这样吧\footnote{时虽困也,处不失义,则其道自亨,困而不失其所亨也。能如是者,其唯君子乎。——程颐。}?守正的大人吉祥,是因为九二九五这两刚爻居中的缘故。有言而人不信,徒尚口说就会陷入穷途。

象辞说:困卦下为坎为水,上为兑为泽,水在泽下,泽无水也。君子应当致命遂志\footnote{君子之处困也,命在天而致之,志在我则遂之,困而安于困者,命之致也。困而有不困者,志之遂也。——冯当可。}\footnote{程颐此处解为当知命之意,这里不太同意,应取致命的献命之意。}。


初六:初六屁股困坐于木桩之上,[欲见九四而]入于幽谷之中,多年不得相见。入于幽谷,前路幽暗而不明啊。

九二:九二困于酒食之事,[九五送来的]朱绂才来,利于用来进行祭祀之事,征凶\footnote{与九五并非正应,不可征进。},无咎。困于酒食,只要九二守其中正之道则必有喜庆之事\footnote{虽困于所欲,未能施惠于人,然守其刚中之德,必能致亨而有福庆也。——程颐。}。

六三:六三困于石头之中,据于蒺藜之上,进入其宫而不见其妻子,凶。据于蒺藜,是指六三柔爻乘九二刚爻。入于其宫,不见其妻,不祥之兆啊\footnote{困于刚爻之中,进退不得,而与上六无正应,凶险。}。

九四:九四与初六相应但却徐徐而来,因困于九二金车\footnote{此处程颐和王弼皆解为困于金车指欲见初六而被九二所阻。}。吝,有终。来徐徐,其志欲与下面的初六相应是也。虽不当位,但还是有初六这个相与\footnote{四应于初,而隔于二。志在下求,故徐徐而来。虽居不当位,为未善,然其正应相与,故有终也。 ——程颐。}。

九五:九五劓刖之象\footnote{截鼻曰劓,伤于上也。去足为刖,伤于下也。上下皆揜于阴,为其伤害,劓刖之象也。——程颐。},困于诸侯臣子之中,而后慢慢有喜悦之事。利于用来进行祭祀之事。劓刖,刚开始并不是得志之时。乃徐有说,是因为九五坚守中直之道\footnote{始为阴掩,无上下之与,方困未得志之时也。徐而有说,以中直之道,得在下之贤,其济于困也。——程颐。}。利用祭祀,而后享受其福是也\footnote{然以求天下之贤,则能亨天下之困,而享受其福庆也。——程颐。}。

上六:上六处久困之时,若困于葛藟,动摇不安的样子,说动辄就有悔恨之事发生,有悔恨,则征进是吉祥的\footnote{曰如是动,皆得悔。当变前之所为,有悔也。能悔,则往而得吉也。——程颐。}。困于葛藟,是之前所处不当的缘故\footnote{所处未当,故致此困也。——王弼。}。动悔有悔,吉行也\footnote{知动则得悔,遂有悔而去之,可出于困,是其行而吉也。——程颐。}。


\chapter{水风井卦}
井 {\Large ䷯}

\section{原文}

\subsection{经}
井:改邑不改井,无丧无得,往来井井。汔至亦未繘井\footnote{qì,汔,水涸也。——说文。繘(jú):井上的汲绳。繘井此处应解为挖井,古代的井是需要时常维护的,井上的汲绳平时确实主要用于汲取井水,但另一个更重要的作用就是挖去井泥。其他按照汲取井水的解释实在算不上什么大事,一次没打上来,再打一次就是了,配不上后面的凶字。},羸\footnote{léi,羸,瘦也。——说文。}其瓶,凶。

初六:井泥不食,旧井无禽。

九二:井谷射鲋,瓮敝漏。

九三:井渫不食,为我心恻,可用汲,王明,并受其福。

六四:井甃,无咎。

九五:井冽,寒泉,食。

上六:井收,勿幕,有孚,元吉。

\subsection{彖}
巽乎水而上水,井,井养而不穷也。改邑不改井,乃以刚中也。汔至亦未繘井,未有功也。羸其瓶,是以凶也。

\subsection{象}
木上有水,井。君子以劳民劝相。

初六:井泥不食,下也。旧井无禽,时舍也。

九二:井谷射鲋,无与也。

九三:井渫不食,行恻也。求王明,受福也。

六四:井甃无咎,修井也。

九五:寒泉之食,中正也。

上六:元吉在上,大成也。

\section{初讲}
卦辞说:井卦,国邑所属变动而市井旁居住的小民不变,既没有减少也没有增加,在井水旁来来往往井井有条的样子。井水将干涸也没有去挖井,让储水用瓶羸弱空虚,凶险。

彖说:

象辞说:

初六:

九二:

九三:

六四:

九五:

上六:



\chapter{泽火革卦}
革 {\Large ䷰}

\section{原文}

\subsection{经}
革:巳日乃孚\footnote{此处巳日的巳字汉典并未给出合适的解释,从程颐的“故必巳日”看来古代巳日就是一个他们熟知的词语,意思是还要过段时日。故此处我认为巳字可通现代的伺,巳日即伺日,需等候一段时日的意思。},元亨利贞,悔亡。

初九:巩用黄牛之革。

六二:巳日乃革之,征吉,无咎。

九三:征凶,贞厉,革言三就,有孚。

九四:悔亡,有孚,改命吉。

九五:大人虎变,未占有孚。

上六:君子豹变,小人革面,征凶,居贞吉。

\subsection{彖}
革,水火相息\footnote{此处息通熄。},二女同居,其志不相得,曰革。巳日乃孚,革而信之。文明以说,大亨以正,革而当,其悔乃亡。天地革而四时成,汤武革命,顺乎天而应乎人,革之时大矣哉!

\subsection{象}
泽中有火,革。君子以治历明时。

初九:巩用黄牛,不可以有为也。

六二:巳日革之,行有嘉也。

九三:革言三就,又何之\footnote{之,往也。}矣。

九四:改命之吉,信志也。

九五:大人虎变,其文炳也。

上六:君子豹变,其文蔚也。小人革面,顺以从君也。

\section{初讲}
卦辞说:革卦,要等待一些时日人们才会相信\footnote{革者,变其故也。变其故,则人未能遽信,故必巳日,然后人心信从。——程颐。},大亨通,利于正道,悔恨消失。

彖说:革卦上为兑为泽,下为离为火,泽水润下,离火炎上,故曰水火相互熄灭。兑为少女离为中女,故曰两女同居。如此上下之志皆不相得,故要变革。需要等待一些时日人们才会相信,这是因为只有变革之后人们才会相信。离为文明兑为愉悦,故变革之道是文明事理而人心愉悦的,大亨通于正道,变革得当,则悔恨消亡\footnote{以卦才言革之道也。离为文明,兑为说。文明则理无不尽,事无不察;说则人心和顺。革而能照察事理,和顺人心,可致大亨,而得贞正。如是,变革得其至当,故悔亡也。天下之事,革之不得其道,则反致弊害,故革有悔之道。惟革之至当,则新旧之悔皆亡也。——程颐。}。天地变革而四时形成,商汤王周文武王革命,是上顺天命而下应人心,革卦运作之时很重大啊\footnote{推革之道,极乎天地变易,时运终始也。天地阴阳推迁改易而成四时,万物于是生长成终,各得其宜,革而后四时成也。时运既终,必有革而新之者。王者之兴,受命于天,故易世谓之革命。汤武之王,上顺天命,下应人心,顺乎天而应乎人也。天道变改,世故迁易,革之至大也,故赞之曰,革之时大矣哉!——程颐。}。

象辞说:革卦上为兑为泽,下为离为火,泽中有火,这就是革卦的卦象了。君子应当修治历史而明时世变化\footnote{此处历字多有解为历法的,从卦辞的天地革和汤武革命来看,此处当解为历史;此处时字亦多有解为四时的,也是小了。我这里解为历史和时世变化从含义来说也有点小,但勉强够了,更合适的解是治所历之事而明变革之时的要义。只是这个解所谈有点抽象,放在这里可作参考。}。

初九:初九当用黄牛之革[中顺之道]自我巩固\footnote{九,以时则初也,动于事初,则无审慎之意,而有躁易之象;以位则下也,无时无援而动于下,则有僭妄之咎,而无体势之重;以才则离体而阳也,离性上而刚体健,皆速于动也。其才如此,有为则凶咎至矣,盖刚不中而体躁,所不足者中与顺也,当以中顺自固而无妄动则可也。巩,局束也。革,所以包束。黄,中色。牛,顺物。巩用黄牛之革,谓以中顺之道自固,不妄动也。——程颐。}。巩用黄牛,不可以有为也。


六二:六二等候一些时日然后变革之,征进吉祥,无咎\footnote{阴之为物,不能先唱,顺从者也。不能自革,革已乃能从之,故曰“巳日乃革之”也。二与五虽有水火殊体之异,同处厥中,阴阳相应,往必合志不忧咎也,是以征吉而无咎——王弼。}。巳日革之,行则有嘉庆也。

九三:九三征进凶险,贞固自守危险,有关变革的言论多次写就,为人所信\footnote{革言,谓当革之论。就,成也,合也。审察当革之言,至于三而皆合,则可信也。言重慎之至能如是,则必得至当乃有孚也。己可信而众所信也,如此则可以革矣。在革之时,居下之上,事之当革,若畏惧而不为,则失时为害;唯当慎重之至,不自任其刚明,审稽公论,至于三就而后革之,则无过矣。——程颐。}。革言三就,还能往哪里去呢。

九四:九四悔恨消亡,有诚信,改命吉祥。改命之吉,信志而行之也。

九五:九五大人变革如同虎皮之纹理,未占卜而有人信\footnote{九五以刚阳之才,中正之德,居尊位,大人也。以大人之道,革天下之事,无不当也,无不时也。所过变化,事理炳著,如虎之文采,故云虎变。龙虎,大人之象也。变者,事物之变。曰虎,何也?曰:大人变之,乃大人之变也。以大人中正之道变革之,炳然昭著,不待占决,知其至当而天下必信也。天下蒙大人之革,不待占决,知其至当而信之也。——程颐。}。大人虎变,若虎纹炳焕明盛也。

上六:上六君子变革如同豹皮之纹理,小人亦洗心革面之,征进凶险,安居守正吉祥\footnote{故至革之终而又征,则凶也,当贞固以自守。革至于极,而不守以贞,则所革随复变矣。天下之事,始则患乎难革,已革则患乎不能守也,故革之终戒以居贞则吉也。——程颐。}。君子豹变,若豹纹彬蔚。小人革面,顺从君上是也。


\chapter{火风鼎卦}
鼎 {\large ䷱}

\section{原文}

\subsection{经}
鼎:元吉,亨。

初六:鼎颠趾,利出否\footnote{否,不善之物也。},得妾\footnote{正室虽亡,妾犹不得为室主。妾为室主,亦犹鼎之颠趾而有咎过。若有贤子,则以子贵。以之继室,则得无咎——孔颖达。}以其子,无咎。

九二:鼎有实,我仇\footnote{通逑,即窈窕淑女,君子好逑的逑。}有疾,不我能即\footnote{即,就也。不我能即可解为不能即我。},吉。

九三:鼎耳革,其行塞,雉膏\footnote{肥美的野鸡肉,取泛指美味之意也可。}不食,方雨亏悔,终吉。

九四:鼎折足,覆公餗\footnote{(sù),鼎中的食物。此处公有解为公上之意的,但个人认为此处当取众,即大家的之意。},其形渥\footnote{(wò),渥,沾濡之貌也。——王弼。},凶。

六五:鼎黄耳金铉,利贞。

上九:鼎玉铉,大吉,无不利。

\subsection{彖}
鼎,象也。以木巽\footnote{巽,入也。}火,亨饪也。圣人亨以享上帝,而大亨以养圣贤。巽而耳目聪明,柔进而上行,得中而应乎刚,是以元亨。

\subsection{象}
木上有火,鼎。君子以正位凝命。

初六:鼎颠趾,未悖也。利出否,以从贵也。

九二:鼎有实,慎所之也。我仇有疾,终无尤也。

九三:鼎耳革,失其义也。

九四:覆公餗,信如何也。

六五:鼎黄耳,中以为实也。

上九:玉铉在上,刚柔节也。

\section{初讲}
卦辞说:鼎卦,大吉祥,亨通\footnote{革去故而鼎取新,取新而当其人,易故而法制齐明,吉然后乃亨,故先元吉而后亨也。——王弼。}。

彖说:鼎卦,鼎器之象是也。鼎卦下为巽为木,上为离为火,将木放入火中,是要进行烹饪之事是也。圣人烹饪以祭享上帝,继而大\footnote{大,言其广。}烹以养圣贤。巽顺而又耳目聪明\footnote{上既言鼎之用矣,复以卦才言人。能如卦之才,可以致元亨也。下体巽,为巽顺于理,离明而中虚于上,为耳目聪明之象。——程颐。},六五柔爻进而居上尊位行之,得中位而与九二刚爻相应,是以能大亨通\footnote{以明居尊而得中道,应乎刚能用刚,阳之道也。五居中而又以柔而应刚,为得中道。其才如是,所以能元亨也。}。

象辞说:鼎卦下为巽为木,上为离为火,木上有火,这就是鼎卦的卦象了。君子应当端正自己的位置,凝重自己的使命\footnote{鼎者法象之器,其形端正,其体安重,取其端正之象,则以正其位,谓正其所居之位。取其安重之象,则凝其命令,安重其命令也。——程颐。}。


初六:初六若鼎之趾,[因上应于九四而让]鼎覆趾颠,利于倾出不善之物,得到初六这个小妾为正室,因为其生了九四这个儿子,无咎。鼎颠趾,看似悖于常理实则未然也\footnote{鼎颠而趾颠,悖道也。然非必为悖者,盖有倾出否恶之时也。——程颐。}。利出否,以上从九四贵人是也。

九二:九二刚爻居于内卦之中若鼎中有实,和我匹配的六五\footnote{此处并不同意程颐的初六之说,相比作用并不大,当以正应为主。}有乘阳之疾,不能与我接近,吉祥。鼎有实,当慎重选择所居之处。我仇有疾,终无怨尤是也。

九三:九三[与上九鼎铉不相正应]若鼎耳革变走样,不能过铉从而使鼎行动滞塞\footnote{而三处下体之上,以阳居阳,守实无应,无所纳受。耳宜空以待铉,而反全其实塞,故曰鼎耳革,其行塞——王弼。},虽有美味却不得食用,将有雨下然后悔恨有所减少,终吉\footnote{方雨,且将雨也,言五与三方将和合,亏悔终吉,谓不足之悔终当获吉也。——程颐。}。鼎耳革,也就是失去了其承载鼎铉之义了。

九四:九四[与初六正应]若鼎折其足\footnote{四下应于初,初阴柔小人,不可用者也,而四用之,其不胜任而败事,犹鼎之折足也。——程颐。},倾覆了鼎中大家的食物,鼎体渥沾,凶\footnote{居大臣之位,当天下之任而所用非人,至于覆败,乃不胜其任,可羞愧之甚也。其形渥,谓赧污也,其凶可知。系辞曰:德薄而位尊,知小而谋大,力小而任重,鲜不及矣。言不胜其任也,蔽于所私,德薄知小也。——程颐。}。覆公餗,其信誉又将如何呢\footnote{大臣当天下之任,必能成天下之治,安则不误君上之所倚,下民之所望,与己致身任道之志不失所,斯乃所谓信也,不然则失其职,误上之委任,得为信乎?故曰:信如何也。——程颐。}。

六五:六五若鼎黄耳,戴金铉,利贞\footnote{五在鼎上,耳之象也。鼎之举措在耳,为鼎之主也。五有中德,故云黄耳,铉加耳者也。二应于五,来从于耳者,铉也,二有刚中之德,阳体刚中色黄,故为金铉。——程颐。}。鼎黄耳,是因为六五以中为实德也。

上九:上九若鼎之玉铉,大吉,无不利。玉铉在上,刚柔节也\footnote{处鼎之终,鼎道之成也。居鼎之成,体刚履柔,用劲施铉,以斯处上,高不诫亢,得夫刚柔之节,能举其任者也。应不在一,则应所不举,故曰大吉,无不利也。——王弼。}。




\chapter{震卦}
震 {\large ䷲}
\section{原文}

\subsection{经}
震:亨。震来虩虩,笑言哑哑。震惊百里,不丧匕鬯。

初九:震来虩虩,后笑言哑哑,吉。

六二:震来厉,亿丧贝,跻于九陵,勿逐,七日得。

六三:震苏苏,震行无眚。

九四:震遂泥。

六五:震往来厉,亿无丧,有事。

上六:震索索,视矍矍,征凶。震不于其躬,于其邻,无咎。婚媾有言。

\subsection{彖}
震,亨。震来虩虩,恐致福也。笑言哑哑,后有则也。震惊百里,惊远而惧迩也。 出可以守宗庙社稷,以为祭主也。

\subsection{象}
洊雷,震。君子以恐惧修省。

初九:震来虩虩,恐致福也。笑言哑哑,后有则也。

六二:震来厉,乘刚也。

六三:震苏苏,位不当也。

九四:震遂泥,未光也。

六五:震往来厉,危行也。其事在中,大无丧也。

上六:震索索,中未得也。虽凶无咎,畏邻戒也。


\section{初讲}
卦辞说:

彖说:

象辞说:



\chapter{艮卦}
艮 {\Large ䷳}


\section{原文}

\subsection{经}
艮:艮\footnote{艮者,止也。}其背,不获其身,行其庭,不见其人,无咎。

初六:艮其趾,无咎,利永贞。

六二:艮其腓\footnote{(féi),小腿肚子肉。},不拯\footnote{拯,举也。——孔颖达。}其随,其心不快。

九三:艮其限\footnote{上下身之界。},列\footnote{通裂。}其夤\footnote{(yín),夹脊肉。},厉薰心。

六四:艮其身,无咎。

六五:艮其辅\footnote{面颊部位。},言有序,悔亡。

上九:敦艮,吉。


\subsection{彖}
艮,止也。时止则止,时行则行,动静不失其时,其道光明。艮其背\footnote{朱熹、俞樾、朱骏声等认为“艮其止”当作“艮其背”。背,古本作“北”。帛《易》卦辞也作“北”。北、止二字字形相近,恐转抄“北”为“止”。故当以“背”为是。——\href{https://www.vsucai.cn/yizhuan/content-52.html}{参考网页} ,经过分析我也是同意这个说法的,所以此处就直接修改为艮其背了。},止其所也。上下敌应,不相与也。是以不获其身,行其庭不见其人,无咎也。

\subsection{象}
兼山,艮。君子以思不出其位。

初六:艮其趾,未失正也。

六二:不拯其随,未退听也。

九三:艮其限,危薰心也。

六四:艮其身,止诸躬也。

六五:艮其辅,以中正也。

上九:敦艮之吉,以厚终也。

\section{初讲}
卦辞说:艮止在背,不得见艮止之物\footnote{人之所以不能安其止者,动于欲也。 欲牵于前,而求其止不可得也。 故艮之道,当艮其背,所见者在前,而背乃背之,是所不见也。止于所不见,则无欲以乱其心,而止乃安。——程颐。}。行走于艮止之物的庭院之间,亦不见其人踪迹,无咎\footnote{行其庭,不见其人,庭除之间,至近也。在背则虽至近不见,谓不交于物也。外物不接,内欲不萌,如是而止,乃得止之道于止,为无咎也。——程颐。}。

彖说:艮是静止的意思。该静止之时则静止,该行动之时则行动,动静不失其时,其道光明。艮止在背,止得其所是也。艮卦上下六爻皆不相应,不相亲与\footnote{各不相应,各相其背,而得其止。}。是故不获其身,行其庭不见其人,无咎是也\footnote{相背,故不获其身。不见其人,是以能止。能止,则无咎也。——程颐。}。

象辞说:艮卦上为艮为山,下为艮为山,两山相并,这就是艮卦的卦象了。君子应当思想不超出其所处之位。

初六:初六艮止在脚趾,无咎,利永守贞正之道\footnote{艮其趾,止于动之初也。事止于初,未至失正,故无咎也。以柔处下,当趾之时也,行则失其正矣,故止乃无咎。阴柔患其不能常也,不能固也,故方止之初,戒以利在常永贞固,则不失止之道也。——程颐。}。初六艮其趾,其行还未失正。

六二:六二艮止在腓,但却不能拯举其随初六,其心不快。不拯其随,是因为初六并未退听顺从于六二是也。

九三:九三艮止在限,撕裂其夹脊肉,危厉薰心。艮其限,危薰心也\footnote{三以刚居刚而不中,为成艮之主,决止之极也。已在下体之上而隔上下之限,皆为止义,故为艮其限,是确乎止而不复能进退者也。在人身如列其夤。夤,膂也,上下之际也。列绝其夤,则上下不相从属,言止于下之坚也。止道贵乎得宜,行止不能以时而定于一,其坚强如此,则处世乖戾,与物睽绝,其危甚矣。人之固止一隅而举世莫与宜者,则艰蹇忿畏焚挠其中,岂有安裕之理?厉薰心,谓不安之势,薰烁其中也。}。

六四:六四艮止在身,无咎\footnote{中上称身,履得其位,止求诸身,得其所处,故不陷于咎也。——王弼。}。艮其身,六四自止其身是也。

六五:六五艮止在辅,言语有条有理,悔亡。艮其辅,是因为六五能以得中为正道\footnote{五之所善者,中也。艮其辅,谓止于中也。言以得中为正。止之于辅,使不失中,乃得正也。——程颐。}。

上九:上九敦厚地艮止,吉祥。敦艮之吉,是因为上九能以敦厚至终\footnote{天下之事,唯终守之为难。能敦于止,有终者也。上之吉,以其能厚于终也。——程颐。}。


\chapter{风山渐卦}
渐 {\Large ䷴}


\section{原文}

\subsection{经}
渐:女归吉,利贞。

初六:鸿渐于干,小子厉,有言,无咎。

六二:鸿渐于磐,饮食衎衎,吉。

九三:鸿渐于陆,夫征不复,妇孕不育,凶。利御寇。

六四:鸿渐于木,或得其桷,无咎。

九五:鸿渐于陵,妇三岁不孕,终莫之胜,吉。

上九:鸿渐于陆,其羽可用为仪,吉。

\subsection{彖}
渐之进也,女归吉也。进得位,往有功也。进以正,可以正邦也。其位刚得中也。止而巽,动不穷也。

\subsection{象}
山上有木,渐。君子以居贤德善俗。

初六:小子之厉,义无咎也。

六二:饮食衎衎,不素饱也。

九三:夫征不复,离群丑也。妇孕不育,失其道也。利用御寇,顺相保也。

六四:或得其桷,顺以巽也。

九五:终莫之胜吉,得所愿也。

上九:其羽可用为仪吉,不可乱也。


\section{初讲}
卦辞说:

彖说:

象辞说:

\chapter{雷泽归妹卦}
归妹 {\Large ䷵}


\section{原文}

\subsection{经}
归妹\footnote{女子以夫为家,在男曰娶,在女曰归。——高岛断易。妹者,少女之称也。——王弼。归妹即少女出嫁之意也。}:征凶,无攸利。

初九:归妹以娣\footnote{娣,(dì),姐妹同嫁一夫,姐姐称姒,妹妹谓娣。——执象易注。},跛能履,征吉。

九二:眇\footnote{眇,(miǎo),一目小也——《说文》}能视,利幽人之贞。

六三:归妹以须\footnote{须,侍也。——尔雅。},反归以娣。

九四:归妹愆期,迟归有时。

六五:帝乙\footnote{帝乙,商朝第三十代君王,文丁之子。文丁诛杀季历,商周交恶,为化仇隙,帝乙嫁妹于姬昌,是帝乙归妹之事。——执象易注。}归妹,其君之袂\footnote{(mèi),衣袖。},不如其娣之袂良,月几望,吉。

上六:女承\footnote{承,奉也。受也。——说文。自下受上称承。——虞注。}筐无实,士刲\footnote{(kuī),宰杀。}羊无血,无攸利。

\subsection{彖}
归妹,天地之大义也\footnote{归妹,女归于男也。故云天地之大义也。男在女上,阴从阳动,故为女归之象。——程颐。兑少女也,震长男也。}。天地不交而万物不兴,归妹人之终始也。说以动,所归妹也。征凶,位不当也。无攸利,柔乘刚也。

\subsection{象}
泽上有雷,归妹。君子以永终知敝。

初九:归妹以娣,以恒\footnote{恒,常也。——说文。}也。跛能履吉,相承也。

九二:利幽人之贞,未变常也。

六三:归妹以须,未当也。

九四:愆期之志,有待而行也。

六五:帝乙归妹,不如其娣之袂良也。其位在中,以贵行也。

上六:上六无实,承虚筐也。


\section{初讲}
卦辞说:归妹卦,征凶,无所利。

彖说:归妹,天地之大义也。天地[阴阳之气]不交则万物不兴,归妹,人伦终始相续之道是也\footnote{男女交,而后有生息,有生息而后其终不穷。前者有终,而后者有始,相续不穷,是人之终始也。——程颐。}。归妹卦下为兑为悦,上为震为动,愉悦地行动,这就是归妹卦了。征凶,是因为九四爻不当位。无所利,是因为六三爻以柔爻乘刚爻之上\footnote{不当位,诸家多解为中间的四爻皆不当位,但彖辞在谈及爻位常常是有所侧重的,这里以震卦之主九四爻言之。柔乘刚这里以兑卦之主六三爻言之。}。

象辞说:归妹卦下为兑为泽,上为震为雷,泽上有雷,这就是归妹卦的卦象了。君子观此卦象而知道永终知敝\footnote{永终,谓生息嗣续,永久其传也。知敝,谓知物有敝坏,而为相继之道也。……天下之事,莫不有终、有敝,莫不有可继、可久之道。观归妹,则当思永终之戒也。——程颐。知敝者一敝坏自在相继之道之中;二相继之道一样也可能会发生敝坏。}的道理。

初九:初九爻好比归妹中随同的娣,虽跛脚但能走路,征吉。归妹以娣,以此为常也。跛能履吉,相承助其君是也\footnote{九乃刚阳有贤贞之德,虽娣之微,乃能以常者也。虽在下,不能有所为,如跛者之能履。然征而吉者,以其能相承助也,能助其君,娣之吉也。——程颐。}。

九二:九二爻眼睛有病但勉强能看,利幽人之贞\footnote{九二阳刚而得中,女之贤正者也。……[六]五虽不正,二自守其幽静,贞正乃所利也。——程颐。}。利幽人之贞,未失夫妇常正之道是也\footnote{守其幽贞,未失夫妇常正之道也。——程颐。}。

六三:六三爻好比归妹中随同的侍女,[想取悦而求归,但并无正应,最后只好]返回而以娣的身份归妹。归妹以须,六三爻未得其当啊\footnote{六三失位、乘刚、无应,位不当也。——执象易注。}。

九四:九四爻归妹的日期耽误了,归妹的时候虽然来迟了些但总会来的。愆期之志,是因为九四爻有所等待再行动是也\footnote{九以阳居四,四上体地之高也。阳刚在女子为正德贤明者也。无正应,未得其归也。过时未归,故云愆期。女子居贵高之地,有贤明之资,人情所愿娶,故其愆期乃为有时,盖自有待,非不售也。待得佳配,而后行也。——程颐。}。

六五:帝乙归妹,女君的衣袖服饰不如其娣的精良,月亮快到望日的时候,吉祥。帝乙归妹,不如其娣之袂良也。是因为六五爻其位在中,以尊贵而行中道是也\footnote{柔顺降屈,尚礼而不尚饰,乃中道也。——程颐。}。

上六:女子承着空筐,士夫宰杀羊却没有血,无所利。上六无实,是因为上六爻承着空筐是也。【归妹之极则无所归,无所归的根源是因为无实,空筐,无所终者也。】



\chapter{雷火丰卦}
丰 {\Large ䷶}

\section{原文}

\subsection{经}
丰:亨,王假之,勿忧,宜日中。

初九:遇其配主,虽旬无咎,往有尚。

六二:丰其蔀,日中见斗。往得疑疾,有孚发若,吉。

九三:丰其沛,日中见沫。折其右肱,无咎。

九四:丰其蔀,日中见斗。遇其夷主,吉。

六五:来章,有庆誉,吉。

上六:丰其屋,蔀其家,窥其户,阒其无人,三岁不觌,凶。

\subsection{彖}
丰,大也。明以动,故丰。王假之,尚大也。勿忧宜日中,宜照天下也。日中则昃,月盈则食,天地盈虚,与时消息,而况于人乎?况于鬼神乎?

\subsection{象}
雷电皆至,丰。君子以折狱致刑。

初九:虽旬无咎,过旬灾也。

六二:虽旬无咎,过旬灾也。

九三:丰其沛,不可大事也。折其右肱,终不可用也。

九四:丰其蔀,位不当也。日中见斗,幽不明也。遇其夷主,吉行也。

六五:六五之吉,有庆也。

上六:丰其屋,天际翔也。窥其户,阒其无人,自藏也。

\section{初讲}
卦辞说:

彖说:

象辞说:

\chapter{火山旅卦}
旅 {\Large ䷷}


\section{原文}

\subsection{经}
旅:小亨,旅贞吉。

初六:旅琐琐,斯其所取灾。

六二:旅即次,怀其资,得童仆贞。

九三:旅焚其次,丧其童仆,贞厉。

九四:旅于处,得其资斧,我心不快。

六五:射雉一矢,亡,终以誉命。

上九:鸟焚其巢,旅人先笑后号啕。丧牛于易,凶。

\subsection{彖}
旅,小亨,柔得中乎外,而顺乎刚,止而丽乎明,是以小亨,旅贞吉也。旅之时义大矣哉。

\subsection{象}
山上有火,旅。君子以明慎用刑,而不留狱。

初六:旅琐琐,志穷灾也。

六二:得童仆贞,终无尤也。

九三:旅焚其次,亦以伤矣。以旅与下,其义丧也。

九四:旅于处,未得位也。得其资斧,心未快也。

六五:终以誉命,上逮也。

上九:以旅在上,其义焚也。丧牛于易,终莫之闻也。

\section{初讲}


\chapter{巽(xùn)卦}
巽 {\Large ䷸}

\section{原文}

\subsection{经}
巽:小亨,利攸往,利见大人。

初六:进退,利武人之贞。

九二:巽在床下,用史巫纷若,吉无咎。

九三:频巽,吝。

六四:悔亡,田获三品。

九五:贞吉,悔亡,无不利。 无初有终,先庚三日,后庚三日,吉。

上九:巽在床下,丧其资斧,贞凶。

\subsection{彖}
重巽以申命,刚巽乎中正而志行。柔皆顺乎刚,是以小亨。利有攸往,利见大人。

\subsection{象}
随风,巽。君子以申命行事。

初六:进退,志疑也。利武人之贞,志治也。

九二:纷若之吉,得中也。

九三:频巽之吝,志穷也。

六四:田获三品,有功也。

九五:九五之吉,位正中也。

上九:巽在床下,上穷也。丧其资斧,正乎凶也。

\section{初讲}
卦辞说:

彖说:

象辞说:


\chapter{兑卦}
兑 {\Large ䷹}

\section{原文}

\subsection{经}
兑:亨,利贞。

初九:和兑,吉。

九二:孚兑,吉,悔亡。

六三:来兑,凶。

九四:商兑未宁,介疾有喜。

九五:孚于剥,有厉。

上六:引兑。

\subsection{彖}
兑,说也。刚中而柔外,说以利贞。是以顺乎天而应乎人,说以先民,民忘其劳。说以犯难,民忘其死。说之大,民劝矣哉。

\subsection{象}
丽泽,兑。君子以朋友讲习。

初九:和兑之吉,行未疑也。

九二:孚兑之吉,信志也。

六三:来兑之凶,位不当也。

九四:九四之喜,有庆也。

九五:孚于剥,位正当也。

上六:上六引兑,未光也。

\section{初讲}
卦辞说:

彖说:

象辞说:

\chapter{风水涣卦}
涣 {\Large ䷺}

\section{原文}

\subsection{经}
涣:亨。王假有庙,利涉大川,利贞。

初六:用拯马壮,吉。

九二:涣奔其机,悔亡。

六三:涣其躬,无悔。

六四:涣其群,元吉。涣有丘,匪夷所思。

九五:涣汗其大号,涣王居,无咎。

上九:涣其血,去逖出,无咎。

\subsection{彖}
涣,亨。刚来而不穷,柔得位乎外而上同。王假有庙,王乃在中也。利涉大川,乘木有功也。

\subsection{象}
风行水上,涣。先王以享于帝,立庙。

初六:初六之吉,顺也。

九二:涣奔其机,得愿也。

六三:涣其躬,志在外也。

六四:涣其群,元吉,光大也。

九五:王居无咎,正位也。

上九:涣其血,远害也。

\section{初讲}
卦辞说:

彖说:

象辞说:


\chapter{水泽节卦}
节 {\Large ䷻}


\section{原文}

\subsection{经}
节:亨。苦节不可贞。

初九:不出户庭,无咎。

九二:不出门庭,凶。

六三:不节若,则嗟\footnote{嗟(jiē),嗟叹。}若,无咎。

六四:安节,亨。

九五:甘节,吉,往有尚。

上六:苦节,贞凶,悔亡。


\subsection{彖}
节亨,刚柔分而刚得中。苦节不可贞,其道穷也。说以行险,当位以节,中正以通。天地节而四时成,节以制度,不伤财,不害民。

\subsection{象}
泽上有水,节。君子以制数度,议德行。

初九:不出户庭,知通塞也。

九二:不出门庭凶,失时极也。

六三:不节之嗟,又谁咎也。

六四:安节之亨,承上道也。

九五:甘节之吉,居位中也。

上六:苦节贞凶,其道穷也。


\section{初讲}
卦辞说:节卦,亨通。不可苦苦节制,这也不是正道。

彖说:节卦亨通,是因为节卦的刚爻和柔爻相间分开而九二九五两刚爻具得中位。苦节不可贞,因为这样节道必穷,不可久也\footnote{节卦的内卦为兑为悦,也就是行节道必须是内心愉悦的,否则必不可久也。}。节卦下为兑为悦上为坎为险,愉悦地行走于坎险之地,九五当得尊位而行之以节道,居中得正而能至亨通。天地有节而四时成,[王道者]节以制度,不伤财,不害民。

象辞说:节卦上为坎为水,下为兑为泽,泽上有水,这就是节卦的卦象了。君子应当制定礼数法度,评议德行\footnote{议德行者,存诸中为德,发于外为行。人之德行,当义则中节。议,谓商度求中节也。 ——程颐。即舆论监督以求中心内自节制也。}。


初九:初九不出户庭\footnote{初九得位而当行节道,因与六四相应而内心蠢蠢欲动,故戒之曰不出户庭。},无咎\footnote{户庭,户外之庭。门庭,门内之庭。 初以阳在下,上复有应,非能节者也,又当节之初,故戒之谨守,至于不出户庭,则无咎也。初能固守,终或渝之,不谨于初,安能有卒,故于节之初,为戒甚严也。——程颐。}。初九虽足不出户庭,但须知时之通塞\footnote{爻辞,于节之初,戒之谨守,故云不出户庭,则无咎也。 象恐人之泥于言也,故复明之,云:虽当谨守不出户庭,又必知时之通塞也。通则行,塞则止,义当出则出矣。——程颐。}。

九二:九二不出门庭,凶\footnote{参看上面的户庭门庭之解,九二连门庭都不出,节之过矣。九二不得位悖于节道偏过节,又不与九五相应,从之节之中道,有凶将至矣。}。不出门庭凶,九二失去好时机太多了\footnote{不能上从九五刚中正之道,成节之功,乃系于私暱之阴柔,是失时之至极,所以凶也。失时,失其所宜也。——程颐。}。

六三:六三不节制,则致独自嗟叹,没有谁可怪咎的。不节之嗟,又有谁可怪咎呢\footnote{节则可以免过,而不能自节以致可嗟,将谁咎乎。——程颐。}。


六四:六四安于节道,亨通。安节之亨,是因为六四上承九五之中正节道\footnote{四能安节之义,非一象,独举其重者。上承九五刚中正之道,以为节,足以亨矣。——程颐。}。

九五:九五甘美于节道,吉祥,前往有嘉尚\footnote{尚字名词作嘉尚解程颐是如此认为的,我查阅了字典并找不到这个说法,所以一开始我是反对的,认为此处应解为志尚。但在坎卦中有:行有尚,而彖继而解到,行有尚,往有功也。如此则嘉尚比志尚更合适一些。所以至少在周易语境下尚作名词可解为嘉尚。同时孔子说的好仁者,无以尚之,似乎该解为都不知道该怎么嘉尚他。}。甘节之吉,是因为九五居尊位而得中道是也。

上六:上六苦苦节制,贞凶,悔恨消失\footnote{上六苦节,贞凶,其道穷也,都容易解。只是这个悔亡程颐认为和其他的卦的悔亡的意思都不同,是悔则凶亡的意思。个人不是很认同,怎么可能通篇唯独此卦这里相同的一个词就成了另外的意思了,这是强行解释了。王弼的解释可作参考:过节之中,以致亢极,苦节者也。以斯施人,物所不堪,正之凶也。以斯修身,行在无妄,故得悔亡。上六得位,故其苦节的根源不是自己所为不合节道,其正常守节道也是凶险的,原因在于节道过极将穷,故终能得悔亡两字。}。苦节贞凶,其道穷也。


\chapter{风泽中孚(fú)卦}
中孚 {\LARGE ䷼}


\section{原文}

\subsection{经}
中孚:豚鱼吉,利涉大川,利贞。

初九:虞吉,有它不燕。

九二:鹤鸣在阴,其子和之。我有好爵,吾与尔靡之。

六三:得敌,或鼓或罢,或泣或歌。

六四:月几望,马匹亡,无咎。

九五:有孚挛如,无咎。

九六:翰音登于天,贞凶。

\subsection{彖}
中孚,柔在内而刚得中。说而巽,孚,乃化邦也。豚鱼吉,信及豚鱼也。利涉大川,乘木舟虚也。中孚以利贞,乃应乎天也。

\subsection{象}
泽上有风,中孚。君子以议狱缓死。

初九:初九虞吉,志未变也。

九二:其子和之,中心愿也。

六三:或鼓或罢,位不当也。

六四:马匹亡,绝类上也。

九五:有孚挛如,位正当也。

九六:翰音登于天,何可长也。

\section{初讲}


\chapter{雷山小过卦}
小过 {\LARGE ䷽}

\section{原文}
\subsection{经}
小过:亨,利贞。可小事,不可大事。飞鸟遗之音,不宜上,宜下,大吉。

初六:飞鸟以凶。

六二:过其祖,遇其妣。不及其君,遇其臣,无咎。

九三:弗过防之,从或戕之,凶。

九四:无咎,弗过遇之,往厉必戒,勿用永贞。

六五:密云不雨,自我西郊,公弋取彼在穴。

上六:弗遇过之,飞鸟离之,凶,是谓灾眚。

\subsection{彖}
小过,小者过而亨也。过以利贞,与时行也。柔得中,是以小事吉也。刚失位而不中,是以不可大事也。有飞鸟之象焉,飞鸟遗之音,不宜上,宜下,大吉,上逆而下顺也。

\subsection{象}
山上有雷,小过。君子以行过乎恭,丧过乎哀,用过乎俭。

初六:飞鸟以凶,不可如何也。

六二:不及其君,臣不可过也。

九三:从或戕之,凶如何也。

九四:弗过遇之,位不当也。往厉必戒,终不可长也。

六五:密云不雨,已上也。

上六:弗遇过之,已亢也。

\section{初讲}



\chapter{水火既济卦}
既济 {\Large ䷾}

\section{原文}

\subsection{经}
既济:亨小,利贞。初吉,终乱。

初九:曳其轮,濡\footnote{濡有濡润之意外还有濡滞之意,参考汉典,此处取濡滞之意较为合适。}其尾,无咎。

六二:妇丧其茀\footnote{fú,车蔽,古代妇女乘车不露于世,车之前后设障以自隐蔽。——汉典。},勿逐,七日得。

九三:高宗伐鬼方,三年克之,小人勿用。

六四:繻\footnote{rú,细密的丝织品——汉典。}有衣袽\footnote{rú,烂衣服或破旧棉絮——汉典。},终日戒。

九五:东邻杀牛,不如西邻之禴祭\footnote{yuè,禴祭,薄祭也——程颐。},实受其福。

上六:濡其首,厉。

\subsection{彖}
既济亨小者,亨也。利贞,刚柔正而位当也。初吉,柔得中也。终止则乱,其道穷也。

\subsection{象}
水在火上,既济。君子以思患而豫\footnote{同预}防之。

初九:曳其轮,义无咎也。

六二:七日得,以中道也。

九三:三年克之,惫也。

六四:终日戒,有所疑也。

九五:东邻杀牛,不如西邻之时也。实受其福,吉大来也。

上六:濡其首厉,何可久也。

\section{初讲}
卦辞说:既济卦,可亨通小事,利于贞道。起初吉祥,终有变乱。

彖说:卦辞说既济亨小者,是大的方面都是亨通的,还有小的方面可宁其亨通的意思。利于贞固守之,因为既济卦各刚爻柔爻都是得正而当位的\footnote{既济之时,大者固已亨矣,唯有小者亨也,时既济矣。固宜贞固以守之。 卦才刚柔正当其位,当位者,其常也,乃正固之义,利于如是之贞也。 阴阳各得正位,所以为既济也。——程颐。}。起初吉祥,是因为六二柔爻得中也\footnote{二以柔顺文明而得中,故能成既济之功。二居下体,方济之初也,而又善处,是以吉也。——程颐。};终至于变乱,是因为既济之道已穷,穷则生变\footnote{天下之事,不进则退,无一定之理。济之终,不进而止矣,无常止也。衰乱至矣,盖其道已穷极也。——程颐。}。

象辞说:既济卦上为坎为水,下为离为火,水在火上,这就是既济卦的卦象了。君子应当思患而预防之。

初九:拖曳初九的车轮,濡滞其尾,能止其进,乃得无咎\footnote{初以阳居下,上应于四,又火体其进之志锐也。然时既济矣,进不已则及于悔咎,故曳其轮,濡其尾,乃得无咎。 ...方既济之初,能止其进,乃得无咎。不知已,则至于咎也。——程颐。}。拖曳其轮而止其进,则义无咎也。

六二:六二妇人丢失车蔽,别去追逐它,七日后自会失而复得。七日后复得,是因为六二居中而得既济之道\footnote{自守不失,则七日当复得也。卦有六位,七则变矣。七日得,谓时变也。——程颐。}。

九三:天下既济,九三高宗远伐鬼方,多年征战才战胜之,能力小的人不可如此行事。三年克之,疲惫啊\footnote{此时仍为既济之事,九三应上六,思远患而长谋略,也不是不可以,但这会弄得人很疲惫,对个人能力有很高的要求。国家也是如此,如果国本不强,强行远征,则是疲惫国民之策。}。

六四:六四细密的丝织品旁还有破衣服,终日戒惧。终日戒,是因为六四思远患而有所疑也\footnote{此处解的各家多取舟遗漏之象,而这里就简单的取了六四思患而过度节俭之象,请关注这里的重点是描述六四的思患。}。

九五:九五东邻杀牛盛大祭祀,不如六二西邻之薄祭,因为六二将实受上天福佑。九五盛祭,既济之时已过半,说六二实受其福,是因为吉祥的大好事将要到来是也\footnote{六二疏于向上天祷告而自得福佑,到了九五已受福佑却又沉迷其中,而大张旗鼓的举行盛大的祭祀,人性如此。}。

上六:上六濡滞其首,危险。濡其首厉,是因为既济之道将穷,不可长久矣\footnote{此时上六当听象辞建言,思患而预防之,既济之道将穷,自不用再囿于既济贞固之道,此时正是大变革之时。}。



\chapter{火水未济卦}
未济 {\Large ䷿}
\section{原文}

\subsection{经}
未济:亨,小狐汔济,濡(rú)其尾,无攸利。

初六:濡其尾,吝。

九二:曳其轮,贞吉。

六三:未济,征凶。利涉大川。

九四:贞吉,悔亡。震用伐鬼方,三年有赏于大国。

六五:贞吉,无悔。君子之光,有孚,吉。

上九:有孚于饮酒,无咎。濡其首,有孚,失是。

\subsection{彖}
未济,亨,柔得中也。小狐汔济,未出中也。濡其尾,无攸利,不续终也。虽不当位,刚柔应也。

\subsection{象}
火在水上,未济。君子以慎辨物居方。

初六:濡其尾,亦不知极也。

九二:九二贞吉,中以行正也。

六三:未济征凶,位不当也。

九四:贞吉悔亡,志行也。

六五:君子之光,其晖吉也。

上九:饮酒濡首,亦不知节也。

\section{初讲}








\part{附录}
\chapter{周易批判}
查理·芒格批评中国人的两大致命伤,其一就是思维方式过于相信运气和神秘而不是相信几率。很是佩服芒格目光如炬一下切中要害。

没有谁能够预测未来,但是未来总有一天以某种状态到来,所以总有一种预测是对的。不同的预测方法是有着不同的成功几率的。如果一定要预测,就要选用最大成功几率的预测方法。理性分析最大成功纪律的预测方式是这样的:\textbf{首先要有对过往历史知识的吸收学习,其次还要有对世界本质运行规律的理解,最后作出合乎逻辑的分析推演}。

周易作为众经之首,前前后后几千年,贯穿在中国人的文化基因中,在文字中,在思维方式中,恐怕这不是简单的视而不见的问题。任何民族在历史的某个阶段都会发展出宗教信仰,靠着宗教信仰来对抗着人对于世界那些不可知部分的恐惧和不安情绪。对于周易的批判和中国人在这方面的反思,要把握好这两个点:一是不能全盘否定,全盘否定就好比全盘否定任何一个民族的宗教一样,就好比全盘否定未来的不可知性,全盘否定个体人对于世界的大部分都是不可知的这个事实,这是认知论的严重错误;其二就是要进行一些批判,咬准这两个字,不能\emph{过于} 。所以对于周易接下来内容的讨论,是被限定在这个范围内的:如果关于某个问题的思考,能够诉诸理性,逻辑思维,概率分析,那么应当用理性的思维去解答这些问题。而对于那些实在茫茫无从得知的问题,即使是科学研究中,有时也要靠一点科学家本人的信仰和偏执,对于这些问题,可以用到本书中讨论的周易的那种神秘的直觉的思维模式。

心怀神明,必有神佑;但行善事,必有福至。如此心无疑虑,又何须占之,何须卜之。对于人子心中的绝大部分疑虑、困惑,实际上都没有必要上升到理性反复求索而不得的地步。这也就是善易不卜的意思,因为人子心中真心怀神明,真的善心善愿善行,虽不至于忘我,但对于小我的大部分荣辱得失,自然会逍遥应对,而不会心有疑虑畏惧等等,不知所终了。




\section{周易的本质}
天人感应是周易预测的基石,只有良好的天人感应,上天以卦数示人,人才可能继续做出正确的预测。没有天人感应学说,后面的一切都是站不住脚的。

我们学到的各个周易预测学派都是前人内心修身,天人感应,外在体察的经验总结。各个学派彼此是不相干的,甚至是彼此跨越几百年完全不搭架的,将这些不同的学派和不同领域的应用企图统一起来的想法是极具野心的,学者应该慎重。

我发觉试图将五行和八卦混合起来的做法可能是错误的,五行学说更适合的是在中医领域,而观星领域发展起来的紫微斗数,或者和建筑相关的奇门之术。它们在具体领域对于从卦象到更具体的物象都有各种各样的约定,所有的这些约定都是源于前人先贤们内在修身,天人感应,体察外在万物,然后解读卦象的结果。所有这些学派的发展都是遵循如下过程:\textbf{内在修身静心,诚心发问,起卦得卦象或者观察万物得卦象,用心体察解读卦象}。

这个过程是最核心本质的部分,而至于具体得出来的结论和前人总结的各个约定反倒只是一些术了。这些术你甚至可以看作某种统计规律,但离周易的本质已经很远了。

所以就以最简单的周易六十四卦预测来说,个人的修身静心和用心体察解读卦象,这个过程是缺一不可的。没有这个过程,而是试图从周易预测的各个术中寻找科学或者某种规律的东西,那就实在是南辕北辙了。



\section{阴阳学说}
\begin{quote}
有太极,是生两仪, 两仪生四象, 四象生八卦,八卦定吉凶,吉凶生大业。
\end{quote}


阴阳学说是如此的浅显,以至于我觉得无法再补充说些什么了,就好像就是那样,它就在哪里,存在于我们的基因中,存在于我们的记忆中,觉得不需要再多说点什么了。但实际上阴阳学说是很深的哲理,西方哲学家们从莱布尼兹到黑格尔到马克思,慢慢把辩证法发展起来付出了极大的努力,而这个努力的基石,莱布尼兹自己也承认过,是受到了中国阴阳学说的启发。

任何事物内部都存在着矛盾对立面,这个矛盾对立面就是阴和阳。然后阴阳衍生出四象。四象很是巧妙地把事物内部矛盾演化的过程进一步细化,于是可比春夏秋冬,可比市场经济繁荣危机的GDP波动图等。实际上任何事物发展演化都可以进一步细化分析,从而得到四象。

\section{八卦和物象}
\begin{quote}
乾(qián)、坤(kūn)、坎(kǎn)、离(lí)、兑(duì)、艮(gèn)、巽(xùn)、震(zhèn)
\end{quote}


假如我们问上天的问题是“是或者不是”这样简单的问题,那么是就是阳,而不是就是阴,一个爻的卦象就够用了。但某些问题必然要求具有更大的信息量,于是人们提出八卦,并将八卦和这个世界不同的万事万物的物象对应起来。读者可以随便搜一搜就可以搜到八卦和物象的映射关系,实际上我们如果问上天的问题最后答案是指明某个物的话,那么分析卦象里面的八卦对应的物象,就能够大致猜测出来上天启示你的那个物品是什么。

我对八卦和这些物象的映射关系持保留意见,可做参考吧。类似的还有后天八卦映射方位的理论,可以在你想特别预测某个方位的时候权做参考吧。



\section{如何摇卦}
蓍草占卦就不说了,要铜钱六十四卦方法因为以前有兴趣学了一下,下面说明一下。

我看到有人批判说电脑计算随机数摇卦是不准的,一定要用铜钱。我觉得说这话的人一定要用蓍草占卜,否则对不起他的纯真坚持的心。重要的是天人感应,是铜钱也好,电脑生成的随机数也好,都不过是过程是手段是术罢了,只要保证这个过程不是掩耳盗铃,就是可行的。

具体是一个爻要抛三次硬币或者生成三个0或1的随机数:

\begin{itemize}
\item 如果加和为3,也就是都是阳,则为老阳爻,本卦得阳爻,变卦得阴爻
\item 如果加和为2,也就是两阳一阴,则为少阳爻,本卦得阳爻,变卦得阳爻
\item 如果加和为1,也就是一阳两阴,则为少阴爻,本卦得阴爻,变卦得阴爻
\item 如果加和为0,也就是都是阴,则为老阴爻,本卦为阴爻,变卦为阳爻
\end{itemize}


如此重复得到周易六十四卦预测的本卦和变卦。

\section{易者谨记}
除了自身的修身静心和用心体察解读卦象之外,无疑不占也是一大准则。若是你的内心的疑问困惑并不是一个真正的问题,你的内心是知道答案的,那么强行去占卦只是试图遮蔽自己的内心,占卦也将是没有意义的;因为天人感应指的是上天和你的内心发生感应。

天道无亲,常与善人,常罚恶人。

周易预测可知天命,天道有常然人命无常,人子善恶一念皆可改命。此易者不可不牢记在心!

王阳明:“人但得好善如好好色,恶恶如恶恶臭,便是圣人。”

善善恶恶,还有更多的道理吗,我看是没有了啊。





\section{六十四卦解卦入门}
在得到本卦和变卦之后,我们就可以查阅周易一书里面的内容了。从我个人的经验来看,变爻的一般为一个到两个,这些情况基本上是没有争议的:

\begin{itemize}
\item 无变爻 以本卦卦辞断之
\item 一个变爻 以该爻的爻辞断之
\item 两个变爻 以两个爻的爻辞共同断之,这里朱熹说 \verb+以上者断之+ ,我觉得说得不够确切,因为周易的各个爻辞明显可以看出是在讲述一个事件的发展经过的过程,所以更确切来说是这两个爻是你预测的该事件在那两个发展阶段存在变数的情况,\verb+以上为主+ 含义是以事情最终那个变化情况为主,但我觉得最好说成是由这两个爻的爻辞共同断之。
\item 更多的变爻的时候我会觉得这次摇卦不是很准了,那个时候我还没有接触朱熹的变爻断法,我觉得可以作为参考\footnote{但也只是参考,除开预测者的非常规预测需求,以一般的事件预测来说,我觉得还是要以本卦为主,只是事情会在很短时间内发生很多变化,所以该卦的时效性可能会很短} 。

具体朱熹关于多个变爻的断法如下所示:
\begin{quotation}
三个变爻以本卦与变卦卦辞断;本卦为贞(体),变卦为悔(用)

四个变爻以变卦之两不变爻爻辞断,但以下者为主

五个变爻以变卦之不变爻爻辞断

六个变爻以变卦之卦辞断,乾坤两卦则以「用」辞断
\end{quotation}

\end{itemize}



\chapter{基本术语}
\section{易经}
易经原有三,连山易,归藏易,周易,前两易已失传,

\section{经彖象}
经是周易最开始的内容,具体解卦最先的是卦象,根据你提出的问题分析卦象。卦象也就是那两个八卦叠起来的象。相传卦辞是文王写的,爻辞是周公写的。然后下面还有彖(tuàn)辞、象辞是孔子和他的门人写的。

\section{爻的当位和不当位}
认为爻位从下往上数,奇数为阳,偶数为阴,于是奇数位为阳位,偶数位为阴位,若阳爻居阳位则为当位,若阴爻居阴位也为当位,反之为不当位。

\section{相应}
初爻与四爻相应,二爻与五爻相应,三爻与六爻相应。两爻相应指这两爻作用明显。阴爻和阳爻相应为正应,反之两阳爻或阴爻为不是正应,也称为无应。正应作用强,无应作用弱。

\section{相比}
两爻相邻称之为相比,比如初爻与二爻相比。两爻相比彼此会更容易接近一些。

\section{卦彖象}
严格意义上来说周易的作者不详,只能说是成书于周朝,最大的可能是周文王指定,其朝中文官收集史料编纂而成。其他彖、象、文言、系、序等是孔子所作的注解,当然可能会有其门人的部分贡献,但绝大部分应该都是孔子所作。

\section{变爻}
爻分为二,若为少阴少阳则该爻不动。若为老阴老阳则该爻为变爻,变动之爻。

\section{先天六十四卦顺序和后天六十四卦顺序}
周易一书的顺序或者说按照序卦传而来的顺序通常被人们称为后天六十四卦顺序,这个顺序更多的反应了作者认为事物发展的一种哲理性解释。

通常预测会按照先天六十四卦顺序来,先天六十四卦顺序就是在变爻到六爻的阶段,最终发展到不可逆转进而形成变卦。先天六十四卦更多的是揭示天理自然规律,而后天六十四严格意义上来说并没有顺序一说,只是方便大家阅读而解释出来的那个顺序。

\section{尊卑贵贱上下}
周易里面尊卑,贵贱,上下各自是分开的,虽然人们常谈尊贵,卑下,下贱,但正所谓上位也可能不尊,下位也有高贵之民,看看周易里面系辞部分说的很清楚:“天尊地卑,乾坤定矣。卑高以陈,贵贱位矣。”尊卑本只是无褒贬含义的高和低的意思,天高高在上,地低低在下,乾坤就这样确定下来了。然后注意下面的\emph{卑高}这个词的顺序 ,正是从人的角度去看,先看到地,再看到天,是言卑高。关于这块南怀瑾说的很好,人性就是如此,容易摸得着的东西就轻贱它,总是得不到的东西就觉得很珍贵。周易在这里说的很明白,本来天尊地卑,并没有贵贱概念,因为人性,所以出现了贵贱的概念了。

在周易里面上下有统治或管理上的上下位之分,再一次将上下和贵贱和尊卑混为一谈那是后来人的私心想法。周易里面谈上下更多是让人们注意到管理上的上下位之别,比如履卦的“君子辩上下”,其正对应孔子谈为政的核心原则就是:“君君,臣臣,父父,子子”。当然孔子的谈论更多的局限在古代社会的君臣父子这几大关系上做了比喻和延伸,就现代社会而言管理上的核心原则其实仍然是一致的,即上位要有上位的样子,下位要有下位的样子。各尽分工,各尽其职。

\section{大人}
周易谈论的大人不是单指官位上的大人,从周易常现的利见大人这几个字可以推断大人是指在本卦象上有能力帮助你的人,或者有能力脱困之人。

\section{元亨利贞}
元更多的形容词含义,大的意思。比如“元亨”就是大亨通,“元吉”就是大吉祥。亨则是亨通的含义,并没有太大问题。利就是有利的意思。就是这个贞在解读上还是有点歧义的,比如有的地方是“小贞吉,大贞凶。”,有的地方是“利牝马之贞”。

贞字一般解读为贞正之道,略带有点褒义,在某些地方可能并没有褒义,仅仅只是表示个人贞固自守其道。

然后我否定了认为贞就是占卜的意思这一说法,古人写文特别讲究言简意赅,本就在行占卜之事,若吉祥则最多用一个字,吉,而不会赘言贞吉。

周易并没有一味要求人们去坚守贞正之道,在某些命运乖巧时运不济的时候,是不利于人们坚守贞正之道的,这是很实际的,同时在某些卦里面特别强调贞正之道的某些方面,比如“牝马之贞”或者“安贞”等。

那么贞字还有另外的含义吗,比如“小贞吉,大贞凶。”很多人都解释为小事吉祥,大事凶,又有人将贞字解释为占卜,我们现在假设小在古代还有小事的意思,那么他如果要表达小事吉祥,则只需要两个字小吉即可,而事实是小和大古今差异都不大,都是一个形容程度的词语。

要理解贞字最关键的是弄明白这个贞字原在人们中的头脑具象是何种情景,后面的名词动词形容词含义都是由这个头脑具象衍生出来的。我观察汉典上的字源字形,大体可以推测这更多的也是一个青铜器形象,比鼎小,考虑到真的形象也是类似的青铜器形象,而贞和蒸在读音上的接近,这些都不是偶然。贞字有些字形象会发现去掉了火,或者两个脚抬得很高,我可以推测这个青铜器后面渐渐成了一种祭祀相关用器,而且下面不再直接生火了。不管怎么说,我都不肯定贞字有占卜的含义在里面,唯一把贞字做占卜解释的那本书就是《周礼》,而这本书目前似乎人们已经断定它是成书于两汉之间,我们看到,在周易行文的时候,贞字基本上已经抽象为一种贞正的美德的含义了。所以我是赞同《周礼》这本书是伪托性质的,至少是成书在周易之后,然后作者因此误将贞字衍生出了占卜的意思。但我从贞字的形象是怎么看不到占卜一事的,至于说文解字将贞解作卜贝,又是希望将贞这个词的含义往占卜上靠,也可能是谬误的。

总的来说,我否认了贞字有占卜含义在里面。继而得出结论,贞作动词就是行贞正之道,作名词就是贞正之道的意思。我不管贞这个容器在很久以前具体代表是什么礼仪或者祭祀事宜,这个实在难以考究了。但至少在周易成书之时,该字已经完全抽象化了,而至于认为贞为占卜意思的说法都不过是后人杜撰假想出来的。

那么为什么有的时候是利贞,有的时候贞凶呢?文天祥从容就义是完全了贞正之道,贞正之道简言之是坚持了人的贞固之气。天地人以人为重为尊,然非常态。在某些形势下,天地太和之气不调,人坚守自己不一定是吉利的。贞固然是值得夸奖的人的美德,但有时不合于天道,此亦非善也。【一般在解白话的时候如果带有一点褒义可解为贞正之道,如果没有褒义则最好解为贞固自守。】


\section{贞吝}
贞吝这个吝不是吝啬的意思,吝通遴\footnote{遴,行难也。——说文。},难行也。即守正道会很难行。

\section{贞厉}
贞厉,厉通危也,故贞厉的意思是守正道会很危险[某些时候适合解为贞固自守]。

\section{咎}
咎这个字在周易中出现了很多次,有的地方可以简单解为灾祸的意思;有的地方需要解为过错,罪过的意思;还有的地方会做动词,则当解为怪罪,罪责的意思。

\section{利涉大川}
利涉大川,指当前形势利于克服艰难险阻。涉大川,指如同跋涉大山大川一般困难的事情,同时利涉大川似乎还有利于远行之意。

\section{征凶}
周易里面的征凶的征字不应该取征战之意而是取尔雅里面的意思:征,行也。——尔雅。所以征凶的意思是征行则凶险之意。

\section{先天八卦和后天八卦的区别}
先天八卦顺序并不是那么重要,先天八卦也没有方位一说。

后天八卦有顺序有方位,其可用于预测。


\section{皇极经世的时间刻度}
除去乾坤坎离四卦的先天六十四卦按照顺序每一卦管六运总共三百六十运。具体就是六十卦按照变爻从初爻变动到六爻分为六个阶段也就是这六个卦,这样三百六十运就分别对应了三百六十个卦,这个卦叫做值运之卦。一运360年。

在此值运之卦下,继续按照变爻从初爻变动到六爻分为六个阶段从而得到值世之卦,这个值世之卦管两世也就是六十年。

将这个值世之卦按照上面讨论的六十卦顺序从第一个值世之卦开始算起,一年一卦得值年之卦。

\chapter{八卦物象}
以下内容引用自 \href{https://www.douban.com/note/684840565/}{这个网页} 。

\section{乾卦}
乾为天

卦象为三阳爻纯阳之卦、其数一、五行属金居西北方色白。天、圆、君、父、刚强坚固、金、玉、冰、火红、马、总之凡是积极的、向上的、刚健有力的、权威的、圆形的、男性长辈、珍贵的、富有的、寒冷的,坚硬易碎的等等事物都归于乾卦。

象意: 老成、激烈、决断、威严、统一、老、扩大、任性、惩罚、制裁、强制、压抑、专横、独霸等。


\begin{description}
\item[人物] 上层人物、有地位的人、实权者、君王、圣人、厂长、经理、书记、一把手、名人、恶人、祖父、父亲、军警、执法者、经济工作者、管钱的人。
\item[性格] 刚健武勇、重义气、动、威严、自尊、正直、勤勉、骄傲、霸道。
\item[人体] 头、首、胸部、大肠、肺、右足、右下腹、精液、男性生殖器、身体健壮、体寒骨瘦之人。
\item[病象] 头面之疾、筋骨疾、肺疾、骨病、寒症、硬化性疾病、老病、急性病、变化异常之病、结肠病、便闭壅结。
\item[动物] 龙、马、天鹅、狮、象。
\item[天象] 太阳、晴、冰、雹、寒、凉。
\item[物象] 金、玉、珠宝、玛瑙、宝物宝器、高档用品、金钱、钟表、镜子、眼镜、古董文物、神物、首饰、高级车辆、火车、飞机、水果瓜、珍珠、帽子、机器、实心金属制品、圆形物体、辛辣之物。
\item[场所] 皇宫、京城、都市、博物馆、寺院、名胜、古迹、政府机构、大会堂、广场、车站、弯曲的大道、郊野、远处、学校楼。
\item[有利时间] 庚辛申酉,戊己辰戌丑未年月日时,不利时间数丙丁巳午年月日时,次壬癸亥子年月日时。
\item[有利方位] 西北、西、东北、西南方,忌南方、次北方。
\item[有利颜色] 黄、白色、忌红色、次黑色。
\end{description}

【为管理】

\section{坤卦}
坤为地

卦象为三阴爻,纯阴之卦,其数八,五行属土,居西南方色黄。

“坤为地、为母、为布、为釜、为吝啬、为均、为子母牛、为大兴、为文、为众、为柄、其于地也为黑。”

坤卦纯阴,性柔顺、象大发、万物生于地、人资生于母、故为母。阴柔故为布。阴虚能容物、故为锅子。阳大阴小、坤阴为小,故为吝啬。万物均生于地,故为均。坤为牛、生生相继,故为子母牛。地载万物如车载,故为大车。地生万物,故为众。操纵万物,故为柄。阴则暗,故为黑。由此可知,凡是消极的、阴柔的、方形的(古天圆地方)、软弱无力的、众多的、厚德的、承载的、辛劳的、静止的、裂开的(卦象三个阴爻中间全部断裂)等等事物都属于坤卦。

象意:谨慎正直、勤劳忍耐、复杂、吝啬、优柔寡断、穷闭沉默、逆来顺受、懦弱迟缓、依赖衰缓、敬奉神佛、恭敬抚养、伏藏疑惑、思想狭小及死丧过错等事物。


\begin{description}
\item[人物] 皇纪、臣子、国民大众、祖母、老母、后母、妻子、女主人、妇女、阴气盛之人、忠厚之人、大腹之人、农夫俗子、小气者、消极者、胆怯者、房地产者、泥瓦工、小人、尸。
\item[性格] 为多重性格,温厚柔顺、恭敬谦让、贞节、俭约、守信诚实、吝啬、懦弱、卑贱狭小、感情暖昧、虚耗嫌恶、固执迟钝、邪恶。
\item[人体] 腹部、胃、消化器、肉、右肩。
\item[病象] 腹部、肠胃、消化道之疾、饮食停滞、湿重浮肪、皮肤、肌肤病、湿疹、疣、晕病、中气虚弱、劳累疲乏、慢性病、癌症。
\item[动物] 牛、母马、百禽、雌性百兽、地下虫类、猫类等夜行动物。
\item[天象] 云、阴天、雾气、露、潮湿气候、低气压。
\item[物象] 城市居民点、国邦、田、土、窖、方形物、柔软之物、布帛丝锦、衣服被褥、妇女用品、文章、书报、纸张、箱。包袋子、轿子大车、车轮冲之物、陶器制品、石灰水泥、砖、砂。
\item[场所] 国郡城廓、乡村田野、平原平地、郊外、牧场、庄稼地、原籍、老家故乡、操坪广场空地、平房农舍、旧屋粮库、贮藏室、农贸市场、市场、肉类加工厂、鸡窝猪舍兔笼等。
\item[有利时间] 戊己辰戌丑未,丙丁巳午年月日时,忌甲乙寅卯年月日时,次忌庚辛申酉年月日时。
\item[有利方位] 西南、东北、南方。不利方位东、西、东南、西北方。
\item[有利颜色] 红、黄、忌绿色。
\end{description}


【为历史】


\section{震卦}
震为雷

震卦 初爻为阳爻,二、三爻为阴爻,其数四五行属木,居东方,色碧青。

“震为雷,为龙、为玄黄、为青、为大途、为长子、为决躁、为苍莨竹、为萑苇;其于马也、为善鸣、为弁足、为作足、为的颡;其于稼也、为反生;其究为健,为蕃鲜。”

震卦两阴爻在上,一阳爻在下,表示一种向上、向外发展的趋势。震为动、为雷。阴在上,有动荡不已的样子,故为龙。天玄地黄,震,乾坤始交,故为黑黄色。一阳在下,二阴在上,故有大道之象。一阳在下专静而动,故为专。阳爻动于初,锐利进取:故决断躁动。震为青绿色,故为小青竹。芦苇上干虚,下茎实,象震阳在下,阴在上之象。震为动,马善动善鸣,为弁足。震阳刚,燥动,故健。初阳在下,故象花生、洋芋、地瓜之土中物。


象意:上升、进步,出发,兴起,新生,勇敢,高,功名大,仁慈,追求,勤思,影响广,意气风发,好动,愤怒,惊恐,虚掠,粗心,轻举妄动,性急,冲突,夸大无礼,行走,出征,响动,高声等事物。

\begin{description}
\item[人物] 长男、大男、乘务员、指挥员、当头的、行政人员。
\item[性格] 动而少静、勤奋、有才干、好动、仁慈直爽、性急易怒、脾气大、心烦急燥、暴燥、倔犟、自尊心强、虚惊。
\item[人体] 足、腿部、肝脏、神经、筋、左肋、右肩臂、头发。
\item[病象] 足疾、肝经之疾、肝火旺、肝炎、精神病、狂躁病、多动症;神经衰弱、歇斯底理症、羊癫病、神经过敏、惊吓病、妇科病、疼痛性症状、剧烈性症状、咳嗽、声带咽喉病症。
\item[动物] 龙、蛇、鹰、鹫、善鸣弁足之马,善鸣之鸟、蜂,百虫、鹤鹄。
\item[天象] 雷、雷雨、雷鸣、地震、火山喷发。
\item[物象] 树木、竹子、鲜花、蔬菜、多节物、嫩芽、青绿色物、茶货、鞭炮、乐器音响、广播电话、行走的车类。火箭、飞机、飞船、大炮枪剑武器;裙、裤、蹄、鲜肉;闹钟。
\item[场所] 山林野地、林区、东向屋舍;茶地、菜市场、地震源、火山口;演凑会场、广播电台、邮电局、音像电器乐器店、歌舞厅、音乐茶座、杂技场、花店;闹市、噪声大之场所、喧哗地、游乐场所、大道、机场、发射场、战场靶场;军警公安部门,营房,军队;车场、车站。
\item[有利时间] 甲乙寅卯壬癸亥子年月日时,忌庚辛申酉年月日时,次忌丙丁巳年月日时。
\item[有利方位] 东、北、东南方;忌西、西北方,次忌南方。
\item[有利颜色] 
\end{description}



【为体育】



\section{巽卦}
巽为风

巽卦初爻为阴爻,二阳爻在上,其数五,五行属木,居东南方。白色

“巽为木、为风、为长女、为绳直、为工、为白、为长、为高、为进退、为木果、为实;其于人也,为寡发,为广颡,为多白眼,为利市三倍,其究为躁卦。”

巽卦一阴爻在下,有一种深人地下,向内发展的趋势,表示一种飘动而有渗透性的事物。巽为木为风,树木根善伸人地下,针大的眼斗大的风,风无孔不人,故巽为入。木又称为曲直,木匠用黑线绳取直制木,故巽为绳直,为工作工匠。风无色无味,在高空中飘拂,来往不定,故巽为高为白为进退为不果为臭。巽二阳一阴,阳多阴少,故为头发稀少,额宽大,眼白多。巽由乾卦初爻变阴而来,乾为金玉,故作生意能获三倍巨利。巽为震的旁通卦,震阳决躁,故为躁卦。

象意:基础不稳、直爽、涣散、清洁干净、整齐、附和、传达、营业生意、繁荣昌盛、交流、新鲜、书信、教令、捷报、号召、举荐、奔波、薄情、悭吝、幻觉、忙碌、轻浮、扫荡、忧疑、烦燥、胆略魄力、多欲、权谋、数术等。


\begin{description}
\item[人物] 长女、处女、寡妇、僧尼、仙道之人、气功师、练功者、商人、教师、医生、技术人员、教室、手艺人、科技工作者、作家、宗教人员、设计师、公关交际人员、文秀之人、造谣者、传令者;外刚内柔,优柔寡断之人,额阔者。头发细直而少者,下肢无力者。
\item[性格] 柔和、细心、责任心强,反复无定无决断,心志不定,仁慈直爽、谄庾、奸佞、多欲、薄情、极爱清洁、疑惑隐伏、说谎。
\item[人体] 头发、神经、气管、胆经、肱股、呼吸器官、食道、肠道、左肩、淋巴系统、血管。
\item[病象] 胆疾、股肱之疾、中风、肠疾胀气,伤风、感冒、受风、风湿、传染、坐骨神经痛、神经痛、神经炎、寒痹症、抽筋、胯股病、支气管炎、哮喘、左肩痛、淋巴疾病、忧郁症、血管症、病情不稳定。
\item[动物] 鸡鸭鹅,羽禽、山林禽虫,蚯蚓等地虫、蛇、蝴蝶、蜻蜒、带鱼、鳗鱼、鳝鱼等细长鱼类。虎、猫、斑马等条纹之兽,勇猛带响声之兽。
\item[天象] 刮风,各种大风,高空带长条的云。
\item[物象] 木材、木制品、纤维品、丝线、绳子、麻、扇、邮件;旗杆、长条桌柜、床、标枪、笔、管形物、刀斧类;薄的器物、裤带、桑帛、气球、气艇、帆船、赛艇、飞机、飞船、救生圈。草木之香,有香味之花草树木、香料、草药、蚊香、花草、柴薪、竹、枝叶、海带、柳、羽毛、风机、干燥机等。
\item[场所] 竹林草原,直而宽的道路、过道、长廊、寺观;各种线路,管道、通风、通气、出入通道;邮局、商店、码头、港口、机场、发射场、索场、升降机、传送带、工艺工厂、设计院。
\item[有利时间] 与震同
\item[有利方位] 与震同
\item[有利颜色] 与震同
\end{description}



【为科技 为教师 为医学 为技术】


\section{坎卦}
坎为水

坎卦阳爻居中,上下各为阴爻,五行属水居北方,色黑。

“坎为水,为沟渎、为隐伏、为矫柔、为弓轮、其于人也,为加忧、为心痛、为耳痛、为血卦、为床。其于马也,为美脊、为函心、为下首、为薄蹄、为曳;其于舆也,为多眚、为通、为月、为盗;其于木也,为坚多心。”

坎卦阳爻居中,阴爻在上下,则外柔内刚,四面向中心性发展的趋卦。坎为水,无处不流不渗入,故为沟渎、隐伏、险陷、加忧、心痛的现象。水能任意曲直矫柔,弯弓车轮为矫柔所成,坎又为车象。故为弓轮。坎为耳、心痛则耳痛。坎为水,为红色故为血卦。坎从乾卦变化来,乾为大赤,故坎为赤。乾为马,坎得乾中爻来,坎阳在中为脊背,阳为美,故为美脊。阴爻在上,所以为下首,阴爻在下,所以为薄蹄。水擦地而行,故为曳。对于车来说,坎为沟渎,为险陷,故多凶。水流通畅,故坎为通。坎中满,又水寒,故为月之象,为险陷,为盗贼。对于木来说,则内阳刚在中,故有坚硬木心之象。

象意:聪明、智慧、善谋、有主张、坚持不懈、以柔胜刚、多心劳碌、曲折坎坷、漂泊多变、暗昧不变、灾难患病、哭泣涕凄、欺诈狡猾、疑虑多心、阴冷会聚、算计、淫欲、讼狱、狠毒、破坏、罪恶、进入、接纳、险、疾、难、法律、流血、月、酒、丛棘、桎梏。


\begin{description}
\item[人物] 中男、江湖之人、船上工作人员、思想家、发明家、数学家、书法家、心理学家、安全保安人员、自来水公司工人、劳苦者、劳务者、印刷工人、贫困者、水货商、冒险者、酒鬼病人、多情轻浮者、诱惑者、诈骗者、有犯罪历史者、失败破产者、中毒者、娼妇、受灾者、流亡者。
\item[性格] 外柔和内刚厉、善谋多智、多欲、喜算计、追求时尚、多心计、阴险卑鄙、城府深、奸诈、捧上压下、作事自有主见;随波逐流。
\item[人体] 肾脏、膀胱、泌尿系统、性器官、血液、血液循环系统、耳、背、腰、背脊骨。
\item[病象] 肾、膀胱、泌尿系统疾病,肾冷、水泻、消渴症、血液病、出血症、免疫系统疾病、性病、遗精、阳萎、生殖器疾病、中毒、病毒性疾病、耳痛、腰背疾病、心脏病、水肿病。
\item[动物] 猪、鱼、水中物、水鸟、鼠、四足动物、脊椎动物、驾辕之马。
\item[天象] 雨、雪、霜、露、寒冷、阴湿、满月、积雨云,半夜、水灾。
\item[物象] 带核之物、桃杏李梅果实、油酒醋、饮料、脂肪、液体物质、染料、涂料、药品毒物;酒具、水车、车、车轮、弓箭;法律法则经典、刑具;冷藏设备、供排水设备、海味、淹藏物、潜艇、计算机、磁盘、录音录相带、激光视盘、黑色物、煤、弓形变曲物。
\item[场所] 大川、江湖海河、溪涧泉水、湿泥泞地、水道;酒吧、冷饮店、浴所、澡池、鱼市、鱼塘;水厂、自来水公司、漆脂厂、冷库、水族馆、车站、车库、地下室、暗室、黑暗场所、牢狱、妓院。
\item[有利时间] 庚辛壬癸申酉亥子年月日时,忌戊己辰戌丑未年月日时,次忌甲乙寅卯年月日时。
\item[有利方位] 北、西、西北;不利方位西南、东北方,次东、东南方。
\item[有利颜色] 黑、白色、忌黄色。
\end{description}


【为数学 为心理学 为思想】


\section{离卦}
离为火

离卦一阴爻居中,二阳爻居外,其数三,五行属火,居南方、色红。

离为火、为日、为电、为中女、为甲胄、为戈兵;其于人也,为大腹,为干燥、为鳖、为蟹、为赢、为蚌、为龟;其于木也、为科上槁。

离卦与坎卦旁通,正好相反,一阴爻居中,二阳爻在外,为外刚内柔,外硬内软之性情,有由中心向外发展的趋势、有离散之象。一切鳖、蟹、龟、贝类,土兵的衣甲胃帽等等外刚内柔之物均归类于离卦。离为火,故为干燥卦。离中虚,对于人来说就象一个大腹便便者。它为日、为火,故象闪电。火性炎上,故对于树木来言,象枝干枯槁。

象意:明、进升、依附、华丽、鲜艳、文明、礼仪、明察、焦躁、煽动、排斥、抗拒、否定、批判、流行、检举、侦察、轻浮、显示、自满、花言巧语、抗上、撒谎、干枯、枯燥、空虚等。

\begin{description}
\item[人物] 中女、文人、大腹人、目疾人、戴头盔者、兵。
\item[性格] 重礼、好美、有依赖性、聪明好学、虚心处事、知书达理、内心空虚、爱好书面和文章、性急、易冲动、好动、孝顺、邪恶。
\item[人体] 眼目、心脏、视力、红血球、血液、乳房、上焦、头面、喉、小肠。
\item[病象] 眼病、视力疾病、心脏病、火烧伤、烫伤、灼伤、放射性疾病、乳腺疾病、发烧、热病、炎症、尿赤黄、血液病、妇科病、囊肿扩散性病疾、肥大症(前列腺肥大、增生、乳腺增生、心脏肥大),血压疾病。
\item[动物] 雉、孔雀、凤凰、鹑及美丽的羽毛类鸟禽、金鱼、热带鱼、“变色龙”、虾、蟹、贝类、龟、鳖、荧火虫、硬壳虫。
\item[天象] 晴天、热天、酷暑、烈日、干旱、丽日、彩虹、光、云霞、闪电。
\item[物象] 文学艺术、美术字画、文科、医科、文件、文章、书报杂志、地图课本、文书印章、证件、证券、信、合同、花、鲜艳物品、旗帜、广告、奖状、电话、电报、火柴、打火机、火炉、锅炉、电动机、发动机、空船、玻璃门窗、火车厢、电车、轿车、火焰喷射器、燃烧弹、焊枪、干肉、果脯、煎炒、烧烤物品、液化气灶、烤箱、笼子、瓶罐、网袋、花衣服。
\item[场所] 朝阳的土地、名胜地、圣地、教堂、殿堂、大会堂、学校、博物馆、展览馆、影剧院、证券交易所、银行、图书馆、书画店;电厂、印刷厂、医院、放射科、检验科、厨房、华丽的大厅、火山、喷火口、火灾场所、阳台;部队、军营、派出所、公安局、法院、检察院、窑炉、冶炼厂、场、仓库、空房屋、桥梁、立交桥、轿子、棚子、火车站;监视塔、电视台、广告塔(牌)、猎场、钓鱼场所。
\item[有利时间] 丙丁巳午甲乙寅卯年月日时,忌壬癸亥子年月日时。
\item[有利方位] 东、南、东南方,忌北方,次忌东北方,西南方。
\item[有利颜色] 喜红、绿色。忌黑色。
\end{description}



【为文学】


\section{艮卦}
艮为山

艮卦一阳爻在上,二阴爻在下,其数七,五行属土,居东北方。色黄。

“艮为山、为径路,为小石、为门、为果瓜、为阍寺、为狗、为指、为鼠、为黔喙之属,其于木也,为坚多节。”

艮卦一阳爻在上,二阴爻在下,表示表面实内里虚,上实下虚的事物。也表示事物一种向下的发展趋卦,事物发展到了顶点,必须谨慎,否则就要向相反方向发展。并且表示事物有困难,停止不前。艮为山为止,一阳爻在坤土之上,故有小石的象征。上画阳爻相连,下二阴爻中间虚空,就如门的象征。木草的果实均在上部,不在根,为阳之象,所以艮卦为果实之象。阍寺视作门卫,禁止人入内,故艮为止像。手能止住物体,狗吠使人惊吓止住不前,老鼠牙齿尖刚,鸟刚在喙,均为艮之象。艮为小石象,故坚硬多节的木象小石一样,故也为艮之象。


象意:禁止、阻滞、阻挡、静止、慎守、界限、抑止、安居、沉着、冷静、更替、隐藏、固执、主观、任性、分水岭、重新开始、标准、独立、转变、转折、讼狱、笃实、消亡、叮咛、等待、厚重、表皮背、至少、顶多。

\begin{description}
\item[人物] 小儿子、门卫、领头的。
\item[性格] 憨厚、安静、笃实、保守、固执、诚实、守信。迟滞、审慎、乖戾。
\item[人体] 鼻、背、手背、指关节、骨、脾、趾、皮、手、脚背、膝关节、肘关节、左足、颧骨、乳房。
\item[病象] 脾骨病、不食虚胀、鼻炎、手脚背之疾、麻木病、关节病、手指疾、肿瘤、结石、消化系统病、气血不通症、血液循环不定。
\item[动物] 有牙、有角的动物,狗、鼠、狼、熊等百兽,喜鹊、鹬鹘等能喙之物,爬虫类、昆虫、家畜、有尾动物。
\item[天象] 有云无雨,多云间阴,山风雾气,气候转折点。
\item[物象] 岩石、石块、门板、凳子、床、柜子、桌子、石碑、硬木、硬的果皮、土坑、柜台、磁器、伞、鞋、钱袋、列车、金库、坟墓、土堆、山坡、座位、屏风、手套、门坎、墙壁、阶梯、药。
\item[场所] 山、土包、土墩、假山、丘陵、坟墓、堤坝、交叉点、最高点、境界、山路、小路、矿山、采石场、阁、寺、房屋、门闩、贮藏室、宗庙、祠堂、帐蓬、影壁、城墙、围墙、大楼、仓库、银行、车站、岗位、监狱。
\item[有利时间] 戊己辰戌丑未,丙丁巳午年月日时,忌甲乙寅卯年月日时,次庚申辛酉年月日时。
\item[有利方位] 东北、西南、南方、忌东、东南方。
\item[有利颜色] 喜红、黄色,忌绿色。
\end{description}




\section{兑卦}
兑为泽

兑卦一阴爻在上,二阳爻在下,其数二,五行属金,居西方,色白。

“兑为泽,为少女、为巫、为口舌、为毁折、为附决,其于地也,为刚卤、为妾、为羊。”

兑卦与艮卦旁通正相反,一阴爻在上,二阳在下,表示一种向上发展的趋势的事物,外柔内里刚硬,外虚内实的事物。兑为泽,故有吸收功能,容易与外界周边事物信息沟通,兑为泽为少女。阴爻见于外,有口舌的现象,少女快乐无忧,喜为悦,兑为口为悦为少女,故为跳大神的巫师。兑居西为申酉金之秋月,故主肃杀,万物毁折,故为毁折。兑柔附于二阳刚上,故为附决。为金为西方之卦,西方多盐卤地。故为刚卤。为少女,有为妾之象。兑为悦,故于动物如羊的欢叫活蹦乱跳。

兑还有如下意象:说、雄辩、讲演、告知、言谈话语、议论、笑、骂、吵闹、叹息、毁谤、叫卖、仰视、魅力、爱欲、服用、口舌、不足、不便、商量、音乐、信仰、破损、刑、破坏、右边的、外软内坚实的事物、上面开口的器物、敞开的器物。

\begin{description}
\item[人物] 少小的女孩,可爱的女孩、少女、朋友,与用口或说、唱有关的职业的人,欢乐性职业者,破坏性职业者,巫师、巫婆、老师、教授、演讲者、解说员、翻译、外科、牙科医生、食品厂工人、饭店工人、金器加工者、秘书、娼妓、非处女、妾、亲戚、和蔼可亲的人、撒娇的人、性魅力者、小人、媒人、刑官县令、副手、二把手、邻居、垃圾工、卫生清洁工、传达人员、服务员、话务员、歌唱者、演员、钢琴家、音乐家、娱乐场所人员、小丑、歌女、金融界人物,经销人员,失败者、破坏者。

\item[性格] 喜悦、吵架、毁谤、拍马屁、卑劣、奉承、色情、亲热和乐、温和、善言、喜唱歌、活跃、温厚、重感情、感召力强、重义气、忧愁、破坏性、口谗。
\item[人体] 口、舌、肺、痰、气管、口角、咽喉、颊骨、牙齿、右肋、肛门、右肩臂。
\item[病象] 口、舌、喉、牙齿之疾、咳嗽、痰喘、胸部、肺部疾病、食欲不振、膀胱疾病、外伤,肛门疾病、性病、贫血、低血压、手术、金属刃具致伤、皮肤病、气管疾病、头部疾病。
\item[动物] 羊、豹、猿猴、兔子、沼泽中之物。
\item[天象] 小雨、潮湿天气、低气压、毛雨、短期气象情况、新月、星星、露水。
\item[物象] 饮食用具、食品、盛水用具、金属币、刀剑、剪子、玩具、开口瓶罐、破损物、欠缺物、修理品、无头物、装饰金属制品、软的金属、废物、乐器、带口的器物、石榴、胡桃、垃圾箱、成形器皿、各种表类。
\item[场所] 沼泽地、沼泽、坑洼地、凹地、水井、浅沟、湖泊地潭、溜冰场、游乐园、会议厅、音乐厅、饮食店、饭馆、门口、路口、垃圾站、废墟、井坑、旧屋宅、洞穴、巢穴、山洞、墓穴、山坑、山口、演出厅、工会、公关部、交谊所。
\item[有利时间] 庚辛申酉戊己辰戌丑未年月日时,忌丙丁巳午年月日时,次忌壬癸亥子年月日时。
\item[有利方位] 西、西北、东北、西南方、忌南方、次忌北方。
\item[有利颜色] 白、黄色、忌红色。
\end{description}


【为一般的言谈,讲演,杂文,为音乐,】

\chapter{个人占卜}

\section{今后三十年世界经济占}
占卜日: 20201028

是日,为世界经济今后三十年占上一卦,以解我心中困惑。

得\textbf{讼卦变爻23456}。

卦辞说:有诚信,内心恐惧,中吉,终凶。利见大人,不利涉大川。

今后三十年世界经济一个关键词就是争讼争端。商人们是有诚信的,但内心是充满了恐惧的。

彖说:讼卦上面为乾卦为刚,下面为坎卦为险。虽遇险但有刚健,这就是讼卦了。讼,有孚,窒惕,中吉,这是因为上面刚健的三爻来到坎险之地而居九二之中的缘故。终凶,争讼之事没有成功。利见大人,这是因为崇尚九五的中正之德。不利涉大川,这是因为入于坎险的深渊之地。

象辞说:坎卦上为天下为水,天从东向西转动,水从西向东流,天水相背而行,这就是讼卦的卦象了。君子观此卦象而知道 \textbf{作事须谋始} 。

2020-2025年,此爻是唯一的不变爻,说明这段时间世界经济基本运势已经确定。后面二十五年爻爻皆变,此亦稀有哉。因天道运行虽有常,人子善恶一念皆可改命,故上天以变爻示人,也是世界运行劝人向善的本义。

基本爻辞白话为:不纠缠于争讼之事,虽然小人有所言论,终吉。不永所事,此争讼之事并不长久。虽有小小的责难之言,但通过辩解就能将是非曲直说明白了。

说明这段时间世界经济争讼争端频出,小人指责言论多有,不过争讼争端一般不长久,彼此辩解沟通交流是能够说明白的。此段时间从整个卦象来看是入于坎险之中的,说明世界经济整体形势正在进入一个坎险的艰难境地。

2025-2030年 变爻指的是需要特别注意应对的局面。此爻辞白话为:不能胜讼,回来后又马上逃亡出去了。他所在村邑下有三百户人家,并没有受到牵连的灾祸。不克讼,是故回来后又马上逃窜中。以九二之下和上面有权有势的人争讼,灾祸是自己找的啊。此爻有商人克冲上面权贵然后流亡他国之象。

说明这段时间世界经济争讼争端频出,其中有一方和另外一个更有权势的人争讼,然后败诉了。败诉的弱势方国家或者阵营下面的老百姓还好没有受到牵连的灾祸。此段时间从整个卦象来看是入于坎险之中,说明世界经济整体形势已经进入了一个坎险的艰难境地。

2030-2035年 此爻辞白话为:享用旧有的德业,贞厉,守正道是危险的,但最终会获得吉祥。可能会随君王作事,但并没有功名成就。食旧德,六三柔顺顺从上面的刚健三爻,还是能收获吉祥的。

这段时间世界经济在吃老本,贞固自守是危险的,最终还是会吉祥的,要最终获得吉祥,就是要顺从后面十五年的刚健而行的计划,按照这个计划来调整世界经济。此段时间世界经济仍然处在坎险之地。

2035-2040年 此爻辞白话为:不能胜讼,继而复归到原来的命运状态,改变后,固守安贞而得吉祥。“复即命,渝,安贞”,并没有什么损失。

这段时间世界经济整体进入了一个刚健而行的阶段,正慢慢走出坎险之地。突出一个复归二字。固守安贞,即不要随便到处乱跑,安于原来的现状即可获得吉祥。

2040-2045年 此爻辞白话为:争讼之事结果大吉祥。讼元吉,是因为九五的中正之德。

这段时间世界经济最关键的地方在于坚守九五的中正之德,如此则大吉祥。此时利见大人,也就是利于见世界各国的大领导人。

2045-2050年 此爻辞白话为:可能会受到君王赐赏官服大带,一天之内却多次被褫夺。因为争讼而受到赏赐,这没什么值得尊敬的。

这段时间世界经济已经进入讼卦之末尾了,此时商人如果继续争讼,可能会得到上面领导人的嘉奖,但过一会又翻脸了。因为世界经济整体氛围已经不再鼓励争讼争端了,就算因为争讼得到赏赐,也没什么好夸耀的。

\section{中共运占}
占卜日:20200929

是日,多年心中翻来覆去的那些东西,何不向上天发问以得中共一世运占之。

得 \textbf{复卦变爻23}。

复卦曰 “七日来复” ,故计时必以七为乘数。继而我发现4*7=28年之数与爻辞描述颇为吻合。

卦辞说:复卦,亨通。出入无疾病,朋友来了言行也无过失。阴阳消长反复之道,七日为一来复,利于有所前往。

彖说:复卦亨通,初九刚爻复返于初,下为震为动,上为坤为顺,震动而以坤顺之德行之,是以出入无疾,朋来无咎。反复其道,七日来复,天地运行如是也。利于有所前往,阳刚君子之道盛长,复卦其内君可见天地之心乎?。

象辞说:复卦下为震为雷,上为坤为地,雷在地中,这就是复卦的卦象了。先王于冬至日闭关,商旅不得通行,君王也不省察四方。


\begin{enumerate}
\item 1921-1949 爻辞白话是:初九爻不远而复返,不至于有悔,大吉。不远之复,以修身也。解为:此时的中共改过颇速,是以元吉。

\item 1949-1977 此爻是一个变爻,也就是某些事情是存在变数的,这个变数是从中共一世整体来看的,现在我们已处于2020年,看起来很多东西已经固定了。爻辞白话为:六二爻休美之复,吉祥。休复之吉,是因为其能下亲于初九。解为:此时中共仍能休美地复于正道,因为其能下亲于天下人民【或者中共基层】。

\item 1977-2005 此爻是一个变爻。爻辞白话为:六三爻频复频失,危险,无咎。虽有频复频失之厉,但复善则义无咎也。解为:此时中共在这个时期频繁地犯错,但又能频繁地改过。虽危险,但大体能频繁改过也能得到无咎二字。

\item 2005-2033 此爻不是变爻,后面的三爻都不是变爻,盖中共后面的命运已定。爻辞白话为:六四与初九正应,行于群阴之中而独自复于正道。中行独复,以从于阳刚君子之善道也。解为:此时中共想要下亲于天下人民【或者中共基层】,但于群阴之中,困难重重,加上自身以阴居阴,甚是柔弱,终不能接近天下人民【或者中共基层】。没有无咎亦不谈吉凶,与初九正应说明此时中共复于正道的心还是在的。

\item 2033-2061 此爻不是变爻。爻辞白话为:六五爻敦厚地复于正道,无悔。敦复无悔,居中而自我考查是也。解为:此时中共能够敦厚地复于正道,能得无悔二字。但此时已经所复甚微了,不谈亨通不谈吉,仅仅是无悔而已。

\item 2061-2089 此爻不是变爻。爻辞白话为:上六爻迷而不复,凶险,有灾眚,用之行师出兵,终有大败,以它为国君,凶险。以至于十年都不能征战胜利。迷复之凶,以其违反为君之道是也。解为:此时中共执迷而不知复于正道,凶险,外有灾而内有眚。兴兵打仗,将会大败,以中共为国君,凶险啊。十年而不得征战胜利,这是因为中共违反了为君之道。

\end{enumerate}

\chapter{参考资料}
\begin{itemize}
\item \href{https://www.eee-learning.com/article/897}{伊川易传}
\item \href{http://www.quanxue.cn/QT_MingXiang/ZhouYiZhuIndex.html}{王弼周易注}
\item \href{https://www.zdic.net/}{汉典}
\item \href{http://www.quanxue.cn/QT_XiaoYa/YiJingIndex.html}{劝学网小雅易经入门学习教程}
\item \href{https://ctext.org/book-of-changes/zhs}{中国哲学书电子化计划周易}
\item \href{http://www.xshiqi.com/}{易经网}
\end{itemize}



\end{document}


