% !Mode:: "TeX:UTF-8"

\documentclass[12pt,oneside]{book}

\newlength{\textpt}
\setlength{\textpt}{11pt}
    
\newcommand{\flypage}[1]{\begin{titlepage}\begin{center}\vspace*{\stretch{1}}#1\vspace*{\stretch{1}}\end{center}\end{titlepage}}
    
%========基本必备的宏包========%
\RequirePackage{calc,float,moresize}
%\RequirePackage[onehalfspacing]{setspace}
\linespread{1.5}
%1.3 onehalfspacing
%试卷或需要文字紧凑的
%1.6 doublespacing

%===========加入目录 某章或某节=====%
\makeatletter

\newcommand{\addchtoc}[1]{
        \cleardoublepage
        \phantomsection
        \addcontentsline{toc}{chapter}{#1}}

\newcommand{\addsectoc}[1]{
        \phantomsection
        \addcontentsline{toc}{section}{#1}}

%===========全文基本格式==========%
\setlength{\parskip}{1.6ex plus 0.2ex minus 0.2ex}   %段落間距
\setlength{\parindent}{\textpt * \real{2}}

%=========页面设置=========%
\RequirePackage[a4paper, %a4paper size 297:210 mm
  bindingoffset=10mm,%裝訂線
  top=35mm,  %上邊距 包括頁眉
  bottom=30mm,%下邊距 包括頁腳
  inner=10mm,  %左邊距or inner
  outer=10mm,  %右邊距or  outer
  headheight=10mm,%頁眉
  headsep=15mm,%
  footskip=15mm,%
  marginparsep=10pt, %旁註與正文間距
  marginparwidth=6em,includemp=true% 旁註寬度計入width%旁註寬度
  ]{geometry}

%color
\RequirePackage[table,svgnames]{xcolor}

%================字體================%
%设置数学字体
\RequirePackage{amssymb,amsmath}
\RequirePackage{stmaryrd}
\everymath{\displaystyle}

\RequirePackage{fontspec}
%設置英文字體
\setmainfont[Mapping=tex-text]{DejaVu Serif}
\setsansfont[Mapping=tex-text]{DejaVu Sans}
\setmonofont[Mapping=tex-text]{DejaVu Sans Mono}


%中文環境
\RequirePackage[]{xeCJK}
\xeCJKsetup{PunctStyle=plain}
\xeCJKDeclareSubCJKBlock{LIUSHISIGUA}{ "4DC0 -> "4DFF}
\setCJKmainfont[FallBack=DejaVu Serif, ItalicFont=方正楷体简体,LIUSHISIGUA=DejaVu Sans]{Source Han Serif CN}
\setCJKsansfont[FallBack=DejaVu Sans]{Source Han Sans CN}
\setCJKmonofont[FallBack=DejaVu Sans Mono]{方正楷体简体}


%%===============中文化=========%
\renewcommand\contentsname{目~录}
\renewcommand\listfigurename{插图目录}
\renewcommand\listtablename{表格目录}
\renewcommand\bibname{参~考~文~献}
\renewcommand\indexname{索~引}
\renewcommand\figurename{图}
\renewcommand\tablename{表}
\renewcommand\partname{部分}
\renewcommand\appendixname{附录}
\renewcommand{\today}{\number\year{}年\number\month{}月\number\day{}日}


%=======页眉页脚格式=========%
\RequirePackage{fancyhdr}   %頁眉頁腳
\RequirePackage{zhnumber}  %计数器中文化
\pagestyle{fancy}
\renewcommand{\sectionmark}[1]
{\markright{第\zhnumber{\arabic{section}}节~~#1}{}}

\fancypagestyle{plain}{%
    \fancyhf{}
    \renewcommand{\headrulewidth}{0pt}
    \renewcommand{\footrulewidth}{0pt}
    \fancyhf[HR]{\ttfamily \footnotesize \rightmark }
    \fancyhf[FR]{\thepage}}
\pagestyle{plain}


%=========章節標題設計=========%
\RequirePackage{titlesec}
%修改part
\titleformat{\part}{\huge\sffamily}{}{0em}{}
%修改chapter
\titleformat{\chapter}{\LARGE\sffamily}{}{0em}{}
%修改section
\titleformat{\section}{\Large\sffamily}{}{0em}{}
%修改subsection
\titleformat{\subsection}{\large\sffamily}{}{0em}{}
%修改subsubsection
\titleformat{\subsubsection}{\normalsize\sffamily}{}{0em}{}


%================目录===============%
%toc label to contents space   dynamic adjust
\RequirePackage{tocloft}%
\renewcommand{\numberline}[1]{%
  \@cftbsnum #1\@cftasnum~\@cftasnumb%
}

%==============超鏈接===============%
\RequirePackage[colorlinks=true,linkcolor=blue,citecolor=blue]{hyperref} %設置書簽和目錄鏈接等
\newcommand{\hlabel}[1]{\phantomsection \label{#1}}%某一小段的引用


%=================文字強調=========%
\RequirePackage{ulem} %下劃線,加點
\normalem%normal em , not instead of the uline

%modified udot command from the dotuline
\def\udot{\bgroup
  \UL@setULdepth
  \markoverwith{\begingroup
     \advance\ULdepth0.1ex
     \lower\ULdepth\hbox{\kern.25em . \kern.045em}%
     \endgroup}%
  \ULon}
\renewcommand\emshape{\color{red}}

%==================插入圖片=======%
\RequirePackage{wrapfig}
\RequirePackage{graphicx}
\graphicspath{{figures/}}
%change the caption style a little like 1-1
\renewcommand{\thefigure}{\arabic{chapter}-\arabic{figure}}


%==============插入表格========%
\RequirePackage{booktabs}
\renewcommand{\thetable}{\arabic{chapter}-\arabic{table}}
\RequirePackage{caption}
%\renewcommand{\arraystretch}{1.3}
%如果用setspace宏包而不是linespread调整行间距,那么才需要额外的表格行距拉大。

%插入代码
\RequirePackage{fancyvrb} 
\fvset{frame=lines,tabsize=4 ,baselinestretch=1.8, fontsize=\footnotesize}


%==========其他宏包===========%
\RequirePackage{tikz} 
\usetikzlibrary{calc}

%========脚注=========%
\newcommand*\circled[1]{%
  \tikz[baseline=(char.base)]\node[shape=circle,draw,inner sep=0.4pt,minimum size=4pt] (char) {#1};}
\newcommand*\circledarabic[1]{\circled{\arabic{#1}}}

\RequirePackage{perpage} %the perpage package
\MakePerPage{footnote} %the perpage package command

\renewcommand*{\thefootnote}{\protect\circledarabic{footnote}}


\renewcommand\@makefntext[1]
{\vspace{5pt}
\noindent
\makebox[20pt][c]{\@makefnmark}
\fontsize{10pt}{12pt}\selectfont #1}

\setlength{\skip\footins}{20pt plus 10pt}
%main body 与脚注之间的距离


%framed环境
\RequirePackage{framed}

\newenvironment{shici}{
\begin{verse}
\centering\large\hspace{12pt}}
{\end{verse}}

\RequirePackage{indentfirst} 





\makeatother





\title{周易}
\author{万泽}
\hypersetup{
  pdfkeywords={},
  pdfsubject={制作者邮箱:a358003542@outlook.com},
  pdfcreator={万泽}}
  
\begin{document}
\frontmatter 

\maketitle

\flypage{感谢上天}


\addchtoc{引言}
\chapter*{引言}
天人感应是周易预测的基石,只有良好的天人感应,上天以卦数示人,人才可能继续做出正确的预测。没有天人感应学说,后面的一切都是站不住脚的。

我们学到的各个周易预测学派都是前人内心修身,天人感应,外在体察的经验总结。各个学派彼此是不相干的,甚至是彼此跨越几百年完全不搭架的,将这些不同的学派和不同领域的应用企图统一起来的想法是极具野心的,初学者应该慎重。

我发觉试图将五行和八卦混合起来的做法可能是错误的,五行学说更适合的是在中医领域,而观星领域发展起来的紫微斗数,或者和建筑相关的奇门之术。它们在具体领域对于从卦象到更具体的物象都有各种各样的约定,所有的这些约定都是源于前人先贤们内在修身,天人感应,体察外在万物,然后解读卦象的结果。所有这些学派的发展都是遵循如下过程:\textbf{内在修身静心,诚心发问,起卦得卦象或者观察万物得卦象,用心体察解读卦象}。

这个过程是最核心本质的部分,而至于具体得出来的结论和前人总结的各个约定反倒只是一些术了。这些术你甚至可以看到有点类似于科学的统计,但离周易的本质已经很远了。

所以就以最简单的周易六十四卦预测来说,个人的修身静心和用心体察解读卦象,这个过程是缺一不可的。没有这个过程,而是试图从周易预测的各个术中寻找科学或者某种规律的东西,那就实在是南辕北辙了。

中国人的这种直觉思维模式是很有别于西方的那种逻辑思维模式的,目前来看西方的逻辑思维对于目前我们的科学贡献更大,但是也许那也只是对于宇宙存在的某一方面而言确实西方的逻辑体系说准了宇宙存在的某些东西,能够更好地对外部存在进行建模和判断。但谁也无法保证中国人的这种整体的直觉思维模式不会在下个阶段,对于宇宙存在做出更好的解释。

\section{阴阳学说}
\begin{quote}
有太极,是生两仪, 两仪生四象, 四象生八卦,八卦定吉凶,吉凶生大业。
\end{quote}


阴阳学说是如此的浅显,以至于我觉得无法再补充说些什么了,就好像就是那样,它就在哪里,存在于我们的基因中,存在于我们的记忆中,觉得不需要再多说点什么了。但实际上阴阳学说是很深的哲理,西方哲学家们从莱布尼兹到黑格尔到马克思,慢慢把辩证法发展起来付出了极大的努力,而这个努力的基石,莱布尼兹自己也承认过,是受到了中国阴阳学说的启发。

任何事物内部都存在着矛盾对立面,这个矛盾对立面就是阴和阳。然后阴阳衍生出四象。四象很是巧妙地把事物内部矛盾演化的过程进一步细化,于是可比春夏秋冬,可比市场经济繁荣危机的GDP波动图等。实际上任何事物发展演化都可以进一步细化分析,从而得到四象。

\section{八卦和物象}
\begin{quote}
乾(qián)、坤(kūn)、坎(kǎn)、离(lí)、兑(duì)、艮(gèn)、巽(xùn)、震(zhèn)
\end{quote}


假如我们问上天的问题是“是或者不是”这样简单的问题,那么是就是阳,而不是就是阴,一个爻的卦象就够用了。但某些问题必然要求具有更大的信息量,于是人们提出八卦,并将八卦和这个世界不同的万事万物的物象对应起来。读者可以随便搜一搜就可以搜到八卦和物象的映射关系,实际上我们如果问上天的问题最后答案是指明某个物的话,那么分析卦象里面的八卦对应的物象,就能够大致猜测出来上天启示你的那个物品是什么。

我对八卦和这些物象的映射关系持保留意见,可做参考吧。类似的还有后天八卦映射方位的理论,可以在你想特别预测某个方位的时候权做参考吧。

\section{如何摇卦}
蓍草占卦就不说了,要铜钱六十四卦方法因为以前有兴趣学了一下,下面说明一下。

我看到有人批判说电脑计算随机数摇卦是不准的,一定要用铜钱。我觉得说这话的人一定要用蓍草占卜,否则对不起他的纯真坚持的心。重要的是天人感应,是铜钱也好,电脑生成的随机数也好,都不过是过程是手段是术罢了,只要保证这个过程不是掩耳盗铃,就是可行的。

具体是一个爻要抛三次硬币或者生成三个0或1的随机数:

\begin{itemize}
\item 如果加和为3,也就是都是阳,则为老阳爻,本卦得阳爻,变卦得阴爻
\item 如果加和为2,也就是两阳一阴,则为少阳爻,本卦得阳爻,变卦得阳爻
\item 如果加和为1,也就是一阳两阴,则为少阴爻,本卦得阴爻,变卦得阴爻
\item 如果加和为0,也就是都是阴,则为老阴爻,本卦为阴爻,变卦为阳爻
\end{itemize}


如此重复得到周易六十四卦预测的本卦和变卦。

\section{六十四卦解卦入门}

在得到本卦和变卦之后,我们就可以查阅周易一书里面的内容了。从我个人的经验来看,变爻的一般为一个到两个,这些情况基本上是没有争议的:

\begin{itemize}
\item 无变爻 以本卦卦辞断之
\item 一个变爻 以该爻的爻辞断之
\item 两个变爻 以两个爻的爻辞共同断之,这里朱熹说 \verb+以上者断之+ ,我觉得说得不够确切,因为周易的各个爻辞明显可以看出是在讲述一个事件的发展经过的过程,所以更确切来说是这两个爻是你预测的该事件在那两个发展阶段存在变数的情况,\verb+以上为主+ 含义是以事情最终那个变化情况为主,但我觉得最好说成是由这两个爻的爻辞共同断之。
\item 更多的变爻的时候我会觉得这次摇卦不是很准了,那个时候我还没有接触朱熹的变爻断法,我觉得可以作为参考\footnote{但也只是参考,除开预测者的非常规预测需求,以一般的事件预测来说,我觉得还是要以本卦为主,只是事情会在很短时间内发生很多变化,所以该卦的时效性可能会很短} 。

具体朱熹关于多个变爻的断法如下所示:
\begin{quotation}
三个变爻以本卦与变卦卦辞断;本卦为贞(体),变卦为悔(用)

四个变爻以变卦之两不变爻爻辞断,但以下者为主

五个变爻以变卦之不变爻爻辞断

六个变爻以变卦之卦辞断,乾坤两卦则以「用」辞断
\end{quotation}

\end{itemize}

\section{关于本书基本术语}
经是周易最开始的内容,

具体解卦最先的是卦象,根据你提出的问题分析卦象。卦象也就是那两个八卦叠起来的象。相传卦辞是文王写的,爻辞是周公写的。然后下面还有彖(tuàn)辞、象辞是孔子和他的门人写的。


\addchtoc{目录}
\setcounter{tocdepth}{2}    
\tableofcontents



\mainmatter
\part{周易}

\chapter{乾卦}
\section{原文}
乾  {\Large ䷀}


\subsection{经}
乾:元亨,利贞。

初九:潜龙,勿用。

九二:见龙在田,利见大人。

九三:君子终日乾乾,夕惕若,厉,无咎。

九四:或跃在渊,无咎。

九五:飞龙在天,利见大人。

上九:亢龙有悔。

用九:见群龙无首,吉。


\subsection{彖}
大哉乾元,万物资始,乃统天。云行雨施,品物流形。大明始终,六位时成,时乘六龙以御天。乾道变化,各正性命,保合大和,乃利贞。首出庶物,万国咸宁。

\subsection{象}
天行健,君子以自强不息。潜龙勿用,阳在下也。见龙再田,德施普也。终日乾乾,反复道也。或跃在渊,进无咎也。飞龙在天,大人造也。亢龙有悔,盈不可久也。用九,天德不可为首也。

初九:潜龙勿用,阳在下也。

九二:见龙在田,德施普也。

九三:终日乾乾,反复道也。

九四:或跃在渊,进无咎也。

九五:飞龙在天,大人造也。

上九:亢龙有悔,盈不可久也。

用九:用九,天德不可为首也。


\chapter{坤卦}
坤 {\Large ䷁}

\section{原文}
\subsection{经}
坤:元亨,利牝马之贞。君子有攸往,先迷后得主,利西南得朋,东北丧朋。安贞,吉。

初六:履霜,坚冰至。

六二:直,方,大,不习无不利。

六三:含章可贞。 或从王事,无成有终。

六四:括囊;无咎,无誉。

六五:黄裳,元吉。

上六:龙战于野,其血玄黄。

用六:利永贞。

\subsection{彖}
至哉坤元,万物资生,乃顺承天。坤厚载物,德合无疆。含弘光大,品物咸亨。牝马地类,行地无疆,柔顺利贞。君子攸行,先迷失道,后顺得常。西南得朋,乃与类行;东北丧朋,乃终有庆。安贞之吉,应地无疆。

\subsection{象}
地势坤,君子以厚德载物。

初六:履霜坚冰,阴始凝也。驯致其道,至坚冰也。

六二:六二之动,直以方也。 不习无不利,地道光也。

六三:含章可贞;以时发也。 或从王事,知光大也。

六四:括囊无咎,慎不害也。

六五:黄裳元吉,文在中也。

上六:龙战于野,其道穷也。

用六:用六永贞,以大终也。



\chapter{水雷屯(zhūn)卦}
屯 {\Large ䷂}

\section{原文}
\subsection{经}
屯:元亨,利贞,勿用有攸往,利建侯。

初九:磐桓;利居贞,利建侯。

六二:屯如邅如,乘马班如。 匪寇婚媾,女子贞不字,十年乃字。

六三:既鹿无虞,惟入于林中,君子几不如舍,往吝。

六四:乘马班如,求婚媾,往吉,无不利。

九五:屯其膏,小贞吉,大贞凶。

上六:乘马班如,泣血涟如。

\subsection{彖}
屯,刚柔始交而难生,动乎险中,大亨贞。雷雨之动满盈,天造草昧,宜建侯而不宁。

\subsection{象}
云,雷,屯;君子以经纶。

初九:虽磐桓,志行正也。 以贵下贱,大得民也。

六二:六二之难,乘刚也。 十年乃字,反常也。

六三:既鹿无虞,以从禽也。 君子舍之,往吝穷也。

六四:求而往,明也。

九五:屯其膏,施未光也。

上六:泣血涟如,何可长也。

\section{讲解}
卦辞说:屯卦,大亨通,利于坚守正道。不要有所前往,适宜建国立候。【说的再直白就是屯卦还是很吉祥的,也利于君子去坚守正道,只是不要轻举妄动,适合做一些大事情的筹备工作。】

彖说:屯卦,阳刚之气和阴柔之气刚开始交汇,困难也随之而生,危险也在其中涌动,但终是大亨通,利贞的。雷雨之动满布大地,而上天的造化开物还处于蒙昧状态,适宜建国立候而不宁。

象辞说:屯卦上面是云下面是雷,君子观此象而经纶天下。

【屯卦指事情才刚开始,寓意天地也是云雷互动而困难重重,此时不要轻举妄动,但也不要倦怠安宁,而应该效法此时云雷互动中上天经纬造化万物之象,立大志,做大梦,定好大的方向和规划。经天纬地,如此安能不是大亨通和利于坚守贞正之道。】


国易堂说:在求名方面,具有坚忍不拔的毅力和锲而不舍的奋斗精神,前途不可估量。若是积极争取,主动追求,可以成功。

占得此卦者,不宜外出,宜改日动身。

在婚恋方面,虽然会出现挫折,但是好事多磨,如果大胆追求,能够成功。



\chapter{山水蒙卦}
蒙 {\Large ䷃}

\section{原文}
\subsection{经}
蒙:亨。 匪我求童蒙,童蒙求我。 初噬告,再三渎,渎则不告。利贞。

初六:发蒙,利用刑人,用说桎梏,以往吝。

九二:包蒙吉;纳妇吉;子克家。

六三:勿用娶女;见金夫,不有躬,无攸利。

六四:困蒙,吝。

六五:童蒙,吉。

上九:击蒙;不利为寇,利御寇。

\subsection{彖}
蒙,山下有险,险而止,蒙。 蒙亨,以亨行时中也。匪我求童蒙,童蒙求我,志应也。初噬告,以刚中也。再三渎, 渎则不告,渎蒙也。 蒙以养正,圣功也。


\subsection{象}
山下出泉,蒙;君子以果行育德。

初六:利用刑人,以正法也。

九二:子克家,刚柔节(节=接)也。

六三:勿用娶女,行不顺也。

六四:困蒙之吝,独远实也。

六五:童蒙之吉,顺以巽(巽=逊)也。

上九:利用御寇,上下顺也。


\chapter{水天需卦}
需 {\Large ䷄}

\section{原文}

\subsection{经}
需:有孚,光亨,贞吉。 利涉大川。

初九:需于郊。 利用恒,无咎。

九二:需于沙。 小有言,终吉。

九三:需于泥,致寇至。

六四:需于血,出自穴。

九五:需于酒食,贞吉。

上六:入于穴,有不速之客三人来,敬之终吉。

\subsection{彖}
需,须也;险在前也。刚健而不陷,其义不困穷矣。需有孚,光亨,贞吉。位乎天位,以正中也。 利涉大川,往有功也。

\subsection{象}
云上于天,需;君子以饮食宴乐。

初九:需于郊,不犯难行也。 利用恒,无咎;未失常也。

九二:需于沙,衍在中也。 虽小有言,以吉终也。

九三:需于泥,灾在外也。 自我致寇,敬慎不败也。

六四:需于血,顺以听也。

九五:酒食贞吉,以中正也。

上六:不速之客来,敬之终吉。 虽不当位,未大失也。


\chapter{天水讼卦}
讼 {\Large ䷅}

\section{原文}
\subsection{经}
讼:有孚,窒惕,中吉。终凶。利见大人,不利涉大川。

初六:不永所事,小有言,终吉。

九二:不克讼,归而逋其邑,人三百户,无眚。

六三:食旧德,贞厉,终吉,或从王事,无成。

九四:不克讼,复自命,渝,安贞,吉。

九五:讼元吉。

上九:或锡之鞶带,终朝三褫之。

\subsection{彖}
讼,上刚下险,险而健,讼。讼有孚,窒惕中吉,刚来而得中也。终凶;讼不可成也。 利见大人;尚中正也。不利涉大川;入于渊也。

\subsection{象}
天与水违行,讼;君子以作事谋始。

初六:不永所事,讼不可长也。虽有小言,其辩明也。

九二:不克讼,归而逋归逋窜也。自下讼上,患至掇也。

六三:食旧德,从上吉也。

九四:复即命渝,安贞;不失也。

九五:讼元吉,以中正也。

上九:以讼受服,亦不足敬也。

\chapter{地水师卦}
师 {\Large ䷆}

\section{原文}
\subsection{经}
师:贞,丈人吉,无咎。

初六:师出以律,否臧凶。

九二:在师中,吉无咎,王三锡命。

六三:师或舆尸,凶。

六四:师左次,无咎。

六五:田有禽,利执言,无咎。长子帅师,弟子舆尸,贞凶。

上六:大君有命,开国承家,小人勿用。

\subsection{彖}
师,众也,贞正也,能以众正,可以王矣。刚中而应,行险而顺,以此毒天下,而民从之,吉又何咎矣。

\subsection{象}
地中有水,师;君子以容民畜众。

初六:师出以律,失律凶也。

九二:在师中吉,承天宠也。王三锡命,怀万邦也。

六三:师或舆尸,大无功也。

六四:左次无咎,未失常也。

六五:长子帅师,以中行也。弟子舆尸,使不当也。

上六:大君有命,以正功也。小人勿用,必乱邦也。


\chapter{水地比卦}
比 {\Large ䷇}

\section{原文}
\subsection{经}
比:吉。原筮元永贞,无咎。不宁方来,后夫凶。

初六:有孚比之,无咎。有孚盈缶,终来有它,吉。

六二:比之自内,贞吉。

六三:比之匪人。

六四:外比之,贞吉。

九五:显比,王用三驱,失前禽。 邑人不诫,吉。

上六:比之无首,凶。

\subsection{彖}
比,吉也,比,辅也,下顺从也。原筮元永贞,无咎,以刚中也。不宁方来,上下应也。后夫凶,其道穷也。

\subsection{象}
地上有水,比;先王以建万国,亲诸侯。

初六:比之初六,有它吉也。

六二:比之自内,不自失也。

六三:比之匪人,不亦伤乎!

六四:外比於贤,以从上也。

九五:显比之吉,位正中也。舍逆取顺,失前禽也。 邑人不诫,上使中也。

上六:比之无首,无所终也。



\chapter{风天小畜(xù)卦}
小畜 {\Large ䷈}

\section{原文}

\subsection{经}
小畜:亨。 密云不雨,自我西郊。

初九:复自道,何其咎,吉。

九二:牵复,吉。

九三:舆说辐,夫妻反目。

六四:有孚,血去惕出,无咎。

九五:有孚挛如,富以其邻。

上九:既雨既处,尚德载,妇贞厉。 月几望,君子征凶

\subsection{彖}
小畜;柔得位,而上下应之,曰小畜。 健而巽,刚中而志行,乃亨。 密云不雨,尚往也。 自我西郊,施未行也。

\subsection{象}
风行天上,小畜;君子以懿文德。

初九:复自道,其义吉也。

九二:牵复在中,亦不自失也。

九三:夫妻反目,不能正室也。

六四:有孚惕出,上合志也。

九五:有孚挛如,不独富也。

上九:既雨既处,德积载也。 君子征凶,有所疑也。


\chapter{天泽履(lǚ)卦}
履 {\Large ䷉}

\section{原文}

\subsection{经}
履:履虎尾,不咥人,亨。

初九:素履往,无咎。

九二:履道坦坦,幽人贞吉。

六三:眇能视,跛能履,履虎尾,咥人,凶。武人为于大君。

九四:履虎尾,愬愬,终吉。

九五:夬履,贞厉。

上九:视履考祥,其旋元吉。

\subsection{彖}
履,柔履刚也。说而应乎乾,是以履虎尾,不咥人,亨。刚中正,履帝位而不疚,光明也。

\subsection{象}
上天下泽,履;君子以辩上下,定民志。

初九:素履之往,独行愿也。

九二:幽人贞吉,中不自乱也。

六三:眇能视;不足以有明也。跛能履;不足以与行也。咥人之凶;位不当也。武人为于大君;志刚也。

九四:愬愬终吉,志行也。

九五:夬履贞厉,位正当也。

上九:元吉在上,大有庆也。


\chapter{地天泰卦}
泰 {\Large ䷊}

\section{原文}

\subsection{经}
泰:小往大来,吉亨。

初九:拔茅茹,以其汇,征吉。

九二:包荒,用冯河,不遐遗,朋亡,得尚于中行。

九三:无平不陂,无往不复,艰贞无咎。 勿恤其孚,于食有福。

六四:翩翩不富,以其邻,不戒以孚。

六五:帝乙归妹,以祉元吉。

上六:城复于隍,勿用师。 自邑告命,贞吝。

\subsection{彖}
泰,小往大来,吉亨。则是天地交,而万物通也;上下交,而其志同也。内阳而外阴,内健而外顺,内君子而外小人,君子道长,小人道消也。

\subsection{象}
天地交,泰,后以财(=裁)成天地之道,辅相天地之宜,以左右民。

初九:拔茅征吉,志在外也。

九二:包荒,得尚于中行,以光大也。

九三:无往不复,天地际也。

六四:翩翩不富,皆失实也。不戒以孚,中心愿也。

六五:以祉元吉,中以行愿也。

上六:城复于隍,其命乱也。


\chapter{天地否(pǐ)卦}
否 {\Large ䷋}

\section{原文}

\subsection{经}
否:否之匪人,不利君子贞,大往小来。

初六:拔茅茹,以其汇,贞吉亨。

六二:包承。小人吉,大人否亨。

六三:包羞。

九四:有命无咎,畴离祉。

九五:休否,大人吉。其亡其亡,系于苞桑。

上九:倾否,先否后喜。

\subsection{彖}
否之匪人,不利君子贞。大往小来,则是天地不交,而万物不通也;上下不交,而天下无邦也。内阴而外阳,内柔而外刚,内小人而外君子。小人道长,君子道消也。

\subsection{象}
天地不交,否;君子以俭德辟难,不可荣以禄。

初六:拔茅贞吉,志在君也。

六二:大人否亨,不乱群也。

六三:包羞,位不当也。

九四:有命无咎,志行也。

九五:大人之吉,位正当也。

上九:否终则倾,何可长也。


\chapter{天火同人卦}
同人 {\Large ䷌}

\section{原文}

\subsection{经}
同人:同人于野,亨。 利涉大川,利君子贞。

初九:同人于门,无咎。

六二:同人于宗,吝。

九三:伏戎于莽,升其高陵,三岁不兴。

九四:乘其墉,弗克攻,吉。

九五:同人,先号啕而后笑。大师克相遇。

上九:同人于郊,无悔。

\subsection{彖}
同人,柔得位得中,而应乎乾,曰同人。同人曰,同人于野,亨。利涉大川,乾行也。文明以健,中正而应,君子正也。 唯君子为能通天下之志。

\subsection{象}
天与火,同人;君子以类族辨物。

初九:出门同人,又谁咎也。

六二:同人于宗,吝道也。

九三:伏戎于莽,敌刚也。三岁不兴,安行也。

九四:乘其墉,义弗克也,其吉,则困而反则也。

九五:同人之先,以中直也。大师相遇,言相克也。

上九:同人于郊,志未得也。


\chapter{火天大有卦}
大有 {\Large ䷍}

\section{原文}

\subsection{经}
大有:元亨。

初九:无交害,匪咎,艰则无咎。

九二:大车以载,有攸往,无咎。

九三:公用亨于天子,小人弗克。

九四:匪其彭,无咎。

六五:厥孚交加,威如;吉。

上九:自天祐之,吉无不利。

\subsection{彖}
大有,柔得尊位,大中而上下应之,曰大有。其德刚健而文明,应乎天而时行,是以元亨。

\subsection{象}
火在天上,大有;君子以遏恶扬善,顺天休命。

初九:大有初九,无交害也。

九二:大车以载,积中不败也。

九三:公用亨于天子,小人害也。

九四:匪其彭,无咎;明辨晢也。

六五:厥孚交加,信以发志也。威如之吉,易而无备也。

上九:大有上吉,自天祐也。


\chapter{地山谦卦}
谦 {\Large ䷎}

\section{原文}

\subsection{经}
谦:亨,君子有终。

初六:谦谦君子,用涉大川,吉。

六二:鸣谦,贞吉。

九三:劳谦,君子有终,吉。

六四:无不利,撝谦。

六五:不富,以其邻,利用侵伐,无不利。

上六:鸣谦,利用行师,征邑国。

\subsection{彖}
谦,亨,天道下济而光明,地道卑而上行。天道亏盈而益谦,地道变盈而流谦,鬼神害盈而福谦,人道恶盈而好谦。谦尊而光,卑而不可逾,君子之终也。

\subsection{象}
地中有山,谦;君子以裒多益寡,称物平施。

初六:谦谦君子,卑以自牧也。

六二:鸣谦贞吉,中心得也。

九三:劳谦君子,万民服也。

六四:无不利,撝谦;不违则也。

六五:利用侵伐,征不服也。

上六:鸣谦,志未得也。 可用行师,征邑国也。


\chapter{雷地豫卦}
豫 {\Large ䷏}

\section{原文}

\subsection{经}
豫:利建侯行师。

初六:鸣豫,凶。

六二:介于石,不终日,贞吉。

六三:盱豫,悔。迟有悔。

九四:由豫,大有得。勿疑。朋盍簪。

六五:贞疾,恒不死。

上六:冥豫成,有渝,无咎。

\subsection{彖}
豫,刚应而志行,顺以动,豫。豫,顺以动,故天地如之,而况建侯行师乎?天地以顺动,故日月不过,而四时不忒;圣人以顺动,则刑罚清而民服。 豫之时义大矣哉!

\subsection{象}
雷出地奋,豫。 先王以作乐崇德,殷荐之上帝,以配祖考。

初六:初六鸣豫,志穷凶也。

六二:不终日,贞吉;以中正也。

六三:盱豫有悔,位不当也。

九四:由豫,大有得;志大行也。

六五:六五贞疾,乘刚也。恒不死,中未亡也。

上六:冥豫在上,何可长也。


\chapter{泽雷随卦}
随 {\Large ䷐}

\section{原文}

\subsection{经}
随:元亨利贞,无咎。

初九:官有渝,贞吉。 出门交有功。

六二:系小子,失丈夫。

六三:系丈夫,失小子。 随有求得,利居贞。

九四:随有获,贞凶。有孚在道,以明,何咎。

九五:孚于嘉,吉。

上六:拘系之,乃从维之。 王用亨于西山。

\subsection{彖}
随,刚来而下柔,动而说,随。大亨贞,无咎,而天下随时,随时之义大矣哉!

\subsection{象}
泽中有雷,随;君子以向晦入宴息。

初九:官有渝,从正吉也。 出门交有功,不失也。

六二:系小子,弗兼与也。

六三:系丈夫,志舍下也。

九四:随有获,其义凶也。 有孚在道,明功也。

九五:孚于嘉,吉;位正中也。

上六:拘系之,上穷也。

\chapter{山风蛊(gǔ)卦}
蛊 {\Large ䷑}

\section{原文}

\subsection{经}
蛊:元亨,利涉大川。先甲三日,后甲三日。

初六:干父之蛊,有子,考无咎,厉终吉。

九二:干母之蛊,不可贞。

九三:干父之蛊,小有悔,无大咎。

六四:裕父之蛊,往见吝。

六五:干父之蛊,用誉。

上九:不事王侯,高尚其事。

\subsection{彖}
蛊,刚上而柔下,巽而止,蛊。蛊,元亨,而天下治也。利涉大川,往有事也。先甲三日,后甲三日,终则有始,天行也。

\subsection{象}
山下有风,蛊;君子以振民育德。

初六:干父之蛊,意承考也。

九二:干母之蛊,得中道也。

九三:干父之蛊,终无咎也。

六四:裕父之蛊,往未得也。

六五:干父之蛊;承以德也。

上九:不事王侯,志可则也。

\chapter{地泽临卦}
临 {\Large ䷒}

\section{原文}

\subsection{经}
临:元,亨,利,贞。 至于八月有凶。

初九:咸临,贞吉。

九二:咸临,吉无不利。

六三:甘临,无攸利。 既忧之,无咎。

六四:至临,无咎。

六五:知临,大君之宜,吉。

上六:敦临,吉无咎。

\subsection{彖}
临,刚浸而长。 说而顺,刚中而应,大亨以正,天之道也。 至于八月有凶,消不久也。

\subsection{象}
泽上有地,临; 君子以教思无穷,容保民无疆。

初九:咸临贞吉,志行正也。

九二:咸临,吉无不利;未顺命也。

六三:甘临,位不当也。既忧之,咎不长也。

六四:至临无咎,位当也。

六五:大君之宜,行中之谓也。

上六:敦临之吉,志在内也。


\chapter{风地观卦}
观 {\Large ䷓}

\section{原文}

\subsection{经}
观:盥而不荐,有孚顒若。

初六:童观,小人无咎,君子吝。

六二:窥观,利女贞。

六三:观我生,进退。

六四:观国之光,利用宾于王。

九五:观我生,君子无咎。

上九:观其生,君子无咎。

\subsection{彖}
大观在上,顺而巽,中正以观天下。观,盥而不荐,有孚顒若,下观而化也。观天之神道,而四时不忒, 圣人以神道设教,而天下服矣。

\subsection{象}
风行地上,观;先王以省方,观民设教。

初六:初六童观,小人道也。

六二:窥观女贞,亦可丑也。

六三:观我生,进退;未失道也。

六四:观国之光,尚宾也。

九五:观我生,观民也。

上九:观其生,志未平也。


\chapter{火雷噬嗑(shìhé)卦}
坤 ䷁

\section{原文}
\subsection{经}
\subsection{彖}
\subsection{象}


\chapter{山火贲(bì)卦}
坤 ䷁

\section{原文}
\subsection{经}
\subsection{彖}
\subsection{象}


\chapter{山地剥卦}
坤 ䷁

\section{原文}
\subsection{经}
\subsection{彖}
\subsection{象}


\chapter{地雷复卦}
坤 ䷁

\section{原文}
\subsection{经}
\subsection{彖}
\subsection{象}


\chapter{天雷无妄卦}
坤 ䷁

\section{原文}
\subsection{经}
\subsection{彖}
\subsection{象}


\chapter{山天大畜(xù)卦}
坤 ䷁

\section{原文}
\subsection{经}
\subsection{彖}
\subsection{象}


\chapter{山雷颐(yí)卦}
坤 ䷁

\section{原文}
\subsection{经}
\subsection{彖}
\subsection{象}


\chapter{泽风大过卦}
坤 ䷁

\section{原文}
\subsection{经}
\subsection{彖}
\subsection{象}


\chapter{坎卦}
坤 ䷁

\section{原文}
\subsection{经}
\subsection{彖}
\subsection{象}

\chapter{离卦}
坤 ䷁

\section{原文}
\subsection{经}
\subsection{彖}
\subsection{象}


\chapter{泽山咸卦}
坤 ䷁

\section{原文}
\subsection{经}
\subsection{彖}
\subsection{象}


\chapter{雷风恒卦}
坤 ䷁

\section{原文}
\subsection{经}
\subsection{彖}
\subsection{象}


\chapter{天山遁(dùn)卦}
坤 ䷁

\section{原文}
\subsection{经}
\subsection{彖}
\subsection{象}


\chapter{雷天大壮卦}
坤 ䷁

\section{原文}
\subsection{经}
\subsection{彖}
\subsection{象}


\chapter{火地晋卦}
坤 ䷁

\section{原文}
\subsection{经}
\subsection{彖}
\subsection{象}


\chapter{地火明夷(yí)卦}
坤 ䷁

\section{原文}
\subsection{经}
\subsection{彖}
\subsection{象}


\chapter{风火家人卦}
坤 ䷁

\section{原文}
\subsection{经}
\subsection{彖}
\subsection{象}


\chapter{火泽睽(kuí)卦}
坤 ䷁

\section{原文}
\subsection{经}
\subsection{彖}
\subsection{象}


\chapter{水山蹇(jiǎn)卦}
坤 ䷁

\section{原文}
\subsection{经}
\subsection{彖}
\subsection{象}

\chapter{雷水解(xiè)卦}
\section{原文}
\subsection{经}
\subsection{彖}
\subsection{象}

\chapter{山泽损卦}
\section{原文}
\subsection{经}
\subsection{彖}
\subsection{象}

\chapter{风雷益卦}
\section{原文}
\subsection{经}
\subsection{彖}
\subsection{象}

\chapter{泽天夬(guài)卦}
\section{原文}
\subsection{经}
\subsection{彖}
\subsection{象}



\chapter{天风姤(gòu)卦}
\section{原文}
\subsection{经}
\subsection{彖}
\subsection{象}

\chapter{泽地萃(cuì)卦}
\section{原文}
\subsection{经}
\subsection{彖}
\subsection{象}

\chapter{地风升卦}
\section{原文}
\subsection{经}
\subsection{彖}
\subsection{象}

\chapter{泽水困卦}
\section{原文}
\subsection{经}
\subsection{彖}
\subsection{象}

\chapter{水风井卦}
\section{原文}
\subsection{经}
\subsection{彖}
\subsection{象}

\chapter{泽火革卦}
革 {\Large ䷰}
\section{原文}

\subsection{经}
革:巳日乃孚,元亨利贞,悔亡。

初九:巩用黄牛之革。

六二:巳日乃革之,征吉,无咎。

九三:征凶,贞厉,革言三就,有孚。

九四:悔亡,有孚改命,吉。

九五:大人虎变,未占有孚。

上六:君子豹变,小人革面,征凶,居贞吉。

\subsection{彖}
革,水火相息,二女同居,其志不相得,曰革。巳日乃孚;革而信之。文明以说,大亨以正,革而当,其悔乃亡。天地革而四时成,汤武革命,顺乎天而应乎人,革之时大矣哉!

\subsection{象}
泽中有火,革;君子以治历明时。

初九:巩用黄牛,不可以有为也。

六二:巳日革之,行有嘉也。

九三:革言三就,又何之矣。

九四:改命之吉,信志也。

九五:大人虎变,其文炳也。

上六:君子豹变,其文蔚也。小人革面,顺以从君也。


\chapter{火风鼎卦}
鼎 {\large ䷱}
\section{原文}

\subsection{经}
鼎:元吉,亨。

初六:鼎颠趾,利出否,得妾以其子,无咎。

九二:鼎有实,我仇有疾,不我能即,吉。

九三:鼎耳革,其行塞,雉膏不食,方雨亏悔,终吉。

九四:鼎折足,覆公餗,其形渥,凶。

六五:鼎黄耳金铉,利贞。

上九:鼎玉铉,大吉,无不利。

\subsection{彖}
鼎,象也。以木巽火,亨饪也。圣人亨以享上帝,而大亨以养圣贤。巽而耳目聪明,柔进而上行,得中而应乎刚,是以元亨。

\subsection{象}
木上有火,鼎;君子以正位凝命。

初六:鼎颠趾,未悖也。利出否,以从贵也。

九二:鼎有实,慎所之也。我仇有疾,终无尤也。

九三:鼎耳革,失其义也。

九四:覆公餗,信如何也。

六五:鼎黄耳,中以为实也。

上九:玉铉在上,刚柔节也。

\section{讲解}
卦辞说:鼎卦,大吉祥,亨通。

彖说:鼎卦的形状就像一个鼎,因为巽为木,离为火,以木生火烧鼎,可以进行烹饪之事。圣人用鼎烹饪用来祭拜上天,而大量的烹饪食物是用来供养圣贤。巽为风无孔不入表明此人耳目聪明,风很柔软表明此人柔顺上进,只要行动适中,就会受到刚强者的响应,因此卦辞上说元亨。

象辞说:木上有火,便是鼎卦的卦象。君子从这个卦象得到启发,应当端正自己的位置,重视上天赋予的使命。

国易堂说:占得此卦者,已经具备开拓事业的各种条件。个人条件很好,聪明冷静,但要注意应以端正的态度去为人处世,严于律已,无轻举妄动和邪思,刚中自守。要注意培养和吸收人才,为己所用。要与有才德的人合作,这样更利成功。

在求名上,首先应严于律已,不陷入与他人的怨仇之中,柔而上行,循序渐进,具备随时应变和随势应变的能力。如果得到知人者的善用,更是前途广大。即使暂时不受重视,无出路也无防,最终可实现抱负。

在外出方面,无重大事情不宜外出,如果为了工作和事业而出差,则会很顺

在婚恋方面,应该说个人条件比较不错,但是不要自视甚高,在选择另一半时要切合自己的实际。对于已婚人士来说,要注意不要出现三角恋爱、第三者插足的情况。

在身体健康方面,因为鼎除能烹煮食物外,还像煮药的砂锅,所以自己或家人可能会有因病而吃汤药的。




\chapter{震卦}
震 {\large ䷲}
\section{原文}

\subsection{经}
震:亨。 震来虩虩,笑言哑哑。 震惊百里,不丧匕鬯。

初九:震来虩虩,后笑言哑哑,吉。

六二:震来厉,亿丧贝,跻于九陵,勿逐,七日得。

六三:震苏苏,震行无眚。

九四:震遂泥。

六五:震往来厉,亿无丧,有事。

上六:震索索,视矍矍,征凶。 震不于其躬,于其邻,无咎。 婚媾有言。

\subsection{彖}
震,亨。震来虩虩,恐致福也。笑言哑哑,后有则也。震惊百里,惊远而惧迩也。 出可以守宗庙社稷,以为祭主也。

\subsection{象}
洊雷,震;君子以恐惧修省。

初九:震来虩虩,恐致福也。笑言哑哑,后有则也。

六二:震来厉,乘刚也。

六三:震苏苏,位不当也。

九四:震遂泥,未光也。

六五:震往来厉,危行也。其事在中,大无丧也。

上六:震索索,未得中也。虽凶无咎,畏邻戒也。


\section{讲解}




\chapter{艮卦}
艮 {\Large ䷳}
\section{原文}

\subsection{经}

\subsection{彖}

\subsection{象}

\chapter{风山渐卦}
渐 {\Large ䷴}
\section{原文}

\subsection{经}

\subsection{彖}

\subsection{象}


\chapter{雷泽归妹卦}
归妹 {\Large ䷵}
\section{原文}

\subsection{经}

\subsection{彖}

\subsection{象}

\chapter{雷火丰卦}
丰 {\Large ䷶}
\section{原文}

\subsection{经}

\subsection{彖}

\subsection{象}

\chapter{火山旅卦}
旅 {\Large ䷷}
\section{原文}

\subsection{经}

\subsection{彖}

\subsection{象}

\chapter{巽(xùn)卦}
巽 {\Large ䷸}
\section{原文}

\subsection{经}

\subsection{彖}

\subsection{象}


\chapter{兑卦}
兑 {\Large ䷹}
\section{原文}

\subsection{经}

\subsection{彖}

\subsection{象}

\chapter{风水涣卦}
涣 {\Large ䷺}
\section{原文}

\subsection{经}

\subsection{彖}

\subsection{象}

\chapter{水泽节卦}
节 {\Large ䷻}
\section{原文}

\subsection{经}

\subsection{彖}

\subsection{象}

\chapter{风泽中孚(fú)卦}
中孚 {\LARGE ䷼}


\section{原文}

\subsection{经}
中孚:豚鱼吉,利涉大川,利贞。

\subsection{彖}
中孚,柔在内而刚得中。说而巽,孚,乃化邦也。豚鱼吉,信及豚鱼也。利涉大川,乘木舟虚也。中孚以利贞,乃应乎天也。

\subsection{象}
泽上有风,中孚;君子以议狱缓死。

\chapter{雷山小过卦}
小过 {\LARGE ䷽}

\section{原文}
\subsection{经}
小过:亨,利贞,可小事,不可大事。飞鸟遗之音,不宜上,宜下,大吉。

初六:飞鸟以凶。

六二:过其祖,遇其妣;不及其君,遇其臣;无咎。

九三:弗过防之,从或戕之,凶。

九四:无咎,弗过遇之。 往厉必戒,勿用永贞。

六五:密云不雨,自我西郊,公弋取彼在穴。

上六:弗遇过之,飞鸟离之,凶,是谓灾眚。

\subsection{彖}
小过,小者过而亨也。过以利贞,与时行也。柔得中,是以小事吉也。刚失位而不中,是以不可大事也。有飞鸟之象焉,有飞鸟遗之音,不宜上宜下,大吉;上逆而下顺也。

\subsection{象}
山上有雷,小过;君子以行过乎恭,丧过乎哀,用过乎俭。

初六:飞鸟以凶,不可如何也。

六二:不及其君,臣不可过也。

九三:从或戕之,凶如何也。

九四:弗过遇之,位不当也。 往厉必戒,终不可长也。

六五:密云不雨,已上也。

上六:弗遇过之,已亢也。

\section{释义}
序卦曰:“有其信者,必行之,故受之以小过。” 。有诚信的人必然会行动,因故而得小过卦。

象曰:“山上有雷,小过;”。这里指出小过的卦象就是山上有雷。山上有雷,雷会把山上的树木击坏,但没有雷怎有雨,没有雨山上的树木又怎得滋润和生长。

象曰:“君子以行过乎恭,丧过乎哀,用过乎俭。”。指的是所以君子参见小过的卦象之后应该行为上过于恭敬,丧事上过于悲哀,日用上过于节俭。此为小小过分之理也。

经谈“飞鸟”,大概因为小过卦从卦象看像一只展翅的飞鸟。

小过,亨通,利于坚守中正之道,可以做一些小事,不要去做大事。飞鸟飞过留下声音,不宜向上飞,宜于向下飞,如此则大为吉祥。因为上飞




\chapter{水火既济卦}
既济 {\Large ䷾}
\section{原文}

\subsection{经}

\subsection{彖}

\subsection{象}

\chapter{火水未济卦}
未济 {\Large ䷿}
\section{原文}

\subsection{经}

\subsection{彖}

\subsection{象}





\part{周易相关}
\chapter{基本术语}
\section{易经}
易经原有三,连山易,归藏易,周易,前两易已失传,

\section{爻的当位和不当位}
认为爻位从下往上数,奇数为阳,偶数为阴,于是奇数位为阳位,偶数位为阴位,若阳爻居阳位则为当位,若阴爻居阴位也为当位,反之为不当位。


\section{变爻}
爻分为二,若为少阴少阳则该爻不动。若为老阴老阳则该爻为变爻,变动之爻。

\section{先天六十四卦顺序和后天六十四卦顺序}
周易一书的顺序或者说按照序卦传而来的顺序通常被人们称为后天六十四卦顺序,这个顺序更多的反应了作者认为事物发展的一种哲理性解释。

通常预测会按照先天六十四卦顺序来,先天六十四卦顺序就是在变爻到六爻的阶段,最终发展到不可逆转进而形成变卦。先天六十四卦更多的是揭示天理自然规律,而后天六十四严格意义上来说并没有顺序一说,只是方便大家阅读而解释出来的那个顺序。

\section{尊卑贵贱上下}
周易里面尊卑,贵贱,上下各自是分开的,虽然人们常谈尊贵,卑下,下贱,但正所谓上位也可能不尊,下位也有高贵之民,看看周易里面系辞部分说的很清楚:“天尊地卑,乾坤定矣。卑高以陈,贵贱位矣。”尊卑本只是无褒贬含义的高和低的意思,天高高在上,地低低在下,乾坤就这样确定下来了。然后注意下面的\emph{卑高}这个词的顺序 ,正是从人的角度去看,先看到地,再看到天,是言卑高。关于这块南怀瑾说的很好,人性就是如此,容易摸得着的东西就轻贱它,总是得不到的东西就觉得很珍贵。周易在这里说的很明白,本来天尊地卑,并没有贵贱概念,因为人性,所以出现了贵贱的概念了。

在周易里面上下有统治或管理上的上下位之分,再一次将上下和贵贱和尊卑混为一谈那是后来人的私心想法。周易里面谈上下更多是让人们注意到管理上的上下位之别,比如履卦的“君子辩上下”,其正对应孔子谈为政的核心原则就是:“君君,臣臣,父父,子子”。当然孔子的谈论更多的局限在古代社会的君臣父子这几大关系上做了比喻和延伸,就现代社会而言管理上的核心原则其实仍然是一致的,即上位要有上位的样子,下位要有下位的样子。各尽分工,各尽其职。



\section{皇极经世的时间刻度}
除去乾坤坎离四卦的先天六十四卦按照顺序每一卦管六运总共三百六十运。具体就是六十卦按照变爻从初爻变动到六爻分为六个阶段也就是这六个卦,这样三百六十运就分别对应了三百六十个卦,这个卦叫做值运之卦。一运360年。

在此值运之卦下,继续按照变爻从初爻变动到六爻分为六个阶段从而得到值世之卦,这个值世之卦管两世也就是六十年。

将这个值世之卦按照上面讨论的六十卦顺序从第一个值世之卦开始算起,一年一卦得值年之卦。



\part{附录}
\chapter{参考资料}
\begin{itemize}
\item \href{http://www.quanxue.cn/QT_MingXiang/Index.html}{quanxue网关于周易相关资料}
\item \href{http://www.guoyi360.com/zyqs/}{guoyi360网关于周易相关资料}

\end{itemize}








% 编者:万泽
\end{document}


